% =============================================================================
% Section 8: Python Environment Setup
% Setting up uv, virtual environments, and project structure
% =============================================================================

\section{Python Environment Setup}

% -----------------------------------------------------------------------------
% Opening
% -----------------------------------------------------------------------------
\begin{frame}{Python Environment Setup}
    \begin{columns}[T]
        \begin{column}{0.5\textwidth}
            \begin{funnybox}
                \textit{``Before we teach a machine to learn, we must first teach our computer to find Python. And not just any Python --- the RIGHT Python!''}
            \end{funnybox}
            
            \vspace{0.3cm}
            
            \textbf{Why environments matter:}
            \begin{itemize}
                \item Different projects need different packages
                \item Version conflicts are REAL
                \item Reproducibility is essential
            \end{itemize}
        \end{column}
        \begin{column}{0.5\textwidth}
            \centering
            \begin{tikzpicture}[scale=0.7]
                % Computer
                \draw[thick,accentcyan,rounded corners] (0,0) rectangle (4,3);
                \node[accentcyan] at (2,2.5) {Your Computer};
                
                % Virtual envs
                \draw[thick,neonpink,rounded corners] (0.3,0.3) rectangle (1.8,1.8);
                \node[neonpink,font=\tiny] at (1.05,1.5) {Project A};
                \node[font=\tiny,fgwhite] at (1.05,1.0) {numpy 1.21};
                \node[font=\tiny,fgwhite] at (1.05,0.6) {torch 2.0};
                
                \draw[thick,neongreen,rounded corners] (2.2,0.3) rectangle (3.7,1.8);
                \node[neongreen,font=\tiny] at (2.95,1.5) {Project B};
                \node[font=\tiny,fgwhite] at (2.95,1.0) {numpy 1.24};
                \node[font=\tiny,fgwhite] at (2.95,0.6) {torch 1.9};
            \end{tikzpicture}
            
            \vspace{0.2cm}
            \textit{Isolated environments = no conflicts!}
        \end{column}
    \end{columns}
\end{frame}

% -----------------------------------------------------------------------------
% Why UV?
% -----------------------------------------------------------------------------
\begin{frame}{Enter: \texttt{uv} --- The Fast Python Package Manager}
    \begin{defbox}[What is uv?]
        \texttt{uv} is an extremely fast Python package and project manager, written in Rust.
        
        \begin{itemize}
            \item 10-100x faster than pip
            \item Handles virtual environments automatically
            \item Modern project management
            \item Replaces: pip, pip-tools, virtualenv, poetry, pyenv
        \end{itemize}
    \end{defbox}
    
    \vspace{0.3cm}
    
    \begin{columns}
        \begin{column}{0.5\textwidth}
            \begin{alertbox}[Installing uv]
                \textbf{macOS/Linux:}
                
                \texttt{curl -LsSf https://astral.sh/uv/install.sh | sh}
                
                \vspace{0.2cm}
                \textbf{Windows:}
                
                \texttt{powershell -c "irm https://astral.sh/uv/install.ps1 | iex"}
            \end{alertbox}
        \end{column}
        \begin{column}{0.5\textwidth}
            \begin{funnybox}
                Why ``uv''? 
                
                Because it's so fast it operates at \textbf{UV frequencies} --- ultraviolet light! 
                
                (Actually, it's just a cool name.)
            \end{funnybox}
        \end{column}
    \end{columns}
\end{frame}

% -----------------------------------------------------------------------------
% uv init
% -----------------------------------------------------------------------------
\begin{frame}{Creating a New Project: \texttt{uv init}}
    \begin{defbox}[uv init]
        Creates a new Python project with proper structure:
        
        \texttt{uv init my\_ml\_project}
        
        \texttt{cd my\_ml\_project}
    \end{defbox}
    
    \vspace{0.3cm}
    
    \begin{columns}
        \begin{column}{0.5\textwidth}
            \textbf{What gets created:}
            
            \texttt{my\_ml\_project/}
            
            \quad \texttt{+-- .python-version}
            
            \quad \texttt{+-- pyproject.toml}
            
            \quad \texttt{+-- README.md}
            
            \quad \texttt{+-- hello.py}
        \end{column}
        \begin{column}{0.5\textwidth}
            \begin{keybox}[Key Files]
                \begin{itemize}
                    \item \texttt{.python-version}: Python version
                    \item \texttt{pyproject.toml}: Project config
                    \item \texttt{README.md}: Documentation
                    \item \texttt{hello.py}: Sample code
                \end{itemize}
            \end{keybox}
        \end{column}
    \end{columns}
\end{frame}

% -----------------------------------------------------------------------------
% pyproject.toml
% -----------------------------------------------------------------------------
\begin{frame}{Understanding \texttt{pyproject.toml}}
    \begin{defbox}[pyproject.toml]
        The \textbf{modern standard} for Python project configuration.
        
        Defines: name, version, dependencies, tools, scripts.
    \end{defbox}
    
    \vspace{0.2cm}
    
    \begin{columns}
        \begin{column}{0.55\textwidth}
            {\small\ttfamily
            [project]\\
            name = "ml-workshop"\\
            version = "0.1.0"\\
            requires-python = ">=3.11"\\
            dependencies = [\\
            \quad "numpy>=1.24.0",\\
            \quad "torch>=2.0.0",\\
            ]\\
            \\
            {[tool.uv]}\\
            dev-dependencies = [\\
            \quad "pytest>=7.0.0",\\
            ]
            }
        \end{column}
        \begin{column}{0.45\textwidth}
            \begin{keybox}[Sections]
                \begin{itemize}
                    \item \texttt{[project]}: Metadata
                    \item \texttt{dependencies}: Runtime deps
                    \item \texttt{[tool.uv]}: uv-specific
                    \item \texttt{dev-dependencies}: Dev tools
                \end{itemize}
            \end{keybox}
        \end{column}
    \end{columns}
\end{frame}

% -----------------------------------------------------------------------------
% uv sync
% -----------------------------------------------------------------------------
\begin{frame}{Installing Dependencies: \texttt{uv sync}}
    \begin{defbox}[uv sync]
        The \textbf{magic command} that:
        \begin{itemize}
            \item Creates a virtual environment (if needed)
            \item Installs all dependencies from pyproject.toml
            \item Locks versions in \texttt{uv.lock}
        \end{itemize}
    \end{defbox}
    
    \vspace{0.3cm}
    
    \begin{columns}
        \begin{column}{0.5\textwidth}
            \textbf{Basic usage:}
            
            \texttt{\$ uv sync}
            
            \vspace{0.2cm}
            \textbf{With dev dependencies:}
            
            \texttt{\$ uv sync --dev}
            
            \vspace{0.2cm}
            \textbf{Add new package:}
            
            \texttt{\$ uv add pandas}
        \end{column}
        \begin{column}{0.5\textwidth}
            \begin{keybox}[Lock File]
                \texttt{uv.lock} ensures:
                \begin{itemize}
                    \item Exact reproducibility
                    \item Same versions everywhere
                    \item Commit to git!
                \end{itemize}
            \end{keybox}
        \end{column}
    \end{columns}
\end{frame}

% -----------------------------------------------------------------------------
% Running Python with uv
% -----------------------------------------------------------------------------
\begin{frame}{Running Python with \texttt{uv}}
    \begin{defbox}[uv run]
        Run commands in the project's virtual environment:
        
        \texttt{uv run python script.py}
        
        \texttt{uv run pytest}
        
        \texttt{uv run jupyter notebook}
    \end{defbox}
    
    \vspace{0.3cm}
    
    \begin{columns}
        \begin{column}{0.5\textwidth}
            \begin{alertbox}[No Activation Needed!]
                Unlike traditional venvs:
                \begin{itemize}
                    \item No \texttt{source .venv/bin/activate}
                    \item No \texttt{deactivate}
                    \item Just prefix with \texttt{uv run}
                \end{itemize}
            \end{alertbox}
        \end{column}
        \begin{column}{0.5\textwidth}
            \begin{funnybox}
                Think of \texttt{uv run} as a magic portal that teleports your command into the right Python universe!
            \end{funnybox}
        \end{column}
    \end{columns}
\end{frame}

% -----------------------------------------------------------------------------
% Workshop Setup
% -----------------------------------------------------------------------------
\begin{frame}{Setting Up Our Workshop Environment}
    \begin{keybox}[Step by Step]
        \begin{enumerate}
            \item Install uv (if not done)
            \item Clone/create workshop project
            \item Run \texttt{uv sync}
            \item Start coding!
        \end{enumerate}
    \end{keybox}
    
    \vspace{0.3cm}
    
    \begin{columns}
        \begin{column}{0.5\textwidth}
            \textbf{Our dependencies:}
            \begin{itemize}
                \item \texttt{numpy} --- numerical computing
                \item \texttt{torch} --- deep learning
                \item \texttt{matplotlib} --- plotting
                \item \texttt{scikit-learn} --- ML algorithms
                \item \texttt{jupyter} --- notebooks
            \end{itemize}
        \end{column}
        \begin{column}{0.5\textwidth}
            \centering
            \begin{tikzpicture}[scale=0.8]
                \node[rectangle,draw=accentcyan,fill=darkgray,minimum width=2cm,minimum height=0.6cm] (uv) at (0,2) {\texttt{uv}};
                \node[rectangle,draw=neonpink,fill=darkgray,minimum width=1.5cm,minimum height=0.5cm] (np) at (-1.5,0) {numpy};
                \node[rectangle,draw=neonpink,fill=darkgray,minimum width=1.5cm,minimum height=0.5cm] (torch) at (0,0) {torch};
                \node[rectangle,draw=neonpink,fill=darkgray,minimum width=1.5cm,minimum height=0.5cm] (mpl) at (1.5,0) {mpl};
                
                \draw[->,accentcyan] (uv) -- (np);
                \draw[->,accentcyan] (uv) -- (torch);
                \draw[->,accentcyan] (uv) -- (mpl);
            \end{tikzpicture}
            
            \textit{uv manages everything!}
        \end{column}
    \end{columns}
\end{frame}

% -----------------------------------------------------------------------------
% Quick Reference
% -----------------------------------------------------------------------------
\begin{frame}{Quick Reference: Essential \texttt{uv} Commands}
    \begin{columns}[T]
        \begin{column}{0.5\textwidth}
            \begin{defbox}[Project Setup]
                \texttt{uv init} --- Create project
                
                \texttt{uv sync} --- Install dependencies
                
                \texttt{uv add PKG} --- Add package
                
                \texttt{uv remove PKG} --- Remove package
            \end{defbox}
        \end{column}
        \begin{column}{0.5\textwidth}
            \begin{defbox}[Running Code]
                \texttt{uv run python X.py}
                
                \texttt{uv run pytest}
                
                \texttt{uv run jupyter notebook}
                
                \texttt{uv run CMD}
            \end{defbox}
        \end{column}
    \end{columns}
    
    \vspace{0.3cm}
    
    \begin{keybox}[Remember]
        \begin{itemize}
            \item Always use \texttt{uv run} to execute Python in the project environment
            \item Commit both \texttt{pyproject.toml} AND \texttt{uv.lock} to version control
            \item Run \texttt{uv sync} after pulling changes
        \end{itemize}
    \end{keybox}
\end{frame}

% -----------------------------------------------------------------------------
% Summary
% -----------------------------------------------------------------------------
\begin{frame}{Python Setup: Summary}
    \begin{columns}[T]
        \begin{column}{0.5\textwidth}
            \begin{keybox}[What We Learned]
                \begin{itemize}
                    \item Why virtual environments matter
                    \item How to use \texttt{uv} for fast package management
                    \item Project structure with \texttt{pyproject.toml}
                    \item Running code with \texttt{uv run}
                \end{itemize}
            \end{keybox}
        \end{column}
        \begin{column}{0.5\textwidth}
            \begin{funnybox}
                Now your Python environment is ready for machine learning!
                
                Let's make those neurons fire!
            \end{funnybox}
        \end{column}
    \end{columns}
    
    \vspace{0.5cm}
    
    \centering
    \textbf{Next: Neural Networks!}
\end{frame}
