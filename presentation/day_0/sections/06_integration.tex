% =============================================================================
% Section 6: Integration
% The inverse of differentiation - finding areas and totals
% =============================================================================

\section{Integration}

% -----------------------------------------------------------------------------
% Opening
% -----------------------------------------------------------------------------
\begin{frame}{Integration: Adding Infinitely Many Infinitely Small Pieces}
    \begin{columns}[T]
        \begin{column}{0.5\textwidth}
            \begin{funnybox}
                \textit{``Integration is like eating pizza: one slice at a time, you can consume the whole thing. Except with integration, the slices are infinitely thin!''}
            \end{funnybox}
            
            \vspace{0.3cm}
            
            \textbf{Two Big Ideas:}
            \begin{enumerate}
                \item Finding areas under curves
                \item Reversing differentiation
            \end{enumerate}
        \end{column}
        \begin{column}{0.5\textwidth}
            \centering
            \begin{tikzpicture}[scale=0.8]
                \draw[<->,thick,softgray] (-0.5,0) -- (4,0) node[right] {$x$};
                \draw[<->,thick,softgray] (0,-0.5) -- (0,2.5) node[above] {$y$};
                
                % Fill area
                \fill[accentcyan,opacity=0.3] (0.5,0) -- plot[domain=0.5:3.2,smooth] (\x,{0.3*\x*\x - 0.5*\x + 1}) -- (3.2,0) -- cycle;
                
                % Curve
                \draw[accentcyan,thick,domain=0.3:3.5,smooth] plot (\x,{0.3*\x*\x - 0.5*\x + 1});
                
                % Boundaries
                \draw[neonpink,dashed] (0.5,0) -- (0.5,0.825);
                \draw[neonpink,dashed] (3.2,0) -- (3.2,2.468);
                
                \node[below,fgwhite] at (0.5,0) {$a$};
                \node[below,fgwhite] at (3.2,0) {$b$};
                \node[accentcyan] at (1.8,0.8) {Area};
            \end{tikzpicture}
        \end{column}
    \end{columns}
\end{frame}

% -----------------------------------------------------------------------------
% Riemann Sums
% -----------------------------------------------------------------------------
\begin{frame}{Riemann Sums: Approximating Area}
    \centering
    \begin{tikzpicture}[scale=0.9]
        % Axes
        \draw[<->,thick,softgray] (-0.5,0) -- (5,0) node[right] {$x$};
        \draw[<->,thick,softgray] (0,-0.5) -- (0,3) node[above] {$y$};
        
        % Rectangles
        \foreach \x in {0.5,1,1.5,2,2.5,3,3.5} {
            \pgfmathsetmacro{\height}{0.2*\x*\x + 0.5}
            \fill[neonyellow,opacity=0.4] (\x,0) rectangle (\x+0.5,\height);
            \draw[neonyellow] (\x,0) rectangle (\x+0.5,\height);
        }
        
        % Curve
        \draw[accentcyan,thick,domain=0.3:4.2,smooth] plot (\x,{0.2*\x*\x + 0.5});
        
        % Labels
        \node[below,fgwhite] at (0.5,0) {$a$};
        \node[below,fgwhite] at (4,0) {$b$};
        
        % Delta x
        \draw[<->,neonpink] (1,-0.4) -- (1.5,-0.4) node[midway,below] {$\Delta x$};
    \end{tikzpicture}
    
    \vspace{0.3cm}
    
    \begin{defbox}[Riemann Sum]
        \[
            \text{Area} \approx \sum_{i=1}^{n} f(x_i) \cdot \Delta x
        \]
        
        where $\Delta x = \frac{b-a}{n}$ is the width of each rectangle.
    \end{defbox}
\end{frame}

% -----------------------------------------------------------------------------
% From Riemann Sum to Integral
% -----------------------------------------------------------------------------
\begin{frame}{From Rectangles to Integrals}
    \begin{columns}
        \begin{column}{0.33\textwidth}
            \centering
            \begin{tikzpicture}[scale=0.5]
                \draw[softgray] (0,0) -- (4,0);
                \draw[softgray] (0,0) -- (0,2.5);
                
                \foreach \x in {0,1,2,3} {
                    \pgfmathsetmacro{\h}{0.15*\x*\x + 0.5}
                    \fill[neonyellow,opacity=0.4] (\x,0) rectangle (\x+1,\h);
                    \draw[neonyellow] (\x,0) rectangle (\x+1,\h);
                }
                \draw[accentcyan,thick,domain=0:4,smooth] plot (\x,{0.15*\x*\x + 0.5});
                
                \node[below,fgwhite,font=\small] at (2,-0.3) {$n=4$};
            \end{tikzpicture}
        \end{column}
        \begin{column}{0.33\textwidth}
            \centering
            \begin{tikzpicture}[scale=0.5]
                \draw[softgray] (0,0) -- (4,0);
                \draw[softgray] (0,0) -- (0,2.5);
                
                \foreach \x in {0,0.5,...,3.5} {
                    \pgfmathsetmacro{\h}{0.15*\x*\x + 0.5}
                    \fill[neonyellow,opacity=0.4] (\x,0) rectangle (\x+0.5,\h);
                    \draw[neonyellow] (\x,0) rectangle (\x+0.5,\h);
                }
                \draw[accentcyan,thick,domain=0:4,smooth] plot (\x,{0.15*\x*\x + 0.5});
                
                \node[below,fgwhite,font=\small] at (2,-0.3) {$n=8$};
            \end{tikzpicture}
        \end{column}
        \begin{column}{0.33\textwidth}
            \centering
            \begin{tikzpicture}[scale=0.5]
                \draw[softgray] (0,0) -- (4,0);
                \draw[softgray] (0,0) -- (0,2.5);
                
                \fill[accentcyan,opacity=0.4] (0,0) -- plot[domain=0:4,smooth] (\x,{0.15*\x*\x + 0.5}) -- (4,0) -- cycle;
                \draw[accentcyan,thick,domain=0:4,smooth] plot (\x,{0.15*\x*\x + 0.5});
                
                \node[below,fgwhite,font=\small] at (2,-0.3) {$n \to \infty$};
            \end{tikzpicture}
        \end{column}
    \end{columns}
    
    \vspace{0.5cm}
    
    \begin{thmbox}[Definition of Definite Integral]
        \[
            \int_a^b f(x)\, dx = \lim_{n \to \infty} \sum_{i=1}^{n} f(x_i) \cdot \Delta x
        \]
    \end{thmbox}
    
    \begin{funnybox}
        As we use more and more rectangles ($n \to \infty$), we get the \textit{exact} area!
    \end{funnybox}
\end{frame}

% -----------------------------------------------------------------------------
% Integral Notation
% -----------------------------------------------------------------------------
\begin{frame}{Understanding Integral Notation}
    \[
        \int_a^b f(x)\, dx
    \]
    
    \vspace{0.3cm}
    
    \centering
    \begin{tikzpicture}[scale=0.9]
        % The integral symbol
        \node[font=\Huge,neonpink] at (0,0) {$\displaystyle\int$};
        \node[font=\small,fgwhite] at (0.3,0.8) {$b$};
        \node[font=\small,fgwhite] at (0.3,-0.8) {$a$};
        
        % Annotations
        \draw[->,neonyellow,thick] (-1,0) -- (-0.4,0);
        \node[left,neonyellow,font=\small,align=right] at (-1,0) {``Sum'' symbol\\(elongated S)};
        
        \draw[->,neongreen,thick] (0.6,0.8) -- (0.3,0.6);
        \node[right,neongreen,font=\small] at (0.6,0.8) {Upper limit};
        
        \draw[->,neongreen,thick] (0.6,-0.8) -- (0.3,-0.6);
        \node[right,neongreen,font=\small] at (0.6,-0.8) {Lower limit};
        
        % Function part
        \node[font=\Large,accentcyan] at (1.3,0) {$f(x)$};
        \draw[->,accentcyan,thick] (1.3,-0.5) -- (1.3,-0.2);
        \node[below,accentcyan,font=\small] at (1.3,-0.5) {Function (height)};
        
        % dx part
        \node[font=\Large,neonpink] at (2.3,0) {$dx$};
        \draw[->,neonpink,thick] (2.3,0.5) -- (2.3,0.2);
        \node[above,neonpink,font=\small] at (2.3,0.5) {Infinitesimal width};
    \end{tikzpicture}
    
    \vspace{0.5cm}
    
    \begin{infobox}
        The integral $\int$ is a stretched ``S'' for ``Sum'' — we're adding up infinitely many $f(x) \cdot dx$ pieces!
    \end{infobox}
\end{frame}

% -----------------------------------------------------------------------------
% Antiderivatives
% -----------------------------------------------------------------------------
\begin{frame}{Antiderivatives: Reversing Differentiation}
    \begin{defbox}[Antiderivative]
        $F(x)$ is an \glow{antiderivative} of $f(x)$ if:
        \[
            F'(x) = f(x)
        \]
        
        The antiderivative ``undoes'' differentiation!
    \end{defbox}
    
    \vspace{0.3cm}
    
    \textbf{Examples:}
    \begin{center}
        \begin{tabular}{c|c|c}
            $f(x)$ & $F(x)$ (antiderivative) & Check: $F'(x) = f(x)$? \\
            \hline
            $x^2$ & $\frac{x^3}{3}$ & $\frac{d}{dx}\left(\frac{x^3}{3}\right) = x^2$ (YES) \\
            $\cos x$ & $\sin x$ & $\frac{d}{dx}(\sin x) = \cos x$ (YES) \\
            $e^x$ & $e^x$ & $\frac{d}{dx}(e^x) = e^x$ (YES) \\
        \end{tabular}
    \end{center}
    
    \begin{alertbox}
        But wait — if $F(x)$ is an antiderivative, so is $F(x) + C$ for any constant $C$!
        
        (Because $\frac{d}{dx}(F + C) = F' + 0 = f$)
    \end{alertbox}
\end{frame}

% -----------------------------------------------------------------------------
% Indefinite Integral
% -----------------------------------------------------------------------------
\begin{frame}{Indefinite Integral}
    \begin{defbox}[Indefinite Integral]
        The \glow{indefinite integral} represents all antiderivatives:
        \[
            \int f(x)\, dx = F(x) + C
        \]
        
        where $C$ is the ``constant of integration'' (can be any number).
    \end{defbox}
    
    \vspace{0.3cm}
    
    \textbf{Basic Integrals:}
    \begin{columns}
        \begin{column}{0.5\textwidth}
            \begin{align*}
                \int x^n\, dx &= \frac{x^{n+1}}{n+1} + C \quad (n \neq -1) \\
                \int e^x\, dx &= e^x + C \\
                \int \frac{1}{x}\, dx &= \ln|x| + C
            \end{align*}
        \end{column}
        \begin{column}{0.5\textwidth}
            \begin{align*}
                \int \sin x\, dx &= -\cos x + C \\
                \int \cos x\, dx &= \sin x + C \\
                \int 1\, dx &= x + C
            \end{align*}
        \end{column}
    \end{columns}
    
    \begin{funnybox}
        Don't forget the $+C$! It's the most forgotten symbol in calculus.
    \end{funnybox}
\end{frame}

% -----------------------------------------------------------------------------
% Fundamental Theorem of Calculus Part 1
% -----------------------------------------------------------------------------
\begin{frame}{Fundamental Theorem of Calculus: Part 1}
    \begin{thmbox}[FTC Part 1]
        If $f$ is continuous on $[a, b]$ and:
        \[
            g(x) = \int_a^x f(t)\, dt
        \]
        
        Then $g$ is differentiable and:
        \[
            \glow{g'(x) = f(x)}
        \]
    \end{thmbox}
    
    \vspace{0.3cm}
    
    \begin{columns}
        \begin{column}{0.5\textwidth}
            \textbf{In words:}
            
            The derivative of the ``area so far'' function equals the original function!
            
            \vspace{0.3cm}
            
            \textbf{Symbolically:}
            \[
                \frac{d}{dx}\int_a^x f(t)\, dt = f(x)
            \]
        \end{column}
        \begin{column}{0.5\textwidth}
            \centering
            \begin{tikzpicture}[scale=0.7]
                \draw[<->,softgray] (-0.3,0) -- (4,0);
                \draw[<->,softgray] (0,-0.3) -- (0,2.5);
                
                % Area
                \fill[accentcyan,opacity=0.3] (0.5,0) -- plot[domain=0.5:2.5,smooth] (\x,{0.3*\x + 0.5}) -- (2.5,0) -- cycle;
                
                \draw[accentcyan,thick,domain=0.2:3.5,smooth] plot (\x,{0.3*\x + 0.5});
                
                % x marker
                \draw[neonpink,dashed] (2.5,0) -- (2.5,1.25);
                \node[below,fgwhite] at (2.5,0) {$x$};
                
                % g(x) label
                \node[accentcyan] at (1.5,0.5) {$g(x)$};
            \end{tikzpicture}
            
            {\small Rate of area growth = height of curve!}
        \end{column}
    \end{columns}
\end{frame}

% -----------------------------------------------------------------------------
% Fundamental Theorem of Calculus Part 2
% -----------------------------------------------------------------------------
\begin{frame}{Fundamental Theorem of Calculus: Part 2}
    \begin{thmbox}[FTC Part 2 (Evaluation Theorem)]
        If $f$ is continuous on $[a, b]$ and $F$ is any antiderivative of $f$, then:
        \[
            \mathbf{\int_a^b f(x)\, dx = F(b) - F(a)}
        \]
    \end{thmbox}
    
    \vspace{0.3cm}
    
    \textbf{This is HUGE!} Instead of computing limits of Riemann sums, just:
    \begin{enumerate}
        \item Find an antiderivative $F$
        \item Evaluate at the endpoints
        \item Subtract!
    \end{enumerate}
    
    \vspace{0.3cm}
    
    \textbf{Example:} $\displaystyle\int_0^2 x^2\, dx$
    
    Antiderivative: $F(x) = \frac{x^3}{3}$
    
    Answer: $F(2) - F(0) = \frac{8}{3} - 0 = \mathbf{\frac{8}{3}}$
\end{frame}

% -----------------------------------------------------------------------------
% FTC Connection
% -----------------------------------------------------------------------------
\begin{frame}{The Beautiful Connection}
    \centering
    \begin{tikzpicture}[scale=0.9]
        % Differentiation
        \node[rectangle,draw=neonpink,fill=darkgray,minimum width=2.5cm,minimum height=1cm] (diff) at (-2,0) {\textcolor{neonpink}{Differentiation}};
        
        % Integration
        \node[rectangle,draw=accentcyan,fill=darkgray,minimum width=2.5cm,minimum height=1cm] (int) at (2,0) {\textcolor{accentcyan}{Integration}};
        
        % Functions
        \node[above=0.5cm of diff,neonpink] {$F(x)$};
        \node[above=0.5cm of int,accentcyan] {$f(x)$};
        
        % Arrows
        \draw[->,thick,neonyellow] (diff.east) -- (int.west) node[midway,above] {$\frac{d}{dx}$};
        \draw[->,thick,neonyellow] (int.west) to[bend left=40] node[midway,below] {$\int dx$} (diff.east);
    \end{tikzpicture}
    
    \vspace{0.5cm}
    
    \begin{keybox}
        Differentiation and Integration are \glow{inverse operations}!
        
        \vspace{0.2cm}
        
        \begin{itemize}
            \item Differentiate then integrate: back where you started (+ C)
            \item Integrate then differentiate: exactly back where you started
        \end{itemize}
    \end{keybox}
    
    \begin{funnybox}
        They're like frenemies — opposite but inseparable!
    \end{funnybox}
\end{frame}

% -----------------------------------------------------------------------------
% Integration by Substitution
% -----------------------------------------------------------------------------
\begin{frame}{Integration Technique: Substitution}
    \begin{defbox}[u-Substitution]
        For $\int f(g(x)) \cdot g'(x)\, dx$:
        
        \begin{enumerate}
            \item Let $u = g(x)$
            \item Then $du = g'(x)\, dx$
            \item Substitute: $\int f(u)\, du$
            \item Integrate and substitute back
        \end{enumerate}
    \end{defbox}
    
    \vspace{0.3cm}
    
    \textbf{Example:} $\int 2x \cdot e^{x^2}\, dx$
    
    \begin{itemize}
        \item Let $u = x^2$, so $du = 2x\, dx$
        \item Substitute: $\int e^u\, du = e^u + C$
        \item Substitute back: $\mathbf{e^{x^2} + C}$
    \end{itemize}
    
    \begin{funnybox}
        ``When in doubt, u-sub it out!'' — Every calculus student
    \end{funnybox}
\end{frame}

% -----------------------------------------------------------------------------
% Integration by Parts
% -----------------------------------------------------------------------------
\begin{frame}{Integration Technique: By Parts}
    \begin{thmbox}[Integration by Parts]
        \[
            \int u\, dv = uv - \int v\, du
        \]
        
        Derived from the product rule: $(uv)' = u'v + uv'$
    \end{thmbox}
    
    \vspace{0.3cm}
    
    \textbf{Example:} $\int x \cdot e^x\, dx$
    
    \begin{columns}
        \begin{column}{0.5\textwidth}
            Choose:
            \begin{itemize}
                \item $u = x \Rightarrow du = dx$
                \item $dv = e^x\, dx \Rightarrow v = e^x$
            \end{itemize}
        \end{column}
        \begin{column}{0.5\textwidth}
            Apply:
            \begin{align*}
                &= x \cdot e^x - \int e^x\, dx \\
                &= xe^x - e^x + C \\
                &= \mathbf{e^x(x-1) + C}
            \end{align*}
        \end{column}
    \end{columns}
    
    \begin{infobox}[LIATE Rule]
        Choose $u$ in order: \textbf{L}og, \textbf{I}nverse trig, \textbf{A}lgebraic, \textbf{T}rig, \textbf{E}xponential
    \end{infobox}
\end{frame}

% -----------------------------------------------------------------------------
% Why Integration Matters
% -----------------------------------------------------------------------------
\begin{frame}{Why Integration Matters in ML}
    \begin{keybox}[Applications]
        \begin{enumerate}
            \item \textbf{Probability Distributions}
            \[
                P(a \leq X \leq b) = \int_a^b f(x)\, dx
            \]
            (Area under probability density function)
            
            \item \textbf{Expected Values}
            \[
                E[X] = \int_{-\infty}^{\infty} x \cdot f(x)\, dx
            \]
            
            \item \textbf{Loss Functions} (continuous case)
            
            \item \textbf{Backpropagation} involves computing areas/volumes
            
            \item \textbf{Gaussian Integrals} (normal distribution!)
        \end{enumerate}
    \end{keybox}
\end{frame}

% -----------------------------------------------------------------------------
% Key Takeaways
% -----------------------------------------------------------------------------
\begin{frame}{Key Takeaways: Integration}
    \begin{keybox}
        \begin{enumerate}
            \item \textbf{Riemann sum}: Approximate area with rectangles
            \item \textbf{Definite integral} $\int_a^b f(x)\, dx$: Exact area under curve
            \item \textbf{Antiderivative}: $F$ where $F' = f$
            \item \textbf{Indefinite integral}: $\int f\, dx = F + C$
            \item \textbf{FTC}: $\int_a^b f\, dx = F(b) - F(a)$
            \item \textbf{Integration and differentiation are inverses!}
            \item \textbf{Techniques}: Substitution, by parts
            \item \textbf{ML uses}: Probability, expectations, continuous losses
        \end{enumerate}
    \end{keybox}
    
    \vspace{0.2cm}
    
    \centering
    \textit{Next: Graph Theory — the foundation for neural network structure!}
\end{frame}
