% =============================================================================
% Section 1: Foundations of Numbers
% From counting sheep to conquering mathematics
% =============================================================================

\section{Foundations of Numbers}

% -----------------------------------------------------------------------------
% Opening Slide
% -----------------------------------------------------------------------------
\begin{frame}{The Story of Numbers}
    \begin{columns}[T]
        \begin{column}{0.5\textwidth}
            \begin{funnybox}
                \textit{``In the beginning, there was 1. Then someone wanted more pizza, and mathematics was born.''}
            \end{funnybox}
            \vspace{0.5cm}
            \textbf{Our Journey:}
            \begin{itemize}
                \item Natural numbers $\mathbb{N}$
                \item Integers $\mathbb{Z}$
                \item Rational numbers $\mathbb{Q}$
            \end{itemize}
        \end{column}
        \begin{column}{0.5\textwidth}
            \centering
            \begin{tikzpicture}[scale=0.8]
                % Nested sets visualization
                \fill[neonpink!20] (0,0) ellipse (3.5cm and 2.5cm);
                \fill[accentcyan!20] (0,0) ellipse (2.5cm and 1.8cm);
                \fill[neongreen!20] (0,0) ellipse (1.5cm and 1cm);
                
                \draw[neonpink,thick] (0,0) ellipse (3.5cm and 2.5cm);
                \draw[accentcyan,thick] (0,0) ellipse (2.5cm and 1.8cm);
                \draw[neongreen,thick] (0,0) ellipse (1.5cm and 1cm);
                
                \node[neongreen] at (0,0) {$\mathbb{N}$};
                \node[accentcyan] at (0,-1.4) {$\mathbb{Z}$};
                \node[neonpink] at (0,-2.2) {$\mathbb{Q}$};
            \end{tikzpicture}
        \end{column}
    \end{columns}
\end{frame}

% -----------------------------------------------------------------------------
% Natural Numbers - Intuition
% -----------------------------------------------------------------------------
\begin{frame}{Natural Numbers $\mathbb{N}$: The Universe's Original Counting App}
    \begin{columns}[T]
        \begin{column}{0.55\textwidth}
            \begin{defbox}[Natural Numbers]
                The \glow{natural numbers} are the counting numbers:
                \[
                    \mathbb{N} = \{1, 2, 3, 4, 5, \ldots\}
                \]
                Some mathematicians include 0, giving $\mathbb{N}_0 = \{0, 1, 2, \ldots\}$
            \end{defbox}
            
            \vspace{0.3cm}
            
            \textbf{Cave Person Math:}
            \begin{itemize}
                \item SHEEP One sheep
                \item SHEEPSHEEP Two sheep
                \item SHEEPSHEEPSHEEP Three sheep... \textit{zzz}
            \end{itemize}
        \end{column}
        \begin{column}{0.45\textwidth}
            \centering
            \begin{tikzpicture}
                % Number line
                \draw[->,thick,accentcyan] (0,0) -- (5,0);
                \foreach \x in {1,2,3,4} {
                    \fill[neongreen] (\x,0) circle (3pt);
                    \node[below,fgwhite] at (\x,-0.2) {\x};
                }
                \node[above,accentcyan] at (2.5,0.3) {$\mathbb{N}$};
                
                % Dots indicating continuation
                \node[fgwhite] at (4.5,0) {$\cdots$};
            \end{tikzpicture}
            
            \vspace{0.5cm}
            
            \begin{funnybox}
                ``Counting: so easy a caveman did it!''
            \end{funnybox}
        \end{column}
    \end{columns}
\end{frame}

% -----------------------------------------------------------------------------
% Peano Axioms
% -----------------------------------------------------------------------------
\begin{frame}{Peano Axioms: Making Counting Rigorous}
    \begin{thmbox}[Peano Axioms (1889)]
        The natural numbers satisfy:
        \begin{enumerate}
            \item $1 \in \mathbb{N}$ \hfill \textit{(1 exists)}
            \item $\forall n \in \mathbb{N}: S(n) \in \mathbb{N}$ \hfill \textit{(every number has a successor)}
            \item $\forall n \in \mathbb{N}: S(n) \neq 1$ \hfill \textit{(1 is not a successor)}
            \item $S(n) = S(m) \Rightarrow n = m$ \hfill \textit{(successors are unique)}
            \item \textbf{Induction axiom} \hfill \textit{(if true for 1 and $n \Rightarrow n+1$, true for all)}
        \end{enumerate}
    \end{thmbox}
    
    \vspace{0.3cm}
    
    \begin{columns}
        \begin{column}{0.5\textwidth}
            \centering
            \begin{tikzpicture}[scale=0.9]
                \node[circle,draw=neongreen,fill=darkgray,minimum size=0.8cm] (1) at (0,0) {1};
                \node[circle,draw=accentcyan,fill=darkgray,minimum size=0.8cm] (2) at (1.5,0) {2};
                \node[circle,draw=accentcyan,fill=darkgray,minimum size=0.8cm] (3) at (3,0) {3};
                \node[circle,draw=accentcyan,fill=darkgray,minimum size=0.8cm] (4) at (4.5,0) {4};
                \node[fgwhite] at (5.5,0) {$\cdots$};
                
                \draw[->,neonpink,thick] (1) -- (2) node[midway,above,font=\tiny] {$S$};
                \draw[->,neonpink,thick] (2) -- (3) node[midway,above,font=\tiny] {$S$};
                \draw[->,neonpink,thick] (3) -- (4) node[midway,above,font=\tiny] {$S$};
            \end{tikzpicture}
        \end{column}
        \begin{column}{0.5\textwidth}
            \begin{funnybox}
                Giuseppe Peano basically said: ``Here's how to count. You're welcome, humanity.''
            \end{funnybox}
        \end{column}
    \end{columns}
\end{frame}

% -----------------------------------------------------------------------------
% Properties of Natural Numbers
% -----------------------------------------------------------------------------
\begin{frame}{Properties of Natural Numbers}
    \begin{columns}[T]
        \begin{column}{0.5\textwidth}
            \begin{defbox}[Closure Properties]
                For all $a, b \in \mathbb{N}$:
                \begin{align*}
                    a + b &\in \mathbb{N} \quad \checkmark \\
                    a \times b &\in \mathbb{N} \quad \checkmark \\
                    a - b &\in \mathbb{N} \quad \textcolor{neonpink}{\times} \\
                    a \div b &\in \mathbb{N} \quad \textcolor{neonpink}{\times}
                \end{align*}
            \end{defbox}
        \end{column}
        \begin{column}{0.5\textwidth}
            \begin{alertbox}[The Problem]
                What is $3 - 5$? 
                
                \vspace{0.2cm}
                
                Natural numbers can't handle this! We need... \glow{negative numbers}!
            \end{alertbox}
        \end{column}
    \end{columns}
    
    \vspace{0.5cm}
    
    \centering
    \begin{tikzpicture}
        \node[fgwhite] at (0,0) {$5 - 3 = 2$ \textcolor{neongreen}{\checkmark}};
        \node[fgwhite] at (4,0) {$3 - 5 = $ \textcolor{neonpink}{???}};
    \end{tikzpicture}
\end{frame}

% -----------------------------------------------------------------------------
% Integers
% -----------------------------------------------------------------------------
\begin{frame}{Integers $\mathbb{Z}$: When Mathematicians Discovered Debt}
    \begin{columns}[T]
        \begin{column}{0.55\textwidth}
            \begin{defbox}[Integers]
                The \glow{integers} extend natural numbers with negatives and zero:
                \[
                    \mathbb{Z} = \{\ldots, -3, -2, -1, 0, 1, 2, 3, \ldots\}
                \]
            \end{defbox}
            
            \vspace{0.3cm}
            
            \textbf{Real-world integers:}
            \begin{itemize}
                \item Temperature: $-10°C$ (brrr!)
                \item Bank account: $-\$500$ (oops!)
                \item Elevation: $-100m$ (underwater)
            \end{itemize}
        \end{column}
        \begin{column}{0.45\textwidth}
            \centering
            \begin{tikzpicture}[scale=0.7]
                % Full integer number line
                \draw[<->,thick,accentcyan] (-3.5,0) -- (3.5,0);
                \foreach \x in {-3,-2,-1,0,1,2,3} {
                    \fill[neongreen] (\x,0) circle (3pt);
                    \node[below,fgwhite,font=\small] at (\x,-0.3) {\x};
                }
                \node[above,accentcyan] at (0,0.4) {$\mathbb{Z}$};
                
                % Color coding
                \draw[neonpink,thick,decorate,decoration={brace,amplitude=5pt}] 
                    (-3,0.5) -- (-0.2,0.5) node[midway,above=5pt,font=\small] {negative};
                \draw[neongreen,thick,decorate,decoration={brace,amplitude=5pt}] 
                    (0.2,0.5) -- (3,0.5) node[midway,above=5pt,font=\small] {positive};
            \end{tikzpicture}
            
            \vspace{0.3cm}
            
            \begin{funnybox}
                The name $\mathbb{Z}$ comes from German ``Zahlen'' (numbers). Germans: always precise.
            \end{funnybox}
        \end{column}
    \end{columns}
\end{frame}

% -----------------------------------------------------------------------------
% Integer Properties
% -----------------------------------------------------------------------------
\begin{frame}{Integer Properties}
    \begin{columns}[T]
        \begin{column}{0.5\textwidth}
            \begin{defbox}[Closure Properties]
                For all $a, b \in \mathbb{Z}$:
                \begin{align*}
                    a + b &\in \mathbb{Z} \quad \checkmark \\
                    a - b &\in \mathbb{Z} \quad \checkmark \\
                    a \times b &\in \mathbb{Z} \quad \checkmark \\
                    a \div b &\in \mathbb{Z} \quad \textcolor{neonpink}{\times}
                \end{align*}
            \end{defbox}
            
            \vspace{0.3cm}
            
            Now $3 - 5 = -2 \in \mathbb{Z}$ \textcolor{neongreen}{\checkmark}
        \end{column}
        \begin{column}{0.5\textwidth}
            \begin{alertbox}[Still a Problem!]
                What is $1 \div 2$?
                
                \vspace{0.2cm}
                
                $\frac{1}{2} = 0.5 \notin \mathbb{Z}$
                
                \vspace{0.2cm}
                
                We need \glow{fractions}!
            \end{alertbox}
        \end{column}
    \end{columns}
    
    \vspace{0.5cm}
    
    \begin{keybox}
        Each number system fixes a problem but creates a new one. Mathematics evolves by solving its own limitations!
    \end{keybox}
\end{frame}

% -----------------------------------------------------------------------------
% Rational Numbers
% -----------------------------------------------------------------------------
\begin{frame}{Rational Numbers $\mathbb{Q}$: Sharing Pizza Among Friends}
    \begin{columns}[T]
        \begin{column}{0.55\textwidth}
            \begin{defbox}[Rational Numbers]
                The \glow{rational numbers} are all fractions:
                \[
                    \mathbb{Q} = \left\{ \frac{p}{q} \;\middle|\; p, q \in \mathbb{Z}, q \neq 0 \right\}
                \]
            \end{defbox}
            
            \vspace{0.3cm}
            
            \textbf{Examples:}
            \begin{itemize}
                \item $\frac{1}{2} = 0.5$
                \item $\frac{22}{7} \approx 3.14159...$
                \item $\frac{-3}{4} = -0.75$
                \item $\frac{6}{3} = 2$ (integers are rational too!)
            \end{itemize}
        \end{column}
        \begin{column}{0.45\textwidth}
            \centering
            \begin{tikzpicture}[scale=0.6]
                % Pizza visualization
                \fill[neonyellow!30] (0,0) circle (1.5cm);
                \draw[neonyellow,thick] (0,0) circle (1.5cm);
                \draw[neonyellow] (0,0) -- (0,1.5);
                \draw[neonyellow] (0,0) -- (1.5,0);
                \draw[neonyellow] (0,0) -- (0,-1.5);
                \draw[neonyellow] (0,0) -- (-1.5,0);
                
                \node[fgwhite,font=\small] at (0,-2.2) {$\frac{1}{4}$ each!};
            \end{tikzpicture}
            
            \vspace{0.3cm}
            
            \begin{funnybox}
                ``$\mathbb{Q}$'' for \textbf{Q}uotient. Mathematicians love their abbreviations!
            \end{funnybox}
        \end{column}
    \end{columns}
\end{frame}

% -----------------------------------------------------------------------------
% Density of Rationals
% -----------------------------------------------------------------------------
\begin{frame}{The Density of Rational Numbers}
    \begin{thmbox}[Density of $\mathbb{Q}$]
        Between \textbf{any} two rational numbers, there exists another rational number.
        
        \vspace{0.2cm}
        
        If $a, b \in \mathbb{Q}$ with $a < b$, then $\frac{a+b}{2} \in \mathbb{Q}$ and $a < \frac{a+b}{2} < b$.
    \end{thmbox}
    
    \vspace{0.3cm}
    
    \centering
    \begin{tikzpicture}[scale=0.85]
        % Number line segment
        \draw[thick,accentcyan] (0,0) -- (6,0);
        
        % Points
        \fill[neongreen] (1,0) circle (3pt) node[below=3pt,fgwhite] {$0$};
        \fill[neongreen] (5,0) circle (3pt) node[below=3pt,fgwhite] {$1$};
        
        % Midpoint
        \fill[neonyellow] (3,0) circle (3pt) node[below=3pt,fgwhite] {$\frac{1}{2}$};
        
        % More midpoints
        \fill[neonpink] (2,0) circle (2pt) node[below=3pt,fgwhite,font=\tiny] {$\frac{1}{4}$};
        \fill[neonpink] (4,0) circle (2pt) node[below=3pt,fgwhite,font=\tiny] {$\frac{3}{4}$};
        
        % Even more
        \fill[softgray] (1.5,0) circle (1.5pt);
        \fill[softgray] (2.5,0) circle (1.5pt);
        \fill[softgray] (3.5,0) circle (1.5pt);
        \fill[softgray] (4.5,0) circle (1.5pt);
    \end{tikzpicture}
    
    \vspace{0.3cm}
    
    \begin{funnybox}
        There are \textbf{infinitely many} rationals between any two rationals. Mind = blown! MINDBLOWN
    \end{funnybox}
\end{frame}

% -----------------------------------------------------------------------------
% But Wait - Irrationals Preview
% -----------------------------------------------------------------------------
\begin{frame}{But Wait... Are Rationals Enough?}
    \begin{columns}[T]
        \begin{column}{0.5\textwidth}
            \begin{alertbox}[The Pythagorean Nightmare]
                Consider a square with side 1:
                
                \vspace{0.3cm}
                
                \centering
                \begin{tikzpicture}[scale=1.0]
                    \draw[accentcyan,thick] (0,0) -- (1,0) -- (1,1) -- (0,1) -- cycle;
                    \draw[neonpink,thick] (0,0) -- (1,1);
                    
                    \node[below,fgwhite,font=\small] at (0.5,0) {1};
                    \node[left,fgwhite,font=\small] at (0,0.5) {1};
                    \node[above right,neonpink,font=\small] at (0.5,0.5) {$\sqrt{2}$};
                \end{tikzpicture}
                
                \vspace{0.3cm}
                
                By Pythagoras: $d^2 = 1^2 + 1^2 = 2$
                
                So $d = \sqrt{2}$... but is $\sqrt{2} \in \mathbb{Q}$?
            \end{alertbox}
        \end{column}
        \begin{column}{0.5\textwidth}
            \begin{infobox}[Spoiler Alert]
                $\sqrt{2}$ is \textbf{NOT} rational!
                
                \vspace{0.2cm}
                
                There exist numbers that \textbf{cannot} be written as $\frac{p}{q}$.
                
                \vspace{0.2cm}
                
                These are called \glow[neonpink]{irrational numbers}.
                
                \vspace{0.2cm}
                
                Coming up next: Real numbers $\mathbb{R}$!
            \end{infobox}
        \end{column}
    \end{columns}
\end{frame}

% -----------------------------------------------------------------------------
% Number Hierarchy Summary
% -----------------------------------------------------------------------------
\begin{frame}{The Number Hierarchy}
    \centering
    \begin{tikzpicture}[scale=0.9]
        % Nested sets with better visualization
        \fill[neonpink!10] (0,0) ellipse (5cm and 3cm);
        \fill[accentcyan!15] (0,0) ellipse (3.5cm and 2.2cm);
        \fill[neonyellow!15] (0,0) ellipse (2.2cm and 1.4cm);
        \fill[neongreen!20] (0,0) ellipse (1cm and 0.7cm);
        
        \draw[neonpink,thick] (0,0) ellipse (5cm and 3cm);
        \draw[accentcyan,thick] (0,0) ellipse (3.5cm and 2.2cm);
        \draw[neonyellow,thick] (0,0) ellipse (2.2cm and 1.4cm);
        \draw[neongreen,thick] (0,0) ellipse (1cm and 0.7cm);
        
        % Labels
        \node[neongreen,font=\bfseries] at (0,0) {$\mathbb{N}$};
        \node[neonyellow,font=\bfseries] at (0,-1.1) {$\mathbb{Z}$};
        \node[accentcyan,font=\bfseries] at (0,-1.9) {$\mathbb{Q}$};
        \node[neonpink,font=\bfseries] at (0,-2.7) {$\mathbb{R}$};
        
        % Side labels
        \node[neongreen,right,font=\small] at (1.2,0) {Natural: $1, 2, 3, \ldots$};
        \node[neonyellow,right,font=\small] at (2.4,-0.5) {Integers: $\ldots, -1, 0, 1, \ldots$};
        \node[accentcyan,right,font=\small] at (3.7,-1) {Rationals: $\frac{p}{q}$};
        \node[neonpink,right,font=\small] at (5.2,-1.5) {Reals: $\sqrt{2}, \pi, e$};
    \end{tikzpicture}
    
    \vspace{0.3cm}
    
    \[
        \mathbb{N} \subset \mathbb{Z} \subset \mathbb{Q} \subset \mathbb{R}
    \]
\end{frame}

% -----------------------------------------------------------------------------
% Key Takeaways
% -----------------------------------------------------------------------------
\begin{frame}{Key Takeaways: Numbers}
    \begin{keybox}
        \begin{enumerate}
            \item \textbf{Natural numbers} $\mathbb{N}$: Counting (1, 2, 3, ...)
            \item \textbf{Integers} $\mathbb{Z}$: Add zero and negatives (..., -1, 0, 1, ...)
            \item \textbf{Rationals} $\mathbb{Q}$: All fractions $\frac{p}{q}$
            \item Each extension solves a problem:
            \begin{itemize}
                \item $\mathbb{Z}$ allows subtraction
                \item $\mathbb{Q}$ allows division
            \end{itemize}
            \item Rationals are \textbf{dense} but not \textbf{complete}
            \item Some numbers (like $\sqrt{2}$) are \textbf{irrational}
        \end{enumerate}
    \end{keybox}
    
    \vspace{0.3cm}
    
    \centering
    \textit{Next: Real and Complex Numbers — where it gets \glow{really} interesting!}
\end{frame}
