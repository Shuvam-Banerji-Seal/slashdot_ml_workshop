% =============================================================================
% Section 4: Limits
% The foundation of calculus - getting infinitely close
% =============================================================================

\section{Limits}

% -----------------------------------------------------------------------------
% Opening
% -----------------------------------------------------------------------------
\begin{frame}{Limits: Getting Infinitely Close}
    \begin{columns}[T]
        \begin{column}{0.5\textwidth}
            \begin{funnybox}
                \textit{``A limit is like trying to touch your nose with your tongue — you can get really, really close, but maybe never quite there!''}
            \end{funnybox}
            
            \vspace{0.3cm}
            
            \textbf{The Big Question:}
            
            What happens to $f(x)$ as $x$ gets \glow{infinitely close} to some value $a$?
        \end{column}
        \begin{column}{0.5\textwidth}
            \centering
            \begin{tikzpicture}[scale=0.8]
                \draw[<->,thick,softgray] (-0.5,0) -- (4,0) node[right] {$x$};
                \draw[<->,thick,softgray] (0,-0.5) -- (0,3) node[above] {$y$};
                
                % Function with hole
                \draw[accentcyan,thick,domain=0:1.8,smooth] plot (\x,{0.5*\x + 0.5});
                \draw[accentcyan,thick,domain=2.2:3.5,smooth] plot (\x,{0.5*\x + 0.5});
                
                % Hole at x=2
                \draw[neonpink,thick] (2,1.5) circle (4pt);
                
                % Approaching arrows
                \draw[->,neonyellow,thick] (1.3,0) -- (1.8,0);
                \draw[->,neonyellow,thick] (2.7,0) -- (2.2,0);
                
                % Labels
                \node[below,fgwhite] at (2,0) {$a$};
                \node[right,neonpink] at (2.3,1.5) {$L$};
                
                \node[neonyellow,font=\small] at (2,-0.7) {approaching from both sides};
            \end{tikzpicture}
        \end{column}
    \end{columns}
\end{frame}

% -----------------------------------------------------------------------------
% Intuitive Definition
% -----------------------------------------------------------------------------
\begin{frame}{Intuitive Definition of Limits}
    \begin{defbox}[Informal Limit Definition]
        We write:
        \[
            \lim_{x \to a} f(x) = L
        \]
        
        and say ``the limit of $f(x)$ as $x$ approaches $a$ equals $L$''
        
        \vspace{0.2cm}
        
        \textbf{Meaning:} As $x$ gets closer and closer to $a$, $f(x)$ gets closer and closer to $L$.
    \end{defbox}
    
    \vspace{0.3cm}
    
    \textbf{Example:} $\displaystyle\lim_{x \to 2} (3x + 1) = ?$
    
    \begin{center}
        \begin{tabular}{c|cccccc}
            $x$ & 1.9 & 1.99 & 1.999 & 2.001 & 2.01 & 2.1 \\
            \hline
            $f(x)$ & 6.7 & 6.97 & 6.997 & 7.003 & 7.03 & 7.3
        \end{tabular}
    \end{center}
    
    As $x \to 2$, we have $f(x) \to \glow{7}$!
\end{frame}

% -----------------------------------------------------------------------------
% Epsilon-Delta Definition
% -----------------------------------------------------------------------------
\begin{frame}{The Rigorous Definition: Epsilon-Delta}
    \begin{thmbox}[Epsilon-Delta Definition]
        \[
            \lim_{x \to a} f(x) = L
        \]
        
        means: For every $\glow[neongreen]{\varepsilon > 0}$, there exists $\glow[neonpink]{\delta > 0}$ such that:
        \[
            0 < |x - a| < \delta \implies |f(x) - L| < \varepsilon
        \]
    \end{thmbox}
    
    \vspace{0.3cm}
    
    \begin{columns}
        \begin{column}{0.5\textwidth}
            \textbf{In English:}
            \begin{itemize}
                \item Pick any tolerance $\varepsilon$ (how close to $L$)
                \item I can find a $\delta$ (how close $x$ needs to be to $a$)
                \item Such that staying within $\delta$ of $a$ keeps $f(x)$ within $\varepsilon$ of $L$
            \end{itemize}
        \end{column}
        \begin{column}{0.5\textwidth}
            \begin{funnybox}
                It's like a game:
                
                \vspace{0.1cm}
                
                You: ``Get within $\varepsilon$ of $L$!''
                
                Me: ``Fine, stay within $\delta$ of $a$.''
                
                \vspace{0.1cm}
                
                If I can \textit{always} win, the limit exists!
            \end{funnybox}
        \end{column}
    \end{columns}
\end{frame}

% -----------------------------------------------------------------------------
% Epsilon-Delta Visualization
% -----------------------------------------------------------------------------
\begin{frame}{Visualizing $\varepsilon$-$\delta$}
    \centering
    \begin{tikzpicture}[scale=0.95]
        % Axes
        \draw[<->,thick,softgray] (-0.5,0) -- (5,0) node[right] {$x$};
        \draw[<->,thick,softgray] (0,-0.5) -- (0,3.5) node[above] {$y$};
        
        % Function
        \draw[accentcyan,thick,domain=0.5:4.5,smooth] plot (\x,{0.5*\x + 0.5});
        
        % Point a on x-axis
        \fill[fgwhite] (2.5,0) circle (2pt);
        \node[below,fgwhite] at (2.5,-0.1) {$a$};
        
        % L on y-axis  
        \fill[fgwhite] (0,1.75) circle (2pt);
        \node[left,fgwhite] at (0,1.75) {$L$};
        
        % Epsilon band (horizontal)
        \fill[neongreen,opacity=0.2] (0,1.45) rectangle (5,2.05);
        \draw[neongreen,dashed] (0,2.05) -- (5,2.05);
        \draw[neongreen,dashed] (0,1.45) -- (5,1.45);
        \node[left,neongreen,font=\small] at (0,2.05) {$L+\varepsilon$};
        \node[left,neongreen,font=\small] at (0,1.45) {$L-\varepsilon$};
        
        % Delta band (vertical)
        \fill[neonpink,opacity=0.2] (1.9,0) rectangle (3.1,3.5);
        \draw[neonpink,dashed] (1.9,0) -- (1.9,3.5);
        \draw[neonpink,dashed] (3.1,0) -- (3.1,3.5);
        \node[below,neonpink,font=\small] at (1.9,0) {$a-\delta$};
        \node[below,neonpink,font=\small] at (3.1,0) {$a+\delta$};
        
        % Arrows showing the constraint
        \draw[<->,neonyellow,thick] (2.5,1.45) -- (2.5,2.05);
        \node[right,neonyellow,font=\small] at (2.5,1.75) {$\varepsilon$};
        
        \draw[<->,neonyellow,thick] (1.9,-0.4) -- (3.1,-0.4);
        \node[below,neonyellow,font=\small] at (2.5,-0.4) {$\delta$};
    \end{tikzpicture}
    
    \vspace{0.2cm}
    
    If $x$ is in the \textcolor{neonpink}{pink zone}, $f(x)$ stays in the \textcolor{neongreen}{green zone}!
\end{frame}

% -----------------------------------------------------------------------------
% Limit Laws
% -----------------------------------------------------------------------------
\begin{frame}{Limit Laws: The Algebra of Limits}
    If $\displaystyle\lim_{x \to a} f(x) = L$ and $\displaystyle\lim_{x \to a} g(x) = M$, then:
    
    \vspace{0.3cm}
    
    \begin{columns}[T]
        \begin{column}{0.5\textwidth}
            \begin{defbox}[Addition]
                \[
                    \lim_{x \to a} [f(x) + g(x)] = L + M
                \]
            \end{defbox}
            
            \vspace{0.2cm}
            
            \begin{defbox}[Multiplication]
                \[
                    \lim_{x \to a} [f(x) \cdot g(x)] = L \cdot M
                \]
            \end{defbox}
            
            \vspace{0.2cm}
            
            \begin{defbox}[Constant Multiple]
                \[
                    \lim_{x \to a} [c \cdot f(x)] = c \cdot L
                \]
            \end{defbox}
        \end{column}
        \begin{column}{0.5\textwidth}
            \begin{defbox}[Subtraction]
                \[
                    \lim_{x \to a} [f(x) - g(x)] = L - M
                \]
            \end{defbox}
            
            \vspace{0.2cm}
            
            \begin{defbox}[Division]
                \[
                    \lim_{x \to a} \frac{f(x)}{g(x)} = \frac{L}{M} \quad (M \neq 0)
                \]
            \end{defbox}
            
            \vspace{0.2cm}
            
            \begin{defbox}[Power]
                \[
                    \lim_{x \to a} [f(x)]^n = L^n
                \]
            \end{defbox}
        \end{column}
    \end{columns}
\end{frame}

% -----------------------------------------------------------------------------
% One-Sided Limits
% -----------------------------------------------------------------------------
\begin{frame}{One-Sided Limits}
    \begin{columns}[T]
        \begin{column}{0.5\textwidth}
            \begin{defbox}[Left-Hand Limit]
                \[
                    \lim_{x \to a^-} f(x) = L
                \]
                
                Approaching from the \textbf{left} (values less than $a$)
            \end{defbox}
            
            \vspace{0.3cm}
            
            \begin{defbox}[Right-Hand Limit]
                \[
                    \lim_{x \to a^+} f(x) = L
                \]
                
                Approaching from the \textbf{right} (values greater than $a$)
            \end{defbox}
        \end{column}
        \begin{column}{0.5\textwidth}
            \centering
            \begin{tikzpicture}[scale=0.8]
                \draw[<->,thick,softgray] (-0.5,0) -- (4,0) node[right] {$x$};
                \draw[<->,thick,softgray] (0,-0.5) -- (0,3) node[above] {$y$};
                
                % Jump discontinuity
                \draw[accentcyan,thick,domain=0.3:1.9] plot (\x,{0.5*\x + 0.5});
                \draw[neonpink,thick,domain=2.1:3.5] plot (\x,{0.5*\x + 1.2});
                
                % Points
                \fill[accentcyan] (2,1.5) circle (3pt);
                \draw[neonpink,thick] (2,2.2) circle (3pt);
                
                % Labels
                \node[below,fgwhite] at (2,0) {$a$};
                \node[left,accentcyan,font=\small] at (1.5,1.2) {$\lim_{x\to a^-}$};
                \node[right,neonpink,font=\small] at (2.5,2.5) {$\lim_{x\to a^+}$};
            \end{tikzpicture}
            
            \vspace{0.2cm}
            
            \begin{alertbox}
                Limit exists $\iff$ both one-sided limits exist and are equal!
            \end{alertbox}
        \end{column}
    \end{columns}
\end{frame}

% -----------------------------------------------------------------------------
% Limits at Infinity
% -----------------------------------------------------------------------------
\begin{frame}{Limits at Infinity}
    \begin{defbox}[Limit at Infinity]
        \[
            \lim_{x \to \infty} f(x) = L
        \]
        
        As $x$ grows without bound, $f(x)$ approaches $L$.
    \end{defbox}
    
    \vspace{0.3cm}
    
    \textbf{Examples:}
    
    \begin{columns}
        \begin{column}{0.5\textwidth}
            \[
                \lim_{x \to \infty} \frac{1}{x} = 0
            \]
            
            \centering
            \begin{tikzpicture}[scale=0.5]
                \draw[<->,thick,softgray] (-0.5,0) -- (4,0);
                \draw[<->,thick,softgray] (0,-0.5) -- (0,3);
                \draw[accentcyan,thick,domain=0.4:3.8,smooth] plot (\x,{1/\x});
                \draw[neonpink,dashed] (0,0) -- (4,0);
            \end{tikzpicture}
            
            {\small Horizontal asymptote at $y = 0$}
        \end{column}
        \begin{column}{0.5\textwidth}
            \[
                \lim_{x \to \infty} \frac{2x + 1}{x} = 2
            \]
            
            \centering
            \begin{tikzpicture}[scale=0.5]
                \draw[<->,thick,softgray] (-0.5,0) -- (4,0);
                \draw[<->,thick,softgray] (0,-0.5) -- (0,3);
                \draw[accentcyan,thick,domain=0.4:3.8,smooth] plot (\x,{(2*\x+1)/\x});
                \draw[neonpink,dashed] (0,2) -- (4,2);
            \end{tikzpicture}
            
            {\small Horizontal asymptote at $y = 2$}
        \end{column}
    \end{columns}
\end{frame}

% -----------------------------------------------------------------------------
% Indeterminate Forms
% -----------------------------------------------------------------------------
\begin{frame}{Indeterminate Forms}
    \begin{alertbox}[Danger Zone!]
        Some limits look like they give answers but don't:
        
        \vspace{0.2cm}
        
        \centering
        \begin{tabular}{ccccccc}
            $\frac{0}{0}$ & $\frac{\infty}{\infty}$ & $0 \cdot \infty$ & $\infty - \infty$ & $0^0$ & $1^\infty$ & $\infty^0$
        \end{tabular}
        
        \vspace{0.2cm}
        
        These are \glow[neonpink]{indeterminate} — they need more work!
    \end{alertbox}
    
    \vspace{0.3cm}
    
    \textbf{Example:} $\displaystyle\lim_{x \to 0} \frac{\sin x}{x} = \frac{0}{0}$ ??? 
    
    \vspace{0.2cm}
    
    \begin{columns}
        \begin{column}{0.5\textwidth}
            But actually:
            \[
                \lim_{x \to 0} \frac{\sin x}{x} = 1
            \]
            (This is a famous result!)
        \end{column}
        \begin{column}{0.5\textwidth}
            \begin{funnybox}
                $\frac{0}{0}$ doesn't mean ``zero'' — it means ``figure it out another way!''
            \end{funnybox}
        \end{column}
    \end{columns}
\end{frame}

% -----------------------------------------------------------------------------
% L'Hôpital's Rule Preview
% -----------------------------------------------------------------------------
\begin{frame}{L'Hôpital's Rule (Preview)}
    \begin{thmbox}[L'Hôpital's Rule]
        If $\displaystyle\lim_{x \to a} \frac{f(x)}{g(x)}$ gives $\frac{0}{0}$ or $\frac{\infty}{\infty}$, then:
        \[
            \lim_{x \to a} \frac{f(x)}{g(x)} = \lim_{x \to a} \frac{f'(x)}{g'(x)}
        \]
        
        (provided the right-hand limit exists)
    \end{thmbox}
    
    \vspace{0.3cm}
    
    \textbf{Example:} $\displaystyle\lim_{x \to 0} \frac{\sin x}{x}$
    
    \begin{itemize}
        \item Direct: $\frac{\sin 0}{0} = \frac{0}{0}$ (indeterminate!)
        \item L'Hôpital: $\displaystyle\lim_{x \to 0} \frac{\cos x}{1} = \frac{\cos 0}{1} = 1$ (YES)
    \end{itemize}
    
    \begin{infobox}
        We'll learn about derivatives ($f'$) in the next section — then L'Hôpital makes sense!
    \end{infobox}
\end{frame}

% -----------------------------------------------------------------------------
% Continuity
% -----------------------------------------------------------------------------
\begin{frame}{Continuity: No Jumps, No Holes}
    \begin{defbox}[Continuous Function]
        A function $f$ is \glow{continuous} at $x = a$ if:
        \[
            \lim_{x \to a} f(x) = f(a)
        \]
        
        \textbf{Three conditions:}
        \begin{enumerate}
            \item $f(a)$ exists (defined at $a$)
            \item $\lim_{x \to a} f(x)$ exists
            \item They're equal!
        \end{enumerate}
    \end{defbox}
    
    \vspace{0.3cm}
    
    \begin{columns}
        \begin{column}{0.33\textwidth}
            \centering
            \begin{tikzpicture}[scale=0.5]
                \draw[softgray] (-1,0) -- (2,0);
                \draw[softgray] (0,-0.5) -- (0,2);
                \draw[accentcyan,thick,domain=-0.8:1.8,smooth] plot (\x,{\x+0.5});
                \node[below,fgwhite,font=\tiny] at (0.5,-0.5) {Continuous (YES)};
            \end{tikzpicture}
        \end{column}
        \begin{column}{0.33\textwidth}
            \centering
            \begin{tikzpicture}[scale=0.5]
                \draw[softgray] (-1,0) -- (2,0);
                \draw[softgray] (0,-0.5) -- (0,2);
                \draw[neonpink,thick,domain=-0.8:0.4] plot (\x,{\x+0.5});
                \draw[neonpink,thick,domain=0.6:1.8] plot (\x,{\x+1});
                \draw[neonpink] (0.5,1) circle (3pt);
                \fill[neonpink] (0.5,1.5) circle (3pt);
                \node[below,fgwhite,font=\tiny] at (0.5,-0.5) {Jump (NO)};
            \end{tikzpicture}
        \end{column}
        \begin{column}{0.33\textwidth}
            \centering
            \begin{tikzpicture}[scale=0.5]
                \draw[softgray] (-1,0) -- (2,0);
                \draw[softgray] (0,-0.5) -- (0,2);
                \draw[neonyellow,thick,domain=-0.8:0.4] plot (\x,{\x+0.5});
                \draw[neonyellow,thick,domain=0.6:1.8] plot (\x,{\x+0.5});
                \draw[neonyellow] (0.5,1) circle (3pt);
                \node[below,fgwhite,font=\tiny] at (0.5,-0.5) {Hole (NO)};
            \end{tikzpicture}
        \end{column}
    \end{columns}
\end{frame}

% -----------------------------------------------------------------------------
% Why Limits Matter in ML
% -----------------------------------------------------------------------------
\begin{frame}{Why Limits Matter in Machine Learning}
    \begin{keybox}[Applications in ML]
        \begin{enumerate}
            \item \textbf{Derivatives} (next section!) are defined using limits
            \begin{itemize}
                \item Gradients for backpropagation!
            \end{itemize}
            
            \item \textbf{Convergence of training}
            \begin{itemize}
                \item Does the loss approach a minimum?
            \end{itemize}
            
            \item \textbf{Asymptotic analysis}
            \begin{itemize}
                \item Algorithm complexity: $O(n)$, $O(n^2)$
            \end{itemize}
            
            \item \textbf{Activation functions}
            \begin{itemize}
                \item $\lim_{x \to \infty} \sigma(x) = 1$ (sigmoid saturation)
            \end{itemize}
        \end{enumerate}
    \end{keybox}
\end{frame}

% -----------------------------------------------------------------------------
% Key Takeaways
% -----------------------------------------------------------------------------
\begin{frame}{Key Takeaways: Limits}
    \begin{keybox}
        \begin{enumerate}
            \item \textbf{Limit}: What $f(x)$ approaches as $x \to a$
            \item \textbf{Notation}: $\lim_{x \to a} f(x) = L$
            \item \textbf{$\varepsilon$-$\delta$ definition}: The rigorous foundation
            \item \textbf{Limit laws}: Add, subtract, multiply, divide limits
            \item \textbf{One-sided limits}: From left ($a^-$) or right ($a^+$)
            \item \textbf{Limits at infinity}: Behavior as $x \to \pm\infty$
            \item \textbf{Indeterminate forms}: $\frac{0}{0}$, $\frac{\infty}{\infty}$, etc. need special handling
            \item \textbf{Continuity}: No jumps, no holes, no surprises
        \end{enumerate}
    \end{keybox}
    
    \vspace{0.2cm}
    
    \centering
    \textit{Next: Differentiation — the heart of calculus!}
\end{frame}
