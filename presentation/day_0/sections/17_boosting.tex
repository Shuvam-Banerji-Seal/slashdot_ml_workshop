% =============================================================================
% Section 17: Ensemble Methods & Boosting
% Random Forests, Bagging, AdaBoost, Gradient Boosting
% =============================================================================

\section{Ensemble Methods \& Boosting}

% -----------------------------------------------------------------------------
% Opening
% -----------------------------------------------------------------------------
\begin{frame}{Ensemble Methods: Wisdom of the Crowd}
    \begin{columns}[T]
        \begin{column}{0.5\textwidth}
            \begin{funnybox}
                \textit{``One tree might be wrong, but a whole forest? Much harder to fool!''}
            \end{funnybox}
            
            \vspace{0.3cm}
            
            \textbf{Core Idea:}
            \begin{itemize}
                \item Train multiple ``weak'' models
                \item Combine their predictions
                \item Get a ``strong'' model!
            \end{itemize}
        \end{column}
        \begin{column}{0.5\textwidth}
            \centering
            \begin{tikzpicture}[scale=0.7]
                % Multiple trees
                \foreach \x in {0,1.5,3} {
                    \begin{scope}[xshift=\x cm]
                        \node[draw=accentcyan,rounded corners,fill=darkgray,minimum size=0.4cm] at (0,0) {};
                        \draw[accentcyan] (-0.3,-0.5) -- (0,-0.3);
                        \draw[accentcyan] (0.3,-0.5) -- (0,-0.3);
                        \draw[accentcyan] (-0.5,-1) -- (-0.3,-0.5);
                        \draw[accentcyan] (-0.1,-1) -- (-0.3,-0.5);
                        \draw[accentcyan] (0.1,-1) -- (0.3,-0.5);
                        \draw[accentcyan] (0.5,-1) -- (0.3,-0.5);
                    \end{scope}
                }
                
                % Voting
                \draw[->,thick,neonyellow] (0.75,-1.3) -- (0.75,-2);
                \draw[->,thick,neonyellow] (1.5,-1.3) -- (1.5,-2);
                \draw[->,thick,neonyellow] (2.25,-1.3) -- (2.25,-2);
                
                % Combined prediction
                \node[draw=neonpink,rounded corners,fill=darkgray,minimum width=2.5cm] at (1.5,-2.5) {Vote/Average};
                
                \draw[->,thick,neonyellow] (1.5,-3) -- (1.5,-3.5);
                \node[neongreen,font=\small] at (1.5,-4) {Strong Prediction};
            \end{tikzpicture}
        \end{column}
    \end{columns}
\end{frame}

% -----------------------------------------------------------------------------
% Bagging
% -----------------------------------------------------------------------------
\begin{frame}{Bagging: Bootstrap Aggregating}
    \begin{defbox}[Bagging (Breiman, 1996)]
        \begin{enumerate}
            \item Create $B$ bootstrap samples (sample with replacement)
            \item Train one model on each bootstrap sample
            \item Aggregate predictions:
            \begin{itemize}
                \item Classification: majority vote
                \item Regression: average
            \end{itemize}
        \end{enumerate}
    \end{defbox}
    
    \centering
    \begin{tikzpicture}[scale=0.7]
        % Original data
        \node[draw=accentcyan,rounded corners,fill=darkgray] (data) at (0,0) {Data: $\{1,2,3,4,5\}$};
        
        % Bootstrap samples
        \node[draw=neonpink,rounded corners,fill=darkgray,font=\small] (b1) at (-2.5,-1.5) {$\{1,1,3,4,5\}$};
        \node[draw=neonpink,rounded corners,fill=darkgray,font=\small] (b2) at (0,-1.5) {$\{2,2,3,3,5\}$};
        \node[draw=neonpink,rounded corners,fill=darkgray,font=\small] (b3) at (2.5,-1.5) {$\{1,3,4,5,5\}$};
        
        \draw[->,thick] (data) -- (b1);
        \draw[->,thick] (data) -- (b2);
        \draw[->,thick] (data) -- (b3);
        
        % Models
        \node[draw=neongreen,rounded corners,fill=darkgray,font=\small] (m1) at (-2.5,-3) {Tree 1};
        \node[draw=neongreen,rounded corners,fill=darkgray,font=\small] (m2) at (0,-3) {Tree 2};
        \node[draw=neongreen,rounded corners,fill=darkgray,font=\small] (m3) at (2.5,-3) {Tree 3};
        
        \draw[->,thick] (b1) -- (m1);
        \draw[->,thick] (b2) -- (m2);
        \draw[->,thick] (b3) -- (m3);
        
        % Aggregate
        \draw[->,thick] (m1) -- (0,-4);
        \draw[->,thick] (m2) -- (0,-4);
        \draw[->,thick] (m3) -- (0,-4);
        \node[draw=neonyellow,rounded corners,fill=darkgray] at (0,-4.5) {Aggregate};
    \end{tikzpicture}
\end{frame}

% -----------------------------------------------------------------------------
% Random Forest
% -----------------------------------------------------------------------------
\begin{frame}{Random Forest: Bagging + Feature Randomness}
    \begin{defbox}[Random Forest (Breiman, 2001)]
        Bagging + random feature selection at each split:
        
        \begin{itemize}
            \item At each node, consider only $m$ random features
            \item Typical $m$: $\sqrt{p}$ for classification, $p/3$ for regression
            \item This \textbf{decorrelates} the trees!
        \end{itemize}
    \end{defbox}
    
    \vspace{0.2cm}
    
    \begin{columns}
        \begin{column}{0.5\textwidth}
            \begin{keybox}[Why Random Features?]
                Without it:
                \begin{itemize}
                    \item All trees use same strong features
                    \item Trees are correlated
                    \item Averaging doesn't help much
                \end{itemize}
                
                With it:
                \begin{itemize}
                    \item Trees are diverse
                    \item Errors cancel out!
                \end{itemize}
            \end{keybox}
        \end{column}
        \begin{column}{0.5\textwidth}
            \begin{infobox}[Hyperparameters]
                \begin{itemize}
                    \item \texttt{n\_estimators}: \# of trees
                    \item \texttt{max\_features}: $m$
                    \item \texttt{max\_depth}: Tree depth
                    \item \texttt{min\_samples\_leaf}
                \end{itemize}
                
                More trees = better (diminishing returns)
            \end{infobox}
        \end{column}
    \end{columns}
\end{frame}

% -----------------------------------------------------------------------------
% Boosting Intuition
% -----------------------------------------------------------------------------
\begin{frame}{Boosting: Sequential Learning}
    \begin{defbox}[Boosting Intuition]
        Train models \textbf{sequentially}, each focusing on mistakes of previous ones.
        
        \centering
        \begin{tikzpicture}[scale=0.7]
            % Model 1
            \node[draw=accentcyan,rounded corners,fill=darkgray] (m1) at (0,0) {Model 1};
            \node[below,font=\small,fgwhite] at (0,-0.5) {Train on all data};
            
            % Errors
            \draw[->,thick,neonpink] (1.5,0) -- (3,0) node[midway,above,font=\tiny] {errors};
            
            % Model 2
            \node[draw=accentcyan,rounded corners,fill=darkgray] (m2) at (4.5,0) {Model 2};
            \node[below,font=\small,fgwhite] at (4.5,-0.5) {Focus on errors};
            
            % More errors
            \draw[->,thick,neonpink] (6,0) -- (7.5,0) node[midway,above,font=\tiny] {errors};
            
            % Model 3
            \node[draw=accentcyan,rounded corners,fill=darkgray] (m3) at (9,0) {Model 3};
            \node[below,font=\small,fgwhite] at (9,-0.5) {...};
        \end{tikzpicture}
    \end{defbox}
    
    \vspace{0.3cm}
    
    \begin{columns}
        \begin{column}{0.5\textwidth}
            \begin{keybox}[Bagging vs Boosting]
                \textbf{Bagging}: Parallel, reduce variance
                
                \textbf{Boosting}: Sequential, reduce bias
            \end{keybox}
        \end{column}
        \begin{column}{0.5\textwidth}
            \begin{funnybox}
                Boosting: ``You got this wrong? Let me send my friend who's good at exactly that!''
            \end{funnybox}
        \end{column}
    \end{columns}
\end{frame}

% -----------------------------------------------------------------------------
% AdaBoost
% -----------------------------------------------------------------------------
\begin{frame}{AdaBoost: Adaptive Boosting}
    \begin{defbox}[AdaBoost (Freund \& Schapire, 1997)]
        \begin{enumerate}
            \item Initialize sample weights: $w_i = 1/N$
            \item For $t = 1, \ldots, T$:
            \begin{enumerate}
                \item Train weak learner on weighted data
                \item Compute weighted error: $\epsilon_t = \sum_{wrong} w_i$
                \item Compute model weight: $\alpha_t = \frac{1}{2} \ln\left(\frac{1-\epsilon_t}{\epsilon_t}\right)$
                \item Update sample weights: increase for wrong, decrease for correct
            \end{enumerate}
            \item Final: $H(x) = \text{sign}\left(\sum_t \alpha_t h_t(x)\right)$
        \end{enumerate}
    \end{defbox}
    
    \begin{keybox}
        Better weak learners get higher $\alpha$. Misclassified samples get higher weight.
    \end{keybox}
\end{frame}

% -----------------------------------------------------------------------------
% Gradient Boosting
% -----------------------------------------------------------------------------
\begin{frame}{Gradient Boosting: Boosting as Gradient Descent}
    \begin{defbox}[Gradient Boosting]
        Each new model fits the \textbf{residuals} (gradient of loss):
        
        \begin{enumerate}
            \item Initialize: $F_0(x) = \arg\min_c \sum_i L(y_i, c)$
            \item For $t = 1, \ldots, T$:
            \begin{enumerate}
                \item Compute residuals: $r_i = -\frac{\partial L(y_i, F_{t-1}(x_i))}{\partial F_{t-1}}$
                \item Fit tree $h_t$ to residuals
                \item Update: $F_t(x) = F_{t-1}(x) + \eta \cdot h_t(x)$
            \end{enumerate}
        \end{enumerate}
    \end{defbox}
    
    \vspace{0.2cm}
    
    \begin{columns}
        \begin{column}{0.5\textwidth}
            \textbf{For MSE loss:}
            
            Residual = $y_i - F_{t-1}(x_i)$
            
            (actual minus prediction)
        \end{column}
        \begin{column}{0.5\textwidth}
            \begin{funnybox}
                Each tree says: ``Here's how much you're off by. Let me fix that!''
            \end{funnybox}
        \end{column}
    \end{columns}
\end{frame}

% -----------------------------------------------------------------------------
% XGBoost
% -----------------------------------------------------------------------------
\begin{frame}{XGBoost: Gradient Boosting on Steroids}
    \begin{defbox}[XGBoost (Chen \& Guestrin, 2016)]
        Key improvements over vanilla gradient boosting:
        
        \begin{itemize}
            \item \textbf{Regularized objective}: $L = \sum_i l(y_i, \hat{y}_i) + \sum_k \Omega(f_k)$
            \item \textbf{Second-order approximation}: Uses Hessian
            \item \textbf{Efficient implementation}: Parallel, cache-aware
            \item \textbf{Handling missing values}: Built-in
            \item \textbf{Column subsampling}: Like random forest
        \end{itemize}
    \end{defbox}
    
    \vspace{0.2cm}
    
    \begin{successbox}
        XGBoost often wins Kaggle competitions!
        
        Also: \textbf{LightGBM} (Microsoft) and \textbf{CatBoost} (Yandex) are popular alternatives.
    \end{successbox}
\end{frame}

% -----------------------------------------------------------------------------
% Comparison
% -----------------------------------------------------------------------------
\begin{frame}{Ensemble Methods Comparison}
    \centering
    \begin{tabular}{lccc}
        & \textbf{Random Forest} & \textbf{AdaBoost} & \textbf{XGBoost} \\
        \hline
        Training & Parallel & Sequential & Sequential \\
        Reduces & Variance & Bias & Both \\
        Overfitting & Robust & Can overfit & Regularized \\
        Tuning & Easy & Medium & Complex \\
        Interpretable & Somewhat & Somewhat & Less \\
        Speed & Fast & Medium & Fast \\
    \end{tabular}
    
    \vspace{0.5cm}
    
    \begin{keybox}[Rules of Thumb]
        \begin{itemize}
            \item \textbf{Start with}: Random Forest (robust, few hyperparams)
            \item \textbf{For maximum performance}: XGBoost/LightGBM
            \item \textbf{For interpretability}: Single tree (with pruning)
        \end{itemize}
    \end{keybox}
\end{frame}

% -----------------------------------------------------------------------------
% Code
% -----------------------------------------------------------------------------
\begin{frame}{Ensemble Methods in Python}
    \begin{successbox}[Scikit-learn \& XGBoost]
        {\small\ttfamily
        from sklearn.ensemble import RandomForestClassifier\\
        from sklearn.ensemble import GradientBoostingClassifier\\
        import xgboost as xgb\\[0.3em]
        \# Random Forest\\
        rf = RandomForestClassifier(n\_estimators=100, max\_depth=10)\\
        rf.fit(X\_train, y\_train)\\[0.3em]
        \# Gradient Boosting\\
        gb = GradientBoostingClassifier(n\_estimators=100,\\
        \quad\quad\quad\quad\quad\quad\quad\quad\quad\quad learning\_rate=0.1)\\
        gb.fit(X\_train, y\_train)\\[0.3em]
        \# XGBoost\\
        xgb\_model = xgb.XGBClassifier(n\_estimators=100,\\
        \quad\quad\quad\quad\quad\quad\quad\quad\quad\quad learning\_rate=0.1, max\_depth=6)\\
        xgb\_model.fit(X\_train, y\_train)
        }
    \end{successbox}
\end{frame}

% -----------------------------------------------------------------------------
% Key Takeaways
% -----------------------------------------------------------------------------
\begin{frame}{Key Takeaways: Ensemble Methods}
    \begin{keybox}
        \begin{enumerate}
            \item \textbf{Ensemble} = combine multiple weak learners
            \item \textbf{Bagging}: Train in parallel on bootstrap samples
            \begin{itemize}
                \item Reduces variance
            \end{itemize}
            \item \textbf{Random Forest}: Bagging + random feature subsets
            \begin{itemize}
                \item Decorrelates trees
            \end{itemize}
            \item \textbf{Boosting}: Train sequentially, focus on errors
            \begin{itemize}
                \item Reduces bias
            \end{itemize}
            \item \textbf{AdaBoost}: Weight samples and models
            \item \textbf{Gradient Boosting}: Fit residuals (gradients)
            \item \textbf{XGBoost}: Regularized, efficient GB
        \end{enumerate}
    \end{keybox}
    
    \centering
    \textit{Next: Statistical Learning Theory — why does any of this work?}
\end{frame}
