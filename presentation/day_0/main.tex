% =============================================================================
% ML Workshop - Day 0: From Numbers to Neural Networks
% Author: Shuvam Banerji Seal
% A comprehensive journey from counting numbers to machine learning
% =============================================================================
\documentclass[aspectratio=169,9pt]{beamer}

% Load our custom dark theme
\usepackage{../global/sbs_dark_beamer}

% Additional packages for this presentation
\usepackage{animate}
\usepackage{media9}
\usepackage{pgfplots}
\pgfplotsset{compat=1.18}
\usepgfplotslibrary{fillbetween}

% For neural network diagrams
\usetikzlibrary{positioning,arrows.meta,calc,decorations.markings,shapes.geometric}

% =============================================================================
% DOCUMENT METADATA
% =============================================================================
\title{From Numbers to Neural Networks}
\subtitle{A Mathematical Journey into Machine Learning}

% Add author using the custom author system
\addauthor{Shuvam Banerji Seal}{Slashdot - The Programming Club}{sbs22ms076@iiserkol.ac.in}

\date{ML Workshop 2026}

% Short versions for footer (use renewcommand for beamer-defined commands)
\renewcommand{\insertshorttitle}{Numbers to Neural Networks}
\renewcommand{\insertshortauthor}{SBS}
\renewcommand{\insertshortdate}{2026}

% =============================================================================
% CUSTOM COMMANDS FOR THIS PRESENTATION
% =============================================================================

% Funny quote boxes
\newtcolorbox{funnybox}[1][]{
    enhanced,
    colback=bgdark,
    colframe=neonyellow,
    coltext=fgwhite,
    coltitle=bgblack,
    colbacktitle=neonyellow,
    fonttitle=\bfseries\small,
    title={Fun Fact},
    attach boxed title to top left={yshift=-1.5mm,xshift=3mm},
    boxed title style={arc=1.5mm,boxrule=0pt,colback=neonyellow},
    boxrule=0.8pt,
    arc=2mm,
    left=2mm,right=2mm,top=2mm,bottom=2mm,
    drop fuzzy shadow=black,
    fontupper=\small,
    #1
}

% Definition boxes
\newtcolorbox{defbox}[1][Definition]{
    enhanced,
    colback=darkgray,
    colframe=accentcyan,
    coltext=fgwhite,
    coltitle=bgblack,
    colbacktitle=accentcyan,
    fonttitle=\bfseries\small,
    title={#1},
    attach boxed title to top left={yshift=-1.5mm,xshift=3mm},
    boxed title style={arc=1.5mm,boxrule=0pt,colback=accentcyan},
    boxrule=0.8pt,
    arc=2mm,
    left=2mm,right=2mm,top=2mm,bottom=2mm,
    drop fuzzy shadow=black,
    fontupper=\small
}

% Theorem boxes
\newtcolorbox{thmbox}[1][Theorem]{
    enhanced,
    colback=darkgray,
    colframe=neongreen,
    coltext=fgwhite,
    coltitle=bgblack,
    colbacktitle=neongreen,
    fonttitle=\bfseries\small,
    title={#1},
    attach boxed title to top left={yshift=-1.5mm,xshift=3mm},
    boxed title style={arc=1.5mm,boxrule=0pt,colback=neongreen},
    boxrule=0.8pt,
    arc=2mm,
    left=2mm,right=2mm,top=2mm,bottom=2mm,
    drop fuzzy shadow=black,
    fontupper=\small
}

% Key takeaway boxes
\newtcolorbox{keybox}[1][Key Takeaway]{
    enhanced,
    colback=darkgray,
    colframe=neonpink,
    coltext=fgwhite,
    coltitle=bgblack,
    colbacktitle=neonpink,
    fonttitle=\bfseries\small,
    title={#1},
    attach boxed title to top left={yshift=-1.5mm,xshift=3mm},
    boxed title style={arc=1.5mm,boxrule=0pt,colback=neonpink},
    boxrule=0.8pt,
    arc=2mm,
    left=2mm,right=2mm,top=2mm,bottom=2mm,
    drop fuzzy shadow=black,
    fontupper=\small
}

% Neural network node styles - Darker, more saturated colors
\tikzset{
    neuron/.style={
        circle,
        draw=accentcyan,
        fill=accentcyan!25!darkgray,
        minimum size=0.7cm,
        line width=1pt,
        font=\scriptsize,
        text=fgwhite
    },
    input neuron/.style={neuron, draw=neongreen, fill=neongreen!30!darkgray, text=fgwhite},
    hidden neuron/.style={neuron, draw=accentcyan, fill=accentcyan!30!darkgray, text=fgwhite},
    output neuron/.style={neuron, draw=neonpink, fill=neonpink!30!darkgray, text=fgwhite},
    connection/.style={->,>=stealth,line width=0.6pt,draw=softgray!80},
    strong connection/.style={connection, draw=accentcyan, line width=1.2pt}
}

% TikZ default text color for contrast - use white on dark backgrounds
\tikzset{
    every node/.append style={font=\footnotesize, text=fgwhite},
    darktext/.style={text=tikztext},
    lighttext/.style={text=fgwhite}
}

% =============================================================================
% BEGIN DOCUMENT
% =============================================================================
\begin{document}

% Title page
\begin{frame}[plain]
    \titlepage
\end{frame}

% Table of Contents
\tocframe

% =============================================================================
% PART I: MATHEMATICAL FOUNDATIONS
% =============================================================================

% Section 1: Numbers & Counting
% =============================================================================
% Section 1: Foundations of Numbers
% From counting sheep to conquering mathematics
% =============================================================================

\section{Foundations of Numbers}

% -----------------------------------------------------------------------------
% Opening Slide
% -----------------------------------------------------------------------------
\begin{frame}{The Story of Numbers}
    \begin{columns}[T]
        \begin{column}{0.5\textwidth}
            \begin{funnybox}
                \textit{``In the beginning, there was 1. Then someone wanted more pizza, and mathematics was born.''}
            \end{funnybox}
            \vspace{0.5cm}
            \textbf{Our Journey:}
            \begin{itemize}
                \item Natural numbers $\mathbb{N}$
                \item Integers $\mathbb{Z}$
                \item Rational numbers $\mathbb{Q}$
            \end{itemize}
        \end{column}
        \begin{column}{0.5\textwidth}
            \centering
            \begin{tikzpicture}[scale=0.8]
                % Nested sets visualization
                \fill[neonpink!20] (0,0) ellipse (3.5cm and 2.5cm);
                \fill[accentcyan!20] (0,0) ellipse (2.5cm and 1.8cm);
                \fill[neongreen!20] (0,0) ellipse (1.5cm and 1cm);
                
                \draw[neonpink,thick] (0,0) ellipse (3.5cm and 2.5cm);
                \draw[accentcyan,thick] (0,0) ellipse (2.5cm and 1.8cm);
                \draw[neongreen,thick] (0,0) ellipse (1.5cm and 1cm);
                
                \node[neongreen] at (0,0) {$\mathbb{N}$};
                \node[accentcyan] at (0,-1.4) {$\mathbb{Z}$};
                \node[neonpink] at (0,-2.2) {$\mathbb{Q}$};
            \end{tikzpicture}
        \end{column}
    \end{columns}
\end{frame}

% -----------------------------------------------------------------------------
% Natural Numbers - Intuition
% -----------------------------------------------------------------------------
\begin{frame}{Natural Numbers $\mathbb{N}$: The Universe's Original Counting App}
    \begin{columns}[T]
        \begin{column}{0.55\textwidth}
            \begin{defbox}[Natural Numbers]
                The \glow{natural numbers} are the counting numbers:
                \[
                    \mathbb{N} = \{1, 2, 3, 4, 5, \ldots\}
                \]
                Some mathematicians include 0, giving $\mathbb{N}_0 = \{0, 1, 2, \ldots\}$
            \end{defbox}
            
            \vspace{0.3cm}
            
            \textbf{Cave Person Math:}
            \begin{itemize}
                \item SHEEP One sheep
                \item SHEEPSHEEP Two sheep
                \item SHEEPSHEEPSHEEP Three sheep... \textit{zzz}
            \end{itemize}
        \end{column}
        \begin{column}{0.45\textwidth}
            \centering
            \begin{tikzpicture}
                % Number line
                \draw[->,thick,accentcyan] (0,0) -- (5,0);
                \foreach \x in {1,2,3,4} {
                    \fill[neongreen] (\x,0) circle (3pt);
                    \node[below,fgwhite] at (\x,-0.2) {\x};
                }
                \node[above,accentcyan] at (2.5,0.3) {$\mathbb{N}$};
                
                % Dots indicating continuation
                \node[fgwhite] at (4.5,0) {$\cdots$};
            \end{tikzpicture}
            
            \vspace{0.5cm}
            
            \begin{funnybox}
                ``Counting: so easy a caveman did it!''
            \end{funnybox}
        \end{column}
    \end{columns}
\end{frame}

% -----------------------------------------------------------------------------
% Peano Axioms
% -----------------------------------------------------------------------------
\begin{frame}{Peano Axioms: Making Counting Rigorous}
    \begin{thmbox}[Peano Axioms (1889)]
        The natural numbers satisfy:
        \begin{enumerate}
            \item $1 \in \mathbb{N}$ \hfill \textit{(1 exists)}
            \item $\forall n \in \mathbb{N}: S(n) \in \mathbb{N}$ \hfill \textit{(every number has a successor)}
            \item $\forall n \in \mathbb{N}: S(n) \neq 1$ \hfill \textit{(1 is not a successor)}
            \item $S(n) = S(m) \Rightarrow n = m$ \hfill \textit{(successors are unique)}
            \item \textbf{Induction axiom} \hfill \textit{(if true for 1 and $n \Rightarrow n+1$, true for all)}
        \end{enumerate}
    \end{thmbox}
    
    \vspace{0.3cm}
    
    \begin{columns}
        \begin{column}{0.5\textwidth}
            \centering
            \begin{tikzpicture}[scale=0.9]
                \node[circle,draw=neongreen,fill=darkgray,minimum size=0.8cm] (1) at (0,0) {1};
                \node[circle,draw=accentcyan,fill=darkgray,minimum size=0.8cm] (2) at (1.5,0) {2};
                \node[circle,draw=accentcyan,fill=darkgray,minimum size=0.8cm] (3) at (3,0) {3};
                \node[circle,draw=accentcyan,fill=darkgray,minimum size=0.8cm] (4) at (4.5,0) {4};
                \node[fgwhite] at (5.5,0) {$\cdots$};
                
                \draw[->,neonpink,thick] (1) -- (2) node[midway,above,font=\tiny] {$S$};
                \draw[->,neonpink,thick] (2) -- (3) node[midway,above,font=\tiny] {$S$};
                \draw[->,neonpink,thick] (3) -- (4) node[midway,above,font=\tiny] {$S$};
            \end{tikzpicture}
        \end{column}
        \begin{column}{0.5\textwidth}
            \begin{funnybox}
                Giuseppe Peano basically said: ``Here's how to count. You're welcome, humanity.''
            \end{funnybox}
        \end{column}
    \end{columns}
\end{frame}

% -----------------------------------------------------------------------------
% Properties of Natural Numbers
% -----------------------------------------------------------------------------
\begin{frame}{Properties of Natural Numbers}
    \begin{columns}[T]
        \begin{column}{0.5\textwidth}
            \begin{defbox}[Closure Properties]
                For all $a, b \in \mathbb{N}$:
                \begin{align*}
                    a + b &\in \mathbb{N} \quad \checkmark \\
                    a \times b &\in \mathbb{N} \quad \checkmark \\
                    a - b &\in \mathbb{N} \quad \textcolor{neonpink}{\times} \\
                    a \div b &\in \mathbb{N} \quad \textcolor{neonpink}{\times}
                \end{align*}
            \end{defbox}
        \end{column}
        \begin{column}{0.5\textwidth}
            \begin{alertbox}[The Problem]
                What is $3 - 5$? 
                
                \vspace{0.2cm}
                
                Natural numbers can't handle this! We need... \glow{negative numbers}!
            \end{alertbox}
        \end{column}
    \end{columns}
    
    \vspace{0.5cm}
    
    \centering
    \begin{tikzpicture}
        \node[fgwhite] at (0,0) {$5 - 3 = 2$ \textcolor{neongreen}{\checkmark}};
        \node[fgwhite] at (4,0) {$3 - 5 = $ \textcolor{neonpink}{???}};
    \end{tikzpicture}
\end{frame}

% -----------------------------------------------------------------------------
% Integers
% -----------------------------------------------------------------------------
\begin{frame}{Integers $\mathbb{Z}$: When Mathematicians Discovered Debt}
    \begin{columns}[T]
        \begin{column}{0.55\textwidth}
            \begin{defbox}[Integers]
                The \glow{integers} extend natural numbers with negatives and zero:
                \[
                    \mathbb{Z} = \{\ldots, -3, -2, -1, 0, 1, 2, 3, \ldots\}
                \]
            \end{defbox}
            
            \vspace{0.3cm}
            
            \textbf{Real-world integers:}
            \begin{itemize}
                \item Temperature: $-10°C$ (brrr!)
                \item Bank account: $-\$500$ (oops!)
                \item Elevation: $-100m$ (underwater)
            \end{itemize}
        \end{column}
        \begin{column}{0.45\textwidth}
            \centering
            \begin{tikzpicture}[scale=0.7]
                % Full integer number line
                \draw[<->,thick,accentcyan] (-3.5,0) -- (3.5,0);
                \foreach \x in {-3,-2,-1,0,1,2,3} {
                    \fill[neongreen] (\x,0) circle (3pt);
                    \node[below,fgwhite,font=\small] at (\x,-0.3) {\x};
                }
                \node[above,accentcyan] at (0,0.4) {$\mathbb{Z}$};
                
                % Color coding
                \draw[neonpink,thick,decorate,decoration={brace,amplitude=5pt}] 
                    (-3,0.5) -- (-0.2,0.5) node[midway,above=5pt,font=\small] {negative};
                \draw[neongreen,thick,decorate,decoration={brace,amplitude=5pt}] 
                    (0.2,0.5) -- (3,0.5) node[midway,above=5pt,font=\small] {positive};
            \end{tikzpicture}
            
            \vspace{0.3cm}
            
            \begin{funnybox}
                The name $\mathbb{Z}$ comes from German ``Zahlen'' (numbers). Germans: always precise.
            \end{funnybox}
        \end{column}
    \end{columns}
\end{frame}

% -----------------------------------------------------------------------------
% Integer Properties
% -----------------------------------------------------------------------------
\begin{frame}{Integer Properties}
    \begin{columns}[T]
        \begin{column}{0.5\textwidth}
            \begin{defbox}[Closure Properties]
                For all $a, b \in \mathbb{Z}$:
                \begin{align*}
                    a + b &\in \mathbb{Z} \quad \checkmark \\
                    a - b &\in \mathbb{Z} \quad \checkmark \\
                    a \times b &\in \mathbb{Z} \quad \checkmark \\
                    a \div b &\in \mathbb{Z} \quad \textcolor{neonpink}{\times}
                \end{align*}
            \end{defbox}
            
            \vspace{0.3cm}
            
            Now $3 - 5 = -2 \in \mathbb{Z}$ \textcolor{neongreen}{\checkmark}
        \end{column}
        \begin{column}{0.5\textwidth}
            \begin{alertbox}[Still a Problem!]
                What is $1 \div 2$?
                
                \vspace{0.2cm}
                
                $\frac{1}{2} = 0.5 \notin \mathbb{Z}$
                
                \vspace{0.2cm}
                
                We need \glow{fractions}!
            \end{alertbox}
        \end{column}
    \end{columns}
    
    \vspace{0.5cm}
    
    \begin{keybox}
        Each number system fixes a problem but creates a new one. Mathematics evolves by solving its own limitations!
    \end{keybox}
\end{frame}

% -----------------------------------------------------------------------------
% Rational Numbers
% -----------------------------------------------------------------------------
\begin{frame}{Rational Numbers $\mathbb{Q}$: Sharing Pizza Among Friends}
    \begin{columns}[T]
        \begin{column}{0.55\textwidth}
            \begin{defbox}[Rational Numbers]
                The \glow{rational numbers} are all fractions:
                \[
                    \mathbb{Q} = \left\{ \frac{p}{q} \;\middle|\; p, q \in \mathbb{Z}, q \neq 0 \right\}
                \]
            \end{defbox}
            
            \vspace{0.3cm}
            
            \textbf{Examples:}
            \begin{itemize}
                \item $\frac{1}{2} = 0.5$
                \item $\frac{22}{7} \approx 3.14159...$
                \item $\frac{-3}{4} = -0.75$
                \item $\frac{6}{3} = 2$ (integers are rational too!)
            \end{itemize}
        \end{column}
        \begin{column}{0.45\textwidth}
            \centering
            \begin{tikzpicture}[scale=0.6]
                % Pizza visualization
                \fill[neonyellow!30] (0,0) circle (1.5cm);
                \draw[neonyellow,thick] (0,0) circle (1.5cm);
                \draw[neonyellow] (0,0) -- (0,1.5);
                \draw[neonyellow] (0,0) -- (1.5,0);
                \draw[neonyellow] (0,0) -- (0,-1.5);
                \draw[neonyellow] (0,0) -- (-1.5,0);
                
                \node[fgwhite,font=\small] at (0,-2.2) {$\frac{1}{4}$ each!};
            \end{tikzpicture}
            
            \vspace{0.3cm}
            
            \begin{funnybox}
                ``$\mathbb{Q}$'' for \textbf{Q}uotient. Mathematicians love their abbreviations!
            \end{funnybox}
        \end{column}
    \end{columns}
\end{frame}

% -----------------------------------------------------------------------------
% Density of Rationals
% -----------------------------------------------------------------------------
\begin{frame}{The Density of Rational Numbers}
    \begin{thmbox}[Density of $\mathbb{Q}$]
        Between \textbf{any} two rational numbers, there exists another rational number.
        
        \vspace{0.2cm}
        
        If $a, b \in \mathbb{Q}$ with $a < b$, then $\frac{a+b}{2} \in \mathbb{Q}$ and $a < \frac{a+b}{2} < b$.
    \end{thmbox}
    
    \vspace{0.3cm}
    
    \centering
    \begin{tikzpicture}[scale=0.85]
        % Number line segment
        \draw[thick,accentcyan] (0,0) -- (6,0);
        
        % Points
        \fill[neongreen] (1,0) circle (3pt) node[below=3pt,fgwhite] {$0$};
        \fill[neongreen] (5,0) circle (3pt) node[below=3pt,fgwhite] {$1$};
        
        % Midpoint
        \fill[neonyellow] (3,0) circle (3pt) node[below=3pt,fgwhite] {$\frac{1}{2}$};
        
        % More midpoints
        \fill[neonpink] (2,0) circle (2pt) node[below=3pt,fgwhite,font=\tiny] {$\frac{1}{4}$};
        \fill[neonpink] (4,0) circle (2pt) node[below=3pt,fgwhite,font=\tiny] {$\frac{3}{4}$};
        
        % Even more
        \fill[softgray] (1.5,0) circle (1.5pt);
        \fill[softgray] (2.5,0) circle (1.5pt);
        \fill[softgray] (3.5,0) circle (1.5pt);
        \fill[softgray] (4.5,0) circle (1.5pt);
    \end{tikzpicture}
    
    \vspace{0.3cm}
    
    \begin{funnybox}
        There are \textbf{infinitely many} rationals between any two rationals. Mind = blown! MINDBLOWN
    \end{funnybox}
\end{frame}

% -----------------------------------------------------------------------------
% But Wait - Irrationals Preview
% -----------------------------------------------------------------------------
\begin{frame}{But Wait... Are Rationals Enough?}
    \begin{columns}[T]
        \begin{column}{0.5\textwidth}
            \begin{alertbox}[The Pythagorean Nightmare]
                Consider a square with side 1:
                
                \vspace{0.3cm}
                
                \centering
                \begin{tikzpicture}[scale=1.0]
                    \draw[accentcyan,thick] (0,0) -- (1,0) -- (1,1) -- (0,1) -- cycle;
                    \draw[neonpink,thick] (0,0) -- (1,1);
                    
                    \node[below,fgwhite,font=\small] at (0.5,0) {1};
                    \node[left,fgwhite,font=\small] at (0,0.5) {1};
                    \node[above right,neonpink,font=\small] at (0.5,0.5) {$\sqrt{2}$};
                \end{tikzpicture}
                
                \vspace{0.3cm}
                
                By Pythagoras: $d^2 = 1^2 + 1^2 = 2$
                
                So $d = \sqrt{2}$... but is $\sqrt{2} \in \mathbb{Q}$?
            \end{alertbox}
        \end{column}
        \begin{column}{0.5\textwidth}
            \begin{infobox}[Spoiler Alert]
                $\sqrt{2}$ is \textbf{NOT} rational!
                
                \vspace{0.2cm}
                
                There exist numbers that \textbf{cannot} be written as $\frac{p}{q}$.
                
                \vspace{0.2cm}
                
                These are called \glow[neonpink]{irrational numbers}.
                
                \vspace{0.2cm}
                
                Coming up next: Real numbers $\mathbb{R}$!
            \end{infobox}
        \end{column}
    \end{columns}
\end{frame}

% -----------------------------------------------------------------------------
% Number Hierarchy Summary
% -----------------------------------------------------------------------------
\begin{frame}{The Number Hierarchy}
    \centering
    \begin{tikzpicture}[scale=0.9]
        % Nested sets with better visualization
        \fill[neonpink!10] (0,0) ellipse (5cm and 3cm);
        \fill[accentcyan!15] (0,0) ellipse (3.5cm and 2.2cm);
        \fill[neonyellow!15] (0,0) ellipse (2.2cm and 1.4cm);
        \fill[neongreen!20] (0,0) ellipse (1cm and 0.7cm);
        
        \draw[neonpink,thick] (0,0) ellipse (5cm and 3cm);
        \draw[accentcyan,thick] (0,0) ellipse (3.5cm and 2.2cm);
        \draw[neonyellow,thick] (0,0) ellipse (2.2cm and 1.4cm);
        \draw[neongreen,thick] (0,0) ellipse (1cm and 0.7cm);
        
        % Labels
        \node[neongreen,font=\bfseries] at (0,0) {$\mathbb{N}$};
        \node[neonyellow,font=\bfseries] at (0,-1.1) {$\mathbb{Z}$};
        \node[accentcyan,font=\bfseries] at (0,-1.9) {$\mathbb{Q}$};
        \node[neonpink,font=\bfseries] at (0,-2.7) {$\mathbb{R}$};
        
        % Side labels
        \node[neongreen,right,font=\small] at (1.2,0) {Natural: $1, 2, 3, \ldots$};
        \node[neonyellow,right,font=\small] at (2.4,-0.5) {Integers: $\ldots, -1, 0, 1, \ldots$};
        \node[accentcyan,right,font=\small] at (3.7,-1) {Rationals: $\frac{p}{q}$};
        \node[neonpink,right,font=\small] at (5.2,-1.5) {Reals: $\sqrt{2}, \pi, e$};
    \end{tikzpicture}
    
    \vspace{0.3cm}
    
    \[
        \mathbb{N} \subset \mathbb{Z} \subset \mathbb{Q} \subset \mathbb{R}
    \]
\end{frame}

% -----------------------------------------------------------------------------
% Key Takeaways
% -----------------------------------------------------------------------------
\begin{frame}{Key Takeaways: Numbers}
    \begin{keybox}
        \begin{enumerate}
            \item \textbf{Natural numbers} $\mathbb{N}$: Counting (1, 2, 3, ...)
            \item \textbf{Integers} $\mathbb{Z}$: Add zero and negatives (..., -1, 0, 1, ...)
            \item \textbf{Rationals} $\mathbb{Q}$: All fractions $\frac{p}{q}$
            \item Each extension solves a problem:
            \begin{itemize}
                \item $\mathbb{Z}$ allows subtraction
                \item $\mathbb{Q}$ allows division
            \end{itemize}
            \item Rationals are \textbf{dense} but not \textbf{complete}
            \item Some numbers (like $\sqrt{2}$) are \textbf{irrational}
        \end{enumerate}
    \end{keybox}
    
    \vspace{0.3cm}
    
    \centering
    \textit{Next: Real and Complex Numbers — where it gets \glow{really} interesting!}
\end{frame}


% Section 2: Real & Complex Numbers
% =============================================================================
% Section 2: Real & Complex Numbers
% Where mathematics gets truly beautiful
% =============================================================================

\section{Real \& Complex Numbers}

% -----------------------------------------------------------------------------
% Opening
% -----------------------------------------------------------------------------
\begin{frame}{Beyond Rationals: The Irrational Truth}
    \begin{columns}[T]
        \begin{column}{0.5\textwidth}
            \begin{funnybox}
                \textit{``When the Pythagoreans discovered irrational numbers, they were so upset they allegedly drowned the messenger. Math: serious business since 500 BC.''}
            \end{funnybox}
            
            \vspace{0.3cm}
            
            \textbf{The Problem:}
            
            Not all numbers can be written as $\frac{p}{q}$!
        \end{column}
        \begin{column}{0.5\textwidth}
            \centering
            \begin{tikzpicture}[scale=0.85]
                % Unit square with diagonal
                \draw[accentcyan,thick] (0,0) rectangle (2,2);
                \draw[neonpink,very thick] (0,0) -- (2,2);
                
                \node[below,fgwhite] at (1,0) {$1$};
                \node[left,fgwhite] at (0,1) {$1$};
                \node[above right,neonpink] at (1,1) {$\sqrt{2}$};
                
                % Question mark
                \node[neonyellow,font=\Huge] at (3,1) {?};
            \end{tikzpicture}
        \end{column}
    \end{columns}
\end{frame}

% -----------------------------------------------------------------------------
% Proof that sqrt(2) is irrational
% -----------------------------------------------------------------------------
\begin{frame}{Proof: $\sqrt{2}$ is Irrational}
    \begin{thmbox}[Theorem]
        $\sqrt{2}$ cannot be expressed as a fraction $\frac{p}{q}$ where $p, q \in \mathbb{Z}$.
    \end{thmbox}
    
    \textbf{Proof by Contradiction:}
    \begin{enumerate}
        \item Assume $\sqrt{2} = \frac{p}{q}$ in lowest terms (gcd$(p,q) = 1$)
        \item Then $2 = \frac{p^2}{q^2}$, so $p^2 = 2q^2$
        \item Therefore $p^2$ is even, which means $p$ is even
        \item Let $p = 2k$, then $(2k)^2 = 2q^2 \Rightarrow 4k^2 = 2q^2 \Rightarrow q^2 = 2k^2$
        \item So $q^2$ is even, meaning $q$ is even
        \item \textcolor{neonpink}{Contradiction!} Both $p$ and $q$ are even, but we said gcd$(p,q) = 1$
    \end{enumerate}
    
    \begin{keybox}
        $\sqrt{2} \approx 1.41421356...$ goes on forever without repeating! \qed
    \end{keybox}
\end{frame}

% -----------------------------------------------------------------------------
% Famous Irrationals
% -----------------------------------------------------------------------------
\begin{frame}{Famous Irrational Numbers}
    \begin{columns}[T]
        \begin{column}{0.33\textwidth}
            \centering
            \begin{tikzpicture}
                \fill[neonpink!20] (0,0) circle (1cm);
                \draw[neonpink,thick] (0,0) circle (1cm);
                \node[neonpink,font=\huge] at (0,0) {$\pi$};
            \end{tikzpicture}
            
            \vspace{0.2cm}
            
            $\pi = 3.14159...$
            
            \vspace{0.1cm}
            
            {\small Ratio of circumference to diameter}
        \end{column}
        \begin{column}{0.33\textwidth}
            \centering
            \begin{tikzpicture}
                \draw[neongreen,thick,domain=0:1.5,smooth,variable=\x] 
                    plot ({\x},{exp(\x)/3});
                \node[neongreen,font=\huge] at (0.75,1.5) {$e$};
            \end{tikzpicture}
            
            \vspace{0.2cm}
            
            $e = 2.71828...$
            
            \vspace{0.1cm}
            
            {\small Base of natural logarithm}
        \end{column}
        \begin{column}{0.33\textwidth}
            \centering
            \begin{tikzpicture}
                \draw[neonyellow,thick] (0,0) -- (1.618,0) -- (1.618,1) -- (0,1) -- cycle;
                \draw[neonyellow] (1,0) -- (1,1);
                \node[neonyellow,font=\huge] at (0.8,1.5) {$\phi$};
            \end{tikzpicture}
            
            \vspace{0.2cm}
            
            $\phi = 1.61803...$
            
            \vspace{0.1cm}
            
            {\small Golden ratio: $\frac{1+\sqrt{5}}{2}$}
        \end{column}
    \end{columns}
    
    \vspace{0.5cm}
    
    \begin{funnybox}
        These numbers show up \textit{everywhere}: nature, art, physics, and yes... machine learning!
    \end{funnybox}
\end{frame}

% -----------------------------------------------------------------------------
% Real Numbers Definition
% -----------------------------------------------------------------------------
\begin{frame}{Real Numbers $\mathbb{R}$: Filling the Gaps}
    \begin{defbox}[Real Numbers]
        The \glow{real numbers} $\mathbb{R}$ include all rationals AND all irrationals:
        \[
            \mathbb{R} = \mathbb{Q} \cup \{\text{irrational numbers}\}
        \]
        
        They form a \textbf{complete}, \textbf{ordered} field — every point on the number line!
    \end{defbox}
    
    \vspace{0.3cm}
    
    \centering
    \begin{tikzpicture}[scale=0.85]
        % Continuous number line
        \shade[left color=neonpink!50,right color=accentcyan!50] (-4,-0.1) rectangle (4,0.1);
        \draw[<->,thick,fgwhite] (-4.5,0) -- (4.5,0);
        
        % Mark some points
        \fill[neonyellow] (-3.14159,0) circle (3pt) node[above=3pt,font=\small] {$-\pi$};
        \fill[neongreen] (-1,0) circle (3pt) node[below=3pt,font=\small] {$-1$};
        \fill[accentcyan] (0,0) circle (3pt) node[below=3pt,font=\small] {$0$};
        \fill[neongreen] (1,0) circle (3pt) node[below=3pt,font=\small] {$1$};
        \fill[neonpink] (1.414,0) circle (3pt) node[above=3pt,font=\small] {$\sqrt{2}$};
        \fill[neonyellow] (2.718,0) circle (3pt) node[above=3pt,font=\small] {$e$};
        \fill[neonyellow] (3.14159,0) circle (3pt) node[below=3pt,font=\small] {$\pi$};
    \end{tikzpicture}
    
    \vspace{0.3cm}
    
    \begin{funnybox}
        ``Nature hates empty spaces'' — the number line is now completely filled!
    \end{funnybox}
\end{frame}

% -----------------------------------------------------------------------------
% Completeness Axiom
% -----------------------------------------------------------------------------
\begin{frame}{The Completeness Axiom}
    \begin{thmbox}[Completeness of $\mathbb{R}$]
        Every non-empty subset of $\mathbb{R}$ that is bounded above has a \textbf{least upper bound} (supremum) in $\mathbb{R}$.
    \end{thmbox}
    
    \vspace{0.3cm}
    
    \textbf{What this means:}
    \begin{itemize}
        \item No ``holes'' in the number line
        \item Limits of convergent sequences always exist
        \item This is what makes calculus possible!
    \end{itemize}
    
    \vspace{0.3cm}
    
    \begin{columns}
        \begin{column}{0.5\textwidth}
            \begin{alertbox}[$\mathbb{Q}$ is NOT complete]
                The sequence $1, 1.4, 1.41, 1.414, ...$
                
                converges to $\sqrt{2}$, but $\sqrt{2} \notin \mathbb{Q}$!
            \end{alertbox}
        \end{column}
        \begin{column}{0.5\textwidth}
            \begin{successbox}[$\mathbb{R}$ IS complete]
                Every convergent sequence of reals has its limit in $\mathbb{R}$.
            \end{successbox}
        \end{column}
    \end{columns}
\end{frame}

% -----------------------------------------------------------------------------
% Complex Numbers - The Final Frontier
% -----------------------------------------------------------------------------
\begin{frame}{Complex Numbers $\mathbb{C}$: Inventing the Impossible}
    \begin{columns}[T]
        \begin{column}{0.55\textwidth}
            \begin{alertbox}[The Last Problem]
                What is $\sqrt{-1}$?
                
                \vspace{0.2cm}
                
                No real number squared gives $-1$!
                
                \vspace{0.2cm}
                
                $x^2 = -1 \Rightarrow x = $ \textcolor{neonpink}{???}
            \end{alertbox}
            
            \vspace{0.3cm}
            
            \begin{defbox}[Imaginary Unit]
                Define $i$ such that:
                \[
                    \mathbf{i^2 = -1}
                \]
                
                Then $\sqrt{-1} = i$
            \end{defbox}
        \end{column}
        \begin{column}{0.45\textwidth}
            \begin{funnybox}
                ``Can't solve it? Just \textbf{invent} a number!''
                
                \vspace{0.2cm}
                
                — Mathematicians, probably
            \end{funnybox}
            
            \vspace{0.3cm}
            
            \textbf{Powers of $i$:}
            \begin{align*}
                i^0 &= 1 \\
                i^1 &= i \\
                i^2 &= -1 \\
                i^3 &= -i \\
                i^4 &= 1 \text{ (cycle repeats!)}
            \end{align*}
        \end{column}
    \end{columns}
\end{frame}

% -----------------------------------------------------------------------------
% Complex Number Definition
% -----------------------------------------------------------------------------
\begin{frame}{Complex Numbers: Definition}
    \begin{defbox}[Complex Numbers]
        A \glow{complex number} has the form:
        \[
            z = a + bi
        \]
        where $a, b \in \mathbb{R}$ and $i^2 = -1$.
        
        \vspace{0.2cm}
        
        \begin{itemize}
            \item $a$ = \textbf{real part}: Re$(z)$
            \item $b$ = \textbf{imaginary part}: Im$(z)$
        \end{itemize}
        
        \vspace{0.2cm}
        
        The set of all complex numbers:
        \[
            \mathbb{C} = \{a + bi \mid a, b \in \mathbb{R}\}
        \]
    \end{defbox}
    
    \vspace{0.3cm}
    
    \textbf{Examples:}
    \begin{itemize}
        \item $3 + 2i$ (real part 3, imaginary part 2)
        \item $-1 - 4i$ (real part $-1$, imaginary part $-4$)
        \item $5 = 5 + 0i$ (real numbers are complex too!)
        \item $2i = 0 + 2i$ (purely imaginary)
    \end{itemize}
\end{frame}

% -----------------------------------------------------------------------------
% Complex Plane
% -----------------------------------------------------------------------------
\begin{frame}{The Complex Plane}
    \begin{columns}[T]
        \begin{column}{0.5\textwidth}
            \centering
            \begin{tikzpicture}[scale=0.9]
                % Axes
                \draw[<->,thick,accentcyan] (-3,0) -- (3,0) node[right] {Re};
                \draw[<->,thick,accentcyan] (0,-3) -- (0,3) node[above] {Im};
                
                % Grid
                \draw[softgray,very thin] (-2.5,-2.5) grid (2.5,2.5);
                
                % Complex number
                \fill[neonpink] (2,1.5) circle (4pt);
                \draw[neonpink,thick,->] (0,0) -- (2,1.5);
                \node[neonpink,above right] at (2,1.5) {$z = 2 + 1.5i$};
                
                % Projections
                \draw[neonyellow,dashed] (2,0) -- (2,1.5);
                \draw[neonyellow,dashed] (0,1.5) -- (2,1.5);
                
                % Labels
                \node[below,fgwhite] at (2,0) {$2$};
                \node[left,fgwhite] at (0,1.5) {$1.5i$};
                
                % Magnitude
                \node[neongreen,font=\small] at (1.3,0.4) {$|z|$};
            \end{tikzpicture}
        \end{column}
        \begin{column}{0.5\textwidth}
            \begin{infobox}[Argand Diagram]
                Complex numbers live on a 2D plane!
                
                \vspace{0.2cm}
                
                \begin{itemize}
                    \item x-axis: real part
                    \item y-axis: imaginary part
                \end{itemize}
                
                \vspace{0.2cm}
                
                \textbf{Modulus} (magnitude):
                \[
                    |z| = \sqrt{a^2 + b^2}
                \]
                
                \textbf{Argument} (angle):
                \[
                    \arg(z) = \tan^{-1}\left(\frac{b}{a}\right)
                \]
            \end{infobox}
        \end{column}
    \end{columns}
\end{frame}

% -----------------------------------------------------------------------------
% Complex Arithmetic
% -----------------------------------------------------------------------------
\begin{frame}{Complex Arithmetic}
    Let $z_1 = a + bi$ and $z_2 = c + di$
    
    \vspace{0.3cm}
    
    \begin{columns}[T]
        \begin{column}{0.5\textwidth}
            \begin{defbox}[Addition]
                \[
                    z_1 + z_2 = (a+c) + (b+d)i
                \]
                
                Just add real and imaginary parts separately!
            \end{defbox}
            
            \vspace{0.3cm}
            
            \begin{defbox}[Subtraction]
                \[
                    z_1 - z_2 = (a-c) + (b-d)i
                \]
            \end{defbox}
        \end{column}
        \begin{column}{0.5\textwidth}
            \begin{defbox}[Multiplication]
                \begin{align*}
                    z_1 \cdot z_2 &= (a+bi)(c+di) \\
                    &= ac + adi + bci + bdi^2 \\
                    &= (ac - bd) + (ad + bc)i
                \end{align*}
                
                Remember: $i^2 = -1$
            \end{defbox}
        \end{column}
    \end{columns}
    
    \vspace{0.3cm}
    
    \textbf{Example:} $(2 + 3i)(1 - i) = 2 - 2i + 3i - 3i^2 = 2 + i + 3 = 5 + i$
\end{frame}

% -----------------------------------------------------------------------------
% Euler's Formula
% -----------------------------------------------------------------------------
\begin{frame}{Euler's Formula: The Most Beautiful Equation}
    \begin{thmbox}[Euler's Formula]
        \[
            \mathbf{e^{i\theta} = \cos\theta + i\sin\theta}
        \]
    \end{thmbox}
    
    \vspace{0.3cm}
    
    \begin{columns}[T]
        \begin{column}{0.5\textwidth}
            \centering
            \begin{tikzpicture}[scale=0.85]
                % Unit circle
                \draw[accentcyan,thick] (0,0) circle (1.5cm);
                \draw[<->,thick,softgray] (-2,0) -- (2,0) node[right] {Re};
                \draw[<->,thick,softgray] (0,-2) -- (0,2) node[above] {Im};
                
                % Point on circle
                \coordinate (P) at (60:1.5);
                \fill[neonpink] (P) circle (3pt);
                \draw[neonpink,thick,->] (0,0) -- (P);
                
                % Angle
                \draw[neonyellow,thick] (0.4,0) arc (0:60:0.4);
                \node[neonyellow,font=\small] at (0.6,0.25) {$\theta$};
                
                % Projections
                \draw[neongreen,dashed] (P) -- (60:1.5 |- 0,0);
                \draw[accentcyan,dashed] (P) -- (0,0 |- P);
                
                \node[below,neongreen,font=\small] at (0.75,0) {$\cos\theta$};
                \node[left,accentcyan,font=\small] at (0,1) {$\sin\theta$};
                
                \node[neonpink,above right] at (P) {$e^{i\theta}$};
            \end{tikzpicture}
        \end{column}
        \begin{column}{0.5\textwidth}
            When $\theta = \pi$:
            \[
                e^{i\pi} = \cos\pi + i\sin\pi = -1
            \]
            
            \begin{keybox}[Euler's Identity]
                \[
                    \mathbf{e^{i\pi} + 1 = 0}
                \]
                
                Links five fundamental constants:
                \begin{itemize}
                    \item $e$ (natural base)
                    \item $i$ (imaginary unit)
                    \item $\pi$ (pi)
                    \item $1$ (multiplicative identity)
                    \item $0$ (additive identity)
                \end{itemize}
            \end{keybox}
        \end{column}
    \end{columns}
\end{frame}

% -----------------------------------------------------------------------------
% Why Complex Numbers Matter
% -----------------------------------------------------------------------------
\begin{frame}{Why Complex Numbers Matter in ML}
    \begin{columns}[T]
        \begin{column}{0.5\textwidth}
            \begin{infobox}[Applications]
                \begin{itemize}
                    \item \textbf{Fourier Transforms}
                    \begin{itemize}
                        \item Signal processing
                        \item Audio/image analysis
                    \end{itemize}
                    \item \textbf{Quantum Computing}
                    \begin{itemize}
                        \item Quantum states use $\mathbb{C}$
                    \end{itemize}
                    \item \textbf{Eigenvalues}
                    \begin{itemize}
                        \item Can be complex!
                        \item PCA, neural network analysis
                    \end{itemize}
                    \item \textbf{Rotations}
                    \begin{itemize}
                        \item Complex multiplication = rotation
                    \end{itemize}
                \end{itemize}
            \end{infobox}
        \end{column}
        \begin{column}{0.5\textwidth}
            \centering
            \begin{tikzpicture}[scale=0.8]
                % Show multiplication as rotation
                \draw[softgray] (0,0) circle (2cm);
                \draw[<->,softgray] (-2.5,0) -- (2.5,0);
                \draw[<->,softgray] (0,-2.5) -- (0,2.5);
                
                % Original vector
                \draw[accentcyan,thick,->] (0,0) -- (2,0);
                \node[accentcyan,below] at (2,0) {$z$};
                
                % Rotated vector
                \draw[neonpink,thick,->] (0,0) -- (1.414,1.414);
                \node[neonpink,above right] at (1.414,1.414) {$z \cdot i$};
                
                % Angle
                \draw[neonyellow] (0.5,0) arc (0:45:0.5);
                \node[neonyellow,font=\small] at (0.8,0.3) {$90°$};
            \end{tikzpicture}
            
            \vspace{0.3cm}
            
            \begin{funnybox}
                Multiplying by $i$ rotates by 90°!
                
                Complex numbers = rotation superpowers!
            \end{funnybox}
        \end{column}
    \end{columns}
\end{frame}

% -----------------------------------------------------------------------------
% Complete Number Hierarchy
% -----------------------------------------------------------------------------
\begin{frame}{The Complete Number Hierarchy}
    \centering
    \begin{tikzpicture}[scale=0.85]
        % Nested sets
        \fill[neonblue!10] (0,0) ellipse (6cm and 3.5cm);
        \fill[neonpink!10] (0,0) ellipse (4.5cm and 2.7cm);
        \fill[accentcyan!15] (0,0) ellipse (3.2cm and 2cm);
        \fill[neonyellow!15] (0,0) ellipse (2cm and 1.3cm);
        \fill[neongreen!20] (0,0) ellipse (0.9cm and 0.6cm);
        
        \draw[neonblue,thick] (0,0) ellipse (6cm and 3.5cm);
        \draw[neonpink,thick] (0,0) ellipse (4.5cm and 2.7cm);
        \draw[accentcyan,thick] (0,0) ellipse (3.2cm and 2cm);
        \draw[neonyellow,thick] (0,0) ellipse (2cm and 1.3cm);
        \draw[neongreen,thick] (0,0) ellipse (0.9cm and 0.6cm);
        
        % Labels
        \node[neongreen,font=\bfseries] at (0,0) {$\mathbb{N}$};
        \node[neonyellow,font=\bfseries] at (0,-1) {$\mathbb{Z}$};
        \node[accentcyan,font=\bfseries] at (0,-1.7) {$\mathbb{Q}$};
        \node[neonpink,font=\bfseries] at (0,-2.4) {$\mathbb{R}$};
        \node[neonblue,font=\bfseries] at (0,-3.2) {$\mathbb{C}$};
        
        % Side annotations
        \node[neongreen,right,font=\small] at (1.1,0) {$1, 2, 3, ...$};
        \node[neonyellow,right,font=\small] at (2.2,-0.3) {$..., -1, 0, 1, ...$};
        \node[accentcyan,right,font=\small] at (3.4,-0.8) {$\frac{p}{q}$};
        \node[neonpink,right,font=\small] at (4.7,-1.3) {$\sqrt{2}, \pi, e$};
        \node[neonblue,right,font=\small] at (6.2,-1.8) {$a + bi$};
    \end{tikzpicture}
    
    \vspace{0.3cm}
    
    \[
        \mathbb{N} \subset \mathbb{Z} \subset \mathbb{Q} \subset \mathbb{R} \subset \mathbb{C}
    \]
\end{frame}

% -----------------------------------------------------------------------------
% Key Takeaways
% -----------------------------------------------------------------------------
\begin{frame}{Key Takeaways: Real \& Complex Numbers}
    \begin{keybox}
        \begin{enumerate}
            \item \textbf{Irrational numbers} ($\sqrt{2}, \pi, e$) cannot be written as fractions
            \item \textbf{Real numbers} $\mathbb{R}$ = rationals + irrationals (complete number line)
            \item \textbf{Complex numbers} $\mathbb{C}$: $z = a + bi$ where $i^2 = -1$
            \item \textbf{Euler's formula}: $e^{i\theta} = \cos\theta + i\sin\theta$
            \item \textbf{Euler's identity}: $e^{i\pi} + 1 = 0$ (the most beautiful equation!)
            \item Complex numbers enable:
            \begin{itemize}
                \item Fourier transforms (signal processing)
                \item Quantum computing
                \item Elegant rotations
            \end{itemize}
        \end{enumerate}
    \end{keybox}
    
    \vspace{0.2cm}
    
    \centering
    \textit{Next: Functions — the building blocks of everything!}
\end{frame}


% Section 3: Functions & Parameters
% =============================================================================
% Section 3: Functions & Parameters
% The building blocks of mathematical relationships
% =============================================================================

\section{Functions \& Parameters}

% -----------------------------------------------------------------------------
% Opening
% -----------------------------------------------------------------------------
\begin{frame}{Functions: Mathematical Vending Machines}
    \begin{columns}[T]
        \begin{column}{0.5\textwidth}
            \begin{funnybox}
                \textit{``A function is like a vending machine: put in coins (input), get a snack (output). No coins? No snack. Wrong coins? Error!''}
            \end{funnybox}
            
            \vspace{0.3cm}
            
            \textbf{Key Questions:}
            \begin{itemize}
                \item What goes in? (Domain)
                \item What comes out? (Range)
                \item What's the rule? (Function definition)
            \end{itemize}
        \end{column}
        \begin{column}{0.5\textwidth}
            \centering
            \begin{tikzpicture}[scale=0.8]
                % Vending machine
                \draw[accentcyan,thick,rounded corners] (-1.5,-2) rectangle (1.5,2);
                
                % Input slot
                \draw[neongreen,thick,fill=darkgray] (-0.5,1.5) rectangle (0.5,1.8);
                \node[neongreen,above,font=\small] at (0,1.8) {Input $x$};
                
                % Function box
                \node[accentcyan,font=\Large] at (0,0.5) {$f$};
                \draw[accentcyan] (-0.8,0) rectangle (0.8,1);
                
                % Output slot
                \draw[neonpink,thick,fill=darkgray] (-0.5,-1.8) rectangle (0.5,-1.5);
                \node[neonpink,below,font=\small] at (0,-1.8) {Output $f(x)$};
                
                % Arrow
                \draw[->,neonyellow,thick] (0,1.5) -- (0,1);
                \draw[->,neonyellow,thick] (0,0) -- (0,-1.5);
            \end{tikzpicture}
        \end{column}
    \end{columns}
\end{frame}

% -----------------------------------------------------------------------------
% Formal Definition
% -----------------------------------------------------------------------------
\begin{frame}{Formal Definition of a Function}
    \begin{defbox}[Function]
        A \glow{function} $f$ from set $A$ to set $B$, written $f: A \rightarrow B$, is a rule that assigns to \textbf{each} element $x \in A$ \textbf{exactly one} element $f(x) \in B$.
        
        \vspace{0.3cm}
        
        \begin{itemize}
            \item $A$ = \textbf{Domain} (all valid inputs)
            \item $B$ = \textbf{Codomain} (possible outputs)
            \item $\{f(x) : x \in A\}$ = \textbf{Range} (actual outputs)
        \end{itemize}
    \end{defbox}
    
    \vspace{0.3cm}
    
    \begin{columns}
        \begin{column}{0.5\textwidth}
            \centering
            \begin{tikzpicture}[scale=0.7]
                % Domain
                \draw[neongreen,thick] (-2,0) ellipse (1cm and 1.5cm);
                \node[neongreen,above] at (-2,1.7) {Domain $A$};
                \foreach \y in {0.8,0,-0.8} {
                    \fill[neongreen] (-2,\y) circle (3pt);
                }
                
                % Codomain
                \draw[neonpink,thick] (2,0) ellipse (1cm and 1.5cm);
                \node[neonpink,above] at (2,1.7) {Codomain $B$};
                \foreach \y in {0.8,0,-0.8} {
                    \fill[neonpink] (2,\y) circle (3pt);
                }
                
                % Arrows
                \draw[->,accentcyan,thick] (-1.2,0.8) -- (1.2,0);
                \draw[->,accentcyan,thick] (-1.2,0) -- (1.2,0.8);
                \draw[->,accentcyan,thick] (-1.2,-0.8) -- (1.2,-0.8);
                
                \node[accentcyan,above] at (0,0.5) {$f$};
            \end{tikzpicture}
        \end{column}
        \begin{column}{0.5\textwidth}
            \begin{alertbox}[Key Rule]
                Each input maps to \textbf{exactly one} output!
                
                \vspace{0.2cm}
                
                One input $\rightarrow$ multiple outputs = NOT a function!
            \end{alertbox}
        \end{column}
    \end{columns}
\end{frame}

% -----------------------------------------------------------------------------
% Function Notation
% -----------------------------------------------------------------------------
\begin{frame}{Function Notation}
    \begin{columns}[T]
        \begin{column}{0.5\textwidth}
            \begin{defbox}[Standard Notation]
                \textbf{Defining a function:}
                \begin{align*}
                    f(x) &= x^2 \\
                    g(x) &= 2x + 1 \\
                    h(x, y) &= x^2 + y^2
                \end{align*}
                
                \textbf{Evaluating:}
                \begin{align*}
                    f(3) &= 3^2 = 9 \\
                    g(-1) &= 2(-1) + 1 = -1 \\
                    h(1, 2) &= 1 + 4 = 5
                \end{align*}
            \end{defbox}
        \end{column}
        \begin{column}{0.5\textwidth}
            \begin{defbox}[Arrow Notation]
                \[
                    f: \mathbb{R} \rightarrow \mathbb{R}
                \]
                \[
                    x \mapsto x^2
                \]
                
                ``$f$ takes real numbers to real numbers, mapping $x$ to $x^2$''
            \end{defbox}
            
            \vspace{0.3cm}
            
            \begin{funnybox}
                Same thing, fancier packaging. Mathematicians love options!
            \end{funnybox}
        \end{column}
    \end{columns}
\end{frame}

% -----------------------------------------------------------------------------
% Domain and Range
% -----------------------------------------------------------------------------
\begin{frame}{Domain, Codomain, and Range}
    \textbf{Example:} $f(x) = \sqrt{x}$ where $f: \mathbb{R}_{\geq 0} \rightarrow \mathbb{R}$
    
    \vspace{0.3cm}
    
    \begin{columns}[T]
        \begin{column}{0.33\textwidth}
            \begin{infobox}[Domain]
                All valid inputs
                
                \vspace{0.2cm}
                
                For $\sqrt{x}$: $x \geq 0$
                
                (Can't sqrt negative reals!)
                
                \[
                    \text{Dom}(f) = [0, \infty)
                \]
            \end{infobox}
        \end{column}
        \begin{column}{0.33\textwidth}
            \begin{infobox}[Codomain]
                Set of possible outputs
                
                \vspace{0.2cm}
                
                Here: all real numbers $\mathbb{R}$
                
                (We \textit{could} land anywhere)
                
                \[
                    \text{Codomain} = \mathbb{R}
                \]
            \end{infobox}
        \end{column}
        \begin{column}{0.33\textwidth}
            \begin{infobox}[Range]
                Actual outputs produced
                
                \vspace{0.2cm}
                
                $\sqrt{x} \geq 0$ always!
                
                Range $\subseteq$ Codomain
                
                \[
                    \text{Range}(f) = [0, \infty)
                \]
            \end{infobox}
        \end{column}
    \end{columns}
    
    \vspace{0.3cm}
    
    \centering
    \begin{tikzpicture}[scale=0.8]
        \draw[<->,thick,softgray] (-0.5,0) -- (4,0) node[right] {$x$};
        \draw[<->,thick,softgray] (0,-0.5) -- (0,2.5) node[above] {$y$};
        
        \draw[accentcyan,thick,domain=0:3.5,smooth,samples=50] plot (\x,{sqrt(\x)});
        
        \node[accentcyan,right] at (3.5,1.87) {$f(x) = \sqrt{x}$};
    \end{tikzpicture}
\end{frame}

% -----------------------------------------------------------------------------
% Parameters vs Variables
% -----------------------------------------------------------------------------
\begin{frame}{Parameters vs Variables: The Director and the Actor}
    \begin{columns}[T]
        \begin{column}{0.5\textwidth}
            \begin{defbox}[Variables]
                \glow[neongreen]{Variables} are the \textbf{inputs} to a function.
                
                \vspace{0.2cm}
                
                They vary — that's why we call them variables!
                
                \vspace{0.2cm}
                
                Example: In $f(x) = x^2$, $x$ is the variable.
            \end{defbox}
            
            \vspace{0.3cm}
            
            \centering
            {\Large Variables = \textbf{Actors}}
            
            {\small (They perform based on the script)}
        \end{column}
        \begin{column}{0.5\textwidth}
            \begin{defbox}[Parameters]
                \glow[neonpink]{Parameters} are \textbf{constants} that control behavior.
                
                \vspace{0.2cm}
                
                They're fixed during evaluation but can be tuned!
                
                \vspace{0.2cm}
                
                Example: In $f(x) = mx + b$, $m$ and $b$ are parameters.
            \end{defbox}
            
            \vspace{0.3cm}
            
            \centering
            {\Large Parameters = \textbf{Director}}
            
            {\small (They control how the show runs)}
        \end{column}
    \end{columns}
\end{frame}

% -----------------------------------------------------------------------------
% Parametric Functions
% -----------------------------------------------------------------------------
\begin{frame}{Parametric Functions: $f(x; \theta)$}
    \begin{defbox}[Parametric Notation]
        We write $f(x; \theta)$ to show:
        \begin{itemize}
            \item $x$ = variable (input)
            \item $\theta$ = parameter(s) (controls shape/behavior)
        \end{itemize}
        
        The semicolon separates variables from parameters.
    \end{defbox}
    
    \vspace{0.3cm}
    
    \textbf{Example: Linear Function}
    \[
        f(x; m, b) = mx + b
    \]
    
    \begin{columns}
        \begin{column}{0.5\textwidth}
            \centering
            \begin{tikzpicture}[scale=0.6]
                \draw[<->,thick,softgray] (-2,0) -- (3,0) node[right] {$x$};
                \draw[<->,thick,softgray] (0,-1) -- (0,3) node[above] {$y$};
                
                % Different slopes
                \draw[neongreen,thick] (-1.5,-0.5) -- (2.5,2.5);
                \draw[accentcyan,thick] (-1.5,0.5) -- (2.5,1.5);
                \draw[neonpink,thick] (-1.5,2) -- (2.5,0);
                
                \node[neongreen,right,font=\small] at (2,2) {$m=1$};
                \node[accentcyan,right,font=\small] at (2,1.3) {$m=0.25$};
                \node[neonpink,right,font=\small] at (2,0.3) {$m=-0.5$};
            \end{tikzpicture}
            
            Varying $m$ (slope)
        \end{column}
        \begin{column}{0.5\textwidth}
            \centering
            \begin{tikzpicture}[scale=0.6]
                \draw[<->,thick,softgray] (-2,0) -- (3,0) node[right] {$x$};
                \draw[<->,thick,softgray] (0,-1) -- (0,3) node[above] {$y$};
                
                % Different intercepts
                \draw[neongreen,thick] (-1.5,-0.5) -- (2.5,1.5);
                \draw[accentcyan,thick] (-1.5,0.5) -- (2.5,2.5);
                \draw[neonpink,thick] (-1.5,-1) -- (2.5,1);
                
                \node[neongreen,right,font=\small] at (2,1.2) {$b=0$};
                \node[accentcyan,right,font=\small] at (2,2.2) {$b=1$};
                \node[neonpink,right,font=\small] at (2,0.7) {$b=-0.5$};
            \end{tikzpicture}
            
            Varying $b$ (intercept)
        \end{column}
    \end{columns}
\end{frame}

% -----------------------------------------------------------------------------
% Why Parameters Matter in ML
% -----------------------------------------------------------------------------
\begin{frame}{Why Parameters Matter in Machine Learning}
    \begin{keybox}[The ML Connection]
        In Machine Learning:
        \begin{itemize}
            \item \textbf{Variables} ($x$) = your data (inputs/features)
            \item \textbf{Parameters} ($\theta$) = what the model \glow{learns}!
        \end{itemize}
        
        \vspace{0.2cm}
        
        Training a neural network = finding the best parameters $\theta^*$
    \end{keybox}
    
    \vspace{0.3cm}
    
    \textbf{Example: Neural Network}
    \[
        \hat{y} = f(x; W, b) = \sigma(Wx + b)
    \]
    
    \begin{itemize}
        \item $x$ = input data (fixed during prediction)
        \item $W$ = weight matrix (learned)
        \item $b$ = bias vector (learned)
        \item $\sigma$ = activation function
    \end{itemize}
    
    \begin{funnybox}
        Training = Turning knobs ($\theta$) until the output looks right!
    \end{funnybox}
\end{frame}

% -----------------------------------------------------------------------------
% Types of Functions
% -----------------------------------------------------------------------------
\begin{frame}{Common Function Types}
    \begin{columns}[T]
        \begin{column}{0.5\textwidth}
            \begin{defbox}[Linear]
                $f(x) = mx + b$
                
                \centering
                \begin{tikzpicture}[scale=0.4]
                    \draw[<->,softgray] (-2,0) -- (2,0);
                    \draw[<->,softgray] (0,-1.5) -- (0,2);
                    \draw[accentcyan,thick] (-1.5,-1) -- (1.5,2);
                \end{tikzpicture}
            \end{defbox}
            
            \vspace{0.2cm}
            
            \begin{defbox}[Polynomial]
                $f(x) = a_nx^n + ... + a_1x + a_0$
                
                \centering
                \begin{tikzpicture}[scale=0.4]
                    \draw[<->,softgray] (-2,0) -- (2,0);
                    \draw[<->,softgray] (0,-1) -- (0,2.5);
                    \draw[neongreen,thick,domain=-1.3:1.3,smooth] plot (\x,{\x*\x});
                \end{tikzpicture}
            \end{defbox}
        \end{column}
        \begin{column}{0.5\textwidth}
            \begin{defbox}[Exponential]
                $f(x) = a^x$ or $e^x$
                
                \centering
                \begin{tikzpicture}[scale=0.4]
                    \draw[<->,softgray] (-2,0) -- (2,0);
                    \draw[<->,softgray] (0,-0.5) -- (0,2.5);
                    \draw[neonyellow,thick,domain=-1.5:1,smooth] plot (\x,{exp(\x)});
                \end{tikzpicture}
            \end{defbox}
            
            \vspace{0.2cm}
            
            \begin{defbox}[Trigonometric]
                $\sin(x), \cos(x), \tan(x)$
                
                \centering
                \begin{tikzpicture}[scale=0.4]
                    \draw[<->,softgray] (-2,0) -- (2,0);
                    \draw[<->,softgray] (0,-1.5) -- (0,1.5);
                    \draw[neonpink,thick,domain=-1.8:1.8,smooth,samples=50] plot (\x,{sin(2*\x r)});
                \end{tikzpicture}
            \end{defbox}
        \end{column}
    \end{columns}
\end{frame}

% -----------------------------------------------------------------------------
% Function Composition
% -----------------------------------------------------------------------------
\begin{frame}{Function Composition: Chaining Functions}
    \begin{defbox}[Composition]
        The \glow{composition} of $f$ and $g$, written $(f \circ g)(x)$:
        \[
            (f \circ g)(x) = f(g(x))
        \]
        
        ``Apply $g$ first, then apply $f$ to the result''
    \end{defbox}
    
    \vspace{0.3cm}
    
    \begin{columns}[T]
        \begin{column}{0.5\textwidth}
            \textbf{Example:}
            \begin{align*}
                f(x) &= x^2 \\
                g(x) &= x + 1 \\
                (f \circ g)(x) &= f(g(x)) = f(x+1) = (x+1)^2 \\
                (g \circ f)(x) &= g(f(x)) = g(x^2) = x^2 + 1
            \end{align*}
            
            \begin{alertbox}
                Order matters! $f \circ g \neq g \circ f$ in general!
            \end{alertbox}
        \end{column}
        \begin{column}{0.5\textwidth}
            \centering
            \begin{tikzpicture}[scale=0.7]
                % Input
                \node[circle,draw=neongreen,fill=darkgray,minimum size=1cm] (x) at (0,0) {$x$};
                
                % g box
                \node[rectangle,draw=accentcyan,fill=darkgray,minimum size=0.8cm] (g) at (2,0) {$g$};
                
                % f box
                \node[rectangle,draw=neonpink,fill=darkgray,minimum size=0.8cm] (f) at (4,0) {$f$};
                
                % Output
                \node[circle,draw=neonyellow,fill=darkgray,minimum size=1cm] (y) at (6,0) {$y$};
                
                % Arrows
                \draw[->,thick,fgwhite] (x) -- (g);
                \draw[->,thick,fgwhite] (g) -- (f) node[midway,above,font=\small] {$g(x)$};
                \draw[->,thick,fgwhite] (f) -- (y);
                
                % Label
                \node[below,fgwhite,font=\small] at (3,-1) {$y = (f \circ g)(x) = f(g(x))$};
            \end{tikzpicture}
        \end{column}
    \end{columns}
\end{frame}

% -----------------------------------------------------------------------------
% Composition in Neural Networks
% -----------------------------------------------------------------------------
\begin{frame}{Composition in Neural Networks}
    \begin{keybox}[Neural Networks = Composed Functions]
        A neural network is just a big composition of simple functions!
    \end{keybox}
    
    \vspace{0.3cm}
    
    \centering
    \begin{tikzpicture}[scale=0.8]
        % Layers as function boxes
        \node[rectangle,draw=neongreen,fill=darkgray,minimum width=1.5cm,minimum height=0.8cm] (l1) at (0,0) {$f_1$};
        \node[rectangle,draw=accentcyan,fill=darkgray,minimum width=1.5cm,minimum height=0.8cm] (l2) at (2.5,0) {$f_2$};
        \node[rectangle,draw=accentcyan,fill=darkgray,minimum width=1.5cm,minimum height=0.8cm] (l3) at (5,0) {$f_3$};
        \node[rectangle,draw=neonpink,fill=darkgray,minimum width=1.5cm,minimum height=0.8cm] (l4) at (7.5,0) {$f_4$};
        
        % Input/Output
        \node[left=0.5cm of l1,fgwhite] (x) {$x$};
        \node[right=0.5cm of l4,fgwhite] (y) {$\hat{y}$};
        
        % Arrows
        \draw[->,thick,fgwhite] (x) -- (l1);
        \draw[->,thick,fgwhite] (l1) -- (l2);
        \draw[->,thick,fgwhite] (l2) -- (l3);
        \draw[->,thick,fgwhite] (l3) -- (l4);
        \draw[->,thick,fgwhite] (l4) -- (y);
        
        % Labels
        \node[below=0.3cm,neongreen,font=\small] at (l1) {Input};
        \node[below=0.3cm,accentcyan,font=\small] at (l2) {Hidden};
        \node[below=0.3cm,accentcyan,font=\small] at (l3) {Hidden};
        \node[below=0.3cm,neonpink,font=\small] at (l4) {Output};
    \end{tikzpicture}
    
    \vspace{0.3cm}
    
    \[
        \hat{y} = (f_4 \circ f_3 \circ f_2 \circ f_1)(x) = f_4(f_3(f_2(f_1(x))))
    \]
    
    \vspace{0.2cm}
    
    Where each $f_i(x) = \sigma(W_ix + b_i)$ is a simple layer operation!
\end{frame}

% -----------------------------------------------------------------------------
% Inverse Functions
% -----------------------------------------------------------------------------
\begin{frame}{Inverse Functions}
    \begin{defbox}[Inverse Function]
        If $f: A \rightarrow B$, the \glow{inverse} $f^{-1}: B \rightarrow A$ satisfies:
        \[
            f^{-1}(f(x)) = x \quad \text{and} \quad f(f^{-1}(y)) = y
        \]
        
        The inverse ``undoes'' what the function does.
    \end{defbox}
    
    \vspace{0.3cm}
    
    \begin{columns}
        \begin{column}{0.5\textwidth}
            \textbf{Examples:}
            \begin{itemize}
                \item $f(x) = x + 3 \Rightarrow f^{-1}(x) = x - 3$
                \item $f(x) = 2x \Rightarrow f^{-1}(x) = \frac{x}{2}$
                \item $f(x) = e^x \Rightarrow f^{-1}(x) = \ln(x)$
                \item $f(x) = x^2 \Rightarrow f^{-1}(x) = \sqrt{x}$ (for $x \geq 0$)
            \end{itemize}
        \end{column}
        \begin{column}{0.5\textwidth}
            \centering
            \begin{tikzpicture}[scale=0.6]
                \draw[<->,softgray] (-0.5,0) -- (3,0) node[right] {$x$};
                \draw[<->,softgray] (0,-0.5) -- (0,3) node[above] {$y$};
                
                % y = x line
                \draw[softgray,dashed] (0,0) -- (2.5,2.5);
                
                % f(x) = e^x (scaled)
                \draw[accentcyan,thick,domain=0:1.2,smooth] plot (\x,{exp(\x)/1.5});
                
                % f^{-1}(x) = ln(x) (scaled)
                \draw[neonpink,thick,domain=0.7:2.7,smooth] plot (\x,{ln(\x*1.5)});
                
                \node[accentcyan,right,font=\small] at (1.2,1.8) {$e^x$};
                \node[neonpink,below,font=\small] at (2.5,0.8) {$\ln x$};
            \end{tikzpicture}
            
            {\small Inverse reflects across $y = x$}
        \end{column}
    \end{columns}
\end{frame}

% -----------------------------------------------------------------------------
% Key Takeaways
% -----------------------------------------------------------------------------
\begin{frame}{Key Takeaways: Functions \& Parameters}
    \begin{keybox}
        \begin{enumerate}
            \item A \textbf{function} maps inputs to outputs: $f: A \rightarrow B$
            \item \textbf{Domain} = valid inputs, \textbf{Range} = actual outputs
            \item \textbf{Variables} ($x$) = inputs that vary
            \item \textbf{Parameters} ($\theta$) = constants that control behavior
            \item In ML: $x$ = data, $\theta$ = what we \textbf{learn}!
            \item \textbf{Composition}: $(f \circ g)(x) = f(g(x))$
            \item Neural networks = composed functions with learnable parameters
            \item \textbf{Inverse} $f^{-1}$ undoes the function
        \end{enumerate}
    \end{keybox}
    
    \vspace{0.2cm}
    
    \centering
    \textit{Next: Limits — the foundation of calculus!}
\end{frame}


% Section 4: Limits
% =============================================================================
% Section 4: Limits
% The foundation of calculus - getting infinitely close
% =============================================================================

\section{Limits}

% -----------------------------------------------------------------------------
% Opening
% -----------------------------------------------------------------------------
\begin{frame}{Limits: Getting Infinitely Close}
    \begin{columns}[T]
        \begin{column}{0.5\textwidth}
            \begin{funnybox}
                \textit{``A limit is like trying to touch your nose with your tongue — you can get really, really close, but maybe never quite there!''}
            \end{funnybox}
            
            \vspace{0.3cm}
            
            \textbf{The Big Question:}
            
            What happens to $f(x)$ as $x$ gets \glow{infinitely close} to some value $a$?
        \end{column}
        \begin{column}{0.5\textwidth}
            \centering
            \begin{tikzpicture}[scale=0.8]
                \draw[<->,thick,softgray] (-0.5,0) -- (4,0) node[right] {$x$};
                \draw[<->,thick,softgray] (0,-0.5) -- (0,3) node[above] {$y$};
                
                % Function with hole
                \draw[accentcyan,thick,domain=0:1.8,smooth] plot (\x,{0.5*\x + 0.5});
                \draw[accentcyan,thick,domain=2.2:3.5,smooth] plot (\x,{0.5*\x + 0.5});
                
                % Hole at x=2
                \draw[neonpink,thick] (2,1.5) circle (4pt);
                
                % Approaching arrows
                \draw[->,neonyellow,thick] (1.3,0) -- (1.8,0);
                \draw[->,neonyellow,thick] (2.7,0) -- (2.2,0);
                
                % Labels
                \node[below,fgwhite] at (2,0) {$a$};
                \node[right,neonpink] at (2.3,1.5) {$L$};
                
                \node[neonyellow,font=\small] at (2,-0.7) {approaching from both sides};
            \end{tikzpicture}
        \end{column}
    \end{columns}
\end{frame}

% -----------------------------------------------------------------------------
% Intuitive Definition
% -----------------------------------------------------------------------------
\begin{frame}{Intuitive Definition of Limits}
    \begin{defbox}[Informal Limit Definition]
        We write:
        \[
            \lim_{x \to a} f(x) = L
        \]
        
        and say ``the limit of $f(x)$ as $x$ approaches $a$ equals $L$''
        
        \vspace{0.2cm}
        
        \textbf{Meaning:} As $x$ gets closer and closer to $a$, $f(x)$ gets closer and closer to $L$.
    \end{defbox}
    
    \vspace{0.3cm}
    
    \textbf{Example:} $\displaystyle\lim_{x \to 2} (3x + 1) = ?$
    
    \begin{center}
        \begin{tabular}{c|cccccc}
            $x$ & 1.9 & 1.99 & 1.999 & 2.001 & 2.01 & 2.1 \\
            \hline
            $f(x)$ & 6.7 & 6.97 & 6.997 & 7.003 & 7.03 & 7.3
        \end{tabular}
    \end{center}
    
    As $x \to 2$, we have $f(x) \to \glow{7}$!
\end{frame}

% -----------------------------------------------------------------------------
% Epsilon-Delta Definition
% -----------------------------------------------------------------------------
\begin{frame}{The Rigorous Definition: Epsilon-Delta}
    \begin{thmbox}[Epsilon-Delta Definition]
        \[
            \lim_{x \to a} f(x) = L
        \]
        
        means: For every $\glow[neongreen]{\varepsilon > 0}$, there exists $\glow[neonpink]{\delta > 0}$ such that:
        \[
            0 < |x - a| < \delta \implies |f(x) - L| < \varepsilon
        \]
    \end{thmbox}
    
    \vspace{0.3cm}
    
    \begin{columns}
        \begin{column}{0.5\textwidth}
            \textbf{In English:}
            \begin{itemize}
                \item Pick any tolerance $\varepsilon$ (how close to $L$)
                \item I can find a $\delta$ (how close $x$ needs to be to $a$)
                \item Such that staying within $\delta$ of $a$ keeps $f(x)$ within $\varepsilon$ of $L$
            \end{itemize}
        \end{column}
        \begin{column}{0.5\textwidth}
            \begin{funnybox}
                It's like a game:
                
                \vspace{0.1cm}
                
                You: ``Get within $\varepsilon$ of $L$!''
                
                Me: ``Fine, stay within $\delta$ of $a$.''
                
                \vspace{0.1cm}
                
                If I can \textit{always} win, the limit exists!
            \end{funnybox}
        \end{column}
    \end{columns}
\end{frame}

% -----------------------------------------------------------------------------
% Epsilon-Delta Visualization
% -----------------------------------------------------------------------------
\begin{frame}{Visualizing $\varepsilon$-$\delta$}
    \centering
    \begin{tikzpicture}[scale=0.95]
        % Axes
        \draw[<->,thick,softgray] (-0.5,0) -- (5,0) node[right] {$x$};
        \draw[<->,thick,softgray] (0,-0.5) -- (0,3.5) node[above] {$y$};
        
        % Function
        \draw[accentcyan,thick,domain=0.5:4.5,smooth] plot (\x,{0.5*\x + 0.5});
        
        % Point a on x-axis
        \fill[fgwhite] (2.5,0) circle (2pt);
        \node[below,fgwhite] at (2.5,-0.1) {$a$};
        
        % L on y-axis  
        \fill[fgwhite] (0,1.75) circle (2pt);
        \node[left,fgwhite] at (0,1.75) {$L$};
        
        % Epsilon band (horizontal)
        \fill[neongreen,opacity=0.2] (0,1.45) rectangle (5,2.05);
        \draw[neongreen,dashed] (0,2.05) -- (5,2.05);
        \draw[neongreen,dashed] (0,1.45) -- (5,1.45);
        \node[left,neongreen,font=\small] at (0,2.05) {$L+\varepsilon$};
        \node[left,neongreen,font=\small] at (0,1.45) {$L-\varepsilon$};
        
        % Delta band (vertical)
        \fill[neonpink,opacity=0.2] (1.9,0) rectangle (3.1,3.5);
        \draw[neonpink,dashed] (1.9,0) -- (1.9,3.5);
        \draw[neonpink,dashed] (3.1,0) -- (3.1,3.5);
        \node[below,neonpink,font=\small] at (1.9,0) {$a-\delta$};
        \node[below,neonpink,font=\small] at (3.1,0) {$a+\delta$};
        
        % Arrows showing the constraint
        \draw[<->,neonyellow,thick] (2.5,1.45) -- (2.5,2.05);
        \node[right,neonyellow,font=\small] at (2.5,1.75) {$\varepsilon$};
        
        \draw[<->,neonyellow,thick] (1.9,-0.4) -- (3.1,-0.4);
        \node[below,neonyellow,font=\small] at (2.5,-0.4) {$\delta$};
    \end{tikzpicture}
    
    \vspace{0.2cm}
    
    If $x$ is in the \textcolor{neonpink}{pink zone}, $f(x)$ stays in the \textcolor{neongreen}{green zone}!
\end{frame}

% -----------------------------------------------------------------------------
% Limit Laws
% -----------------------------------------------------------------------------
\begin{frame}{Limit Laws: The Algebra of Limits}
    If $\displaystyle\lim_{x \to a} f(x) = L$ and $\displaystyle\lim_{x \to a} g(x) = M$, then:
    
    \vspace{0.3cm}
    
    \begin{columns}[T]
        \begin{column}{0.5\textwidth}
            \begin{defbox}[Addition]
                \[
                    \lim_{x \to a} [f(x) + g(x)] = L + M
                \]
            \end{defbox}
            
            \vspace{0.2cm}
            
            \begin{defbox}[Multiplication]
                \[
                    \lim_{x \to a} [f(x) \cdot g(x)] = L \cdot M
                \]
            \end{defbox}
            
            \vspace{0.2cm}
            
            \begin{defbox}[Constant Multiple]
                \[
                    \lim_{x \to a} [c \cdot f(x)] = c \cdot L
                \]
            \end{defbox}
        \end{column}
        \begin{column}{0.5\textwidth}
            \begin{defbox}[Subtraction]
                \[
                    \lim_{x \to a} [f(x) - g(x)] = L - M
                \]
            \end{defbox}
            
            \vspace{0.2cm}
            
            \begin{defbox}[Division]
                \[
                    \lim_{x \to a} \frac{f(x)}{g(x)} = \frac{L}{M} \quad (M \neq 0)
                \]
            \end{defbox}
            
            \vspace{0.2cm}
            
            \begin{defbox}[Power]
                \[
                    \lim_{x \to a} [f(x)]^n = L^n
                \]
            \end{defbox}
        \end{column}
    \end{columns}
\end{frame}

% -----------------------------------------------------------------------------
% One-Sided Limits
% -----------------------------------------------------------------------------
\begin{frame}{One-Sided Limits}
    \begin{columns}[T]
        \begin{column}{0.5\textwidth}
            \begin{defbox}[Left-Hand Limit]
                \[
                    \lim_{x \to a^-} f(x) = L
                \]
                
                Approaching from the \textbf{left} (values less than $a$)
            \end{defbox}
            
            \vspace{0.3cm}
            
            \begin{defbox}[Right-Hand Limit]
                \[
                    \lim_{x \to a^+} f(x) = L
                \]
                
                Approaching from the \textbf{right} (values greater than $a$)
            \end{defbox}
        \end{column}
        \begin{column}{0.5\textwidth}
            \centering
            \begin{tikzpicture}[scale=0.8]
                \draw[<->,thick,softgray] (-0.5,0) -- (4,0) node[right] {$x$};
                \draw[<->,thick,softgray] (0,-0.5) -- (0,3) node[above] {$y$};
                
                % Jump discontinuity
                \draw[accentcyan,thick,domain=0.3:1.9] plot (\x,{0.5*\x + 0.5});
                \draw[neonpink,thick,domain=2.1:3.5] plot (\x,{0.5*\x + 1.2});
                
                % Points
                \fill[accentcyan] (2,1.5) circle (3pt);
                \draw[neonpink,thick] (2,2.2) circle (3pt);
                
                % Labels
                \node[below,fgwhite] at (2,0) {$a$};
                \node[left,accentcyan,font=\small] at (1.5,1.2) {$\lim_{x\to a^-}$};
                \node[right,neonpink,font=\small] at (2.5,2.5) {$\lim_{x\to a^+}$};
            \end{tikzpicture}
            
            \vspace{0.2cm}
            
            \begin{alertbox}
                Limit exists $\iff$ both one-sided limits exist and are equal!
            \end{alertbox}
        \end{column}
    \end{columns}
\end{frame}

% -----------------------------------------------------------------------------
% Limits at Infinity
% -----------------------------------------------------------------------------
\begin{frame}{Limits at Infinity}
    \begin{defbox}[Limit at Infinity]
        \[
            \lim_{x \to \infty} f(x) = L
        \]
        
        As $x$ grows without bound, $f(x)$ approaches $L$.
    \end{defbox}
    
    \vspace{0.3cm}
    
    \textbf{Examples:}
    
    \begin{columns}
        \begin{column}{0.5\textwidth}
            \[
                \lim_{x \to \infty} \frac{1}{x} = 0
            \]
            
            \centering
            \begin{tikzpicture}[scale=0.5]
                \draw[<->,thick,softgray] (-0.5,0) -- (4,0);
                \draw[<->,thick,softgray] (0,-0.5) -- (0,3);
                \draw[accentcyan,thick,domain=0.4:3.8,smooth] plot (\x,{1/\x});
                \draw[neonpink,dashed] (0,0) -- (4,0);
            \end{tikzpicture}
            
            {\small Horizontal asymptote at $y = 0$}
        \end{column}
        \begin{column}{0.5\textwidth}
            \[
                \lim_{x \to \infty} \frac{2x + 1}{x} = 2
            \]
            
            \centering
            \begin{tikzpicture}[scale=0.5]
                \draw[<->,thick,softgray] (-0.5,0) -- (4,0);
                \draw[<->,thick,softgray] (0,-0.5) -- (0,3);
                \draw[accentcyan,thick,domain=0.4:3.8,smooth] plot (\x,{(2*\x+1)/\x});
                \draw[neonpink,dashed] (0,2) -- (4,2);
            \end{tikzpicture}
            
            {\small Horizontal asymptote at $y = 2$}
        \end{column}
    \end{columns}
\end{frame}

% -----------------------------------------------------------------------------
% Indeterminate Forms
% -----------------------------------------------------------------------------
\begin{frame}{Indeterminate Forms}
    \begin{alertbox}[Danger Zone!]
        Some limits look like they give answers but don't:
        
        \vspace{0.2cm}
        
        \centering
        \begin{tabular}{ccccccc}
            $\frac{0}{0}$ & $\frac{\infty}{\infty}$ & $0 \cdot \infty$ & $\infty - \infty$ & $0^0$ & $1^\infty$ & $\infty^0$
        \end{tabular}
        
        \vspace{0.2cm}
        
        These are \glow[neonpink]{indeterminate} — they need more work!
    \end{alertbox}
    
    \vspace{0.3cm}
    
    \textbf{Example:} $\displaystyle\lim_{x \to 0} \frac{\sin x}{x} = \frac{0}{0}$ ??? 
    
    \vspace{0.2cm}
    
    \begin{columns}
        \begin{column}{0.5\textwidth}
            But actually:
            \[
                \lim_{x \to 0} \frac{\sin x}{x} = 1
            \]
            (This is a famous result!)
        \end{column}
        \begin{column}{0.5\textwidth}
            \begin{funnybox}
                $\frac{0}{0}$ doesn't mean ``zero'' — it means ``figure it out another way!''
            \end{funnybox}
        \end{column}
    \end{columns}
\end{frame}

% -----------------------------------------------------------------------------
% L'Hôpital's Rule Preview
% -----------------------------------------------------------------------------
\begin{frame}{L'Hôpital's Rule (Preview)}
    \begin{thmbox}[L'Hôpital's Rule]
        If $\displaystyle\lim_{x \to a} \frac{f(x)}{g(x)}$ gives $\frac{0}{0}$ or $\frac{\infty}{\infty}$, then:
        \[
            \lim_{x \to a} \frac{f(x)}{g(x)} = \lim_{x \to a} \frac{f'(x)}{g'(x)}
        \]
        
        (provided the right-hand limit exists)
    \end{thmbox}
    
    \vspace{0.3cm}
    
    \textbf{Example:} $\displaystyle\lim_{x \to 0} \frac{\sin x}{x}$
    
    \begin{itemize}
        \item Direct: $\frac{\sin 0}{0} = \frac{0}{0}$ (indeterminate!)
        \item L'Hôpital: $\displaystyle\lim_{x \to 0} \frac{\cos x}{1} = \frac{\cos 0}{1} = 1$ (YES)
    \end{itemize}
    
    \begin{infobox}
        We'll learn about derivatives ($f'$) in the next section — then L'Hôpital makes sense!
    \end{infobox}
\end{frame}

% -----------------------------------------------------------------------------
% Continuity
% -----------------------------------------------------------------------------
\begin{frame}{Continuity: No Jumps, No Holes}
    \begin{defbox}[Continuous Function]
        A function $f$ is \glow{continuous} at $x = a$ if:
        \[
            \lim_{x \to a} f(x) = f(a)
        \]
        
        \textbf{Three conditions:}
        \begin{enumerate}
            \item $f(a)$ exists (defined at $a$)
            \item $\lim_{x \to a} f(x)$ exists
            \item They're equal!
        \end{enumerate}
    \end{defbox}
    
    \vspace{0.3cm}
    
    \begin{columns}
        \begin{column}{0.33\textwidth}
            \centering
            \begin{tikzpicture}[scale=0.5]
                \draw[softgray] (-1,0) -- (2,0);
                \draw[softgray] (0,-0.5) -- (0,2);
                \draw[accentcyan,thick,domain=-0.8:1.8,smooth] plot (\x,{\x+0.5});
                \node[below,fgwhite,font=\tiny] at (0.5,-0.5) {Continuous (YES)};
            \end{tikzpicture}
        \end{column}
        \begin{column}{0.33\textwidth}
            \centering
            \begin{tikzpicture}[scale=0.5]
                \draw[softgray] (-1,0) -- (2,0);
                \draw[softgray] (0,-0.5) -- (0,2);
                \draw[neonpink,thick,domain=-0.8:0.4] plot (\x,{\x+0.5});
                \draw[neonpink,thick,domain=0.6:1.8] plot (\x,{\x+1});
                \draw[neonpink] (0.5,1) circle (3pt);
                \fill[neonpink] (0.5,1.5) circle (3pt);
                \node[below,fgwhite,font=\tiny] at (0.5,-0.5) {Jump (NO)};
            \end{tikzpicture}
        \end{column}
        \begin{column}{0.33\textwidth}
            \centering
            \begin{tikzpicture}[scale=0.5]
                \draw[softgray] (-1,0) -- (2,0);
                \draw[softgray] (0,-0.5) -- (0,2);
                \draw[neonyellow,thick,domain=-0.8:0.4] plot (\x,{\x+0.5});
                \draw[neonyellow,thick,domain=0.6:1.8] plot (\x,{\x+0.5});
                \draw[neonyellow] (0.5,1) circle (3pt);
                \node[below,fgwhite,font=\tiny] at (0.5,-0.5) {Hole (NO)};
            \end{tikzpicture}
        \end{column}
    \end{columns}
\end{frame}

% -----------------------------------------------------------------------------
% Why Limits Matter in ML
% -----------------------------------------------------------------------------
\begin{frame}{Why Limits Matter in Machine Learning}
    \begin{keybox}[Applications in ML]
        \begin{enumerate}
            \item \textbf{Derivatives} (next section!) are defined using limits
            \begin{itemize}
                \item Gradients for backpropagation!
            \end{itemize}
            
            \item \textbf{Convergence of training}
            \begin{itemize}
                \item Does the loss approach a minimum?
            \end{itemize}
            
            \item \textbf{Asymptotic analysis}
            \begin{itemize}
                \item Algorithm complexity: $O(n)$, $O(n^2)$
            \end{itemize}
            
            \item \textbf{Activation functions}
            \begin{itemize}
                \item $\lim_{x \to \infty} \sigma(x) = 1$ (sigmoid saturation)
            \end{itemize}
        \end{enumerate}
    \end{keybox}
\end{frame}

% -----------------------------------------------------------------------------
% Key Takeaways
% -----------------------------------------------------------------------------
\begin{frame}{Key Takeaways: Limits}
    \begin{keybox}
        \begin{enumerate}
            \item \textbf{Limit}: What $f(x)$ approaches as $x \to a$
            \item \textbf{Notation}: $\lim_{x \to a} f(x) = L$
            \item \textbf{$\varepsilon$-$\delta$ definition}: The rigorous foundation
            \item \textbf{Limit laws}: Add, subtract, multiply, divide limits
            \item \textbf{One-sided limits}: From left ($a^-$) or right ($a^+$)
            \item \textbf{Limits at infinity}: Behavior as $x \to \pm\infty$
            \item \textbf{Indeterminate forms}: $\frac{0}{0}$, $\frac{\infty}{\infty}$, etc. need special handling
            \item \textbf{Continuity}: No jumps, no holes, no surprises
        \end{enumerate}
    \end{keybox}
    
    \vspace{0.2cm}
    
    \centering
    \textit{Next: Differentiation — the heart of calculus!}
\end{frame}


% Section 5: Differentiation
% =============================================================================
% Section 5: Differentiation
% Rates of change and the heart of calculus
% =============================================================================

\section{Differentiation}

% -----------------------------------------------------------------------------
% Opening
% -----------------------------------------------------------------------------
\begin{frame}{Differentiation: The Art of Measuring Change}
    \begin{columns}[T]
        \begin{column}{0.5\textwidth}
            \begin{funnybox}
                \textit{``How fast is your position changing? Ask the derivative! How fast is your speed changing? Ask the second derivative! How fast is your bank account changing? Don't ask.''}
            \end{funnybox}
            
            \vspace{0.3cm}
            
            \textbf{The Big Question:}
            
            How fast is $f(x)$ changing at point $x$?
        \end{column}
        \begin{column}{0.5\textwidth}
            \centering
            \begin{tikzpicture}[scale=0.8]
                \draw[<->,thick,softgray] (-0.5,0) -- (4,0) node[right] {$x$};
                \draw[<->,thick,softgray] (0,-0.5) -- (0,3) node[above] {$y$};
                
                % Curve
                \draw[accentcyan,thick,domain=0.3:3.5,smooth] plot (\x,{0.3*(\x-1)*(\x-1)+0.5});
                
                % Tangent line at x=2
                \draw[neonpink,thick] (1,0.2) -- (3,1.4);
                
                % Point
                \fill[neonyellow] (2,0.8) circle (3pt);
                
                % Labels
                \node[accentcyan,right] at (3.2,2) {$f(x)$};
                \node[neonpink,above] at (2.8,1.2) {tangent};
                \node[neonyellow,below right] at (2,0.8) {slope = $f'(x)$};
            \end{tikzpicture}
        \end{column}
    \end{columns}
\end{frame}

% -----------------------------------------------------------------------------
% Secant to Tangent
% -----------------------------------------------------------------------------
\begin{frame}{From Secant to Tangent}
    \centering
    \begin{tikzpicture}[scale=0.85]
        \draw[<->,thick,softgray] (-0.5,0) -- (5,0) node[right] {$x$};
        \draw[<->,thick,softgray] (0,-0.5) -- (0,3.5) node[above] {$y$};
        
        % Curve
        \draw[accentcyan,thick,domain=0.5:4.5,smooth,samples=50] plot (\x,{0.2*(\x-2)*(\x-2)+1});
        
        % Points
        \coordinate (A) at (1.5,1.05);
        \coordinate (B) at (3.5,1.45);
        \coordinate (C) at (2.5,1.05);
        
        % Secant lines (fading)
        \draw[softgray,thick] (0.8,0.55) -- (4.2,1.85);
        \draw[neonyellow,thick] (1,0.75) -- (3.8,1.55);
        \draw[neonpink,thick] (1.5,0.95) -- (3.2,1.25);
        
        % Tangent (final)
        \draw[neongreen,very thick] (1.5,0.85) -- (3.5,1.25);
        
        % Point on curve
        \fill[neongreen] (2.5,1.05) circle (3pt);
        
        % Labels
        \node[below,fgwhite] at (1.5,0) {$a$};
        \node[below,fgwhite] at (2.5,0) {$x$};
        \node[below,fgwhite] at (3.5,0) {$a+h$};
        
        % Arrow showing h shrinking
        \draw[<->,neonyellow] (2.5,-0.3) -- (3.5,-0.3) node[midway,below] {$h \to 0$};
    \end{tikzpicture}
    
    \vspace{0.3cm}
    
    As $h \to 0$, the \textcolor{softgray}{secant line} becomes the \textcolor{neongreen}{tangent line}!
\end{frame}

% -----------------------------------------------------------------------------
% Derivative Definition
% -----------------------------------------------------------------------------
\begin{frame}{The Derivative: Definition}
    \begin{defbox}[Derivative]
        The \glow{derivative} of $f$ at $x$ is:
        \[
            f'(x) = \lim_{h \to 0} \frac{f(x+h) - f(x)}{h}
        \]
        
        provided this limit exists.
        
        \vspace{0.2cm}
        
        \textbf{Alternative notations:}
        \[
            f'(x) = \frac{df}{dx} = \frac{d}{dx}f(x) = Df(x) = \dot{f}
        \]
    \end{defbox}
    
    \vspace{0.3cm}
    
    \begin{columns}
        \begin{column}{0.5\textwidth}
            \textbf{What it measures:}
            \begin{itemize}
                \item Instantaneous rate of change
                \item Slope of tangent line
                \item Sensitivity of output to input
            \end{itemize}
        \end{column}
        \begin{column}{0.5\textwidth}
            \begin{funnybox}
                The derivative is asking: ``If I nudge $x$ a tiny bit, how much does $f(x)$ change?''
            \end{funnybox}
        \end{column}
    \end{columns}
\end{frame}

% -----------------------------------------------------------------------------
% Example Derivation
% -----------------------------------------------------------------------------
\begin{frame}{Example: Deriving $f(x) = x^2$}
    \begin{align*}
        f'(x) &= \lim_{h \to 0} \frac{f(x+h) - f(x)}{h} \\[0.3cm]
        &= \lim_{h \to 0} \frac{(x+h)^2 - x^2}{h} \\[0.3cm]
        &= \lim_{h \to 0} \frac{x^2 + 2xh + h^2 - x^2}{h} \\[0.3cm]
        &= \lim_{h \to 0} \frac{2xh + h^2}{h} \\[0.3cm]
        &= \lim_{h \to 0} (2x + h) \\[0.3cm]
        &= \mathbf{2x}
    \end{align*}
    
    \begin{keybox}
        If $f(x) = x^2$, then $f'(x) = 2x$. At $x=3$: slope $= 6$!
    \end{keybox}
\end{frame}

% -----------------------------------------------------------------------------
% Power Rule
% -----------------------------------------------------------------------------
\begin{frame}{Differentiation Rules: Power Rule}
    \begin{thmbox}[Power Rule]
        If $f(x) = x^n$, then:
        \[
            \mathbf{f'(x) = nx^{n-1}}
        \]
        
        Works for any real $n$!
    \end{thmbox}
    
    \vspace{0.3cm}
    
    \textbf{Examples:}
    \begin{columns}
        \begin{column}{0.5\textwidth}
            \begin{align*}
                \frac{d}{dx}(x^3) &= 3x^2 \\
                \frac{d}{dx}(x^5) &= 5x^4 \\
                \frac{d}{dx}(x^{100}) &= 100x^{99}
            \end{align*}
        \end{column}
        \begin{column}{0.5\textwidth}
            \begin{align*}
                \frac{d}{dx}(x^{-1}) &= -x^{-2} = -\frac{1}{x^2} \\
                \frac{d}{dx}(\sqrt{x}) &= \frac{d}{dx}(x^{1/2}) = \frac{1}{2}x^{-1/2} \\
                \frac{d}{dx}(1) &= \frac{d}{dx}(x^0) = 0
            \end{align*}
        \end{column}
    \end{columns}
    
    \begin{funnybox}
        ``Bring down the power, reduce by one'' — calculus's most satisfying rule!
    \end{funnybox}
\end{frame}

% -----------------------------------------------------------------------------
% Basic Rules
% -----------------------------------------------------------------------------
\begin{frame}{More Differentiation Rules}
    \begin{columns}[T]
        \begin{column}{0.5\textwidth}
            \begin{defbox}[Constant Rule]
                \[
                    \frac{d}{dx}(c) = 0
                \]
                Constants don't change!
            \end{defbox}
            
            \vspace{0.2cm}
            
            \begin{defbox}[Constant Multiple]
                \[
                    \frac{d}{dx}(cf) = c \cdot f'
                \]
            \end{defbox}
            
            \vspace{0.2cm}
            
            \begin{defbox}[Sum Rule]
                \[
                    \frac{d}{dx}(f + g) = f' + g'
                \]
            \end{defbox}
        \end{column}
        \begin{column}{0.5\textwidth}
            \begin{defbox}[Product Rule]
                \[
                    \frac{d}{dx}(fg) = f'g + fg'
                \]
                ``First times derivative of second, plus second times derivative of first''
            \end{defbox}
            
            \vspace{0.2cm}
            
            \begin{defbox}[Quotient Rule]
                \[
                    \frac{d}{dx}\left(\frac{f}{g}\right) = \frac{f'g - fg'}{g^2}
                \]
                ``Low d-high minus high d-low, over low squared''
            \end{defbox}
        \end{column}
    \end{columns}
\end{frame}

% -----------------------------------------------------------------------------
% Chain Rule
% -----------------------------------------------------------------------------
\begin{frame}{The Chain Rule: Derivatives of Compositions}
    \begin{thmbox}[Chain Rule]
        If $y = f(g(x))$, then:
        \[
            \mathbf{\frac{dy}{dx} = f'(g(x)) \cdot g'(x)}
        \]
        
        Or in Leibniz notation:
        \[
            \frac{dy}{dx} = \frac{dy}{du} \cdot \frac{du}{dx}
        \]
        where $u = g(x)$
    \end{thmbox}
    
    \vspace{0.3cm}
    
    \textbf{Example:} Find $\frac{d}{dx}(x^2 + 1)^3$
    
    Let $u = x^2 + 1$, so $y = u^3$
    \begin{align*}
        \frac{dy}{du} &= 3u^2, \quad \frac{du}{dx} = 2x \\
        \frac{dy}{dx} &= 3u^2 \cdot 2x = 3(x^2+1)^2 \cdot 2x = \mathbf{6x(x^2+1)^2}
    \end{align*}
\end{frame}

% -----------------------------------------------------------------------------
% Chain Rule Visualization
% -----------------------------------------------------------------------------
\begin{frame}{Chain Rule: The Pipeline}
    \centering
    \begin{tikzpicture}[scale=0.9]
        % Boxes
        \node[rectangle,draw=neongreen,fill=darkgray,minimum width=1.5cm,minimum height=1cm] (x) at (0,0) {$x$};
        \node[rectangle,draw=accentcyan,fill=darkgray,minimum width=1.5cm,minimum height=1cm] (g) at (3,0) {$g$};
        \node[rectangle,draw=neonpink,fill=darkgray,minimum width=1.5cm,minimum height=1cm] (f) at (6,0) {$f$};
        \node[rectangle,draw=neonyellow,fill=darkgray,minimum width=1.5cm,minimum height=1cm] (y) at (9,0) {$y$};
        
        % Forward arrows
        \draw[->,thick,fgwhite] (x) -- (g) node[midway,above] {$x$};
        \draw[->,thick,fgwhite] (g) -- (f) node[midway,above] {$u=g(x)$};
        \draw[->,thick,fgwhite] (f) -- (y) node[midway,above] {$f(u)$};
        
        % Backward arrows (derivatives)
        \draw[<-,thick,neonpink,dashed] (x.south) -- ++(0,-0.5) -- (g.south |- 0,-0.5) -- (g.south) node[midway,below] {$g'(x)$};
        \draw[<-,thick,accentcyan,dashed] (g.south) -- ++(0,-1) -- (f.south |- 0,-1) -- (f.south) node[midway,below] {$f'(u)$};
        
        % Result
        \node[below=2cm,fgwhite] at (4.5,0) {$\displaystyle\frac{dy}{dx} = \underbrace{f'(g(x))}_{\text{outer derivative}} \cdot \underbrace{g'(x)}_{\text{inner derivative}}$};
    \end{tikzpicture}
    
    \vspace{0.5cm}
    
    \begin{funnybox}
        Chain rule: Multiply all the ``rates of change'' along the pipeline!
    \end{funnybox}
\end{frame}

% -----------------------------------------------------------------------------
% Important Derivatives
% -----------------------------------------------------------------------------
\begin{frame}{Important Derivatives to Know}
    \begin{columns}[T]
        \begin{column}{0.5\textwidth}
            \begin{defbox}[Exponential \& Log]
                \begin{align*}
                    \frac{d}{dx}(e^x) &= e^x \\
                    \frac{d}{dx}(a^x) &= a^x \ln(a) \\
                    \frac{d}{dx}(\ln x) &= \frac{1}{x} \\
                    \frac{d}{dx}(\log_a x) &= \frac{1}{x \ln a}
                \end{align*}
            \end{defbox}
        \end{column}
        \begin{column}{0.5\textwidth}
            \begin{defbox}[Trigonometric]
                \begin{align*}
                    \frac{d}{dx}(\sin x) &= \cos x \\
                    \frac{d}{dx}(\cos x) &= -\sin x \\
                    \frac{d}{dx}(\tan x) &= \sec^2 x
                \end{align*}
            \end{defbox}
        \end{column}
    \end{columns}
    
    \vspace{0.3cm}
    
    \begin{keybox}[Star of the Show]
        $\frac{d}{dx}(e^x) = e^x$ — the exponential is its own derivative!
        
        This is why $e$ appears everywhere in calculus and ML!
    \end{keybox}
\end{frame}

% -----------------------------------------------------------------------------
% Higher Derivatives
% -----------------------------------------------------------------------------
\begin{frame}{Higher-Order Derivatives}
    \begin{defbox}[Higher Derivatives]
        The \textbf{second derivative} is the derivative of the derivative:
        \[
            f''(x) = \frac{d^2f}{dx^2} = \frac{d}{dx}\left(\frac{df}{dx}\right)
        \]
        
        And we can keep going: $f'''(x)$, $f^{(4)}(x)$, etc.
    \end{defbox}
    
    \vspace{0.3cm}
    
    \textbf{Physical Interpretation:}
    \begin{itemize}
        \item $f(t)$ = position at time $t$
        \item $f'(t)$ = velocity (rate of change of position)
        \item $f''(t)$ = acceleration (rate of change of velocity)
        \item $f'''(t)$ = jerk (rate of change of acceleration)
    \end{itemize}
    
    \begin{funnybox}
        Jerk is what makes roller coasters fun... or terrifying!
    \end{funnybox}
\end{frame}

% -----------------------------------------------------------------------------
% Partial Derivatives
% -----------------------------------------------------------------------------
\begin{frame}{Partial Derivatives: Multiple Variables}
    \begin{defbox}[Partial Derivative]
        For $f(x, y)$, the \glow{partial derivative} with respect to $x$:
        \[
            \frac{\partial f}{\partial x} = \lim_{h \to 0} \frac{f(x+h, y) - f(x, y)}{h}
        \]
        
        Treat $y$ as constant, differentiate with respect to $x$ only.
    \end{defbox}
    
    \vspace{0.3cm}
    
    \textbf{Example:} $f(x, y) = x^2y + 3xy^2$
    \begin{align*}
        \frac{\partial f}{\partial x} &= 2xy + 3y^2 \quad \text{(treat $y$ as constant)} \\
        \frac{\partial f}{\partial y} &= x^2 + 6xy \quad \text{(treat $x$ as constant)}
    \end{align*}
    
    \begin{keybox}[ML Connection]
        Neural networks have \textbf{many} parameters. Partial derivatives tell us how to adjust each one!
    \end{keybox}
\end{frame}

% -----------------------------------------------------------------------------
% Gradient
% -----------------------------------------------------------------------------
\begin{frame}{The Gradient: All Partials Together}
    \begin{defbox}[Gradient]
        The \glow{gradient} of $f(x_1, x_2, ..., x_n)$ is a vector of all partial derivatives:
        \[
            \nabla f = \left( \frac{\partial f}{\partial x_1}, \frac{\partial f}{\partial x_2}, \ldots, \frac{\partial f}{\partial x_n} \right)
        \]
    \end{defbox}
    
    \vspace{0.3cm}
    
    \begin{columns}
        \begin{column}{0.5\textwidth}
            \textbf{Properties:}
            \begin{itemize}
                \item Points in direction of steepest increase
                \item Magnitude = rate of steepest increase
                \item Perpendicular to level curves
            \end{itemize}
        \end{column}
        \begin{column}{0.5\textwidth}
            \centering
            \begin{tikzpicture}[scale=0.7]
                % Contours
                \foreach \r in {0.5,1,1.5,2} {
                    \draw[accentcyan,thick] (0,0) circle (\r);
                }
                
                % Gradient arrows
                \foreach \angle in {0,45,90,135,180,225,270,315} {
                    \draw[->,neonpink,thick] (\angle:1) -- (\angle:1.5);
                }
                
                % Center
                \fill[neonyellow] (0,0) circle (3pt);
                \node[neonyellow,above right,font=\small] at (0,0) {min};
                
                \node[below,fgwhite,font=\small] at (0,-2.5) {$\nabla f$ points ``uphill''};
            \end{tikzpicture}
        \end{column}
    \end{columns}
\end{frame}

% -----------------------------------------------------------------------------
% Why Derivatives Matter in ML
% -----------------------------------------------------------------------------
\begin{frame}{Why Derivatives Matter in ML}
    \begin{keybox}[The Core of Machine Learning]
        \textbf{Training = Minimizing a Loss Function}
        
        \vspace{0.2cm}
        
        To find the minimum, we need \glow{derivatives}!
    \end{keybox}
    
    \vspace{0.3cm}
    
    \textbf{Gradient Descent:}
    \[
        \theta_{\text{new}} = \theta_{\text{old}} - \eta \cdot \nabla_\theta L(\theta)
    \]
    
    \begin{itemize}
        \item $\theta$ = model parameters
        \item $L$ = loss function
        \item $\eta$ = learning rate
        \item $\nabla_\theta L$ = gradient (tells us which way is ``downhill'')
    \end{itemize}
    
    \begin{funnybox}
        Gradient descent: ``Follow the slope downhill until you reach the bottom!''
    \end{funnybox}
\end{frame}

% -----------------------------------------------------------------------------
% Key Takeaways
% -----------------------------------------------------------------------------
\begin{frame}{Key Takeaways: Differentiation}
    \begin{keybox}
        \begin{enumerate}
            \item \textbf{Derivative} $f'(x) = \lim_{h \to 0} \frac{f(x+h) - f(x)}{h}$
            \item \textbf{Geometric meaning}: Slope of tangent line
            \item \textbf{Power rule}: $\frac{d}{dx}(x^n) = nx^{n-1}$
            \item \textbf{Chain rule}: $(f \circ g)' = f'(g) \cdot g'$ — crucial for neural nets!
            \item \textbf{Partial derivatives}: Rate of change in one variable
            \item \textbf{Gradient} $\nabla f$: Vector of all partial derivatives
            \item \textbf{ML connection}: Gradients guide learning (gradient descent)
        \end{enumerate}
    \end{keybox}
    
    \vspace{0.2cm}
    
    \centering
    \textit{Next: Integration — the inverse of differentiation!}
\end{frame}


% Section 6: Integration
% =============================================================================
% Section 6: Integration
% The inverse of differentiation - finding areas and totals
% =============================================================================

\section{Integration}

% -----------------------------------------------------------------------------
% Opening
% -----------------------------------------------------------------------------
\begin{frame}{Integration: Adding Infinitely Many Infinitely Small Pieces}
    \begin{columns}[T]
        \begin{column}{0.5\textwidth}
            \begin{funnybox}
                \textit{``Integration is like eating pizza: one slice at a time, you can consume the whole thing. Except with integration, the slices are infinitely thin!''}
            \end{funnybox}
            
            \vspace{0.3cm}
            
            \textbf{Two Big Ideas:}
            \begin{enumerate}
                \item Finding areas under curves
                \item Reversing differentiation
            \end{enumerate}
        \end{column}
        \begin{column}{0.5\textwidth}
            \centering
            \begin{tikzpicture}[scale=0.8]
                \draw[<->,thick,softgray] (-0.5,0) -- (4,0) node[right] {$x$};
                \draw[<->,thick,softgray] (0,-0.5) -- (0,2.5) node[above] {$y$};
                
                % Fill area
                \fill[accentcyan,opacity=0.3] (0.5,0) -- plot[domain=0.5:3.2,smooth] (\x,{0.3*\x*\x - 0.5*\x + 1}) -- (3.2,0) -- cycle;
                
                % Curve
                \draw[accentcyan,thick,domain=0.3:3.5,smooth] plot (\x,{0.3*\x*\x - 0.5*\x + 1});
                
                % Boundaries
                \draw[neonpink,dashed] (0.5,0) -- (0.5,0.825);
                \draw[neonpink,dashed] (3.2,0) -- (3.2,2.468);
                
                \node[below,fgwhite] at (0.5,0) {$a$};
                \node[below,fgwhite] at (3.2,0) {$b$};
                \node[accentcyan] at (1.8,0.8) {Area};
            \end{tikzpicture}
        \end{column}
    \end{columns}
\end{frame}

% -----------------------------------------------------------------------------
% Riemann Sums
% -----------------------------------------------------------------------------
\begin{frame}{Riemann Sums: Approximating Area}
    \centering
    \begin{tikzpicture}[scale=0.9]
        % Axes
        \draw[<->,thick,softgray] (-0.5,0) -- (5,0) node[right] {$x$};
        \draw[<->,thick,softgray] (0,-0.5) -- (0,3) node[above] {$y$};
        
        % Rectangles
        \foreach \x in {0.5,1,1.5,2,2.5,3,3.5} {
            \pgfmathsetmacro{\height}{0.2*\x*\x + 0.5}
            \fill[neonyellow,opacity=0.4] (\x,0) rectangle (\x+0.5,\height);
            \draw[neonyellow] (\x,0) rectangle (\x+0.5,\height);
        }
        
        % Curve
        \draw[accentcyan,thick,domain=0.3:4.2,smooth] plot (\x,{0.2*\x*\x + 0.5});
        
        % Labels
        \node[below,fgwhite] at (0.5,0) {$a$};
        \node[below,fgwhite] at (4,0) {$b$};
        
        % Delta x
        \draw[<->,neonpink] (1,-0.4) -- (1.5,-0.4) node[midway,below] {$\Delta x$};
    \end{tikzpicture}
    
    \vspace{0.3cm}
    
    \begin{defbox}[Riemann Sum]
        \[
            \text{Area} \approx \sum_{i=1}^{n} f(x_i) \cdot \Delta x
        \]
        
        where $\Delta x = \frac{b-a}{n}$ is the width of each rectangle.
    \end{defbox}
\end{frame}

% -----------------------------------------------------------------------------
% From Riemann Sum to Integral
% -----------------------------------------------------------------------------
\begin{frame}{From Rectangles to Integrals}
    \begin{columns}
        \begin{column}{0.33\textwidth}
            \centering
            \begin{tikzpicture}[scale=0.5]
                \draw[softgray] (0,0) -- (4,0);
                \draw[softgray] (0,0) -- (0,2.5);
                
                \foreach \x in {0,1,2,3} {
                    \pgfmathsetmacro{\h}{0.15*\x*\x + 0.5}
                    \fill[neonyellow,opacity=0.4] (\x,0) rectangle (\x+1,\h);
                    \draw[neonyellow] (\x,0) rectangle (\x+1,\h);
                }
                \draw[accentcyan,thick,domain=0:4,smooth] plot (\x,{0.15*\x*\x + 0.5});
                
                \node[below,fgwhite,font=\small] at (2,-0.3) {$n=4$};
            \end{tikzpicture}
        \end{column}
        \begin{column}{0.33\textwidth}
            \centering
            \begin{tikzpicture}[scale=0.5]
                \draw[softgray] (0,0) -- (4,0);
                \draw[softgray] (0,0) -- (0,2.5);
                
                \foreach \x in {0,0.5,...,3.5} {
                    \pgfmathsetmacro{\h}{0.15*\x*\x + 0.5}
                    \fill[neonyellow,opacity=0.4] (\x,0) rectangle (\x+0.5,\h);
                    \draw[neonyellow] (\x,0) rectangle (\x+0.5,\h);
                }
                \draw[accentcyan,thick,domain=0:4,smooth] plot (\x,{0.15*\x*\x + 0.5});
                
                \node[below,fgwhite,font=\small] at (2,-0.3) {$n=8$};
            \end{tikzpicture}
        \end{column}
        \begin{column}{0.33\textwidth}
            \centering
            \begin{tikzpicture}[scale=0.5]
                \draw[softgray] (0,0) -- (4,0);
                \draw[softgray] (0,0) -- (0,2.5);
                
                \fill[accentcyan,opacity=0.4] (0,0) -- plot[domain=0:4,smooth] (\x,{0.15*\x*\x + 0.5}) -- (4,0) -- cycle;
                \draw[accentcyan,thick,domain=0:4,smooth] plot (\x,{0.15*\x*\x + 0.5});
                
                \node[below,fgwhite,font=\small] at (2,-0.3) {$n \to \infty$};
            \end{tikzpicture}
        \end{column}
    \end{columns}
    
    \vspace{0.5cm}
    
    \begin{thmbox}[Definition of Definite Integral]
        \[
            \int_a^b f(x)\, dx = \lim_{n \to \infty} \sum_{i=1}^{n} f(x_i) \cdot \Delta x
        \]
    \end{thmbox}
    
    \begin{funnybox}
        As we use more and more rectangles ($n \to \infty$), we get the \textit{exact} area!
    \end{funnybox}
\end{frame}

% -----------------------------------------------------------------------------
% Integral Notation
% -----------------------------------------------------------------------------
\begin{frame}{Understanding Integral Notation}
    \[
        \int_a^b f(x)\, dx
    \]
    
    \vspace{0.3cm}
    
    \centering
    \begin{tikzpicture}[scale=0.9]
        % The integral symbol
        \node[font=\Huge,neonpink] at (0,0) {$\displaystyle\int$};
        \node[font=\small,fgwhite] at (0.3,0.8) {$b$};
        \node[font=\small,fgwhite] at (0.3,-0.8) {$a$};
        
        % Annotations
        \draw[->,neonyellow,thick] (-1,0) -- (-0.4,0);
        \node[left,neonyellow,font=\small,align=right] at (-1,0) {``Sum'' symbol\\(elongated S)};
        
        \draw[->,neongreen,thick] (0.6,0.8) -- (0.3,0.6);
        \node[right,neongreen,font=\small] at (0.6,0.8) {Upper limit};
        
        \draw[->,neongreen,thick] (0.6,-0.8) -- (0.3,-0.6);
        \node[right,neongreen,font=\small] at (0.6,-0.8) {Lower limit};
        
        % Function part
        \node[font=\Large,accentcyan] at (1.3,0) {$f(x)$};
        \draw[->,accentcyan,thick] (1.3,-0.5) -- (1.3,-0.2);
        \node[below,accentcyan,font=\small] at (1.3,-0.5) {Function (height)};
        
        % dx part
        \node[font=\Large,neonpink] at (2.3,0) {$dx$};
        \draw[->,neonpink,thick] (2.3,0.5) -- (2.3,0.2);
        \node[above,neonpink,font=\small] at (2.3,0.5) {Infinitesimal width};
    \end{tikzpicture}
    
    \vspace{0.5cm}
    
    \begin{infobox}
        The integral $\int$ is a stretched ``S'' for ``Sum'' — we're adding up infinitely many $f(x) \cdot dx$ pieces!
    \end{infobox}
\end{frame}

% -----------------------------------------------------------------------------
% Antiderivatives
% -----------------------------------------------------------------------------
\begin{frame}{Antiderivatives: Reversing Differentiation}
    \begin{defbox}[Antiderivative]
        $F(x)$ is an \glow{antiderivative} of $f(x)$ if:
        \[
            F'(x) = f(x)
        \]
        
        The antiderivative ``undoes'' differentiation!
    \end{defbox}
    
    \vspace{0.3cm}
    
    \textbf{Examples:}
    \begin{center}
        \begin{tabular}{c|c|c}
            $f(x)$ & $F(x)$ (antiderivative) & Check: $F'(x) = f(x)$? \\
            \hline
            $x^2$ & $\frac{x^3}{3}$ & $\frac{d}{dx}\left(\frac{x^3}{3}\right) = x^2$ (YES) \\
            $\cos x$ & $\sin x$ & $\frac{d}{dx}(\sin x) = \cos x$ (YES) \\
            $e^x$ & $e^x$ & $\frac{d}{dx}(e^x) = e^x$ (YES) \\
        \end{tabular}
    \end{center}
    
    \begin{alertbox}
        But wait — if $F(x)$ is an antiderivative, so is $F(x) + C$ for any constant $C$!
        
        (Because $\frac{d}{dx}(F + C) = F' + 0 = f$)
    \end{alertbox}
\end{frame}

% -----------------------------------------------------------------------------
% Indefinite Integral
% -----------------------------------------------------------------------------
\begin{frame}{Indefinite Integral}
    \begin{defbox}[Indefinite Integral]
        The \glow{indefinite integral} represents all antiderivatives:
        \[
            \int f(x)\, dx = F(x) + C
        \]
        
        where $C$ is the ``constant of integration'' (can be any number).
    \end{defbox}
    
    \vspace{0.3cm}
    
    \textbf{Basic Integrals:}
    \begin{columns}
        \begin{column}{0.5\textwidth}
            \begin{align*}
                \int x^n\, dx &= \frac{x^{n+1}}{n+1} + C \quad (n \neq -1) \\
                \int e^x\, dx &= e^x + C \\
                \int \frac{1}{x}\, dx &= \ln|x| + C
            \end{align*}
        \end{column}
        \begin{column}{0.5\textwidth}
            \begin{align*}
                \int \sin x\, dx &= -\cos x + C \\
                \int \cos x\, dx &= \sin x + C \\
                \int 1\, dx &= x + C
            \end{align*}
        \end{column}
    \end{columns}
    
    \begin{funnybox}
        Don't forget the $+C$! It's the most forgotten symbol in calculus.
    \end{funnybox}
\end{frame}

% -----------------------------------------------------------------------------
% Fundamental Theorem of Calculus Part 1
% -----------------------------------------------------------------------------
\begin{frame}{Fundamental Theorem of Calculus: Part 1}
    \begin{thmbox}[FTC Part 1]
        If $f$ is continuous on $[a, b]$ and:
        \[
            g(x) = \int_a^x f(t)\, dt
        \]
        
        Then $g$ is differentiable and:
        \[
            \glow{g'(x) = f(x)}
        \]
    \end{thmbox}
    
    \vspace{0.3cm}
    
    \begin{columns}
        \begin{column}{0.5\textwidth}
            \textbf{In words:}
            
            The derivative of the ``area so far'' function equals the original function!
            
            \vspace{0.3cm}
            
            \textbf{Symbolically:}
            \[
                \frac{d}{dx}\int_a^x f(t)\, dt = f(x)
            \]
        \end{column}
        \begin{column}{0.5\textwidth}
            \centering
            \begin{tikzpicture}[scale=0.7]
                \draw[<->,softgray] (-0.3,0) -- (4,0);
                \draw[<->,softgray] (0,-0.3) -- (0,2.5);
                
                % Area
                \fill[accentcyan,opacity=0.3] (0.5,0) -- plot[domain=0.5:2.5,smooth] (\x,{0.3*\x + 0.5}) -- (2.5,0) -- cycle;
                
                \draw[accentcyan,thick,domain=0.2:3.5,smooth] plot (\x,{0.3*\x + 0.5});
                
                % x marker
                \draw[neonpink,dashed] (2.5,0) -- (2.5,1.25);
                \node[below,fgwhite] at (2.5,0) {$x$};
                
                % g(x) label
                \node[accentcyan] at (1.5,0.5) {$g(x)$};
            \end{tikzpicture}
            
            {\small Rate of area growth = height of curve!}
        \end{column}
    \end{columns}
\end{frame}

% -----------------------------------------------------------------------------
% Fundamental Theorem of Calculus Part 2
% -----------------------------------------------------------------------------
\begin{frame}{Fundamental Theorem of Calculus: Part 2}
    \begin{thmbox}[FTC Part 2 (Evaluation Theorem)]
        If $f$ is continuous on $[a, b]$ and $F$ is any antiderivative of $f$, then:
        \[
            \mathbf{\int_a^b f(x)\, dx = F(b) - F(a)}
        \]
    \end{thmbox}
    
    \vspace{0.3cm}
    
    \textbf{This is HUGE!} Instead of computing limits of Riemann sums, just:
    \begin{enumerate}
        \item Find an antiderivative $F$
        \item Evaluate at the endpoints
        \item Subtract!
    \end{enumerate}
    
    \vspace{0.3cm}
    
    \textbf{Example:} $\displaystyle\int_0^2 x^2\, dx$
    
    Antiderivative: $F(x) = \frac{x^3}{3}$
    
    Answer: $F(2) - F(0) = \frac{8}{3} - 0 = \mathbf{\frac{8}{3}}$
\end{frame}

% -----------------------------------------------------------------------------
% FTC Connection
% -----------------------------------------------------------------------------
\begin{frame}{The Beautiful Connection}
    \centering
    \begin{tikzpicture}[scale=0.9]
        % Differentiation
        \node[rectangle,draw=neonpink,fill=darkgray,minimum width=2.5cm,minimum height=1cm] (diff) at (-2,0) {\textcolor{neonpink}{Differentiation}};
        
        % Integration
        \node[rectangle,draw=accentcyan,fill=darkgray,minimum width=2.5cm,minimum height=1cm] (int) at (2,0) {\textcolor{accentcyan}{Integration}};
        
        % Functions
        \node[above=0.5cm of diff,neonpink] {$F(x)$};
        \node[above=0.5cm of int,accentcyan] {$f(x)$};
        
        % Arrows
        \draw[->,thick,neonyellow] (diff.east) -- (int.west) node[midway,above] {$\frac{d}{dx}$};
        \draw[->,thick,neonyellow] (int.west) to[bend left=40] node[midway,below] {$\int dx$} (diff.east);
    \end{tikzpicture}
    
    \vspace{0.5cm}
    
    \begin{keybox}
        Differentiation and Integration are \glow{inverse operations}!
        
        \vspace{0.2cm}
        
        \begin{itemize}
            \item Differentiate then integrate: back where you started (+ C)
            \item Integrate then differentiate: exactly back where you started
        \end{itemize}
    \end{keybox}
    
    \begin{funnybox}
        They're like frenemies — opposite but inseparable!
    \end{funnybox}
\end{frame}

% -----------------------------------------------------------------------------
% Integration by Substitution
% -----------------------------------------------------------------------------
\begin{frame}{Integration Technique: Substitution}
    \begin{defbox}[u-Substitution]
        For $\int f(g(x)) \cdot g'(x)\, dx$:
        
        \begin{enumerate}
            \item Let $u = g(x)$
            \item Then $du = g'(x)\, dx$
            \item Substitute: $\int f(u)\, du$
            \item Integrate and substitute back
        \end{enumerate}
    \end{defbox}
    
    \vspace{0.3cm}
    
    \textbf{Example:} $\int 2x \cdot e^{x^2}\, dx$
    
    \begin{itemize}
        \item Let $u = x^2$, so $du = 2x\, dx$
        \item Substitute: $\int e^u\, du = e^u + C$
        \item Substitute back: $\mathbf{e^{x^2} + C}$
    \end{itemize}
    
    \begin{funnybox}
        ``When in doubt, u-sub it out!'' — Every calculus student
    \end{funnybox}
\end{frame}

% -----------------------------------------------------------------------------
% Integration by Parts
% -----------------------------------------------------------------------------
\begin{frame}{Integration Technique: By Parts}
    \begin{thmbox}[Integration by Parts]
        \[
            \int u\, dv = uv - \int v\, du
        \]
        
        Derived from the product rule: $(uv)' = u'v + uv'$
    \end{thmbox}
    
    \vspace{0.3cm}
    
    \textbf{Example:} $\int x \cdot e^x\, dx$
    
    \begin{columns}
        \begin{column}{0.5\textwidth}
            Choose:
            \begin{itemize}
                \item $u = x \Rightarrow du = dx$
                \item $dv = e^x\, dx \Rightarrow v = e^x$
            \end{itemize}
        \end{column}
        \begin{column}{0.5\textwidth}
            Apply:
            \begin{align*}
                &= x \cdot e^x - \int e^x\, dx \\
                &= xe^x - e^x + C \\
                &= \mathbf{e^x(x-1) + C}
            \end{align*}
        \end{column}
    \end{columns}
    
    \begin{infobox}[LIATE Rule]
        Choose $u$ in order: \textbf{L}og, \textbf{I}nverse trig, \textbf{A}lgebraic, \textbf{T}rig, \textbf{E}xponential
    \end{infobox}
\end{frame}

% -----------------------------------------------------------------------------
% Why Integration Matters
% -----------------------------------------------------------------------------
\begin{frame}{Why Integration Matters in ML}
    \begin{keybox}[Applications]
        \begin{enumerate}
            \item \textbf{Probability Distributions}
            \[
                P(a \leq X \leq b) = \int_a^b f(x)\, dx
            \]
            (Area under probability density function)
            
            \item \textbf{Expected Values}
            \[
                E[X] = \int_{-\infty}^{\infty} x \cdot f(x)\, dx
            \]
            
            \item \textbf{Loss Functions} (continuous case)
            
            \item \textbf{Backpropagation} involves computing areas/volumes
            
            \item \textbf{Gaussian Integrals} (normal distribution!)
        \end{enumerate}
    \end{keybox}
\end{frame}

% -----------------------------------------------------------------------------
% Key Takeaways
% -----------------------------------------------------------------------------
\begin{frame}{Key Takeaways: Integration}
    \begin{keybox}
        \begin{enumerate}
            \item \textbf{Riemann sum}: Approximate area with rectangles
            \item \textbf{Definite integral} $\int_a^b f(x)\, dx$: Exact area under curve
            \item \textbf{Antiderivative}: $F$ where $F' = f$
            \item \textbf{Indefinite integral}: $\int f\, dx = F + C$
            \item \textbf{FTC}: $\int_a^b f\, dx = F(b) - F(a)$
            \item \textbf{Integration and differentiation are inverses!}
            \item \textbf{Techniques}: Substitution, by parts
            \item \textbf{ML uses}: Probability, expectations, continuous losses
        \end{enumerate}
    \end{keybox}
    
    \vspace{0.2cm}
    
    \centering
    \textit{Next: Graph Theory — the foundation for neural network structure!}
\end{frame}


% =============================================================================
% PART II: FOUNDATIONS FOR ML
% =============================================================================

% Section 7: Graph Theory Basics
% =============================================================================
% Section 7: Graph Theory Basics
% Understanding networks - the foundation for neural networks
% =============================================================================

\section{Graph Theory Basics}

% -----------------------------------------------------------------------------
% Opening
% -----------------------------------------------------------------------------
\begin{frame}{Graph Theory: Networks Before the Internet}
    \begin{columns}[T]
        \begin{column}{0.5\textwidth}
            \begin{funnybox}
                \textit{``Facebook, Twitter, neural networks — they're all just graphs in disguise. Euler figured this out in 1736 while solving a bridge puzzle!''}
            \end{funnybox}
            
            \vspace{0.3cm}
            
            \textbf{Why Graphs for ML?}
            \begin{itemize}
                \item Neural networks ARE graphs
                \item Information flows through edges
                \item Structure determines computation
            \end{itemize}
        \end{column}
        \begin{column}{0.5\textwidth}
            \centering
            \begin{tikzpicture}[scale=0.8]
                % Simple graph
                \node[circle,draw=accentcyan,fill=darkgray,minimum size=0.8cm] (A) at (0,2) {A};
                \node[circle,draw=accentcyan,fill=darkgray,minimum size=0.8cm] (B) at (2,2) {B};
                \node[circle,draw=accentcyan,fill=darkgray,minimum size=0.8cm] (C) at (3,0) {C};
                \node[circle,draw=accentcyan,fill=darkgray,minimum size=0.8cm] (D) at (1,0) {D};
                \node[circle,draw=accentcyan,fill=darkgray,minimum size=0.8cm] (E) at (-1,0) {E};
                
                % Edges
                \draw[neonpink,thick] (A) -- (B);
                \draw[neonpink,thick] (A) -- (D);
                \draw[neonpink,thick] (A) -- (E);
                \draw[neonpink,thick] (B) -- (C);
                \draw[neonpink,thick] (B) -- (D);
                \draw[neonpink,thick] (C) -- (D);
                \draw[neonpink,thick] (D) -- (E);
            \end{tikzpicture}
        \end{column}
    \end{columns}
\end{frame}

% -----------------------------------------------------------------------------
% Graph Definition
% -----------------------------------------------------------------------------
\begin{frame}{What is a Graph?}
    \begin{defbox}[Graph]
        A \glow{graph} $G = (V, E)$ consists of:
        \begin{itemize}
            \item $V$ = set of \textbf{vertices} (or nodes)
            \item $E$ = set of \textbf{edges} (connections between vertices)
        \end{itemize}
        
        Each edge $e \in E$ connects two vertices.
    \end{defbox}
    
    \vspace{0.3cm}
    
    \begin{columns}
        \begin{column}{0.5\textwidth}
            \textbf{Example:}
            \begin{align*}
                V &= \{1, 2, 3, 4\} \\
                E &= \{\{1,2\}, \{1,3\}, \{2,3\}, \{3,4\}\}
            \end{align*}
        \end{column}
        \begin{column}{0.5\textwidth}
            \centering
            \begin{tikzpicture}[scale=0.9]
                \node[circle,draw=neongreen,fill=darkgray] (1) at (0,1) {1};
                \node[circle,draw=neongreen,fill=darkgray] (2) at (2,1) {2};
                \node[circle,draw=neongreen,fill=darkgray] (3) at (1,0) {3};
                \node[circle,draw=neongreen,fill=darkgray] (4) at (3,0) {4};
                
                \draw[accentcyan,thick] (1) -- (2);
                \draw[accentcyan,thick] (1) -- (3);
                \draw[accentcyan,thick] (2) -- (3);
                \draw[accentcyan,thick] (3) -- (4);
            \end{tikzpicture}
        \end{column}
    \end{columns}
\end{frame}

% -----------------------------------------------------------------------------
% Types of Graphs
% -----------------------------------------------------------------------------
\begin{frame}{Types of Graphs}
    \begin{columns}[T]
        \begin{column}{0.5\textwidth}
            \begin{defbox}[Undirected Graph]
                Edges have no direction — connection is mutual.
                
                \centering
                \begin{tikzpicture}[scale=0.6]
                    \node[circle,draw=accentcyan,fill=darkgray,minimum size=0.6cm] (A) at (0,0) {};
                    \node[circle,draw=accentcyan,fill=darkgray,minimum size=0.6cm] (B) at (2,0) {};
                    \draw[neonpink,thick] (A) -- (B);
                \end{tikzpicture}
                
                A and B are connected (symmetric).
            \end{defbox}
            
            \vspace{0.3cm}
            
            \begin{defbox}[Directed Graph (Digraph)]
                Edges have direction — one-way connection.
                
                \centering
                \begin{tikzpicture}[scale=0.6]
                    \node[circle,draw=accentcyan,fill=darkgray,minimum size=0.6cm] (A) at (0,0) {};
                    \node[circle,draw=accentcyan,fill=darkgray,minimum size=0.6cm] (B) at (2,0) {};
                    \draw[->,neonpink,thick] (A) -- (B);
                \end{tikzpicture}
                
                A points to B (asymmetric).
            \end{defbox}
        \end{column}
        \begin{column}{0.5\textwidth}
            \begin{defbox}[Weighted Graph]
                Edges have associated values (weights).
                
                \centering
                \begin{tikzpicture}[scale=0.6]
                    \node[circle,draw=accentcyan,fill=darkgray,minimum size=0.6cm] (A) at (0,0) {};
                    \node[circle,draw=accentcyan,fill=darkgray,minimum size=0.6cm] (B) at (2,0) {};
                    \draw[neonpink,thick] (A) -- (B) node[midway,above,neonyellow] {0.7};
                \end{tikzpicture}
                
                Connection strength = 0.7
            \end{defbox}
            
            \vspace{0.3cm}
            
            \begin{keybox}[Neural Networks]
                Neural networks are \textbf{directed, weighted} graphs!
                
                \begin{itemize}
                    \item Neurons = vertices
                    \item Weights = edge values
                    \item Direction = information flow
                \end{itemize}
            \end{keybox}
        \end{column}
    \end{columns}
\end{frame}

% -----------------------------------------------------------------------------
% Vertex Degree
% -----------------------------------------------------------------------------
\begin{frame}{Degree of a Vertex}
    \begin{defbox}[Degree]
        The \glow{degree} of a vertex is the number of edges connected to it.
        
        For directed graphs:
        \begin{itemize}
            \item \textbf{In-degree}: edges coming IN
            \item \textbf{Out-degree}: edges going OUT
        \end{itemize}
    \end{defbox}
    
    \vspace{0.3cm}
    
    \begin{columns}
        \begin{column}{0.5\textwidth}
            \centering
            \begin{tikzpicture}[scale=0.8]
                \node[circle,draw=neonyellow,fill=darkgray,minimum size=0.8cm] (A) at (0,0) {A};
                \node[circle,draw=accentcyan,fill=darkgray,minimum size=0.8cm] (B) at (2,1) {B};
                \node[circle,draw=accentcyan,fill=darkgray,minimum size=0.8cm] (C) at (2,-1) {C};
                \node[circle,draw=accentcyan,fill=darkgray,minimum size=0.8cm] (D) at (-2,0) {D};
                
                \draw[neonpink,thick] (A) -- (B);
                \draw[neonpink,thick] (A) -- (C);
                \draw[neonpink,thick] (A) -- (D);
                
                \node[below=0.5cm,fgwhite] at (0,-1.5) {deg(A) = 3};
            \end{tikzpicture}
        \end{column}
        \begin{column}{0.5\textwidth}
            \centering
            \begin{tikzpicture}[scale=0.8]
                \node[circle,draw=neonyellow,fill=darkgray,minimum size=0.8cm] (A) at (0,0) {A};
                \node[circle,draw=accentcyan,fill=darkgray,minimum size=0.8cm] (B) at (2,1) {B};
                \node[circle,draw=accentcyan,fill=darkgray,minimum size=0.8cm] (C) at (2,-1) {C};
                \node[circle,draw=accentcyan,fill=darkgray,minimum size=0.8cm] (D) at (-2,0) {D};
                
                \draw[->,neonpink,thick] (B) -- (A);
                \draw[->,neonpink,thick] (A) -- (C);
                \draw[->,neonpink,thick] (D) -- (A);
                
                \node[below=0.3cm,fgwhite,font=\small] at (0,-1.5) {in(A)=2, out(A)=1};
            \end{tikzpicture}
        \end{column}
    \end{columns}
\end{frame}

% -----------------------------------------------------------------------------
% Paths and Connectivity
% -----------------------------------------------------------------------------
\begin{frame}{Paths and Connectivity}
    \begin{defbox}[Path]
        A \glow{path} is a sequence of vertices connected by edges:
        \[
            v_0 \rightarrow v_1 \rightarrow v_2 \rightarrow \cdots \rightarrow v_k
        \]
        
        \textbf{Path length} = number of edges (here: $k$)
    \end{defbox}
    
    \vspace{0.3cm}
    
    \begin{columns}
        \begin{column}{0.5\textwidth}
            \begin{defbox}[Connected Graph]
                A graph is \glow{connected} if there exists a path between every pair of vertices.
            \end{defbox}
            
            \begin{funnybox}
                Can you get from any vertex to any other? If yes: connected!
            \end{funnybox}
        \end{column}
        \begin{column}{0.5\textwidth}
            \centering
            \begin{tikzpicture}[scale=0.7]
                % Connected component 1
                \node[circle,draw=neongreen,fill=darkgray,minimum size=0.6cm] (1) at (0,1) {};
                \node[circle,draw=neongreen,fill=darkgray,minimum size=0.6cm] (2) at (1,0) {};
                \node[circle,draw=neongreen,fill=darkgray,minimum size=0.6cm] (3) at (0,-1) {};
                \draw[neongreen,thick] (1) -- (2) -- (3) -- (1);
                
                % Connected component 2
                \node[circle,draw=neonpink,fill=darkgray,minimum size=0.6cm] (4) at (3,0.5) {};
                \node[circle,draw=neonpink,fill=darkgray,minimum size=0.6cm] (5) at (3,-0.5) {};
                \draw[neonpink,thick] (4) -- (5);
                
                \node[below,fgwhite,font=\small] at (1.5,-1.5) {NOT connected (2 components)};
            \end{tikzpicture}
        \end{column}
    \end{columns}
\end{frame}

% -----------------------------------------------------------------------------
% Adjacency Matrix
% -----------------------------------------------------------------------------
\begin{frame}{Adjacency Matrix: Graphs as Matrices}
    \begin{defbox}[Adjacency Matrix]
        For a graph with $n$ vertices, the \glow{adjacency matrix} $A$ is an $n \times n$ matrix where:
        \[
            A_{ij} = \begin{cases}
                1 & \text{if edge from } i \text{ to } j \\
                w_{ij} & \text{if weighted edge} \\
                0 & \text{otherwise}
            \end{cases}
        \]
    \end{defbox}
    
    \vspace{0.3cm}
    
    \begin{columns}
        \begin{column}{0.4\textwidth}
            \centering
            \begin{tikzpicture}[scale=0.8]
                \node[circle,draw=accentcyan,fill=darkgray] (1) at (0,1.5) {1};
                \node[circle,draw=accentcyan,fill=darkgray] (2) at (1.5,1.5) {2};
                \node[circle,draw=accentcyan,fill=darkgray] (3) at (1.5,0) {3};
                \node[circle,draw=accentcyan,fill=darkgray] (4) at (0,0) {4};
                
                \draw[neonpink,thick] (1) -- (2);
                \draw[neonpink,thick] (1) -- (4);
                \draw[neonpink,thick] (2) -- (3);
                \draw[neonpink,thick] (3) -- (4);
            \end{tikzpicture}
        \end{column}
        \begin{column}{0.6\textwidth}
            \[
                A = \begin{pmatrix}
                    0 & 1 & 0 & 1 \\
                    1 & 0 & 1 & 0 \\
                    0 & 1 & 0 & 1 \\
                    1 & 0 & 1 & 0
                \end{pmatrix}
            \]
            
            Row $i$, Column $j$ = connection from $i$ to $j$
        \end{column}
    \end{columns}
    
    \begin{keybox}
        For undirected graphs: $A$ is symmetric ($A = A^T$)
    \end{keybox}
\end{frame}

% -----------------------------------------------------------------------------
% Neural Network as Graph
% -----------------------------------------------------------------------------
\begin{frame}{Neural Networks AS Graphs}
    \centering
    \begin{tikzpicture}[scale=0.9]
        % Input layer
        \foreach \i in {1,2,3} {
            \node[input neuron] (I\i) at (0, 2-\i) {$x_\i$};
        }
        
        % Hidden layer 1
        \foreach \i in {1,2,3,4} {
            \node[hidden neuron] (H1\i) at (2.5, 2.5-\i) {};
        }
        
        % Hidden layer 2
        \foreach \i in {1,2,3} {
            \node[hidden neuron] (H2\i) at (5, 2-\i) {};
        }
        
        % Output layer
        \foreach \i in {1,2} {
            \node[output neuron] (O\i) at (7.5, 1.5-\i) {$y_\i$};
        }
        
        % Connections (sample)
        \foreach \i in {1,2,3} {
            \foreach \j in {1,2,3,4} {
                \draw[connection] (I\i) -- (H1\j);
            }
        }
        
        \foreach \i in {1,2,3,4} {
            \foreach \j in {1,2,3} {
                \draw[connection] (H1\i) -- (H2\j);
            }
        }
        
        \foreach \i in {1,2,3} {
            \foreach \j in {1,2} {
                \draw[connection] (H2\i) -- (O\j);
            }
        }
        
        % Labels
        \node[neongreen,above,font=\small] at (0,1.5) {Input};
        \node[accentcyan,above,font=\small] at (2.5,2) {Hidden 1};
        \node[accentcyan,above,font=\small] at (5,1.5) {Hidden 2};
        \node[neonpink,above,font=\small] at (7.5,1) {Output};
    \end{tikzpicture}
    
    \vspace{0.3cm}
    
    \begin{columns}
        \begin{column}{0.5\textwidth}
            \begin{itemize}
                \item \textbf{Vertices}: Neurons (circles)
                \item \textbf{Edges}: Connections (arrows)
                \item \textbf{Weights}: Edge values (learned!)
            \end{itemize}
        \end{column}
        \begin{column}{0.5\textwidth}
            \begin{itemize}
                \item \textbf{Direction}: Left to right (feedforward)
                \item \textbf{Layers}: Groups of vertices
                \item \textbf{Computation}: Flows through graph
            \end{itemize}
        \end{column}
    \end{columns}
\end{frame}

% -----------------------------------------------------------------------------
% Information Flow
% -----------------------------------------------------------------------------
\begin{frame}{Information Flow Through Graphs}
    \begin{columns}[T]
        \begin{column}{0.5\textwidth}
            \begin{infobox}[Forward Pass]
                Information flows from input to output:
                
                \begin{enumerate}
                    \item Input enters at source nodes
                    \item Each node computes a value
                    \item Values propagate along edges
                    \item Output emerges at sink nodes
                \end{enumerate}
            \end{infobox}
        \end{column}
        \begin{column}{0.5\textwidth}
            \centering
            \begin{tikzpicture}[scale=0.8]
                \node[circle,draw=neongreen,fill=darkgray] (x1) at (0,1) {$x_1$};
                \node[circle,draw=neongreen,fill=darkgray] (x2) at (0,-1) {$x_2$};
                
                \node[circle,draw=accentcyan,fill=darkgray] (h) at (2,0) {$h$};
                
                \node[circle,draw=neonpink,fill=darkgray] (y) at (4,0) {$y$};
                
                \draw[->,thick,neonyellow] (x1) -- (h) node[midway,above,font=\small] {$w_1$};
                \draw[->,thick,neonyellow] (x2) -- (h) node[midway,below,font=\small] {$w_2$};
                \draw[->,thick,neonyellow] (h) -- (y) node[midway,above,font=\small] {$w_3$};
                
                % Computation
                \node[below=0.8cm,fgwhite,font=\small] at (2,0) {$h = \sigma(w_1 x_1 + w_2 x_2)$};
            \end{tikzpicture}
        \end{column}
    \end{columns}
    
    \vspace{0.5cm}
    
    \begin{keybox}
        The graph structure determines:
        \begin{itemize}
            \item What information each neuron receives
            \item How information is combined
            \item What the network can compute!
        \end{itemize}
    \end{keybox}
\end{frame}

% -----------------------------------------------------------------------------
% Graph Properties Summary
% -----------------------------------------------------------------------------
\begin{frame}{Graph Properties for ML}
    \begin{columns}[T]
        \begin{column}{0.5\textwidth}
            \begin{defbox}[DAG]
                A \glow{Directed Acyclic Graph} has:
                \begin{itemize}
                    \item Directed edges
                    \item No cycles (can't loop back)
                \end{itemize}
                
                Feedforward neural networks are DAGs!
            \end{defbox}
            
            \vspace{0.3cm}
            
            \begin{defbox}[Bipartite Graph]
                Vertices split into two groups — edges only between groups.
                
                Adjacent layers in a NN form bipartite subgraphs!
            \end{defbox}
        \end{column}
        \begin{column}{0.5\textwidth}
            \begin{defbox}[Complete Graph]
                Every vertex connected to every other.
                
                \centering
                \begin{tikzpicture}[scale=0.5]
                    \foreach \i in {1,...,5} {
                        \node[circle,draw=accentcyan,fill=darkgray,minimum size=0.4cm] (\i) at ({90+72*(\i-1)}:1.2) {};
                    }
                    \foreach \i in {1,...,5} {
                        \foreach \j in {\i,...,5} {
                            \draw[neonpink] (\i) -- (\j);
                        }
                    }
                \end{tikzpicture}
                
                ``Fully connected'' layers!
            \end{defbox}
        \end{column}
    \end{columns}
\end{frame}

% -----------------------------------------------------------------------------
% Key Takeaways
% -----------------------------------------------------------------------------
\begin{frame}{Key Takeaways: Graph Theory}
    \begin{keybox}
        \begin{enumerate}
            \item \textbf{Graph} $G = (V, E)$: vertices and edges
            \item \textbf{Types}: undirected, directed, weighted
            \item \textbf{Degree}: number of connections
            \item \textbf{Path}: sequence of connected vertices
            \item \textbf{Adjacency matrix}: graph as a matrix
            \item \textbf{Neural networks are graphs!}
            \begin{itemize}
                \item Neurons = vertices
                \item Connections = weighted, directed edges
                \item Feedforward NN = DAG
            \end{itemize}
            \item Structure determines computation capability
        \end{enumerate}
    \end{keybox}
    
    \vspace{0.2cm}
    
    \centering
    \textit{Next: Python Environment Setup — preparing our tools!}
\end{frame}


% Section 8: Python Environment Setup (MOVED HERE - before NN)
% =============================================================================
% Section 8: Python Environment Setup
% Setting up uv, virtual environments, and project structure
% =============================================================================

\section{Python Environment Setup}

% -----------------------------------------------------------------------------
% Opening
% -----------------------------------------------------------------------------
\begin{frame}{Python Environment Setup}
    \begin{columns}[T]
        \begin{column}{0.5\textwidth}
            \begin{funnybox}
                \textit{``Before we teach a machine to learn, we must first teach our computer to find Python. And not just any Python --- the RIGHT Python!''}
            \end{funnybox}
            
            \vspace{0.3cm}
            
            \textbf{Why environments matter:}
            \begin{itemize}
                \item Different projects need different packages
                \item Version conflicts are REAL
                \item Reproducibility is essential
            \end{itemize}
        \end{column}
        \begin{column}{0.5\textwidth}
            \centering
            \begin{tikzpicture}[scale=0.7]
                % Computer
                \draw[thick,accentcyan,rounded corners] (0,0) rectangle (4,3);
                \node[accentcyan] at (2,2.5) {Your Computer};
                
                % Virtual envs
                \draw[thick,neonpink,rounded corners] (0.3,0.3) rectangle (1.8,1.8);
                \node[neonpink,font=\tiny] at (1.05,1.5) {Project A};
                \node[font=\tiny,fgwhite] at (1.05,1.0) {numpy 1.21};
                \node[font=\tiny,fgwhite] at (1.05,0.6) {torch 2.0};
                
                \draw[thick,neongreen,rounded corners] (2.2,0.3) rectangle (3.7,1.8);
                \node[neongreen,font=\tiny] at (2.95,1.5) {Project B};
                \node[font=\tiny,fgwhite] at (2.95,1.0) {numpy 1.24};
                \node[font=\tiny,fgwhite] at (2.95,0.6) {torch 1.9};
            \end{tikzpicture}
            
            \vspace{0.2cm}
            \textit{Isolated environments = no conflicts!}
        \end{column}
    \end{columns}
\end{frame}

% -----------------------------------------------------------------------------
% Why UV?
% -----------------------------------------------------------------------------
\begin{frame}{Enter: \texttt{uv} --- The Fast Python Package Manager}
    \begin{defbox}[What is uv?]
        \texttt{uv} is an extremely fast Python package and project manager, written in Rust.
        
        \begin{itemize}
            \item 10-100x faster than pip
            \item Handles virtual environments automatically
            \item Modern project management
            \item Replaces: pip, pip-tools, virtualenv, poetry, pyenv
        \end{itemize}
    \end{defbox}
    
    \vspace{0.3cm}
    
    \begin{columns}
        \begin{column}{0.5\textwidth}
            \begin{alertbox}[Installing uv]
                \textbf{macOS/Linux:}
                
                \texttt{curl -LsSf https://astral.sh/uv/install.sh | sh}
                
                \vspace{0.2cm}
                \textbf{Windows:}
                
                \texttt{powershell -c "irm https://astral.sh/uv/install.ps1 | iex"}
            \end{alertbox}
        \end{column}
        \begin{column}{0.5\textwidth}
            \begin{funnybox}
                Why ``uv''? 
                
                Because it's so fast it operates at \textbf{UV frequencies} --- ultraviolet light! 
                
                (Actually, it's just a cool name.)
            \end{funnybox}
        \end{column}
    \end{columns}
\end{frame}

% -----------------------------------------------------------------------------
% uv init
% -----------------------------------------------------------------------------
\begin{frame}{Creating a New Project: \texttt{uv init}}
    \begin{defbox}[uv init]
        Creates a new Python project with proper structure:
        
        \texttt{uv init my\_ml\_project}
        
        \texttt{cd my\_ml\_project}
    \end{defbox}
    
    \vspace{0.3cm}
    
    \begin{columns}
        \begin{column}{0.5\textwidth}
            \textbf{What gets created:}
            
            \texttt{my\_ml\_project/}
            
            \quad \texttt{+-- .python-version}
            
            \quad \texttt{+-- pyproject.toml}
            
            \quad \texttt{+-- README.md}
            
            \quad \texttt{+-- hello.py}
        \end{column}
        \begin{column}{0.5\textwidth}
            \begin{keybox}[Key Files]
                \begin{itemize}
                    \item \texttt{.python-version}: Python version
                    \item \texttt{pyproject.toml}: Project config
                    \item \texttt{README.md}: Documentation
                    \item \texttt{hello.py}: Sample code
                \end{itemize}
            \end{keybox}
        \end{column}
    \end{columns}
\end{frame}

% -----------------------------------------------------------------------------
% pyproject.toml
% -----------------------------------------------------------------------------
\begin{frame}{Understanding \texttt{pyproject.toml}}
    \begin{defbox}[pyproject.toml]
        The \textbf{modern standard} for Python project configuration.
        
        Defines: name, version, dependencies, tools, scripts.
    \end{defbox}
    
    \vspace{0.2cm}
    
    \begin{columns}
        \begin{column}{0.55\textwidth}
            {\small\ttfamily
            [project]\\
            name = "ml-workshop"\\
            version = "0.1.0"\\
            requires-python = ">=3.11"\\
            dependencies = [\\
            \quad "numpy>=1.24.0",\\
            \quad "torch>=2.0.0",\\
            ]\\
            \\
            {[tool.uv]}\\
            dev-dependencies = [\\
            \quad "pytest>=7.0.0",\\
            ]
            }
        \end{column}
        \begin{column}{0.45\textwidth}
            \begin{keybox}[Sections]
                \begin{itemize}
                    \item \texttt{[project]}: Metadata
                    \item \texttt{dependencies}: Runtime deps
                    \item \texttt{[tool.uv]}: uv-specific
                    \item \texttt{dev-dependencies}: Dev tools
                \end{itemize}
            \end{keybox}
        \end{column}
    \end{columns}
\end{frame}

% -----------------------------------------------------------------------------
% uv sync
% -----------------------------------------------------------------------------
\begin{frame}{Installing Dependencies: \texttt{uv sync}}
    \begin{defbox}[uv sync]
        The \textbf{magic command} that:
        \begin{itemize}
            \item Creates a virtual environment (if needed)
            \item Installs all dependencies from pyproject.toml
            \item Locks versions in \texttt{uv.lock}
        \end{itemize}
    \end{defbox}
    
    \vspace{0.3cm}
    
    \begin{columns}
        \begin{column}{0.5\textwidth}
            \textbf{Basic usage:}
            
            \texttt{\$ uv sync}
            
            \vspace{0.2cm}
            \textbf{With dev dependencies:}
            
            \texttt{\$ uv sync --dev}
            
            \vspace{0.2cm}
            \textbf{Add new package:}
            
            \texttt{\$ uv add pandas}
        \end{column}
        \begin{column}{0.5\textwidth}
            \begin{keybox}[Lock File]
                \texttt{uv.lock} ensures:
                \begin{itemize}
                    \item Exact reproducibility
                    \item Same versions everywhere
                    \item Commit to git!
                \end{itemize}
            \end{keybox}
        \end{column}
    \end{columns}
\end{frame}

% -----------------------------------------------------------------------------
% Running Python with uv
% -----------------------------------------------------------------------------
\begin{frame}{Running Python with \texttt{uv}}
    \begin{defbox}[uv run]
        Run commands in the project's virtual environment:
        
        \texttt{uv run python script.py}
        
        \texttt{uv run pytest}
        
        \texttt{uv run jupyter notebook}
    \end{defbox}
    
    \vspace{0.3cm}
    
    \begin{columns}
        \begin{column}{0.5\textwidth}
            \begin{alertbox}[No Activation Needed!]
                Unlike traditional venvs:
                \begin{itemize}
                    \item No \texttt{source .venv/bin/activate}
                    \item No \texttt{deactivate}
                    \item Just prefix with \texttt{uv run}
                \end{itemize}
            \end{alertbox}
        \end{column}
        \begin{column}{0.5\textwidth}
            \begin{funnybox}
                Think of \texttt{uv run} as a magic portal that teleports your command into the right Python universe!
            \end{funnybox}
        \end{column}
    \end{columns}
\end{frame}

% -----------------------------------------------------------------------------
% Workshop Setup
% -----------------------------------------------------------------------------
\begin{frame}{Setting Up Our Workshop Environment}
    \begin{keybox}[Step by Step]
        \begin{enumerate}
            \item Install uv (if not done)
            \item Clone/create workshop project
            \item Run \texttt{uv sync}
            \item Start coding!
        \end{enumerate}
    \end{keybox}
    
    \vspace{0.3cm}
    
    \begin{columns}
        \begin{column}{0.5\textwidth}
            \textbf{Our dependencies:}
            \begin{itemize}
                \item \texttt{numpy} --- numerical computing
                \item \texttt{torch} --- deep learning
                \item \texttt{matplotlib} --- plotting
                \item \texttt{scikit-learn} --- ML algorithms
                \item \texttt{jupyter} --- notebooks
            \end{itemize}
        \end{column}
        \begin{column}{0.5\textwidth}
            \centering
            \begin{tikzpicture}[scale=0.8]
                \node[rectangle,draw=accentcyan,fill=darkgray,minimum width=2cm,minimum height=0.6cm] (uv) at (0,2) {\texttt{uv}};
                \node[rectangle,draw=neonpink,fill=darkgray,minimum width=1.5cm,minimum height=0.5cm] (np) at (-1.5,0) {numpy};
                \node[rectangle,draw=neonpink,fill=darkgray,minimum width=1.5cm,minimum height=0.5cm] (torch) at (0,0) {torch};
                \node[rectangle,draw=neonpink,fill=darkgray,minimum width=1.5cm,minimum height=0.5cm] (mpl) at (1.5,0) {mpl};
                
                \draw[->,accentcyan] (uv) -- (np);
                \draw[->,accentcyan] (uv) -- (torch);
                \draw[->,accentcyan] (uv) -- (mpl);
            \end{tikzpicture}
            
            \textit{uv manages everything!}
        \end{column}
    \end{columns}
\end{frame}

% -----------------------------------------------------------------------------
% Quick Reference
% -----------------------------------------------------------------------------
\begin{frame}{Quick Reference: Essential \texttt{uv} Commands}
    \begin{columns}[T]
        \begin{column}{0.5\textwidth}
            \begin{defbox}[Project Setup]
                \texttt{uv init} --- Create project
                
                \texttt{uv sync} --- Install dependencies
                
                \texttt{uv add PKG} --- Add package
                
                \texttt{uv remove PKG} --- Remove package
            \end{defbox}
        \end{column}
        \begin{column}{0.5\textwidth}
            \begin{defbox}[Running Code]
                \texttt{uv run python X.py}
                
                \texttt{uv run pytest}
                
                \texttt{uv run jupyter notebook}
                
                \texttt{uv run CMD}
            \end{defbox}
        \end{column}
    \end{columns}
    
    \vspace{0.3cm}
    
    \begin{keybox}[Remember]
        \begin{itemize}
            \item Always use \texttt{uv run} to execute Python in the project environment
            \item Commit both \texttt{pyproject.toml} AND \texttt{uv.lock} to version control
            \item Run \texttt{uv sync} after pulling changes
        \end{itemize}
    \end{keybox}
\end{frame}

% -----------------------------------------------------------------------------
% Summary
% -----------------------------------------------------------------------------
\begin{frame}{Python Setup: Summary}
    \begin{columns}[T]
        \begin{column}{0.5\textwidth}
            \begin{keybox}[What We Learned]
                \begin{itemize}
                    \item Why virtual environments matter
                    \item How to use \texttt{uv} for fast package management
                    \item Project structure with \texttt{pyproject.toml}
                    \item Running code with \texttt{uv run}
                \end{itemize}
            \end{keybox}
        \end{column}
        \begin{column}{0.5\textwidth}
            \begin{funnybox}
                Now your Python environment is ready for machine learning!
                
                Let's make those neurons fire!
            \end{funnybox}
        \end{column}
    \end{columns}
    
    \vspace{0.5cm}
    
    \centering
    \textbf{Next: Neural Networks!}
\end{frame}


% =============================================================================
% PART III: NEURAL NETWORKS
% =============================================================================

% Section 9: Neural Networks Introduction
% =============================================================================
% Section 9: Introduction to Neural Networks
% What are neural networks and why do they work?
% =============================================================================

\section{Introduction to Neural Networks}

% -----------------------------------------------------------------------------
% Opening
% -----------------------------------------------------------------------------
\begin{frame}{Neural Networks: Inspired by Biology}
    \begin{columns}[T]
        \begin{column}{0.5\textwidth}
            \begin{funnybox}
                \textit{``Humans tried to build flying machines by copying birds. It kinda worked. Then we tried copying brains. It... also kinda worked?''}
            \end{funnybox}
            
            \vspace{0.3cm}
            
            \textbf{The Big Idea:}
            \begin{itemize}
                \item Brains are amazing at learning
                \item Brains are made of neurons
                \item Let's build artificial neurons!
            \end{itemize}
        \end{column}
        \begin{column}{0.5\textwidth}
            \centering
            \begin{tikzpicture}[scale=0.9]
                % Biological neuron (simplified)
                % Cell body
                \draw[thick,accentcyan,fill=darkgray] (0,0) circle (0.5);
                \node[accentcyan,font=\tiny] at (0,0) {soma};
                
                % Dendrites
                \draw[thick,neongreen] (-1.5,0.5) -- (-0.5,0.2);
                \draw[thick,neongreen] (-1.5,0) -- (-0.5,0);
                \draw[thick,neongreen] (-1.5,-0.5) -- (-0.5,-0.2);
                \node[neongreen,font=\tiny,left] at (-1.5,0.5) {dendrites};
                
                % Axon
                \draw[thick,neonpink] (0.5,0) -- (2,0);
                \draw[thick,neonpink] (2,0) -- (2.5,0.3);
                \draw[thick,neonpink] (2,0) -- (2.5,0);
                \draw[thick,neonpink] (2,0) -- (2.5,-0.3);
                \node[neonpink,font=\tiny,above] at (1.25,0) {axon};
                
                % Arrow
                \draw[->,thick,neonyellow] (0,-1) -- (0,-1.8);
                
                % Artificial neuron
                \node[circle,draw=accentcyan,fill=darkgray,minimum size=1cm] (n) at (0,-2.8) {$\sigma$};
                \draw[thick,neongreen] (-1.5,-2.5) -- (-0.5,-2.7);
                \draw[thick,neongreen] (-1.5,-2.8) -- (-0.5,-2.8);
                \draw[thick,neongreen] (-1.5,-3.1) -- (-0.5,-2.9);
                \draw[->,thick,neonpink] (0.5,-2.8) -- (1.5,-2.8);
                
                \node[fgwhite,font=\tiny] at (0,-3.7) {Artificial Neuron};
            \end{tikzpicture}
        \end{column}
    \end{columns}
\end{frame}

% -----------------------------------------------------------------------------
% The Perceptron
% -----------------------------------------------------------------------------
\begin{frame}{The Perceptron: Simplest Neural Unit}
    \begin{defbox}[Perceptron (Rosenblatt, 1958)]
        A \glow{perceptron} computes:
        \[
            y = \sigma\left(\sum_{i=1}^{n} w_i x_i + b\right) = \sigma(\mathbf{w}^\top \mathbf{x} + b)
        \]
        
        Where:
        \begin{itemize}
            \item $\mathbf{x} = (x_1, \ldots, x_n)$: inputs
            \item $\mathbf{w} = (w_1, \ldots, w_n)$: weights
            \item $b$: bias
            \item $\sigma$: activation function
        \end{itemize}
    \end{defbox}
    
    \centering
    \begin{tikzpicture}[scale=0.7]
        % Inputs
        \node[input neuron] (x1) at (0,1.5) {$x_1$};
        \node[input neuron] (x2) at (0,0) {$x_2$};
        \node[input neuron] (x3) at (0,-1.5) {$x_3$};
        
        % Neuron
        \node[hidden neuron,minimum size=1cm] (n) at (3,0) {$\sigma$};
        
        % Output
        \node[output neuron] (y) at (6,0) {$y$};
        
        % Connections with weights
        \draw[connection] (x1) -- (n) node[midway,above,font=\small] {$w_1$};
        \draw[connection] (x2) -- (n) node[midway,above,font=\small] {$w_2$};
        \draw[connection] (x3) -- (n) node[midway,below,font=\small] {$w_3$};
        \draw[connection] (n) -- (y);
        
        % Bias
        \node[below,neonyellow,font=\small] at (3,-1.2) {$+b$};
    \end{tikzpicture}
\end{frame}

% -----------------------------------------------------------------------------
% Linear Combination
% -----------------------------------------------------------------------------
\begin{frame}{Step 1: Weighted Sum (Linear Combination)}
    \begin{defbox}[Weighted Sum]
        The neuron computes a weighted sum of inputs:
        \[
            z = \sum_{i=1}^{n} w_i x_i + b = w_1 x_1 + w_2 x_2 + \cdots + w_n x_n + b
        \]
    \end{defbox}
    
    \vspace{0.3cm}
    
    \begin{columns}
        \begin{column}{0.5\textwidth}
            \textbf{Example:}
            \begin{align*}
                \mathbf{x} &= (0.5, 0.8, -0.3) \\
                \mathbf{w} &= (0.4, -0.2, 0.6) \\
                b &= 0.1
            \end{align*}
            
            \begin{align*}
                z &= 0.4(0.5) + (-0.2)(0.8) + 0.6(-0.3) + 0.1 \\
                &= 0.2 - 0.16 - 0.18 + 0.1 \\
                &= -0.04
            \end{align*}
        \end{column}
        \begin{column}{0.5\textwidth}
            \begin{funnybox}
                Weighted sum = ``voting with different levels of influence''
                
                \vspace{0.1cm}
                
                Big $|w_i|$ = strong opinion\\
                $w_i > 0$ = votes YES\\
                $w_i < 0$ = votes NO
            \end{funnybox}
        \end{column}
    \end{columns}
\end{frame}

% -----------------------------------------------------------------------------
% Why Activation Functions?
% -----------------------------------------------------------------------------
\begin{frame}{Step 2: Activation Function — Adding Non-Linearity}
    \begin{alertbox}[The Problem]
        Linear functions composed = still linear!
        \[
            f_2(f_1(\mathbf{x})) = W_2(W_1\mathbf{x} + b_1) + b_2 = W_2 W_1 \mathbf{x} + (W_2 b_1 + b_2) = W'\mathbf{x} + b'
        \]
    \end{alertbox}
    
    \vspace{0.3cm}
    
    \begin{keybox}[Solution: Non-Linear Activation]
        Apply a non-linear function $\sigma$ after each linear transform:
        \[
            y = \sigma(z) = \sigma(\mathbf{w}^\top \mathbf{x} + b)
        \]
    \end{keybox}
    
    \vspace{0.3cm}
    
    \begin{funnybox}
        Without activation functions, a 1000-layer network is just... a single matrix multiplication wearing a disguise.
    \end{funnybox}
\end{frame}

% -----------------------------------------------------------------------------
% Common Activation Functions
% -----------------------------------------------------------------------------
\begin{frame}{Common Activation Functions}
    \begin{columns}[T]
        \begin{column}{0.5\textwidth}
            \begin{defbox}[Sigmoid]
                \[
                    \sigma(z) = \frac{1}{1 + e^{-z}}
                \]
                
                Range: $(0, 1)$
                
                \centering
                \begin{tikzpicture}[scale=0.5]
                    \draw[->] (-3,0) -- (3,0) node[right] {$z$};
                    \draw[->] (0,-0.5) -- (0,2) node[above] {$\sigma$};
                    \draw[thick,neonpink,domain=-2.5:2.5,samples=50] plot (\x, {1.5/(1+exp(-\x))});
                    \draw[dashed,fgwhite] (-2.5,1.5) -- (2.5,1.5);
                \end{tikzpicture}
            \end{defbox}
            
            \begin{defbox}[Tanh]
                \[
                    \tanh(z) = \frac{e^z - e^{-z}}{e^z + e^{-z}}
                \]
                
                Range: $(-1, 1)$
            \end{defbox}
        \end{column}
        \begin{column}{0.5\textwidth}
            \begin{defbox}[ReLU (Most Popular!)]
                \[
                    \text{ReLU}(z) = \max(0, z)
                \]
                
                Range: $[0, \infty)$
                
                \centering
                \begin{tikzpicture}[scale=0.5]
                    \draw[->] (-2,0) -- (3,0) node[right] {$z$};
                    \draw[->] (0,-0.5) -- (0,2.5) node[above] {ReLU};
                    \draw[thick,neongreen] (-2,0) -- (0,0) -- (2,2);
                \end{tikzpicture}
            \end{defbox}
            
            \begin{keybox}
                ReLU is fast, simple, and works!
                
                \textit{``If in doubt, use ReLU.''}
            \end{keybox}
        \end{column}
    \end{columns}
\end{frame}

% -----------------------------------------------------------------------------
% Universal Approximation
% -----------------------------------------------------------------------------
\begin{frame}{The Universal Approximation Theorem}
    \begin{thmbox}[Universal Approximation (Cybenko, 1989)]
        A feedforward neural network with a single hidden layer containing a finite number of neurons can approximate any continuous function on compact subsets of $\mathbb{R}^n$ to arbitrary accuracy.
    \end{thmbox}
    
    \vspace{0.3cm}
    
    \begin{columns}
        \begin{column}{0.5\textwidth}
            \begin{funnybox}
                \textit{``Given enough neurons, a neural network can approximate ANY function!''}
                
                \vspace{0.1cm}
                
                The catch? ``Enough'' might be astronomically large.
            \end{funnybox}
        \end{column}
        \begin{column}{0.5\textwidth}
            \begin{infobox}[Implications]
                \begin{itemize}
                    \item NNs are \textit{universal function approximators}
                    \item Existence theorem (not constructive)
                    \item Depth helps: deeper = more efficient
                \end{itemize}
            \end{infobox}
        \end{column}
    \end{columns}
\end{frame}

% -----------------------------------------------------------------------------
% Learning as Optimization
% -----------------------------------------------------------------------------
\begin{frame}{Learning = Finding Good Weights}
    \begin{columns}[T]
        \begin{column}{0.5\textwidth}
            \begin{defbox}[The Learning Problem]
                Given:
                \begin{itemize}
                    \item Data: $\{(\mathbf{x}^{(i)}, y^{(i)})\}_{i=1}^{N}$
                    \item Network architecture
                \end{itemize}
                
                Find: weights $\mathbf{W}$ such that
                \[
                    f_{\mathbf{W}}(\mathbf{x}^{(i)}) \approx y^{(i)} \quad \forall i
                \]
            \end{defbox}
        \end{column}
        \begin{column}{0.5\textwidth}
            \centering
            \begin{tikzpicture}[scale=0.8]
                % Loss landscape (simplified)
                \draw[->] (0,0) -- (4,0) node[right,font=\small] {$w$};
                \draw[->] (0,0) -- (0,3) node[above,font=\small] {Loss};
                
                % Curve
                \draw[thick,accentcyan,domain=0.5:3.5,samples=50] 
                    plot (\x, {0.5 + 1.5*(\x-2)*(\x-2)});
                
                % Minimum
                \fill[neongreen] (2,0.5) circle (2pt);
                \node[below,neongreen,font=\small] at (2,0.3) {optimal $w^*$};
                
                % Current point
                \fill[neonpink] (0.8,1.72) circle (2pt);
                \node[above,neonpink,font=\small] at (0.8,1.72) {current $w$};
                
                % Arrow
                \draw[->,thick,neonyellow] (0.8,1.72) -- (1.4,1.1);
            \end{tikzpicture}
            
            \textit{Minimize loss by adjusting weights}
        \end{column}
    \end{columns}
    
    \vspace{0.3cm}
    
    \begin{keybox}
        \textbf{Training} = searching for weights that minimize prediction error
        
        Tool: \textbf{Gradient Descent} (uses derivatives we learned!)
    \end{keybox}
\end{frame}

% -----------------------------------------------------------------------------
% What Can NNs Do?
% -----------------------------------------------------------------------------
\begin{frame}{What Can Neural Networks Do?}
    \begin{columns}[T]
        \begin{column}{0.33\textwidth}
            \begin{keybox}[Classification]
                Is this a cat or dog?
                
                \centering
                \begin{tikzpicture}[scale=0.5]
                    \draw[thick,accentcyan] (0,0) circle (1);
                    \draw[thick,neonpink] (0,0) ellipse (2 and 1.5);
                    \fill[neongreen] (0.3,0.3) circle (2pt);
                    \fill[neonyellow] (-1.2,0.5) circle (2pt);
                \end{tikzpicture}
                
                \textit{Separate classes}
            \end{keybox}
        \end{column}
        \begin{column}{0.33\textwidth}
            \begin{keybox}[Regression]
                Predict house price
                
                \centering
                \begin{tikzpicture}[scale=0.5]
                    \draw[->] (0,0) -- (2.5,0);
                    \draw[->] (0,0) -- (0,2);
                    \draw[thick,neonpink,domain=0:2,samples=30] plot (\x, {0.5 + 0.6*\x});
                    \foreach \x/\y in {0.3/0.7, 0.8/1.1, 1.2/0.9, 1.6/1.4, 2/1.5} {
                        \fill[accentcyan] (\x,\y) circle (2pt);
                    }
                \end{tikzpicture}
                
                \textit{Continuous output}
            \end{keybox}
        \end{column}
        \begin{column}{0.33\textwidth}
            \begin{keybox}[Generation]
                Create new images/text
                
                \centering
                \begin{tikzpicture}[scale=0.5]
                    \draw[thick,accentcyan,rounded corners] (0,0) rectangle (2,1.5);
                    \node[font=\Large] at (1,0.75) {?};
                    \draw[->,thick,neongreen] (2.2,0.75) -- (2.8,0.75);
                    \draw[thick,neonpink,rounded corners] (3,0) rectangle (5,1.5);
                    \node[font=\Large,neonpink] at (4,0.75) {Art};
                \end{tikzpicture}
                
                \textit{Create content}
            \end{keybox}
        \end{column}
    \end{columns}
    
    \vspace{0.5cm}
    
    \begin{funnybox}
        Neural networks power: image recognition, language translation, game playing, drug discovery, art generation, and... whatever GPT does when nobody's looking.
    \end{funnybox}
\end{frame}

% -----------------------------------------------------------------------------
% Key Takeaways
% -----------------------------------------------------------------------------
\begin{frame}{Key Takeaways: Neural Networks Introduction}
    \begin{keybox}
        \begin{enumerate}
            \item A \textbf{neuron} computes: $y = \sigma(\mathbf{w}^\top \mathbf{x} + b)$
            \item Components:
            \begin{itemize}
                \item \textbf{Weights} ($\mathbf{w}$): learned parameters
                \item \textbf{Bias} ($b$): threshold/offset
                \item \textbf{Activation} ($\sigma$): adds non-linearity
            \end{itemize}
            \item \textbf{ReLU} is the most common activation: $\max(0, z)$
            \item \textbf{Universal Approximation}: NNs can approximate any function
            \item \textbf{Learning} = finding weights that minimize error
            \item Applications: classification, regression, generation
        \end{enumerate}
    \end{keybox}
    
    \vspace{0.2cm}
    
    \centering
    \textit{Next: Building deeper networks with multiple layers!}
\end{frame}


% Section 10: Neural Network Architectures
% =============================================================================
% Section 10: Neural Network Architectures
% Multi-layer perceptrons and network design
% =============================================================================

\section{Neural Network Architectures}

% -----------------------------------------------------------------------------
% From Single Neuron to Layers
% -----------------------------------------------------------------------------
\begin{frame}{From Single Neuron to Networks}
    \begin{columns}[T]
        \begin{column}{0.5\textwidth}
            \begin{funnybox}
                \textit{``One neuron is smart. Many neurons together are... either genius or chaos. Let's aim for genius.''}
            \end{funnybox}
            
            \vspace{0.3cm}
            
            \textbf{Why go deeper?}
            \begin{itemize}
                \item Single neuron: linear decision boundary
                \item Multiple neurons: complex patterns
                \item More layers: hierarchical features
            \end{itemize}
        \end{column}
        \begin{column}{0.5\textwidth}
            \centering
            % Single neuron limitation
            \begin{tikzpicture}[scale=0.8]
                \node[font=\small,accentcyan] at (1.5,2.5) {Single Neuron};
                \draw[->] (0,0) -- (3,0);
                \draw[->] (0,0) -- (0,2);
                
                % XOR problem - can't separate
                \fill[neongreen] (0.5,0.5) circle (3pt);
                \fill[neongreen] (2,1.5) circle (3pt);
                \fill[neonpink] (0.5,1.5) circle (3pt);
                \fill[neonpink] (2,0.5) circle (3pt);
                
                % Linear separator - fails
                \draw[dashed,neonyellow] (0,1) -- (3,1);
                \node[font=\tiny,fgwhite] at (1.5,-0.5) {Can't separate XOR!};
            \end{tikzpicture}
            
            \vspace{0.3cm}
            
            \begin{tikzpicture}[scale=0.6]
                \node[font=\small,neonpink] at (1.5,2.5) {Multi-Layer};
                \draw[->] (0,0) -- (3,0);
                \draw[->] (0,0) -- (0,2);
                
                % XOR problem - separated
                \fill[neongreen] (0.5,0.5) circle (3pt);
                \fill[neongreen] (2,1.5) circle (3pt);
                \fill[neonpink] (0.5,1.5) circle (3pt);
                \fill[neonpink] (2,0.5) circle (3pt);
                
                % Non-linear separator
                \draw[thick,neonyellow] (0.5,1) .. controls (1.25,0.5) .. (2,1);
                \draw[thick,neonyellow] (0.5,1) .. controls (1.25,1.5) .. (2,1);
                
                \node[font=\tiny,fgwhite] at (1.5,-0.5) {Can separate!};
            \end{tikzpicture}
        \end{column}
    \end{columns}
\end{frame}

% -----------------------------------------------------------------------------
% Multi-Layer Perceptron
% -----------------------------------------------------------------------------
\begin{frame}{Multi-Layer Perceptron (MLP)}
    \begin{defbox}[MLP]
        A \glow{Multi-Layer Perceptron} consists of:
        \begin{itemize}
            \item \textbf{Input layer}: receives data
            \item \textbf{Hidden layers}: intermediate computations
            \item \textbf{Output layer}: produces predictions
        \end{itemize}
    \end{defbox}
    
    \vspace{0.2cm}
    
    \centering
    \begin{tikzpicture}[scale=0.80]
        % Input layer
        \foreach \i in {1,2,3,4} {
            \node[input neuron] (I\i) at (0, 3-\i) {$x_\i$};
        }
        
        % Hidden layer 1
        \foreach \i in {1,2,3,4,5} {
            \node[hidden neuron] (H1\i) at (2.5, 3.5-\i) {};
        }
        
        % Hidden layer 2
        \foreach \i in {1,2,3,4} {
            \node[hidden neuron] (H2\i) at (5, 3-\i) {};
        }
        
        % Output layer
        \foreach \i in {1,2} {
            \node[output neuron] (O\i) at (7.5, 1.5-\i+1) {$y_\i$};
        }
        
        % Connections
        \foreach \i in {1,2,3,4} {
            \foreach \j in {1,2,3,4,5} {
                \draw[connection,opacity=0.5] (I\i) -- (H1\j);
            }
        }
        
        \foreach \i in {1,2,3,4,5} {
            \foreach \j in {1,2,3,4} {
                \draw[connection,opacity=0.5] (H1\i) -- (H2\j);
            }
        }
        
        \foreach \i in {1,2,3,4} {
            \foreach \j in {1,2} {
                \draw[connection,opacity=0.5] (H2\i) -- (O\j);
            }
        }
        
        % Labels
        \node[neongreen,below] at (0,-1.2) {Input (4)};
        \node[accentcyan,below] at (2.5,-1.7) {Hidden 1 (5)};
        \node[accentcyan,below] at (5,-1.2) {Hidden 2 (4)};
        \node[neonpink,below] at (7.5,-0.7) {Output (2)};
    \end{tikzpicture}
\end{frame}

% -----------------------------------------------------------------------------
% Layer Math
% -----------------------------------------------------------------------------
\begin{frame}{Layer-wise Computation}
    \begin{defbox}[Layer Computation]
        For layer $l$ with $n_{l-1}$ inputs and $n_l$ outputs:
        \[
            \mathbf{a}^{[l]} = \sigma\left(W^{[l]} \mathbf{a}^{[l-1]} + \mathbf{b}^{[l]}\right)
        \]
        
        Where:
        \begin{itemize}
            \item $W^{[l]} \in \mathbb{R}^{n_l \times n_{l-1}}$: weight matrix
            \item $\mathbf{b}^{[l]} \in \mathbb{R}^{n_l}$: bias vector
            \item $\mathbf{a}^{[l]} \in \mathbb{R}^{n_l}$: activations (output)
        \end{itemize}
    \end{defbox}
    
    \vspace{0.3cm}
    
    \begin{columns}
        \begin{column}{0.5\textwidth}
            \textbf{Dimensions matter!}
            
            \begin{align*}
                \text{Input: } & \mathbf{x} \in \mathbb{R}^{4} \\
                W^{[1]} &\in \mathbb{R}^{5 \times 4} \\
                \mathbf{a}^{[1]} &= \sigma(W^{[1]}\mathbf{x} + \mathbf{b}^{[1]}) \in \mathbb{R}^{5}
            \end{align*}
        \end{column}
        \begin{column}{0.5\textwidth}
            \begin{keybox}
                Matrix multiplication handles all neurons in a layer simultaneously!
                
                \vspace{0.1cm}
                
                \textit{Vectorization = speed}
            \end{keybox}
        \end{column}
    \end{columns}
\end{frame}

% -----------------------------------------------------------------------------
% Notation
% -----------------------------------------------------------------------------
\begin{frame}{Notation Convention}
    \begin{defbox}[Standard Notation]
        \begin{tabular}{ll}
            $L$ & Total number of layers \\
            $n^{[l]}$ & Number of neurons in layer $l$ \\
            $W^{[l]}$ & Weights from layer $l-1$ to $l$ \\
            $\mathbf{b}^{[l]}$ & Biases for layer $l$ \\
            $\mathbf{z}^{[l]}$ & Pre-activation: $W^{[l]}\mathbf{a}^{[l-1]} + \mathbf{b}^{[l]}$ \\
            $\mathbf{a}^{[l]}$ & Post-activation: $\sigma(\mathbf{z}^{[l]})$ \\
            $\mathbf{a}^{[0]} = \mathbf{x}$ & Input \\
            $\mathbf{a}^{[L]} = \hat{\mathbf{y}}$ & Output (prediction)
        \end{tabular}
    \end{defbox}
    
    \vspace{0.3cm}
    
    \begin{funnybox}
        Superscript $[l]$ = layer number, not exponent!
        
        $W^{[2]}$ means ``weights of layer 2'', not ``$W$ squared''
    \end{funnybox}
\end{frame}

% -----------------------------------------------------------------------------
% Full Forward Pass
% -----------------------------------------------------------------------------
\begin{frame}{Full Forward Pass}
    \begin{keybox}[Forward Propagation]
        For a network with $L$ layers:
        
        \vspace{0.2cm}
        
        \begin{enumerate}
            \item Input: $\mathbf{a}^{[0]} = \mathbf{x}$
            \item For $l = 1, 2, \ldots, L$:
            \begin{align*}
                \mathbf{z}^{[l]} &= W^{[l]} \mathbf{a}^{[l-1]} + \mathbf{b}^{[l]} \\
                \mathbf{a}^{[l]} &= \sigma_l(\mathbf{z}^{[l]})
            \end{align*}
            \item Output: $\hat{\mathbf{y}} = \mathbf{a}^{[L]}$
        \end{enumerate}
    \end{keybox}
    
    \vspace{0.3cm}
    
    \centering
    \begin{tikzpicture}[scale=0.7]
        % Flow diagram
        \node[draw=neongreen,rounded corners,fill=darkgray] (x) at (0,0) {$\mathbf{x}$};
        \node[draw=accentcyan,rounded corners,fill=darkgray] (z1) at (2.5,0) {$\mathbf{z}^{[1]}$};
        \node[draw=accentcyan,rounded corners,fill=darkgray] (a1) at (5,0) {$\mathbf{a}^{[1]}$};
        \node[draw=accentcyan,rounded corners,fill=darkgray] (z2) at (7.5,0) {$\mathbf{z}^{[2]}$};
        \node[draw=neonpink,rounded corners,fill=darkgray] (y) at (10,0) {$\hat{\mathbf{y}}$};
        
        \draw[->,thick,neonyellow] (x) -- (z1) node[midway,above,font=\tiny] {$W^{[1]}, \mathbf{b}^{[1]}$};
        \draw[->,thick,neonyellow] (z1) -- (a1) node[midway,above,font=\tiny] {$\sigma$};
        \draw[->,thick,neonyellow] (a1) -- (z2) node[midway,above,font=\tiny] {$W^{[2]}, \mathbf{b}^{[2]}$};
        \draw[->,thick,neonyellow] (z2) -- (y) node[midway,above,font=\tiny] {$\sigma$};
    \end{tikzpicture}
\end{frame}

% -----------------------------------------------------------------------------
% Network Width and Depth
% -----------------------------------------------------------------------------
\begin{frame}{Width vs Depth}
    \begin{columns}[T]
        \begin{column}{0.5\textwidth}
            \begin{defbox}[Width]
                \glow{Width} = number of neurons per layer
                
                \centering
                \begin{tikzpicture}[scale=0.5]
                    \foreach \i in {1,...,6} {
                        \node[circle,draw=accentcyan,fill=darkgray,minimum size=0.4cm] at (0,3-0.5*\i) {};
                    }
                    \foreach \i in {1,...,6} {
                        \node[circle,draw=accentcyan,fill=darkgray,minimum size=0.4cm] at (2,3-0.5*\i) {};
                    }
                \end{tikzpicture}
                
                Wide network
            \end{defbox}
            
            \textbf{More width:}
            \begin{itemize}
                \item More parameters per layer
                \item Can memorize more patterns
                \item Risk of overfitting
            \end{itemize}
        \end{column}
        \begin{column}{0.5\textwidth}
            \begin{defbox}[Depth]
                \glow{Depth} = number of layers
                
                \centering
                \begin{tikzpicture}[scale=0.5]
                    \foreach \l in {0,1.5,3,4.5} {
                        \foreach \i in {1,2,3} {
                            \node[circle,draw=neonpink,fill=darkgray,minimum size=0.4cm] at (\l,1.5-0.5*\i) {};
                        }
                    }
                \end{tikzpicture}
                
                Deep network
            \end{defbox}
            
            \textbf{More depth:}
            \begin{itemize}
                \item Hierarchical features
                \item More expressive (often)
                \item Harder to train
            \end{itemize}
        \end{column}
    \end{columns}
    
    \vspace{0.3cm}
    
    \begin{funnybox}
        \textit{``Deep Learning''} = lots of layers. Who would have guessed?
    \end{funnybox}
\end{frame}

% -----------------------------------------------------------------------------
% Hidden Layer Features
% -----------------------------------------------------------------------------
\begin{frame}{What Do Hidden Layers Learn?}
    \begin{columns}[T]
        \begin{column}{0.5\textwidth}
            \begin{infobox}[Feature Hierarchy]
                Each layer learns increasingly abstract features:
                
                \vspace{0.2cm}
                
                \textbf{Image Recognition Example:}
                \begin{enumerate}
                    \item Layer 1: Edges, gradients
                    \item Layer 2: Textures, patterns
                    \item Layer 3: Parts (eyes, wheels)
                    \item Layer 4+: Objects, concepts
                \end{enumerate}
            \end{infobox}
        \end{column}
        \begin{column}{0.5\textwidth}
            \centering
            \begin{tikzpicture}[scale=0.7]
                % Hierarchy visualization
                \node[draw=neongreen,rounded corners,minimum width=2cm] (in) at (0,0) {Pixels};
                \node[draw=accentcyan,rounded corners,minimum width=2cm] (l1) at (0,-1.2) {Edges};
                \node[draw=accentcyan,rounded corners,minimum width=2cm] (l2) at (0,-2.4) {Textures};
                \node[draw=accentcyan,rounded corners,minimum width=2cm] (l3) at (0,-3.6) {Parts};
                \node[draw=neonpink,rounded corners,minimum width=2cm] (out) at (0,-4.8) {``Cat''};
                
                \draw[->,thick,neonyellow] (in) -- (l1);
                \draw[->,thick,neonyellow] (l1) -- (l2);
                \draw[->,thick,neonyellow] (l2) -- (l3);
                \draw[->,thick,neonyellow] (l3) -- (out);
            \end{tikzpicture}
        \end{column}
    \end{columns}
    
    \vspace{0.3cm}
    
    \begin{keybox}
        Hidden layers = learned feature extractors. They transform raw data into useful representations!
    \end{keybox}
\end{frame}

% -----------------------------------------------------------------------------
% Output Layer Design
% -----------------------------------------------------------------------------
\begin{frame}{Output Layer Design}
    \begin{columns}[T]
        \begin{column}{0.5\textwidth}
            \begin{defbox}[Binary Classification]
                \textbf{1 output neuron} with sigmoid
                
                \[
                    P(y=1|\mathbf{x}) = \sigma(z) \in (0, 1)
                \]
                
                \textit{Is it a cat? Yes/No}
            \end{defbox}
            
            \vspace{0.3cm}
            
            \begin{defbox}[Regression]
                \textbf{1 output neuron} (no activation or linear)
                
                \[
                    \hat{y} = z \in \mathbb{R}
                \]
                
                \textit{Predict house price}
            \end{defbox}
        \end{column}
        \begin{column}{0.5\textwidth}
            \begin{defbox}[Multi-class Classification]
                \textbf{$K$ output neurons} with softmax
                
                \[
                    P(y=k|\mathbf{x}) = \frac{e^{z_k}}{\sum_{j=1}^{K} e^{z_j}}
                \]
                
                Outputs sum to 1!
                
                \textit{Cat, dog, or bird?}
            \end{defbox}
            
            \vspace{0.2cm}
            
            \begin{keybox}
                Softmax = ``soft'' version of argmax
                
                Converts scores to probabilities
            \end{keybox}
        \end{column}
    \end{columns}
\end{frame}

% -----------------------------------------------------------------------------
% Parameter Count
% -----------------------------------------------------------------------------
\begin{frame}{Counting Parameters}
    \begin{defbox}[Parameter Count]
        For a fully connected layer from $n_{in}$ to $n_{out}$:
        
        \[
            \text{Parameters} = n_{out} \times n_{in} + n_{out} = n_{out}(n_{in} + 1)
        \]
        
        (weights + biases)
    \end{defbox}
    
    \vspace{0.3cm}
    
    \textbf{Example: Network [4, 5, 4, 2]}
    
    \begin{columns}
        \begin{column}{0.5\textwidth}
            \begin{align*}
                \text{Layer 1: } & 5 \times 4 + 5 = 25 \\
                \text{Layer 2: } & 4 \times 5 + 4 = 24 \\
                \text{Layer 3: } & 2 \times 4 + 2 = 10 \\
                \hline
                \text{Total: } & 59 \text{ parameters}
            \end{align*}
        \end{column}
        \begin{column}{0.5\textwidth}
            \begin{funnybox}
                GPT-3 has 175 \textbf{billion} parameters.
                
                \vspace{0.1cm}
                
                Our example: 59.
                
                \vspace{0.1cm}
                
                \textit{We all start somewhere!}
            \end{funnybox}
        \end{column}
    \end{columns}
\end{frame}

% -----------------------------------------------------------------------------
% Key Takeaways
% -----------------------------------------------------------------------------
\begin{frame}{Key Takeaways: Network Architectures}
    \begin{keybox}
        \begin{enumerate}
            \item \textbf{MLP}: Input → Hidden layers → Output
            \item \textbf{Layer computation}: $\mathbf{a}^{[l]} = \sigma(W^{[l]} \mathbf{a}^{[l-1]} + \mathbf{b}^{[l]})$
            \item \textbf{Forward pass}: Sequentially apply all layers
            \item \textbf{Width} vs \textbf{Depth}: Different trade-offs
            \item \textbf{Hidden layers learn features} at increasing abstraction
            \item \textbf{Output layer design}:
            \begin{itemize}
                \item Binary: sigmoid (1 output)
                \item Multi-class: softmax ($K$ outputs)
                \item Regression: linear (1 output)
            \end{itemize}
            \item \textbf{Parameters}: Weights + Biases per layer
        \end{enumerate}
    \end{keybox}
    
    \vspace{0.2cm}
    
    \centering
    \textit{Next: How does information flow forward through the network?}
\end{frame}


% Section 11: Forward Propagation
% =============================================================================
% Section 11: Forward Propagation
% How information flows through the network
% =============================================================================

\section{Forward Propagation}

% -----------------------------------------------------------------------------
% Opening
% -----------------------------------------------------------------------------
\begin{frame}{Forward Propagation: Data's Journey}
    \begin{columns}[T]
        \begin{column}{0.5\textwidth}
            \begin{funnybox}
                \textit{``Forward propagation is like a game of telephone, except each person does math before passing the message.''}
            \end{funnybox}
            
            \vspace{0.3cm}
            
            \textbf{The Big Picture:}
            \begin{enumerate}
                \item Input enters the network
                \item Each layer transforms it
                \item Output emerges at the end
            \end{enumerate}
            
            \textit{Direction: Input → Output}
        \end{column}
        \begin{column}{0.5\textwidth}
            \centering
            \begin{tikzpicture}[scale=0.8]
                % Simplified flow
                \node[draw=neongreen,rounded corners,fill=darkgray,minimum width=1.5cm] (in) at (0,3) {Input};
                \node[draw=accentcyan,rounded corners,fill=darkgray,minimum width=1.5cm] (h1) at (0,1.5) {Hidden 1};
                \node[draw=accentcyan,rounded corners,fill=darkgray,minimum width=1.5cm] (h2) at (0,0) {Hidden 2};
                \node[draw=neonpink,rounded corners,fill=darkgray,minimum width=1.5cm] (out) at (0,-1.5) {Output};
                
                \draw[->,ultra thick,neonyellow] (in) -- (h1);
                \draw[->,ultra thick,neonyellow] (h1) -- (h2);
                \draw[->,ultra thick,neonyellow] (h2) -- (out);
                
                % Labels
                \node[right,font=\small,fgwhite] at (1,2.25) {$W^{[1]}, \mathbf{b}^{[1]}, \sigma$};
                \node[right,font=\small,fgwhite] at (1,0.75) {$W^{[2]}, \mathbf{b}^{[2]}, \sigma$};
                \node[right,font=\small,fgwhite] at (1,-0.75) {$W^{[3]}, \mathbf{b}^{[3]}, \sigma$};
            \end{tikzpicture}
        \end{column}
    \end{columns}
\end{frame}

% -----------------------------------------------------------------------------
% Detailed Example Setup
% -----------------------------------------------------------------------------
\begin{frame}{Concrete Example: A 2-Layer Network}
    \begin{defbox}[Network Configuration]
        Architecture: 3 → 2 → 1 (input → hidden → output)
        
        \begin{itemize}
            \item Input: $\mathbf{x} = (0.5, 0.8, -0.3)^T$
            \item Hidden layer: 2 neurons, ReLU activation
            \item Output layer: 1 neuron, sigmoid activation
        \end{itemize}
    \end{defbox}
    
    \vspace{0.2cm}
    
    \centering
    \begin{tikzpicture}[scale=1]
        % Input layer
        \node[input neuron] (x1) at (0,1.5) {$x_1$};
        \node[input neuron] (x2) at (0,0) {$x_2$};
        \node[input neuron] (x3) at (0,-1.5) {$x_3$};
        
        % Hidden layer
        \node[hidden neuron,minimum size=0.9cm] (h1) at (3,0.75) {$h_1$};
        \node[hidden neuron,minimum size=0.9cm] (h2) at (3,-0.75) {$h_2$};
        
        % Output
        \node[output neuron,minimum size=0.9cm] (y) at (6,0) {$\hat{y}$};
        
        % Connections with weights
        \draw[connection] (x1) -- (h1);
        \draw[connection] (x1) -- (h2);
        \draw[connection] (x2) -- (h1);
        \draw[connection] (x2) -- (h2);
        \draw[connection] (x3) -- (h1);
        \draw[connection] (x3) -- (h2);
        \draw[connection] (h1) -- (y);
        \draw[connection] (h2) -- (y);
        
        % Labels
        \node[neongreen,font=\small,below] at (0,-2) {Input (3)};
        \node[accentcyan,font=\small,below] at (3,-1.8) {Hidden (2)};
        \node[neonpink,font=\small,below] at (6,-0.8) {Output (1)};
    \end{tikzpicture}
\end{frame}

% -----------------------------------------------------------------------------
% Layer 1 Computation
% -----------------------------------------------------------------------------
\begin{frame}{Step 1: Hidden Layer Computation}
    \textbf{Given weights and biases:}
    \[
        W^{[1]} = \begin{pmatrix} 0.2 & 0.4 & -0.5 \\ -0.3 & 0.1 & 0.2 \end{pmatrix}, \quad
        \mathbf{b}^{[1]} = \begin{pmatrix} 0.1 \\ -0.2 \end{pmatrix}
    \]
    
    \vspace{0.2cm}
    
    \textbf{Compute pre-activation:}
    \begin{align*}
        \mathbf{z}^{[1]} &= W^{[1]} \mathbf{x} + \mathbf{b}^{[1]} \\
        &= \begin{pmatrix} 0.2 & 0.4 & -0.5 \\ -0.3 & 0.1 & 0.2 \end{pmatrix}
        \begin{pmatrix} 0.5 \\ 0.8 \\ -0.3 \end{pmatrix} + \begin{pmatrix} 0.1 \\ -0.2 \end{pmatrix} \\
        &= \begin{pmatrix} 0.1 + 0.32 + 0.15 + 0.1 \\ -0.15 + 0.08 - 0.06 - 0.2 \end{pmatrix}
        = \begin{pmatrix} 0.67 \\ -0.33 \end{pmatrix}
    \end{align*}
\end{frame}

% -----------------------------------------------------------------------------
% Layer 1 Activation
% -----------------------------------------------------------------------------
\begin{frame}{Step 1 (continued): Apply Activation}
    \textbf{Pre-activation:} $\mathbf{z}^{[1]} = (0.67, -0.33)^T$
    
    \vspace{0.3cm}
    
    \textbf{Apply ReLU:}
    \[
        \mathbf{a}^{[1]} = \text{ReLU}(\mathbf{z}^{[1]}) = \begin{pmatrix} \max(0, 0.67) \\ \max(0, -0.33) \end{pmatrix} = \begin{pmatrix} 0.67 \\ 0 \end{pmatrix}
    \]
    
    \vspace{0.3cm}
    
    \begin{columns}
        \begin{column}{0.5\textwidth}
            \begin{keybox}
                The negative value becomes 0!
                
                ReLU ``kills'' negative activations.
            \end{keybox}
        \end{column}
        \begin{column}{0.5\textwidth}
            \centering
            \begin{tikzpicture}[scale=0.7]
                \node[hidden neuron] (h1) at (0,1) {$0.67$};
                \node[hidden neuron,fill=darkgray!50] (h2) at (0,-1) {$0$};
                
                \node[right,neongreen,font=\small] at (0.8,1) {active};
                \node[right,neonpink,font=\small] at (0.8,-1) {``dead''};
            \end{tikzpicture}
        \end{column}
    \end{columns}
\end{frame}

% -----------------------------------------------------------------------------
% Layer 2 Computation
% -----------------------------------------------------------------------------
\begin{frame}{Step 2: Output Layer Computation}
    \textbf{Output layer weights and biases:}
    \[
        W^{[2]} = \begin{pmatrix} 0.6 & -0.4 \end{pmatrix}, \quad
        b^{[2]} = 0.3
    \]
    
    \vspace{0.2cm}
    
    \textbf{Compute pre-activation:}
    \begin{align*}
        z^{[2]} &= W^{[2]} \mathbf{a}^{[1]} + b^{[2]} \\
        &= \begin{pmatrix} 0.6 & -0.4 \end{pmatrix} \begin{pmatrix} 0.67 \\ 0 \end{pmatrix} + 0.3 \\
        &= 0.402 + 0 + 0.3 = 0.702
    \end{align*}
    
    \vspace{0.2cm}
    
    \textbf{Apply sigmoid:}
    \[
        \hat{y} = \sigma(z^{[2]}) = \frac{1}{1 + e^{-0.702}} \approx \boxed{0.669}
    \]
    
    \begin{keybox}
        Final prediction: 66.9\% probability for class 1
    \end{keybox}
\end{frame}

% -----------------------------------------------------------------------------
% Summary of Forward Pass
% -----------------------------------------------------------------------------
\begin{frame}{Forward Pass Summary}
    \centering
    \begin{tikzpicture}[scale=0.85]
        % Input
        \node[draw=neongreen,rounded corners,fill=darkgray,minimum height=1cm] (x) at (0,0) {$\mathbf{x} = \begin{pmatrix} 0.5 \\ 0.8 \\ -0.3 \end{pmatrix}$};
        
        % z1
        \node[draw=accentcyan,rounded corners,fill=darkgray,minimum height=1cm] (z1) at (3,0) {$\mathbf{z}^{[1]} = \begin{pmatrix} 0.67 \\ -0.33 \end{pmatrix}$};
        
        % a1
        \node[draw=accentcyan,rounded corners,fill=darkgray,minimum height=1cm] (a1) at (6,0) {$\mathbf{a}^{[1]} = \begin{pmatrix} 0.67 \\ 0 \end{pmatrix}$};
        
        % z2
        \node[draw=neonpink,rounded corners,fill=darkgray,minimum height=0.8cm] (z2) at (9,0) {$z^{[2]} = 0.702$};
        
        % y
        \node[draw=neonyellow,rounded corners,fill=darkgray,minimum height=0.8cm] (y) at (12,0) {$\hat{y} = 0.669$};
        
        % Arrows
        \draw[->,thick,fgwhite] (x) -- (z1) node[midway,above,font=\tiny] {$W^{[1]}\mathbf{x} + \mathbf{b}^{[1]}$};
        \draw[->,thick,fgwhite] (z1) -- (a1) node[midway,above,font=\tiny] {ReLU};
        \draw[->,thick,fgwhite] (a1) -- (z2) node[midway,above,font=\tiny] {$W^{[2]}\mathbf{a}^{[1]} + b^{[2]}$};
        \draw[->,thick,fgwhite] (z2) -- (y) node[midway,above,font=\tiny] {sigmoid};
    \end{tikzpicture}
    
    \vspace{0.5cm}
    
    \begin{funnybox}
        From 3 numbers in to 1 number out. That's forward propagation!
        
        \textit{(All the matrix multiplication happens behind the scenes)}
    \end{funnybox}
\end{frame}

% -----------------------------------------------------------------------------
% Batch Processing
% -----------------------------------------------------------------------------
\begin{frame}{Batching: Multiple Samples at Once}
    \begin{defbox}[Batch Forward Pass]
        Instead of one input $\mathbf{x}$, process $m$ samples simultaneously:
        \[
            X = \begin{pmatrix} | & | & & | \\ \mathbf{x}^{(1)} & \mathbf{x}^{(2)} & \cdots & \mathbf{x}^{(m)} \\ | & | & & | \end{pmatrix} \in \mathbb{R}^{n \times m}
        \]
    \end{defbox}
    
    \vspace{0.2cm}
    
    \textbf{Layer computation (vectorized):}
    \[
        Z^{[l]} = W^{[l]} A^{[l-1]} + \mathbf{b}^{[l]}, \quad A^{[l]} = \sigma(Z^{[l]})
    \]
    
    \vspace{0.2cm}
    
    \begin{columns}
        \begin{column}{0.5\textwidth}
            \begin{keybox}[Why batch?]
                \begin{itemize}
                    \item GPU parallelization
                    \item Faster training
                    \item Better gradient estimates
                \end{itemize}
            \end{keybox}
        \end{column}
        \begin{column}{0.5\textwidth}
            \begin{infobox}[Typical batch sizes]
                32, 64, 128, 256...
                
                (powers of 2 for efficiency)
            \end{infobox}
        \end{column}
    \end{columns}
\end{frame}

% -----------------------------------------------------------------------------
% Computational Graph
% -----------------------------------------------------------------------------
\begin{frame}{Computational Graph View}
    \begin{defbox}[Computational Graph]
        A \glow{computational graph} represents the network as a directed graph where:
        \begin{itemize}
            \item Nodes = operations or variables
            \item Edges = data flow
        \end{itemize}
    \end{defbox}
    
    \centering
    \begin{tikzpicture}[scale=0.75]
        % Variables
        \node[draw=neongreen,circle,fill=darkgray] (x) at (0,0) {$\mathbf{x}$};
        \node[draw=accentcyan,circle,fill=darkgray] (W1) at (0,2) {$W^{[1]}$};
        \node[draw=accentcyan,circle,fill=darkgray] (b1) at (2,2) {$\mathbf{b}^{[1]}$};
        
        % Operations
        \node[draw=neonyellow,rectangle,fill=darkgray] (matmul1) at (2,0) {$\times$};
        \node[draw=neonyellow,rectangle,fill=darkgray] (add1) at (4,0) {$+$};
        \node[draw=neonyellow,rectangle,fill=darkgray] (relu) at (6,0) {ReLU};
        
        \node[draw=accentcyan,circle,fill=darkgray] (W2) at (6,2) {$W^{[2]}$};
        \node[draw=accentcyan,circle,fill=darkgray] (b2) at (8,2) {$b^{[2]}$};
        
        \node[draw=neonyellow,rectangle,fill=darkgray] (matmul2) at (8,0) {$\times$};
        \node[draw=neonyellow,rectangle,fill=darkgray] (add2) at (10,0) {$+$};
        \node[draw=neonyellow,rectangle,fill=darkgray] (sig) at (12,0) {$\sigma$};
        
        \node[draw=neonpink,circle,fill=darkgray] (y) at (14,0) {$\hat{y}$};
        
        % Edges
        \draw[->,thick] (x) -- (matmul1);
        \draw[->,thick] (W1) -- (matmul1);
        \draw[->,thick] (matmul1) -- (add1);
        \draw[->,thick] (b1) -- (add1);
        \draw[->,thick] (add1) -- (relu);
        \draw[->,thick] (relu) -- (matmul2);
        \draw[->,thick] (W2) -- (matmul2);
        \draw[->,thick] (matmul2) -- (add2);
        \draw[->,thick] (b2) -- (add2);
        \draw[->,thick] (add2) -- (sig);
        \draw[->,thick] (sig) -- (y);
    \end{tikzpicture}
    
    \vspace{0.3cm}
    
    \begin{keybox}
        PyTorch and TensorFlow build these graphs automatically!
        
        Used for automatic differentiation (backprop).
    \end{keybox}
\end{frame}

% -----------------------------------------------------------------------------
% Code Preview
% -----------------------------------------------------------------------------
\begin{frame}{Forward Pass in PyTorch}
    \begin{successbox}[PyTorch Implementation]
        {\small\ttfamily
        import torch\\
        import torch.nn as nn\\[0.5em]
        class SimpleNet(nn.Module):\\
        \quad def \_\_init\_\_(self):\\
        \quad\quad super().\_\_init\_\_()\\
        \quad\quad self.hidden = nn.Linear(3, 2)\\
        \quad\quad self.output = nn.Linear(2, 1)\\
        \quad\quad self.relu = nn.ReLU()\\
        \quad\quad self.sigmoid = nn.Sigmoid()\\[0.5em]
        \quad def forward(self, x):\\
        \quad\quad x = self.relu(self.hidden(x))\\
        \quad\quad x = self.sigmoid(self.output(x))\\
        \quad\quad return x\\[0.5em]
        model = SimpleNet()\\
        x = torch.tensor([0.5, 0.8, -0.3])\\
        y\_pred = model(x)
        }
    \end{successbox}
\end{frame}

% -----------------------------------------------------------------------------
% Key Takeaways
% -----------------------------------------------------------------------------
\begin{frame}{Key Takeaways: Forward Propagation}
    \begin{keybox}
        \begin{enumerate}
            \item \textbf{Forward propagation} = data flows input → output
            \item Each layer: $\mathbf{a}^{[l]} = \sigma(W^{[l]} \mathbf{a}^{[l-1]} + \mathbf{b}^{[l]})$
            \item Steps per layer:
            \begin{itemize}
                \item Linear: $\mathbf{z} = W\mathbf{a} + \mathbf{b}$
                \item Activation: $\mathbf{a} = \sigma(\mathbf{z})$
            \end{itemize}
            \item \textbf{Batching}: Process multiple samples simultaneously
            \item \textbf{Computational graph}: Framework builds automatically
            \item Forward pass produces \textbf{prediction} $\hat{y}$
        \end{enumerate}
    \end{keybox}
    
    \vspace{0.2cm}
    
    \centering
    \textit{Next: How do we measure if the prediction is any good? Loss functions!}
\end{frame}


% Section 12: Loss Functions
% =============================================================================
% Section 12: Loss Functions
% Measuring how wrong we are
% =============================================================================

\section{Loss Functions}

% -----------------------------------------------------------------------------
% Opening
% -----------------------------------------------------------------------------
\begin{frame}{Loss Functions: Measuring Mistakes}
    \begin{columns}[T]
        \begin{column}{0.5\textwidth}
            \begin{funnybox}
                \textit{``The loss function is like a brutally honest friend. It tells you exactly how wrong you are — numerically.''}
            \end{funnybox}
            
            \vspace{0.3cm}
            
            \textbf{The Goal:}
            \begin{itemize}
                \item Prediction: $\hat{y}$
                \item True value: $y$
                \item Loss: How far off is $\hat{y}$ from $y$?
            \end{itemize}
        \end{column}
        \begin{column}{0.5\textwidth}
            \centering
            \begin{tikzpicture}[scale=0.8]
                % Predicted vs actual
                \draw[->] (0,0) -- (4,0) node[right] {$x$};
                \draw[->] (0,0) -- (0,3) node[above] {$y$};
                
                % Target
                \fill[neongreen] (2,2) circle (4pt);
                \node[neongreen,right,font=\small] at (2.2,2) {$y$ (true)};
                
                % Prediction
                \fill[neonpink] (2,0.8) circle (4pt);
                \node[neonpink,right,font=\small] at (2.2,0.8) {$\hat{y}$ (pred)};
                
                % Error
                \draw[<->,thick,neonyellow] (1.8,0.8) -- (1.8,2);
                \node[neonyellow,left,font=\small] at (1.7,1.4) {error};
            \end{tikzpicture}
        \end{column}
    \end{columns}
\end{frame}

% -----------------------------------------------------------------------------
% Loss vs Cost
% -----------------------------------------------------------------------------
\begin{frame}{Loss vs Cost}
    \begin{columns}[T]
        \begin{column}{0.5\textwidth}
            \begin{defbox}[Loss Function]
                \glow{Loss} $\mathcal{L}(\hat{y}, y)$ measures error for a \textbf{single sample}.
                
                \textit{How wrong is this one prediction?}
            \end{defbox}
        \end{column}
        \begin{column}{0.5\textwidth}
            \begin{defbox}[Cost Function]
                \glow{Cost} $J$ is the \textbf{average loss} over all training samples:
                
                \[
                    J = \frac{1}{N} \sum_{i=1}^{N} \mathcal{L}(\hat{y}^{(i)}, y^{(i)})
                \]
            \end{defbox}
        \end{column}
    \end{columns}
    
    \vspace{0.5cm}
    
    \begin{keybox}
        \textbf{Training goal}: Minimize the cost function $J$
        
        Lower cost = better predictions (on average)
    \end{keybox}
    
    \begin{funnybox}
        Loss is personal. Cost is the team average.
    \end{funnybox}
\end{frame}

% -----------------------------------------------------------------------------
% MSE
% -----------------------------------------------------------------------------
\begin{frame}{Mean Squared Error (MSE)}
    \begin{defbox}[MSE for Regression]
        \[
            \mathcal{L}_{\text{MSE}}(\hat{y}, y) = (\hat{y} - y)^2
        \]
        
        \[
            J_{\text{MSE}} = \frac{1}{N} \sum_{i=1}^{N} (\hat{y}^{(i)} - y^{(i)})^2
        \]
    \end{defbox}
    
    \vspace{0.3cm}
    
    \begin{columns}
        \begin{column}{0.5\textwidth}
            \textbf{Properties:}
            \begin{itemize}
                \item Always $\geq 0$
                \item Zero when $\hat{y} = y$
                \item Penalizes large errors MORE (squared!)
                \item Smooth, differentiable
            \end{itemize}
        \end{column}
        \begin{column}{0.5\textwidth}
            \centering
            \begin{tikzpicture}[scale=0.6]
                \draw[->] (-2,0) -- (2,0) node[right] {$\hat{y} - y$};
                \draw[->] (0,-0.3) -- (0,3) node[above] {Loss};
                
                \draw[thick,neonpink,domain=-1.5:1.5,samples=50] plot (\x, {\x*\x});
                
                \fill[neongreen] (0,0) circle (3pt);
                \node[below,neongreen,font=\small] at (0,-0.3) {optimal};
            \end{tikzpicture}
            
            \textit{Parabola centered at 0}
        \end{column}
    \end{columns}
\end{frame}

% -----------------------------------------------------------------------------
% MAE
% -----------------------------------------------------------------------------
\begin{frame}{Mean Absolute Error (MAE)}
    \begin{defbox}[MAE for Regression]
        \[
            \mathcal{L}_{\text{MAE}}(\hat{y}, y) = |\hat{y} - y|
        \]
        
        \[
            J_{\text{MAE}} = \frac{1}{N} \sum_{i=1}^{N} |\hat{y}^{(i)} - y^{(i)}|
        \]
    \end{defbox}
    
    \vspace{0.3cm}
    
    \begin{columns}
        \begin{column}{0.5\textwidth}
            \textbf{MSE vs MAE:}
            
            \begin{tabular}{lcc}
                & MSE & MAE \\
                \hline
                Outliers & Sensitive & Robust \\
                Gradient & Smooth & Discontinuous at 0 \\
                Use & Default & Noisy data
            \end{tabular}
        \end{column}
        \begin{column}{0.5\textwidth}
            \centering
            \begin{tikzpicture}[scale=0.6]
                \draw[->] (-2,0) -- (2,0) node[right] {$\hat{y} - y$};
                \draw[->] (0,-0.3) -- (0,2.5) node[above] {Loss};
                
                % MAE
                \draw[thick,neongreen] (-1.5,1.5) -- (0,0) -- (1.5,1.5);
                
                \fill[neonyellow] (0,0) circle (3pt);
            \end{tikzpicture}
            
            \textit{V-shaped (not smooth at 0)}
        \end{column}
    \end{columns}
\end{frame}

% -----------------------------------------------------------------------------
% Binary Cross-Entropy
% -----------------------------------------------------------------------------
\begin{frame}{Binary Cross-Entropy (BCE)}
    \begin{defbox}[BCE for Binary Classification]
        For $y \in \{0, 1\}$ and $\hat{y} \in (0, 1)$ (probability):
        \[
            \mathcal{L}_{\text{BCE}}(\hat{y}, y) = -\left[y \log(\hat{y}) + (1-y) \log(1-\hat{y})\right]
        \]
    \end{defbox}
    
    \vspace{0.2cm}
    
    \textbf{Intuition:}
    \begin{itemize}
        \item If $y = 1$: Loss $= -\log(\hat{y})$ → want $\hat{y} \to 1$
        \item If $y = 0$: Loss $= -\log(1-\hat{y})$ → want $\hat{y} \to 0$
    \end{itemize}
    
    \centering
    \begin{tikzpicture}[scale=0.6]
        \draw[->] (0,0) -- (4,0) node[right] {$\hat{y}$};
        \draw[->] (0,0) -- (0,3.5) node[above] {Loss};
        
        % -log(y) for y=1
        \draw[thick,neonpink,domain=0.1:3.5,samples=50] plot (\x, {-ln(\x/3.5)});
        \node[neonpink,font=\small] at (2.5,2.5) {$y=1$};
        
        % -log(1-y) for y=0
        \draw[thick,neongreen,domain=0.1:3.4,samples=50] plot (\x, {-ln(1-\x/3.5)});
        \node[neongreen,font=\small] at (1,2.5) {$y=0$};
        
        \node[below,font=\small] at (0,0) {0};
        \node[below,font=\small] at (3.5,0) {1};
    \end{tikzpicture}
\end{frame}

% -----------------------------------------------------------------------------
% Why Cross-Entropy?
% -----------------------------------------------------------------------------
\begin{frame}{Why Cross-Entropy for Classification?}
    \begin{alertbox}[The Problem with MSE for Classification]
        Using MSE with sigmoid output:
        \begin{itemize}
            \item Gradient $\to 0$ when sigmoid saturates
            \item Very slow learning when $\hat{y} \approx 0$ or $\hat{y} \approx 1$
            \item ``Vanishing gradient'' problem
        \end{itemize}
    \end{alertbox}
    
    \vspace{0.3cm}
    
    \begin{successbox}[Cross-Entropy Solves This]
        The $\log$ in BCE cancels the $\exp$ in sigmoid!
        
        \[
            \frac{\partial \mathcal{L}_{\text{BCE}}}{\partial z} = \hat{y} - y
        \]
        
        Gradient is \textbf{linear} in error — no vanishing!
    \end{successbox}
    
    \begin{keybox}
        \textbf{Rule of thumb}: Use cross-entropy for classification, MSE for regression.
    \end{keybox}
\end{frame}

% -----------------------------------------------------------------------------
% Categorical Cross-Entropy
% -----------------------------------------------------------------------------
\begin{frame}{Categorical Cross-Entropy (Multi-class)}
    \begin{defbox}[Categorical Cross-Entropy]
        For $K$ classes with one-hot encoded $\mathbf{y}$ and softmax output $\hat{\mathbf{y}}$:
        \[
            \mathcal{L}_{\text{CE}}(\hat{\mathbf{y}}, \mathbf{y}) = -\sum_{k=1}^{K} y_k \log(\hat{y}_k)
        \]
        
        Since $\mathbf{y}$ is one-hot (only one $y_k = 1$):
        \[
            \mathcal{L}_{\text{CE}} = -\log(\hat{y}_c) \quad \text{where } c \text{ is the true class}
        \]
    \end{defbox}
    
    \vspace{0.2cm}
    
    \textbf{Example:} True class = 2 (out of 3)
    \begin{align*}
        \mathbf{y} &= (0, 1, 0) \\
        \hat{\mathbf{y}} &= (0.1, 0.7, 0.2) \\
        \mathcal{L} &= -\log(0.7) \approx 0.357
    \end{align*}
\end{frame}

% -----------------------------------------------------------------------------
% Summary Table
% -----------------------------------------------------------------------------
\begin{frame}{Loss Function Summary}
    \begin{tabular}{llll}
        \textbf{Task} & \textbf{Loss} & \textbf{Output Act.} & \textbf{Formula} \\
        \hline
        Regression & MSE & None/Linear & $(\hat{y} - y)^2$ \\[0.3em]
        Regression & MAE & None/Linear & $|\hat{y} - y|$ \\[0.3em]
        Binary Class. & BCE & Sigmoid & $-y\log\hat{y} - (1-y)\log(1-\hat{y})$ \\[0.3em]
        Multi-class & CE & Softmax & $-\sum_k y_k \log \hat{y}_k$ \\
    \end{tabular}
    
    \vspace{0.5cm}
    
    \begin{funnybox}
        Choosing the right loss function is like choosing the right tool:
        
        \begin{itemize}
            \item Hammer (MSE) for nails (regression)
            \item Screwdriver (CE) for screws (classification)
        \end{itemize}
    \end{funnybox}
\end{frame}

% -----------------------------------------------------------------------------
% Loss Landscape
% -----------------------------------------------------------------------------
\begin{frame}{The Loss Landscape}
    \begin{columns}[T]
        \begin{column}{0.5\textwidth}
            \begin{infobox}[Visualization]
                The cost function $J(\mathbf{W})$ defines a ``landscape'' over parameter space.
                
                \begin{itemize}
                    \item Valleys = good parameters
                    \item Peaks = bad parameters
                    \item Training = finding valleys
                \end{itemize}
            \end{infobox}
        \end{column}
        \begin{column}{0.5\textwidth}
            \centering
            \begin{tikzpicture}[scale=0.6]
                % 2D loss landscape
                \draw[->] (-2,0) -- (3,0) node[right] {$w_1$};
                \draw[->] (0,-0.5) -- (0,3) node[above] {$J$};
                
                % Wavy landscape
                \draw[thick,accentcyan,domain=-1.5:2.5,samples=50] 
                    plot (\x, {1.5 + 0.5*sin(3*\x r) + 0.3*\x*\x - 0.3*\x});
                
                % Local min
                \fill[neonyellow] (-0.5,1.1) circle (2pt);
                \node[neonyellow,below,font=\tiny] at (-0.5,1.0) {local};
                
                % Global min
                \fill[neongreen] (1.5,1.2) circle (2pt);
                \node[neongreen,below,font=\tiny] at (1.5,1.1) {global};
            \end{tikzpicture}
            
            \textit{Non-convex landscape}
        \end{column}
    \end{columns}
    
    \vspace{0.3cm}
    
    \begin{keybox}
        Neural network loss landscapes are:
        \begin{itemize}
            \item High-dimensional (millions of parameters!)
            \item Non-convex (multiple minima)
            \item Surprisingly well-behaved (saddle points, not local minima)
        \end{itemize}
    \end{keybox}
\end{frame}

% -----------------------------------------------------------------------------
% Key Takeaways
% -----------------------------------------------------------------------------
\begin{frame}{Key Takeaways: Loss Functions}
    \begin{keybox}
        \begin{enumerate}
            \item \textbf{Loss} = error for one sample, \textbf{Cost} = average over dataset
            \item \textbf{MSE}: $(\hat{y}-y)^2$ — regression, penalizes outliers
            \item \textbf{MAE}: $|\hat{y}-y|$ — regression, robust to outliers
            \item \textbf{Binary Cross-Entropy}: $-y\log\hat{y} - (1-y)\log(1-\hat{y})$
            \begin{itemize}
                \item Use with sigmoid output for binary classification
            \end{itemize}
            \item \textbf{Categorical Cross-Entropy}: $-\sum y_k \log \hat{y}_k$
            \begin{itemize}
                \item Use with softmax output for multi-class
            \end{itemize}
            \item Match loss function to task and output activation!
        \end{enumerate}
    \end{keybox}
    
    \vspace{0.2cm}
    
    \centering
    \textit{Next: How do we minimize the loss? Backpropagation!}
\end{frame}


% Section 13: Backpropagation
% =============================================================================
% Section 13: Backpropagation
% Computing gradients through the chain rule
% =============================================================================

\section{Backpropagation}

% -----------------------------------------------------------------------------
% Opening
% -----------------------------------------------------------------------------
\begin{frame}{Backpropagation: Learning from Mistakes}
    \begin{columns}[T]
        \begin{column}{0.5\textwidth}
            \begin{funnybox}
                \textit{``Backprop is like tracing your steps back through a maze, but instead of finding where you went wrong, you're finding WHO is responsible for the mistake.''}
            \end{funnybox}
            
            \vspace{0.3cm}
            
            \textbf{The Key Insight:}
            \begin{itemize}
                \item Forward: Compute predictions
                \item Backward: Compute gradients
                \item Chain rule does ALL the work!
            \end{itemize}
        \end{column}
        \begin{column}{0.5\textwidth}
            \centering
            \begin{tikzpicture}[scale=0.7]
                % Forward arrow
                \draw[->,ultra thick,neongreen] (0,2) -- (4,2);
                \node[neongreen,above] at (2,2) {Forward};
                
                % Network
                \node[circle,draw=accentcyan,fill=darkgray] (x) at (0,0) {$x$};
                \node[circle,draw=accentcyan,fill=darkgray] (h) at (2,0) {$h$};
                \node[circle,draw=accentcyan,fill=darkgray] (y) at (4,0) {$\hat{y}$};
                \node[circle,draw=neonpink,fill=darkgray] (L) at (6,0) {$\mathcal{L}$};
                
                \draw[->,thick] (x) -- (h);
                \draw[->,thick] (h) -- (y);
                \draw[->,thick] (y) -- (L);
                
                % Backward arrow
                \draw[<-,ultra thick,neonpink] (0,-1.5) -- (4,-1.5);
                \node[neonpink,below] at (2,-1.5) {Backward (gradients)};
            \end{tikzpicture}
        \end{column}
    \end{columns}
\end{frame}

% -----------------------------------------------------------------------------
% Gradient Descent Review
% -----------------------------------------------------------------------------
\begin{frame}{Gradient Descent: The Big Picture}
    \begin{defbox}[Gradient Descent Update]
        To minimize $J(\mathbf{W})$, update parameters:
        \[
            W_{new} = W_{old} - \eta \frac{\partial J}{\partial W}
        \]
        
        where $\eta$ is the \glow{learning rate}.
    \end{defbox}
    
    \vspace{0.3cm}
    
    \begin{columns}
        \begin{column}{0.5\textwidth}
            \begin{keybox}[We Need Gradients!]
                \[
                    \frac{\partial J}{\partial W^{[l]}}, \quad \frac{\partial J}{\partial \mathbf{b}^{[l]}}
                \]
                
                For EVERY parameter in EVERY layer.
                
                How? \textbf{Backpropagation!}
            \end{keybox}
        \end{column}
        \begin{column}{0.5\textwidth}
            \centering
            \begin{tikzpicture}[scale=0.6]
                \draw[->] (0,0) -- (4,0) node[right] {$w$};
                \draw[->] (0,0) -- (0,3) node[above] {$J$};
                
                \draw[thick,accentcyan,domain=0.5:3.5,samples=50] 
                    plot (\x, {0.5 + 1.2*(\x-2)*(\x-2)});
                
                % Current point with gradient
                \fill[neonpink] (0.8,1.93) circle (3pt);
                \draw[->,thick,neonyellow] (0.8,1.93) -- (1.5,1.2);
                \node[neonpink,above,font=\small] at (0.8,2.1) {current};
                \node[neonyellow,right,font=\small] at (1.4,1.4) {$-\eta \nabla J$};
                
                \fill[neongreen] (2,0.5) circle (3pt);
                \node[neongreen,below,font=\small] at (2,0.3) {goal};
            \end{tikzpicture}
        \end{column}
    \end{columns}
\end{frame}

% -----------------------------------------------------------------------------
% Chain Rule
% -----------------------------------------------------------------------------
\begin{frame}{The Chain Rule: Heart of Backprop}
    \begin{thmbox}[Chain Rule]
        If $y = f(g(x))$, then:
        \[
            \frac{dy}{dx} = \frac{dy}{dg} \cdot \frac{dg}{dx}
        \]
        
        \textit{Multiply the derivatives along the path!}
    \end{thmbox}
    
    \vspace{0.3cm}
    
    \textbf{Example:} $y = (2x + 1)^2$
    
    Let $g = 2x + 1$, so $y = g^2$
    
    \begin{align*}
        \frac{dy}{dg} &= 2g = 2(2x+1) \\
        \frac{dg}{dx} &= 2 \\
        \frac{dy}{dx} &= 2(2x+1) \cdot 2 = 4(2x+1)
    \end{align*}
    
    \begin{funnybox}
        Chain rule = ``blame propagation.'' Each step shares responsibility for the final error!
    \end{funnybox}
\end{frame}

% -----------------------------------------------------------------------------
% Multivariate Chain Rule
% -----------------------------------------------------------------------------
\begin{frame}{Multivariate Chain Rule}
    \begin{thmbox}[Multivariate Chain Rule]
        If $L = L(y)$ and $y = y(x_1, x_2, \ldots, x_n)$:
        \[
            \frac{\partial L}{\partial x_i} = \frac{\partial L}{\partial y} \cdot \frac{\partial y}{\partial x_i}
        \]
        
        If there are multiple paths from $x$ to $L$:
        \[
            \frac{\partial L}{\partial x} = \sum_{\text{all paths}} \frac{\partial L}{\partial y_j} \cdot \frac{\partial y_j}{\partial x}
        \]
    \end{thmbox}
    
    \centering
    \begin{tikzpicture}[scale=0.8]
        \node[circle,draw=neongreen,fill=darkgray] (x) at (0,0) {$x$};
        \node[circle,draw=accentcyan,fill=darkgray] (y1) at (2,0.8) {$y_1$};
        \node[circle,draw=accentcyan,fill=darkgray] (y2) at (2,-0.8) {$y_2$};
        \node[circle,draw=neonpink,fill=darkgray] (L) at (4,0) {$L$};
        
        \draw[->,thick] (x) -- (y1) node[midway,above,font=\small] {$\frac{\partial y_1}{\partial x}$};
        \draw[->,thick] (x) -- (y2) node[midway,below,font=\small] {$\frac{\partial y_2}{\partial x}$};
        \draw[->,thick] (y1) -- (L) node[midway,above,font=\small] {$\frac{\partial L}{\partial y_1}$};
        \draw[->,thick] (y2) -- (L) node[midway,below,font=\small] {$\frac{\partial L}{\partial y_2}$};
    \end{tikzpicture}
    
    $\frac{\partial L}{\partial x} = \frac{\partial L}{\partial y_1} \frac{\partial y_1}{\partial x} + \frac{\partial L}{\partial y_2} \frac{\partial y_2}{\partial x}$
\end{frame}

% -----------------------------------------------------------------------------
% Backprop Example Setup
% -----------------------------------------------------------------------------
\begin{frame}{Backprop Example: Setup}
    \textbf{Simple network:} $x \to$ hidden $\to$ output $\to$ loss
    
    \vspace{0.2cm}
    
    \begin{columns}
        \begin{column}{0.5\textwidth}
            \textbf{Forward equations:}
            \begin{align*}
                z &= wx + b \\
                a &= \sigma(z) = \frac{1}{1+e^{-z}} \\
                \mathcal{L} &= (a - y)^2
            \end{align*}
        \end{column}
        \begin{column}{0.5\textwidth}
            \centering
            \begin{tikzpicture}[scale=0.7]
                \node[circle,draw=neongreen,fill=darkgray] (x) at (0,0) {$x$};
                \node[circle,draw=accentcyan,fill=darkgray] (z) at (2,0) {$z$};
                \node[circle,draw=accentcyan,fill=darkgray] (a) at (4,0) {$a$};
                \node[circle,draw=neonpink,fill=darkgray] (L) at (6,0) {$\mathcal{L}$};
                
                \draw[->,thick] (x) -- (z) node[midway,above,font=\small] {$w,b$};
                \draw[->,thick] (z) -- (a) node[midway,above,font=\small] {$\sigma$};
                \draw[->,thick] (a) -- (L) node[midway,above,font=\small] {MSE};
            \end{tikzpicture}
        \end{column}
    \end{columns}
    
    \vspace{0.3cm}
    
    \begin{keybox}[Goal]
        Compute: $\frac{\partial \mathcal{L}}{\partial w}$ and $\frac{\partial \mathcal{L}}{\partial b}$
        
        Then update: $w \leftarrow w - \eta \frac{\partial \mathcal{L}}{\partial w}$
    \end{keybox}
\end{frame}

% -----------------------------------------------------------------------------
% Backprop Example Step 1
% -----------------------------------------------------------------------------
\begin{frame}{Backprop Step 1: Output Layer}
    \textbf{Start at the end:} $\mathcal{L} = (a - y)^2$
    
    \vspace{0.3cm}
    
    \[
        \frac{\partial \mathcal{L}}{\partial a} = 2(a - y)
    \]
    
    \vspace{0.3cm}
    
    \centering
    \begin{tikzpicture}[scale=0.8]
        \node[circle,draw=accentcyan,fill=darkgray] (a) at (0,0) {$a$};
        \node[circle,draw=neonpink,fill=darkgray] (L) at (3,0) {$\mathcal{L}$};
        
        \draw[->,thick,neongreen] (a) -- (L);
        \draw[<-,thick,neonyellow,dashed] (a) to[bend left=30] node[above,font=\small] {$\frac{\partial \mathcal{L}}{\partial a} = 2(a-y)$} (L);
    \end{tikzpicture}
    
    \vspace{0.3cm}
    
    \begin{infobox}
        This gradient tells us: ``How much does $\mathcal{L}$ change if we nudge $a$?''
        
        If $a > y$: positive gradient (decrease $a$!)\\
        If $a < y$: negative gradient (increase $a$!)
    \end{infobox}
\end{frame}

% -----------------------------------------------------------------------------
% Backprop Example Step 2
% -----------------------------------------------------------------------------
\begin{frame}{Backprop Step 2: Through Activation}
    \textbf{Chain through sigmoid:} $a = \sigma(z)$
    
    \vspace{0.2cm}
    
    Sigmoid derivative: $\frac{da}{dz} = \sigma(z)(1 - \sigma(z)) = a(1-a)$
    
    \vspace{0.2cm}
    
    \[
        \frac{\partial \mathcal{L}}{\partial z} = \frac{\partial \mathcal{L}}{\partial a} \cdot \frac{\partial a}{\partial z} = 2(a-y) \cdot a(1-a)
    \]
    
    \vspace{0.2cm}
    
    \centering
    \begin{tikzpicture}[scale=0.8]
        \node[circle,draw=accentcyan,fill=darkgray] (z) at (0,0) {$z$};
        \node[circle,draw=accentcyan,fill=darkgray] (a) at (3,0) {$a$};
        \node[circle,draw=neonpink,fill=darkgray] (L) at (6,0) {$\mathcal{L}$};
        
        \draw[->,thick] (z) -- (a) node[midway,above,font=\small] {$\sigma$};
        \draw[->,thick] (a) -- (L);
        
        \draw[<-,thick,neonyellow,dashed] (z) to[bend left=30] (a);
        \node[above,neonyellow,font=\small] at (1.5,0.8) {$\times a(1-a)$};
    \end{tikzpicture}
    
    \begin{keybox}
        Multiply by local gradient $a(1-a)$ at each step!
    \end{keybox}
\end{frame}

% -----------------------------------------------------------------------------
% Backprop Example Step 3
% -----------------------------------------------------------------------------
\begin{frame}{Backprop Step 3: To Parameters}
    \textbf{Final step:} $z = wx + b$
    
    \vspace{0.3cm}
    
    \begin{align*}
        \frac{\partial z}{\partial w} &= x \\[0.5em]
        \frac{\partial z}{\partial b} &= 1
    \end{align*}
    
    \vspace{0.3cm}
    
    \textbf{Full gradients:}
    \[
        \boxed{\frac{\partial \mathcal{L}}{\partial w} = \frac{\partial \mathcal{L}}{\partial z} \cdot x = 2(a-y) \cdot a(1-a) \cdot x}
    \]
    
    \[
        \boxed{\frac{\partial \mathcal{L}}{\partial b} = \frac{\partial \mathcal{L}}{\partial z} \cdot 1 = 2(a-y) \cdot a(1-a)}
    \]
    
    \begin{successbox}
        Done! We can now update $w$ and $b$ using gradient descent.
    \end{successbox}
\end{frame}

% -----------------------------------------------------------------------------
% General Backprop Algorithm
% -----------------------------------------------------------------------------
\begin{frame}{General Backprop Algorithm}
    \begin{keybox}[Backpropagation for $L$ Layers]
        \textbf{Initialize:} $\delta^{[L]} = \nabla_a \mathcal{L} \odot \sigma'(\mathbf{z}^{[L]})$
        
        \textbf{For} $l = L, L-1, \ldots, 1$:
        \begin{align*}
            \frac{\partial \mathcal{L}}{\partial W^{[l]}} &= \delta^{[l]} (\mathbf{a}^{[l-1]})^T \\[0.5em]
            \frac{\partial \mathcal{L}}{\partial \mathbf{b}^{[l]}} &= \delta^{[l]} \\[0.5em]
            \delta^{[l-1]} &= (W^{[l]})^T \delta^{[l]} \odot \sigma'(\mathbf{z}^{[l-1]})
        \end{align*}
    \end{keybox}
    
    \vspace{0.2cm}
    
    \begin{funnybox}
        $\delta^{[l]}$ = ``error signal'' at layer $l$
        
        It flows backward, getting transformed at each layer!
    \end{funnybox}
\end{frame}

% -----------------------------------------------------------------------------
% Computational Efficiency
% -----------------------------------------------------------------------------
\begin{frame}{Why Backprop is Efficient}
    \begin{columns}[T]
        \begin{column}{0.5\textwidth}
            \begin{alertbox}[Naive Approach]
                Compute $\frac{\partial \mathcal{L}}{\partial w}$ separately for each weight.
                
                \textbf{Cost:} $O(W \cdot N)$ forward passes
                
                For 1M weights: 1 million forward passes!
            \end{alertbox}
        \end{column}
        \begin{column}{0.5\textwidth}
            \begin{successbox}[Backprop]
                One forward + one backward pass.
                
                \textbf{Cost:} $O(2)$ passes total
                
                Same cost regardless of \# weights!
            \end{successbox}
        \end{column}
    \end{columns}
    
    \vspace{0.5cm}
    
    \begin{keybox}
        Backprop reuses intermediate computations!
        
        The ``error'' $\delta^{[l]}$ computed once serves all weights in that layer.
    \end{keybox}
\end{frame}

% -----------------------------------------------------------------------------
% Automatic Differentiation
% -----------------------------------------------------------------------------
\begin{frame}{Automatic Differentiation in Practice}
    \begin{infobox}[PyTorch Does It For You!]
        PyTorch (and other frameworks) compute gradients automatically.
    \end{infobox}
    
    \vspace{0.2cm}
    
    {\small\ttfamily
    import torch\\[0.3em]
    x = torch.tensor([0.5, 0.8, -0.3], requires\_grad=True)\\
    w = torch.tensor([0.2, 0.4, -0.5], requires\_grad=True)\\
    b = torch.tensor(0.1, requires\_grad=True)\\[0.3em]
    z = torch.dot(w, x) + b\\
    a = torch.sigmoid(z)\\
    y\_true = torch.tensor(1.0)\\
    loss = (a - y\_true) ** 2\\[0.3em]
    loss.backward() \quad \# Backward pass - ONE LINE!\\[0.3em]
    print(w.grad) \quad \# dL/dw for each component
    }
\end{frame}

% -----------------------------------------------------------------------------
% Key Takeaways
% -----------------------------------------------------------------------------
\begin{frame}{Key Takeaways: Backpropagation}
    \begin{keybox}
        \begin{enumerate}
            \item \textbf{Backprop} computes $\frac{\partial \mathcal{L}}{\partial W}$ for all parameters
            \item \textbf{Chain rule} is the key: multiply local gradients
            \item Flow: Start at loss, work backward through network
            \item Error signal $\delta^{[l]}$ propagates backward
            \item \textbf{Efficient}: $O(1)$ passes, not $O(\text{params})$
            \item \textbf{Frameworks handle it}: \texttt{loss.backward()} does everything!
            \item Key derivatives to know:
            \begin{itemize}
                \item Sigmoid: $\sigma'(z) = \sigma(z)(1-\sigma(z))$
                \item ReLU: $\text{ReLU}'(z) = \mathbf{1}_{z > 0}$
                \item MSE: $\frac{\partial}{\partial \hat{y}}(\hat{y}-y)^2 = 2(\hat{y}-y)$
            \end{itemize}
        \end{enumerate}
    \end{keybox}
    
    \centering
    \textit{Next: Using gradients to optimize — Training algorithms!}
\end{frame}


% Section 14: Optimization Algorithms
% =============================================================================
% Section 14: Optimization Algorithms
% From vanilla gradient descent to Adam
% =============================================================================

\section{Optimization Algorithms}

% -----------------------------------------------------------------------------
% Opening
% -----------------------------------------------------------------------------
\begin{frame}{Optimization: Finding the Best Weights}
    \begin{columns}[T]
        \begin{column}{0.5\textwidth}
            \begin{funnybox}
                \textit{``Gradient descent is like hiking down a mountain in the fog. You can only feel the slope beneath your feet and hope you're going the right way.''}
            \end{funnybox}
            
            \vspace{0.3cm}
            
            \textbf{The Challenge:}
            \begin{itemize}
                \item Loss landscape is high-dimensional
                \item Non-convex (many local minima)
                \item We only see local gradients
            \end{itemize}
        \end{column}
        \begin{column}{0.5\textwidth}
            \centering
            \begin{tikzpicture}[scale=0.7]
                \draw[->] (0,0) -- (4.5,0) node[right] {$w$};
                \draw[->] (0,0) -- (0,3.5) node[above] {$J$};
                
                % Complex landscape
                \draw[thick,accentcyan,domain=0.3:4,samples=100] 
                    plot (\x, {1.5 + 0.5*sin(4*\x r) + 0.2*\x});
                
                % Current position
                \fill[neonpink] (1,1.8) circle (3pt);
                \draw[->,thick,neonyellow] (1,1.8) -- (1.5,1.4);
                
                % Local minimum
                \fill[neonyellow] (2.3,1.3) circle (2pt);
                
                % Global minimum
                \fill[neongreen] (3.5,1.1) circle (2pt);
                \node[neongreen,below,font=\tiny] at (3.5,0.9) {global};
            \end{tikzpicture}
        \end{column}
    \end{columns}
\end{frame}

% -----------------------------------------------------------------------------
% Vanilla Gradient Descent
% -----------------------------------------------------------------------------
\begin{frame}{Vanilla Gradient Descent}
    \begin{defbox}[Gradient Descent Update]
        \[
            \theta_{t+1} = \theta_t - \eta \nabla_\theta J(\theta_t)
        \]
        
        where $\eta$ is the \glow{learning rate}.
    \end{defbox}
    
    \vspace{0.3cm}
    
    \begin{columns}
        \begin{column}{0.5\textwidth}
            \textbf{Types by batch size:}
            \begin{itemize}
                \item \textbf{Batch GD}: All data at once
                \item \textbf{Stochastic GD}: One sample
                \item \textbf{Mini-batch GD}: Subset (common!)
            \end{itemize}
        \end{column}
        \begin{column}{0.5\textwidth}
            \begin{alertbox}[Problems]
                \begin{itemize}
                    \item Learning rate too high → diverge
                    \item Learning rate too low → slow
                    \item Same rate for all parameters
                    \item Gets stuck in saddle points
                \end{itemize}
            \end{alertbox}
        \end{column}
    \end{columns}
\end{frame}

% -----------------------------------------------------------------------------
% Learning Rate
% -----------------------------------------------------------------------------
\begin{frame}{The Learning Rate Dilemma}
    \centering
    \begin{tikzpicture}[scale=0.8]
        % Too small
        \begin{scope}[xshift=0cm]
            \draw[->] (0,0) -- (3,0) node[right,font=\small] {$w$};
            \draw[->] (0,0) -- (0,2.5) node[above,font=\small] {$J$};
            \draw[thick,accentcyan,domain=0.3:2.7,samples=50] plot (\x, {0.3 + 1.5*(\x-1.5)*(\x-1.5)});
            
            \fill[neonpink] (0.5,2.0) circle (2pt);
            \fill[neonpink] (0.6,1.7) circle (2pt);
            \fill[neonpink] (0.7,1.5) circle (2pt);
            \fill[neonpink] (0.8,1.3) circle (2pt);
            
            \node[below,font=\small] at (1.5,-0.3) {$\eta$ too small};
            \node[below,font=\tiny,fgwhite] at (1.5,-0.7) {(very slow)};
        \end{scope}
        
        % Just right
        \begin{scope}[xshift=4.5cm]
            \draw[->] (0,0) -- (3,0) node[right,font=\small] {$w$};
            \draw[->] (0,0) -- (0,2.5) node[above,font=\small] {$J$};
            \draw[thick,accentcyan,domain=0.3:2.7,samples=50] plot (\x, {0.3 + 1.5*(\x-1.5)*(\x-1.5)});
            
            \fill[neongreen] (0.5,2.0) circle (2pt);
            \fill[neongreen] (1.0,0.9) circle (2pt);
            \fill[neongreen] (1.4,0.35) circle (2pt);
            \fill[neongreen] (1.5,0.3) circle (2pt);
            
            \node[below,font=\small] at (1.5,-0.3) {$\eta$ just right};
            \node[below,font=\tiny,fgwhite] at (1.5,-0.7) {(converges)};
        \end{scope}
        
        % Too large
        \begin{scope}[xshift=9cm]
            \draw[->] (0,0) -- (3,0) node[right,font=\small] {$w$};
            \draw[->] (0,0) -- (0,2.5) node[above,font=\small] {$J$};
            \draw[thick,accentcyan,domain=0.3:2.7,samples=50] plot (\x, {0.3 + 1.5*(\x-1.5)*(\x-1.5)});
            
            \fill[neonpink] (0.5,2.0) circle (2pt);
            \fill[neonpink] (2.5,2.0) circle (2pt);
            \fill[neonpink] (0.6,1.7) circle (2pt);
            \draw[->,thick,neonyellow] (0.5,2.0) -- (2.4,2.0);
            
            \node[below,font=\small] at (1.5,-0.3) {$\eta$ too large};
            \node[below,font=\tiny,fgwhite] at (1.5,-0.7) {(oscillates/diverges)};
        \end{scope}
    \end{tikzpicture}
    
    \vspace{0.3cm}
    
    \begin{keybox}
        \textbf{Typical values}: $\eta \in [10^{-4}, 10^{-1}]$
        
        Start with $0.001$ and adjust based on training curves.
    \end{keybox}
\end{frame}

% -----------------------------------------------------------------------------
% Momentum
% -----------------------------------------------------------------------------
\begin{frame}{Momentum: Building Speed}
    \begin{defbox}[SGD with Momentum]
        \begin{align*}
            v_t &= \beta v_{t-1} + \nabla_\theta J(\theta_t) \\
            \theta_{t+1} &= \theta_t - \eta v_t
        \end{align*}
        
        Typical: $\beta = 0.9$
    \end{defbox}
    
    \vspace{0.2cm}
    
    \begin{columns}
        \begin{column}{0.5\textwidth}
            \begin{infobox}[Intuition]
                Like a ball rolling downhill:
                \begin{itemize}
                    \item Builds up velocity
                    \item Carries through flat regions
                    \item Dampens oscillations
                \end{itemize}
            \end{infobox}
        \end{column}
        \begin{column}{0.5\textwidth}
            \centering
            \begin{tikzpicture}[scale=0.6]
                % Elongated valley
                \draw[thick,accentcyan] (0,2) .. controls (1.5,0) and (2.5,0) .. (4,2);
                \draw[thick,accentcyan] (0,2.5) .. controls (1.5,0.5) and (2.5,0.5) .. (4,2.5);
                
                % Without momentum - zigzag
                \draw[thick,neonpink,dashed] (0.3,2.2) -- (0.8,0.8) -- (1.3,1.8) -- (1.8,0.6);
                \node[neonpink,font=\tiny] at (0.8,2.5) {no mom.};
                
                % With momentum - smooth
                \draw[thick,neongreen] (0.3,1.8) .. controls (1,1) and (1.5,0.5) .. (2,0.25);
                \node[neongreen,font=\tiny] at (2.5,1) {with mom.};
            \end{tikzpicture}
        \end{column}
    \end{columns}
    
    \begin{funnybox}
        Momentum = memory of past gradients. ``Keep going in the general direction!''
    \end{funnybox}
\end{frame}

% -----------------------------------------------------------------------------
% RMSprop
% -----------------------------------------------------------------------------
\begin{frame}{RMSprop: Adaptive Learning Rates}
    \begin{defbox}[RMSprop (Hinton)]
        \begin{align*}
            s_t &= \beta s_{t-1} + (1-\beta) (\nabla_\theta J)^2 \\
            \theta_{t+1} &= \theta_t - \frac{\eta}{\sqrt{s_t + \epsilon}} \nabla_\theta J
        \end{align*}
        
        Typical: $\beta = 0.9$, $\epsilon = 10^{-8}$
    \end{defbox}
    
    \vspace{0.3cm}
    
    \begin{columns}
        \begin{column}{0.5\textwidth}
            \begin{keybox}[Key Insight]
                Divide by running average of gradient magnitudes.
                
                \begin{itemize}
                    \item Large gradients → smaller steps
                    \item Small gradients → larger steps
                \end{itemize}
            \end{keybox}
        \end{column}
        \begin{column}{0.5\textwidth}
            \begin{infobox}
                \textbf{Adapts per-parameter!}
                
                Different learning rates for different weights based on their gradient history.
            \end{infobox}
        \end{column}
    \end{columns}
\end{frame}

% -----------------------------------------------------------------------------
% Adam
% -----------------------------------------------------------------------------
\begin{frame}{Adam: The Best of Both Worlds}
    \begin{defbox}[Adam (Adaptive Moment Estimation)]
        \begin{align*}
            m_t &= \beta_1 m_{t-1} + (1-\beta_1) \nabla_\theta J \quad \text{(momentum)}\\
            v_t &= \beta_2 v_{t-1} + (1-\beta_2) (\nabla_\theta J)^2 \quad \text{(RMSprop)}\\
            \hat{m}_t &= \frac{m_t}{1-\beta_1^t}, \quad \hat{v}_t = \frac{v_t}{1-\beta_2^t} \quad \text{(bias correction)}\\
            \theta_{t+1} &= \theta_t - \frac{\eta}{\sqrt{\hat{v}_t} + \epsilon} \hat{m}_t
        \end{align*}
    \end{defbox}
    
    \vspace{0.2cm}
    
    \begin{columns}
        \begin{column}{0.5\textwidth}
            \textbf{Default hyperparameters:}
            \begin{itemize}
                \item $\eta = 0.001$
                \item $\beta_1 = 0.9$ (momentum)
                \item $\beta_2 = 0.999$ (RMSprop)
                \item $\epsilon = 10^{-8}$
            \end{itemize}
        \end{column}
        \begin{column}{0.5\textwidth}
            \begin{successbox}
                \textbf{Adam is the default choice!}
                
                Works well out of the box for most problems.
            \end{successbox}
        \end{column}
    \end{columns}
\end{frame}

% -----------------------------------------------------------------------------
% Optimizer Comparison
% -----------------------------------------------------------------------------
\begin{frame}{Optimizer Comparison}
    \centering
    \begin{tikzpicture}[scale=0.9]
        % Contour plot (simplified)
        \foreach \r in {0.5,1,1.5,2} {
            \draw[accentcyan,opacity=0.5] (2,1.5) ellipse ({\r} and {\r*0.5});
        }
        
        % SGD path
        \draw[thick,neonpink] (0,1.5) -- (0.5,2) -- (1,1) -- (1.5,2) -- (1.8,1.5);
        \node[neonpink,font=\small] at (0,2.5) {SGD};
        
        % Momentum path
        \draw[thick,neonyellow] (0,0.5) .. controls (1,0.8) and (1.5,1.2) .. (2,1.5);
        \node[neonyellow,font=\small] at (0,-0.2) {Momentum};
        
        % Adam path
        \draw[thick,neongreen] (4,1.5) .. controls (3.5,1.5) and (2.5,1.5) .. (2,1.5);
        \node[neongreen,font=\small] at (4.5,1.5) {Adam};
        
        % Minimum
        \fill[white] (2,1.5) circle (3pt);
        \node[below,fgwhite,font=\small] at (2,1.2) {minimum};
    \end{tikzpicture}
    
    \vspace{0.3cm}
    
    \begin{tabular}{lccc}
        & \textbf{Speed} & \textbf{Stability} & \textbf{Tuning} \\
        \hline
        SGD & Medium & Low & Hard \\
        Momentum & Fast & Medium & Medium \\
        RMSprop & Fast & High & Easy \\
        \textbf{Adam} & Fast & High & Easy \\
    \end{tabular}
\end{frame}

% -----------------------------------------------------------------------------
% Learning Rate Schedules
% -----------------------------------------------------------------------------
\begin{frame}{Learning Rate Schedules}
    \begin{defbox}[Learning Rate Decay]
        Decrease $\eta$ during training for fine-tuning:
        
        \begin{itemize}
            \item \textbf{Step decay}: $\eta_t = \eta_0 \cdot \gamma^{\lfloor t/s \rfloor}$
            \item \textbf{Exponential}: $\eta_t = \eta_0 \cdot e^{-\lambda t}$
            \item \textbf{Cosine annealing}: $\eta_t = \eta_{\min} + \frac{1}{2}(\eta_0 - \eta_{\min})(1 + \cos(\frac{t \pi}{T}))$
        \end{itemize}
    \end{defbox}
    
    \centering
    \begin{tikzpicture}[scale=0.6]
        \draw[->] (0,0) -- (5,0) node[right] {epoch};
        \draw[->] (0,0) -- (0,3) node[above] {$\eta$};
        
        % Step decay
        \draw[thick,neonpink] (0,2.5) -- (1.5,2.5) -- (1.5,1.5) -- (3,1.5) -- (3,0.8) -- (4.5,0.8);
        \node[neonpink,font=\tiny,right] at (4.5,0.8) {step};
        
        % Exponential
        \draw[thick,neongreen,domain=0:4.5,samples=50] plot (\x, {2.5*exp(-0.5*\x)});
        \node[neongreen,font=\tiny,right] at (4.5,0.3) {exp};
        
        % Cosine
        \draw[thick,neonyellow,domain=0:4.5,samples=50] plot (\x, {0.3 + 1.1*(1+cos(\x*40))});
        \node[neonyellow,font=\tiny,right] at (4.5,1.4) {cosine};
    \end{tikzpicture}
    
    \begin{keybox}
        \textbf{Warmup}: Start with small $\eta$, gradually increase, then decay.
    \end{keybox}
\end{frame}

% -----------------------------------------------------------------------------
% PyTorch Code
% -----------------------------------------------------------------------------
\begin{frame}{Optimizers in PyTorch}
    \begin{successbox}[PyTorch Implementation]
        {\small\ttfamily
        import torch.optim as optim\\[0.3em]
        model = MyNetwork()\\[0.3em]
        optimizer = optim.Adam(model.parameters(), lr=0.001)\\
        \# or: optim.SGD(model.parameters(), lr=0.01, momentum=0.9)\\[0.3em]
        for epoch in range(num\_epochs):\\
        \quad for x, y in dataloader:\\
        \quad\quad optimizer.zero\_grad() \quad \# Clear gradients\\
        \quad\quad y\_pred = model(x) \quad\quad \# Forward pass\\
        \quad\quad loss = criterion(y\_pred, y)\\
        \quad\quad loss.backward() \quad\quad\quad \# Compute gradients\\
        \quad\quad optimizer.step() \quad\quad\quad \# Update parameters
        }
    \end{successbox}
\end{frame}

% -----------------------------------------------------------------------------
% Key Takeaways
% -----------------------------------------------------------------------------
\begin{frame}{Key Takeaways: Optimization}
    \begin{keybox}
        \begin{enumerate}
            \item \textbf{Gradient Descent}: $\theta \leftarrow \theta - \eta \nabla J$
            \item \textbf{Mini-batch}: Balance between speed and stability
            \item \textbf{Momentum}: Accumulate velocity, smooth updates
            \item \textbf{RMSprop}: Adapt learning rate per parameter
            \item \textbf{Adam}: Combines momentum + RMSprop
            \begin{itemize}
                \item Default choice: $\eta=0.001$, $\beta_1=0.9$, $\beta_2=0.999$
            \end{itemize}
            \item \textbf{Learning rate schedule}: Start high, decay over time
            \item Key workflow: \texttt{zero\_grad() → forward → backward → step()}
        \end{enumerate}
    \end{keybox}
    
    \vspace{0.2cm}
    
    \centering
    \textit{Next: Preventing overfitting with regularization!}
\end{frame}


% Section 15: Regularization Techniques
% =============================================================================
% Section 15: Regularization Techniques
% Preventing overfitting: L2, Dropout, and more
% =============================================================================

\section{Regularization Techniques}

% -----------------------------------------------------------------------------
% Opening
% -----------------------------------------------------------------------------
\begin{frame}{Regularization: Keeping Models Humble}
    \begin{columns}[T]
        \begin{column}{0.5\textwidth}
            \begin{funnybox}
                \textit{``A model that memorizes the training data is like a student who only memorizes answers — useless on a test with new questions!''}
            \end{funnybox}
            
            \vspace{0.3cm}
            
            \textbf{The Overfitting Problem:}
            \begin{itemize}
                \item Training loss: LOW (YES)
                \item Test loss: HIGH (NO)
                \item Model memorized, didn't learn!
            \end{itemize}
        \end{column}
        \begin{column}{0.5\textwidth}
            \centering
            \begin{tikzpicture}[scale=0.6]
                \draw[->] (0,0) -- (4,0) node[right] {$x$};
                \draw[->] (0,0) -- (0,3) node[above] {$y$};
                
                % Data points
                \foreach \x/\y in {0.5/0.8, 1/1.5, 1.5/1.2, 2/2, 2.5/1.8, 3/2.5, 3.5/2.2} {
                    \fill[accentcyan] (\x,\y) circle (3pt);
                }
                
                % Good fit
                \draw[thick,neongreen,domain=0.3:3.7,samples=50] plot (\x, {0.5 + 0.5*\x});
                
                % Overfitting
                \draw[thick,neonpink,domain=0.4:3.6,samples=100] 
                    plot (\x, {0.8 + 0.5*sin(3*\x r) + 0.4*\x + 0.2*sin(8*\x r)});
                
                \node[neongreen,font=\tiny] at (1,2.5) {good};
                \node[neonpink,font=\tiny] at (3,3) {overfit};
            \end{tikzpicture}
        \end{column}
    \end{columns}
\end{frame}

% -----------------------------------------------------------------------------
% Bias-Variance Tradeoff
% -----------------------------------------------------------------------------
\begin{frame}{Bias-Variance Tradeoff}
    \begin{defbox}[Decomposition of Error]
        \[
            \text{Error} = \text{Bias}^2 + \text{Variance} + \text{Irreducible Noise}
        \]
    \end{defbox}
    
    \vspace{0.2cm}
    
    \begin{columns}
        \begin{column}{0.5\textwidth}
            \begin{alertbox}[High Bias (Underfitting)]
                Model too simple
                \begin{itemize}
                    \item Misses patterns
                    \item Train error: HIGH
                    \item Test error: HIGH
                \end{itemize}
            \end{alertbox}
        \end{column}
        \begin{column}{0.5\textwidth}
            \begin{alertbox}[High Variance (Overfitting)]
                Model too complex
                \begin{itemize}
                    \item Fits noise
                    \item Train error: LOW
                    \item Test error: HIGH
                \end{itemize}
            \end{alertbox}
        \end{column}
    \end{columns}
    
    \vspace{0.3cm}
    
    \begin{keybox}
        \textbf{Regularization} reduces variance at the cost of slightly higher bias.
        
        Goal: Find the sweet spot!
    \end{keybox}
\end{frame}

% -----------------------------------------------------------------------------
% L2 Regularization
% -----------------------------------------------------------------------------
\begin{frame}{L2 Regularization (Weight Decay)}
    \begin{defbox}[L2 Regularization]
        Add penalty for large weights:
        \[
            J_{\text{reg}} = J_{\text{original}} + \frac{\lambda}{2} \sum_{l} \|W^{[l]}\|_F^2
        \]
        
        where $\|W\|_F^2 = \sum_{i,j} W_{ij}^2$ (Frobenius norm)
    \end{defbox}
    
    \vspace{0.2cm}
    
    \begin{columns}
        \begin{column}{0.5\textwidth}
            \textbf{Effect on gradient:}
            \[
                \frac{\partial J_{\text{reg}}}{\partial W} = \frac{\partial J}{\partial W} + \lambda W
            \]
            
            Update becomes:
            \[
                W \leftarrow W(1 - \eta\lambda) - \eta \nabla_W J
            \]
            
            \textit{Weights ``decay'' toward zero}
        \end{column}
        \begin{column}{0.5\textwidth}
            \begin{keybox}[Why it works]
                \begin{itemize}
                    \item Penalizes complex models
                    \item Encourages small weights
                    \item Smooths decision boundary
                    \item Reduces sensitivity to noise
                \end{itemize}
            \end{keybox}
        \end{column}
    \end{columns}
\end{frame}

% -----------------------------------------------------------------------------
% L1 Regularization
% -----------------------------------------------------------------------------
\begin{frame}{L1 Regularization (Lasso)}
    \begin{defbox}[L1 Regularization]
        \[
            J_{\text{reg}} = J_{\text{original}} + \lambda \sum_{l} \|W^{[l]}\|_1
        \]
        
        where $\|W\|_1 = \sum_{i,j} |W_{ij}|$
    \end{defbox}
    
    \vspace{0.2cm}
    
    \begin{columns}
        \begin{column}{0.5\textwidth}
            \begin{infobox}[L1 vs L2]
                \textbf{L1 promotes sparsity!}
                \begin{itemize}
                    \item Drives some weights to exactly 0
                    \item Feature selection built-in
                    \item Simpler models
                \end{itemize}
            \end{infobox}
        \end{column}
        \begin{column}{0.5\textwidth}
            \centering
            \begin{tikzpicture}[scale=0.6]
                \draw[->] (-2,0) -- (2,0) node[right] {$w_1$};
                \draw[->] (0,-2) -- (0,2) node[above] {$w_2$};
                
                % L2 constraint (circle)
                \draw[thick,neongreen] (0,0) circle (1);
                \node[neongreen,font=\tiny] at (1.5,0.8) {L2};
                
                % L1 constraint (diamond)
                \draw[thick,neonpink] (-1,0) -- (0,1) -- (1,0) -- (0,-1) -- cycle;
                \node[neonpink,font=\tiny] at (-1.3,0.8) {L1};
                
                % Optimal hits corner for L1
                \fill[neonyellow] (1,0) circle (3pt);
            \end{tikzpicture}
            
            \textit{L1 tends to hit corners (sparse)}
        \end{column}
    \end{columns}
\end{frame}

% -----------------------------------------------------------------------------
% Dropout
% -----------------------------------------------------------------------------
\begin{frame}{Dropout: Random Deactivation}
    \begin{defbox}[Dropout (Srivastava et al., 2014)]
        During training, randomly set each neuron's output to 0 with probability $p$:
        \[
            \tilde{a}_i = \begin{cases}
                0 & \text{with probability } p \\
                \frac{a_i}{1-p} & \text{with probability } 1-p
            \end{cases}
        \]
    \end{defbox}
    
    \centering
    \begin{tikzpicture}[scale=0.7]
        % Full network
        \begin{scope}[xshift=0cm]
            \node[font=\small,accentcyan] at (1.5,2) {Training};
            \foreach \i in {1,2,3} {
                \node[circle,draw=accentcyan,fill=darkgray,minimum size=0.5cm] (I\i) at (0,1.5-\i*0.8) {};
            }
            \foreach \i in {1,2,3,4} {
                \ifnum\i=2
                    \node[circle,draw=neonpink,fill=neonpink!30,minimum size=0.5cm,text=black,font=\bfseries] (H\i) at (1.5,2-\i*0.7) {$\times$};
                \else\ifnum\i=4
                    \node[circle,draw=neonpink,fill=neonpink!30,minimum size=0.5cm,text=black,font=\bfseries] (H\i) at (1.5,2-\i*0.7) {$\times$};
                \else
                    \node[circle,draw=neongreen,fill=darkgray,minimum size=0.5cm] (H\i) at (1.5,2-\i*0.7) {};
                \fi\fi
            }
            \foreach \i in {1,2} {
                \node[circle,draw=accentcyan,fill=darkgray,minimum size=0.5cm] (O\i) at (3,0.8-\i*0.8) {};
            }
            
            \foreach \i in {1,2,3} {
                \foreach \j in {1,3} {
                    \draw[connection,opacity=0.6] (I\i) -- (H\j);
                }
            }
            \foreach \j in {1,3} {
                \foreach \k in {1,2} {
                    \draw[connection,opacity=0.6] (H\j) -- (O\k);
                }
            }
        \end{scope}
        
        % Test network
        \begin{scope}[xshift=5cm]
            \node[font=\small,neongreen] at (1.5,2) {Testing};
            \foreach \i in {1,2,3} {
                \node[circle,draw=accentcyan,fill=darkgray,minimum size=0.5cm] (I\i) at (0,1.5-\i*0.8) {};
            }
            \foreach \i in {1,2,3,4} {
                \node[circle,draw=neongreen,fill=darkgray,minimum size=0.5cm] (H\i) at (1.5,2-\i*0.7) {};
            }
            \foreach \i in {1,2} {
                \node[circle,draw=accentcyan,fill=darkgray,minimum size=0.5cm] (O\i) at (3,0.8-\i*0.8) {};
            }
            
            \foreach \i in {1,2,3} {
                \foreach \j in {1,2,3,4} {
                    \draw[connection,opacity=0.4] (I\i) -- (H\j);
                }
            }
            \foreach \j in {1,2,3,4} {
                \foreach \k in {1,2} {
                    \draw[connection,opacity=0.4] (H\j) -- (O\k);
                }
            }
            
            \node[below,font=\tiny,fgwhite] at (1.5,-1.5) {(all neurons, scaled)};
        \end{scope}
    \end{tikzpicture}
\end{frame}

% -----------------------------------------------------------------------------
% Why Dropout Works
% -----------------------------------------------------------------------------
\begin{frame}{Why Dropout Works}
    \begin{columns}[T]
        \begin{column}{0.5\textwidth}
            \begin{keybox}[Intuitions]
                \begin{itemize}
                    \item \textbf{Ensemble effect}: Training many sub-networks
                    \item \textbf{Redundancy}: Can't rely on any single feature
                    \item \textbf{Co-adaptation prevention}: Features must be useful alone
                \end{itemize}
            \end{keybox}
        \end{column}
        \begin{column}{0.5\textwidth}
            \begin{infobox}[Typical Values]
                \begin{itemize}
                    \item Input layer: $p = 0.2$
                    \item Hidden layers: $p = 0.5$
                    \item Output layer: No dropout
                \end{itemize}
                
                \vspace{0.2cm}
                
                \textit{Higher $p$ = more regularization}
            \end{infobox}
        \end{column}
    \end{columns}
    
    \vspace{0.3cm}
    
    \begin{funnybox}
        Dropout is like studying for an exam knowing some of your brain cells will randomly fail. You learn to be robust!
    \end{funnybox}
\end{frame}

% -----------------------------------------------------------------------------
% Batch Normalization
% -----------------------------------------------------------------------------
\begin{frame}{Batch Normalization}
    \begin{defbox}[Batch Normalization (Ioffe \& Szegedy, 2015)]
        Normalize activations across the batch:
        \[
            \hat{x}_i = \frac{x_i - \mu_B}{\sqrt{\sigma_B^2 + \epsilon}}
        \]
        
        Then scale and shift:
        \[
            y_i = \gamma \hat{x}_i + \beta
        \]
        
        where $\gamma$ and $\beta$ are learnable parameters.
    \end{defbox}
    
    \vspace{0.2cm}
    
    \begin{columns}
        \begin{column}{0.5\textwidth}
            \textbf{Benefits:}
            \begin{itemize}
                \item Stabilizes training
                \item Allows higher learning rates
                \item Mild regularization effect
                \item Reduces internal covariate shift
            \end{itemize}
        \end{column}
        \begin{column}{0.5\textwidth}
            \begin{keybox}
                Apply BatchNorm \textbf{before} or \textbf{after} activation.
                
                Typical placement: after linear, before ReLU.
            \end{keybox}
        \end{column}
    \end{columns}
\end{frame}

% -----------------------------------------------------------------------------
% Early Stopping
% -----------------------------------------------------------------------------
\begin{frame}{Early Stopping}
    \begin{defbox}[Early Stopping]
        Stop training when validation loss starts increasing.
    \end{defbox}
    
    \centering
    \begin{tikzpicture}[scale=0.8]
        \draw[->] (0,0) -- (6,0) node[right] {epochs};
        \draw[->] (0,0) -- (0,3.5) node[above] {loss};
        
        % Training loss
        \draw[thick,neonpink,domain=0.2:5.5,samples=50] 
            plot (\x, {2.5*exp(-0.4*\x) + 0.2});
        \node[neonpink,font=\small] at (5,0.7) {train};
        
        % Validation loss
        \draw[thick,neongreen,domain=0.2:5.5,samples=50] 
            plot (\x, {2.5*exp(-0.4*\x) + 0.3 + 0.1*max(0,\x-2.5)*max(0,\x-2.5)});
        \node[neongreen,font=\small] at (5,2) {valid};
        
        % Early stopping point
        \draw[dashed,neonyellow] (2.5,0) -- (2.5,3);
        \node[neonyellow,above,font=\small] at (2.5,3) {STOP};
        
        % Overfitting region
        \fill[neonpink,opacity=0.2] (2.5,0) rectangle (5.5,3);
        \node[fgwhite,font=\tiny] at (4,1.5) {overfitting};
    \end{tikzpicture}
    
    \begin{keybox}
        \textbf{Patience}: Wait $k$ epochs before stopping (to avoid noise).
        
        Save model at best validation loss!
    \end{keybox}
\end{frame}

% -----------------------------------------------------------------------------
% Data Augmentation
% -----------------------------------------------------------------------------
\begin{frame}{Data Augmentation}
    \begin{defbox}[Data Augmentation]
        Create more training data by applying transformations:
        \begin{itemize}
            \item Images: rotation, flip, crop, color jitter
            \item Text: synonym replacement, back-translation
            \item Audio: pitch shift, time stretch, noise injection
        \end{itemize}
    \end{defbox}
    
    \vspace{0.3cm}
    
    \centering
    \begin{tikzpicture}[scale=0.7]
        % Original
        \draw[thick,accentcyan,rounded corners] (0,0) rectangle (1.5,1.5);
        \node[font=\Large] at (0.75,0.75) {CAT};
        \node[below,fgwhite,font=\small] at (0.75,-0.2) {original};
        
        % Arrow
        \draw[->,thick,neonyellow] (2,0.75) -- (3,0.75);
        
        % Augmented versions
        \begin{scope}[xshift=3.5cm]
            \draw[thick,neonpink,rounded corners] (0,0) rectangle (1.2,1.2);
            \node[font=\normalsize,rotate=15] at (0.6,0.6) {CAT};
        \end{scope}
        
        \begin{scope}[xshift=5cm]
            \draw[thick,neonpink,rounded corners] (0,0) rectangle (1.2,1.2);
            \node[font=\normalsize,xscale=-1] at (0.6,0.6) {CAT};
        \end{scope}
        
        \begin{scope}[xshift=6.5cm]
            \draw[thick,neonpink,rounded corners] (0,0) rectangle (1.2,1.2);
            \node[font=\small] at (0.6,0.6) {CAT};
        \end{scope}
        
        \begin{scope}[xshift=8cm]
            \draw[thick,neonpink,rounded corners,fill=darkgray!50] (0,0) rectangle (1.2,1.2);
            \node[font=\normalsize,opacity=0.7] at (0.6,0.6) {CAT};
        \end{scope}
        
        \node[below,fgwhite,font=\small] at (6,0) {augmented copies};
    \end{tikzpicture}
    
    \vspace{0.2cm}
    
    \begin{funnybox}
        The same cat is still a cat whether it's rotated, flipped, or darker!
    \end{funnybox}
\end{frame}

% -----------------------------------------------------------------------------
% Summary
% -----------------------------------------------------------------------------
\begin{frame}{Regularization Summary}
    \begin{tabular}{lll}
        \textbf{Technique} & \textbf{Effect} & \textbf{When to Use} \\
        \hline
        L2 (Weight Decay) & Shrinks weights & Almost always \\
        L1 (Lasso) & Sparse weights & Feature selection \\
        Dropout & Random deactivation & Deep networks \\
        Batch Norm & Normalize activations & Deep networks \\
        Early Stopping & Stop training early & Always \\
        Data Augmentation & More training data & Images, audio \\
    \end{tabular}
    
    \vspace{0.5cm}
    
    \begin{keybox}
        \textbf{Best practice}: Combine multiple techniques!
        \begin{itemize}
            \item L2 regularization (almost always)
            \item Dropout (hidden layers)
            \item BatchNorm (deep networks)
            \item Early stopping (monitor validation)
            \item Data augmentation (when possible)
        \end{itemize}
    \end{keybox}
\end{frame}

% -----------------------------------------------------------------------------
% Key Takeaways
% -----------------------------------------------------------------------------
\begin{frame}{Key Takeaways: Regularization}
    \begin{keybox}
        \begin{enumerate}
            \item \textbf{Overfitting}: Low train error, high test error
            \item \textbf{Regularization} trades bias for lower variance
            \item \textbf{L2/L1}: Add weight penalty to loss
            \begin{itemize}
                \item L2: Shrinks all weights
                \item L1: Drives some to zero (sparse)
            \end{itemize}
            \item \textbf{Dropout}: Randomly disable neurons (training only!)
            \item \textbf{BatchNorm}: Normalize activations per batch
            \item \textbf{Early stopping}: Monitor validation loss
            \item \textbf{Data augmentation}: Create more training data
            \item Combine multiple techniques for best results
        \end{enumerate}
    \end{keybox}
    
    \centering
    \textit{Next: Decision Trees — a different approach to learning!}
\end{frame}


% =============================================================================
% PART IV: CLASSICAL ML & THEORY
% =============================================================================

% Section 16: Decision Trees
% =============================================================================
% Section 16: Decision Trees
% Tree-based learning methods
% =============================================================================

\section{Decision Trees}

% -----------------------------------------------------------------------------
% Opening
% -----------------------------------------------------------------------------
\begin{frame}{Decision Trees: Asking the Right Questions}
    \begin{columns}[T]
        \begin{column}{0.5\textwidth}
            \begin{funnybox}
                \textit{``Decision trees are like a game of 20 questions, except the computer picks the questions AND answers them!''}
            \end{funnybox}
            
            \vspace{0.3cm}
            
            \textbf{Key Ideas:}
            \begin{itemize}
                \item Split data with yes/no questions
                \item Each split = one decision
                \item Leaves = final predictions
            \end{itemize}
        \end{column}
        \begin{column}{0.5\textwidth}
            \centering
            \begin{tikzpicture}[
                scale=0.7,
                level distance=1.2cm,
                level 1/.style={sibling distance=3cm},
                level 2/.style={sibling distance=1.5cm}
            ]
                \node[draw=accentcyan,rounded corners,fill=darkgray] {Age $> 30$?}
                    child {
                        node[draw=accentcyan,rounded corners,fill=darkgray] {Income $> 50k$?}
                        child {node[draw=neongreen,rounded corners,fill=darkgray] {Buy}}
                        child {node[draw=neonpink,rounded corners,fill=darkgray] {No}}
                        edge from parent node[left,font=\tiny] {yes}
                    }
                    child {
                        node[draw=accentcyan,rounded corners,fill=darkgray] {Student?}
                        child {node[draw=neongreen,rounded corners,fill=darkgray] {Buy}}
                        child {node[draw=neonpink,rounded corners,fill=darkgray] {No}}
                        edge from parent node[right,font=\tiny] {no}
                    };
            \end{tikzpicture}
        \end{column}
    \end{columns}
\end{frame}

% -----------------------------------------------------------------------------
% Tree Structure
% -----------------------------------------------------------------------------
\begin{frame}{Anatomy of a Decision Tree}
    \begin{defbox}[Decision Tree Components]
        \begin{itemize}
            \item \textbf{Root}: Top node (first question)
            \item \textbf{Internal nodes}: Decision points (questions)
            \item \textbf{Branches}: Answers (yes/no or thresholds)
            \item \textbf{Leaves}: Final predictions
        \end{itemize}
    \end{defbox}
    
    \centering
    \begin{tikzpicture}[scale=0.8]
        % Root
        \node[draw=neonyellow,thick,rounded corners,fill=darkgray] (root) at (0,0) {$x_1 < 5$?};
        \node[neonyellow,left,font=\small] at (-1.5,0) {Root};
        
        % Level 1
        \node[draw=accentcyan,rounded corners,fill=darkgray] (l1) at (-2,-1.5) {$x_2 > 3$?};
        \node[draw=accentcyan,rounded corners,fill=darkgray] (r1) at (2,-1.5) {$x_3 = A$?};
        \node[accentcyan,font=\small] at (-4,-1.5) {Internal};
        
        % Leaves
        \node[draw=neongreen,rounded corners,fill=darkgray] (ll) at (-3,-3) {Class 1};
        \node[draw=neonpink,rounded corners,fill=darkgray] (lr) at (-1,-3) {Class 0};
        \node[draw=neongreen,rounded corners,fill=darkgray] (rl) at (1,-3) {Class 1};
        \node[draw=neonpink,rounded corners,fill=darkgray] (rr) at (3,-3) {Class 0};
        \node[neongreen,font=\small] at (-4.5,-3) {Leaves};
        
        % Edges
        \draw[thick] (root) -- (l1) node[midway,left,font=\tiny] {yes};
        \draw[thick] (root) -- (r1) node[midway,right,font=\tiny] {no};
        \draw[thick] (l1) -- (ll) node[midway,left,font=\tiny] {yes};
        \draw[thick] (l1) -- (lr) node[midway,right,font=\tiny] {no};
        \draw[thick] (r1) -- (rl) node[midway,left,font=\tiny] {yes};
        \draw[thick] (r1) -- (rr) node[midway,right,font=\tiny] {no};
    \end{tikzpicture}
\end{frame}

% -----------------------------------------------------------------------------
% Splitting Criteria - Information Gain
% -----------------------------------------------------------------------------
\begin{frame}{How to Choose Splits: Information Gain}
    \begin{defbox}[Entropy]
        Measure of impurity/uncertainty:
        \[
            H(S) = -\sum_{c=1}^{C} p_c \log_2(p_c)
        \]
        
        where $p_c$ = proportion of class $c$ in set $S$.
    \end{defbox}
    
    \vspace{0.2cm}
    
    \begin{columns}
        \begin{column}{0.5\textwidth}
            \begin{defbox}[Information Gain]
                \[
                    IG(S, A) = H(S) - \sum_{v \in A} \frac{|S_v|}{|S|} H(S_v)
                \]
                
                Choose split that maximizes $IG$!
            \end{defbox}
        \end{column}
        \begin{column}{0.5\textwidth}
            \centering
            \begin{tikzpicture}[scale=0.6]
                \draw[->] (0,0) -- (4,0) node[right] {$p$};
                \draw[->] (0,0) -- (0,2.5) node[above] {$H$};
                
                \draw[thick,neonpink,domain=0.01:0.99,samples=50] 
                    plot (\x*4, {-\x*ln(\x)/ln(2) - (1-\x)*ln(1-\x)/ln(2)});
                
                \node[below,font=\tiny] at (0,0) {0};
                \node[below,font=\tiny] at (2,0) {0.5};
                \node[below,font=\tiny] at (4,0) {1};
                
                \fill[neongreen] (0,0) circle (2pt);
                \fill[neongreen] (4,0) circle (2pt);
                \fill[neonyellow] (2,2) circle (2pt);
                
                \node[font=\tiny,fgwhite] at (2,-0.7) {max entropy at 50/50};
            \end{tikzpicture}
        \end{column}
    \end{columns}
    
    \begin{keybox}
        Low entropy = pure node. High entropy = mixed node.
    \end{keybox}
\end{frame}

% -----------------------------------------------------------------------------
% Gini Impurity
% -----------------------------------------------------------------------------
\begin{frame}{Alternative: Gini Impurity}
    \begin{defbox}[Gini Impurity]
        \[
            Gini(S) = 1 - \sum_{c=1}^{C} p_c^2
        \]
    \end{defbox}
    
    \vspace{0.2cm}
    
    \begin{columns}
        \begin{column}{0.5\textwidth}
            \textbf{Example:}
            
            If $S$ has 70\% class A, 30\% class B:
            \[
                Gini = 1 - (0.7^2 + 0.3^2) = 1 - 0.58 = 0.42
            \]
            
            If $S$ is 100\% class A:
            \[
                Gini = 1 - 1^2 = 0
            \]
        \end{column}
        \begin{column}{0.5\textwidth}
            \begin{infobox}[Entropy vs Gini]
                \begin{itemize}
                    \item Similar results in practice
                    \item Gini slightly faster (no log)
                    \item Gini: default in sklearn
                    \item Entropy: information-theoretic
                \end{itemize}
            \end{infobox}
        \end{column}
    \end{columns}
    
    \begin{funnybox}
        Gini = probability of misclassifying a randomly chosen sample if it was randomly labeled.
    \end{funnybox}
\end{frame}

% -----------------------------------------------------------------------------
% Building the Tree
% -----------------------------------------------------------------------------
\begin{frame}{Building a Decision Tree (ID3/CART)}
    \begin{keybox}[Recursive Algorithm]
        \textbf{BuildTree}($S$):
        \begin{enumerate}
            \item If all samples in $S$ have same class: return Leaf(class)
            \item If no features left: return Leaf(majority class)
            \item Find best feature/threshold to split on (max $IG$ or min Gini)
            \item Split $S$ into subsets $S_1, S_2, \ldots$
            \item Recursively: BuildTree($S_1$), BuildTree($S_2$), ...
        \end{enumerate}
    \end{keybox}
    
    \vspace{0.3cm}
    
    \begin{columns}
        \begin{column}{0.5\textwidth}
            \textbf{For continuous features:}
            
            Try all thresholds:
            $x_i < t$ vs $x_i \geq t$
            
            Choose $t$ with best split.
        \end{column}
        \begin{column}{0.5\textwidth}
            \textbf{For categorical features:}
            
            Try all subsets or
            one-vs-rest splits.
        \end{column}
    \end{columns}
\end{frame}

% -----------------------------------------------------------------------------
% Overfitting in Trees
% -----------------------------------------------------------------------------
\begin{frame}{Overfitting: The Tree's Curse}
    \begin{columns}[T]
        \begin{column}{0.5\textwidth}
            \begin{alertbox}[Deep Trees Overfit!]
                Without limits, a tree can have:
                \begin{itemize}
                    \item One leaf per training sample
                    \item 100\% training accuracy
                    \item Terrible test accuracy
                \end{itemize}
            \end{alertbox}
            
            \vspace{0.2cm}
            
            \begin{keybox}[Solution: Pruning]
                Limit tree growth:
                \begin{itemize}
                    \item Max depth
                    \item Min samples per leaf
                    \item Min samples to split
                \end{itemize}
            \end{keybox}
        \end{column}
        \begin{column}{0.5\textwidth}
            \centering
            \begin{tikzpicture}[scale=0.6]
                \draw[->] (0,0) -- (5,0) node[right,font=\small] {depth};
                \draw[->] (0,0) -- (0,3.5) node[above,font=\small] {error};
                
                % Training error
                \draw[thick,neonpink,domain=0.5:4.5,samples=50] 
                    plot (\x, {2.5*exp(-0.5*\x)});
                \node[neonpink,font=\tiny] at (4.5,0.8) {train};
                
                % Test error (piecewise function)
                \draw[thick,neongreen,domain=0.5:2,samples=25] 
                    plot (\x, {1.5*exp(-0.3*\x)});
                \draw[thick,neongreen,domain=2:4.5,samples=25] 
                    plot (\x, {1.5*exp(-0.3*\x) + 0.2*(\x-2)*(\x-2)});
                \node[neongreen,font=\tiny] at (4.5,2.5) {test};
                
                % Optimal depth
                \draw[dashed,neonyellow] (2,0) -- (2,3);
                \node[neonyellow,above,font=\tiny] at (2,3) {optimal};
            \end{tikzpicture}
        \end{column}
    \end{columns}
\end{frame}

% -----------------------------------------------------------------------------
% Regression Trees
% -----------------------------------------------------------------------------
\begin{frame}{Regression Trees}
    \begin{defbox}[Regression Tree]
        For continuous targets, use \textbf{variance reduction}:
        \[
            \text{Var}(S) = \frac{1}{|S|} \sum_{i \in S} (y_i - \bar{y})^2
        \]
        
        Leaf prediction: mean of samples in that leaf.
    \end{defbox}
    
    \centering
    \begin{tikzpicture}[scale=0.8]
        \draw[->] (0,0) -- (5,0) node[right] {$x$};
        \draw[->] (0,0) -- (0,3) node[above] {$y$};
        
        % Data points
        \foreach \x/\y in {0.5/0.8, 1/1.2, 1.5/1.0, 2.2/1.8, 2.5/2.2, 2.8/2.0, 3.5/2.6, 4/2.4, 4.3/2.8} {
            \fill[accentcyan] (\x,\y) circle (2pt);
        }
        
        % Tree prediction (step function)
        \draw[thick,neonpink] (0,1) -- (1.8,1);
        \draw[thick,neonpink] (1.8,2) -- (3.2,2);
        \draw[thick,neonpink] (3.2,2.6) -- (5,2.6);
        
        % Split lines
        \draw[dashed,neonyellow] (1.8,0) -- (1.8,3);
        \draw[dashed,neonyellow] (3.2,0) -- (3.2,3);
    \end{tikzpicture}
    
    \begin{funnybox}
        Regression trees = piecewise constant approximation. Square-ish, but it works!
    \end{funnybox}
\end{frame}

% -----------------------------------------------------------------------------
% Pros and Cons
% -----------------------------------------------------------------------------
\begin{frame}{Decision Trees: Pros and Cons}
    \begin{columns}[T]
        \begin{column}{0.5\textwidth}
            \begin{successbox}[Advantages]
                \begin{itemize}
                    \item Easy to interpret
                    \item No scaling needed
                    \item Handles mixed features
                    \item Built-in feature selection
                    \item Fast to train
                    \item No assumptions about data
                \end{itemize}
            \end{successbox}
        \end{column}
        \begin{column}{0.5\textwidth}
            \begin{alertbox}[Disadvantages]
                \begin{itemize}
                    \item Prone to overfitting
                    \item Unstable (small data changes → different tree)
                    \item Can't extrapolate
                    \item Biased toward features with many values
                    \item Greedy (local optima)
                \end{itemize}
            \end{alertbox}
        \end{column}
    \end{columns}
    
    \vspace{0.3cm}
    
    \begin{keybox}
        Single trees are limited, but they're the building blocks for powerful \textbf{ensemble methods}!
    \end{keybox}
\end{frame}

% -----------------------------------------------------------------------------
% Code Example
% -----------------------------------------------------------------------------
\begin{frame}{Decision Trees in Scikit-learn}
    \begin{successbox}[Python Implementation]
        {\small\ttfamily
        from sklearn.tree import DecisionTreeClassifier, plot\_tree\\
        import matplotlib.pyplot as plt\\[0.3em]
        clf = DecisionTreeClassifier(\\
        \quad max\_depth=3,\quad\quad\quad \# Prevent overfitting\\
        \quad min\_samples\_leaf=5,\quad \# Min samples per leaf\\
        \quad criterion='gini'\quad\quad \# or 'entropy'\\
        )\\
        clf.fit(X\_train, y\_train)\\[0.3em]
        y\_pred = clf.predict(X\_test)\\[0.3em]
        plt.figure(figsize=(12, 8))\\
        plot\_tree(clf, filled=True, feature\_names=feature\_names)\\
        plt.show()
        }
    \end{successbox}
\end{frame}

% -----------------------------------------------------------------------------
% Key Takeaways
% -----------------------------------------------------------------------------
\begin{frame}{Key Takeaways: Decision Trees}
    \begin{keybox}
        \begin{enumerate}
            \item \textbf{Decision trees} = recursive binary splits
            \item \textbf{Split criteria}: Information gain (entropy) or Gini impurity
            \item \textbf{Entropy}: $H(S) = -\sum p_c \log_2 p_c$
            \item \textbf{Gini}: $G(S) = 1 - \sum p_c^2$
            \item \textbf{Regression}: Use variance, predict mean
            \item \textbf{Overfitting risk}: Always prune!
            \begin{itemize}
                \item Limit depth, min samples per leaf
            \end{itemize}
            \item \textbf{Pros}: Interpretable, fast, no preprocessing
            \item \textbf{Cons}: Unstable, overfit easily
        \end{enumerate}
    \end{keybox}
    
    \centering
    \textit{Next: Combining trees into powerful ensembles!}
\end{frame}


% Section 17: Ensemble Methods & Boosting
% =============================================================================
% Section 17: Ensemble Methods & Boosting
% Random Forests, Bagging, AdaBoost, Gradient Boosting
% =============================================================================

\section{Ensemble Methods \& Boosting}

% -----------------------------------------------------------------------------
% Opening
% -----------------------------------------------------------------------------
\begin{frame}{Ensemble Methods: Wisdom of the Crowd}
    \begin{columns}[T]
        \begin{column}{0.5\textwidth}
            \begin{funnybox}
                \textit{``One tree might be wrong, but a whole forest? Much harder to fool!''}
            \end{funnybox}
            
            \vspace{0.3cm}
            
            \textbf{Core Idea:}
            \begin{itemize}
                \item Train multiple ``weak'' models
                \item Combine their predictions
                \item Get a ``strong'' model!
            \end{itemize}
        \end{column}
        \begin{column}{0.5\textwidth}
            \centering
            \begin{tikzpicture}[scale=0.7]
                % Multiple trees
                \foreach \x in {0,1.5,3} {
                    \begin{scope}[xshift=\x cm]
                        \node[draw=accentcyan,rounded corners,fill=darkgray,minimum size=0.4cm] at (0,0) {};
                        \draw[accentcyan] (-0.3,-0.5) -- (0,-0.3);
                        \draw[accentcyan] (0.3,-0.5) -- (0,-0.3);
                        \draw[accentcyan] (-0.5,-1) -- (-0.3,-0.5);
                        \draw[accentcyan] (-0.1,-1) -- (-0.3,-0.5);
                        \draw[accentcyan] (0.1,-1) -- (0.3,-0.5);
                        \draw[accentcyan] (0.5,-1) -- (0.3,-0.5);
                    \end{scope}
                }
                
                % Voting
                \draw[->,thick,neonyellow] (0.75,-1.3) -- (0.75,-2);
                \draw[->,thick,neonyellow] (1.5,-1.3) -- (1.5,-2);
                \draw[->,thick,neonyellow] (2.25,-1.3) -- (2.25,-2);
                
                % Combined prediction
                \node[draw=neonpink,rounded corners,fill=darkgray,minimum width=2.5cm] at (1.5,-2.5) {Vote/Average};
                
                \draw[->,thick,neonyellow] (1.5,-3) -- (1.5,-3.5);
                \node[neongreen,font=\small] at (1.5,-4) {Strong Prediction};
            \end{tikzpicture}
        \end{column}
    \end{columns}
\end{frame}

% -----------------------------------------------------------------------------
% Bagging
% -----------------------------------------------------------------------------
\begin{frame}{Bagging: Bootstrap Aggregating}
    \begin{defbox}[Bagging (Breiman, 1996)]
        \begin{enumerate}
            \item Create $B$ bootstrap samples (sample with replacement)
            \item Train one model on each bootstrap sample
            \item Aggregate predictions:
            \begin{itemize}
                \item Classification: majority vote
                \item Regression: average
            \end{itemize}
        \end{enumerate}
    \end{defbox}
    
    \centering
    \begin{tikzpicture}[scale=0.7]
        % Original data
        \node[draw=accentcyan,rounded corners,fill=darkgray] (data) at (0,0) {Data: $\{1,2,3,4,5\}$};
        
        % Bootstrap samples
        \node[draw=neonpink,rounded corners,fill=darkgray,font=\small] (b1) at (-2.5,-1.5) {$\{1,1,3,4,5\}$};
        \node[draw=neonpink,rounded corners,fill=darkgray,font=\small] (b2) at (0,-1.5) {$\{2,2,3,3,5\}$};
        \node[draw=neonpink,rounded corners,fill=darkgray,font=\small] (b3) at (2.5,-1.5) {$\{1,3,4,5,5\}$};
        
        \draw[->,thick] (data) -- (b1);
        \draw[->,thick] (data) -- (b2);
        \draw[->,thick] (data) -- (b3);
        
        % Models
        \node[draw=neongreen,rounded corners,fill=darkgray,font=\small] (m1) at (-2.5,-3) {Tree 1};
        \node[draw=neongreen,rounded corners,fill=darkgray,font=\small] (m2) at (0,-3) {Tree 2};
        \node[draw=neongreen,rounded corners,fill=darkgray,font=\small] (m3) at (2.5,-3) {Tree 3};
        
        \draw[->,thick] (b1) -- (m1);
        \draw[->,thick] (b2) -- (m2);
        \draw[->,thick] (b3) -- (m3);
        
        % Aggregate
        \draw[->,thick] (m1) -- (0,-4);
        \draw[->,thick] (m2) -- (0,-4);
        \draw[->,thick] (m3) -- (0,-4);
        \node[draw=neonyellow,rounded corners,fill=darkgray] at (0,-4.5) {Aggregate};
    \end{tikzpicture}
\end{frame}

% -----------------------------------------------------------------------------
% Random Forest
% -----------------------------------------------------------------------------
\begin{frame}{Random Forest: Bagging + Feature Randomness}
    \begin{defbox}[Random Forest (Breiman, 2001)]
        Bagging + random feature selection at each split:
        
        \begin{itemize}
            \item At each node, consider only $m$ random features
            \item Typical $m$: $\sqrt{p}$ for classification, $p/3$ for regression
            \item This \textbf{decorrelates} the trees!
        \end{itemize}
    \end{defbox}
    
    \vspace{0.2cm}
    
    \begin{columns}
        \begin{column}{0.5\textwidth}
            \begin{keybox}[Why Random Features?]
                Without it:
                \begin{itemize}
                    \item All trees use same strong features
                    \item Trees are correlated
                    \item Averaging doesn't help much
                \end{itemize}
                
                With it:
                \begin{itemize}
                    \item Trees are diverse
                    \item Errors cancel out!
                \end{itemize}
            \end{keybox}
        \end{column}
        \begin{column}{0.5\textwidth}
            \begin{infobox}[Hyperparameters]
                \begin{itemize}
                    \item \texttt{n\_estimators}: \# of trees
                    \item \texttt{max\_features}: $m$
                    \item \texttt{max\_depth}: Tree depth
                    \item \texttt{min\_samples\_leaf}
                \end{itemize}
                
                More trees = better (diminishing returns)
            \end{infobox}
        \end{column}
    \end{columns}
\end{frame}

% -----------------------------------------------------------------------------
% Boosting Intuition
% -----------------------------------------------------------------------------
\begin{frame}{Boosting: Sequential Learning}
    \begin{defbox}[Boosting Intuition]
        Train models \textbf{sequentially}, each focusing on mistakes of previous ones.
        
        \centering
        \begin{tikzpicture}[scale=0.7]
            % Model 1
            \node[draw=accentcyan,rounded corners,fill=darkgray] (m1) at (0,0) {Model 1};
            \node[below,font=\small,fgwhite] at (0,-0.5) {Train on all data};
            
            % Errors
            \draw[->,thick,neonpink] (1.5,0) -- (3,0) node[midway,above,font=\tiny] {errors};
            
            % Model 2
            \node[draw=accentcyan,rounded corners,fill=darkgray] (m2) at (4.5,0) {Model 2};
            \node[below,font=\small,fgwhite] at (4.5,-0.5) {Focus on errors};
            
            % More errors
            \draw[->,thick,neonpink] (6,0) -- (7.5,0) node[midway,above,font=\tiny] {errors};
            
            % Model 3
            \node[draw=accentcyan,rounded corners,fill=darkgray] (m3) at (9,0) {Model 3};
            \node[below,font=\small,fgwhite] at (9,-0.5) {...};
        \end{tikzpicture}
    \end{defbox}
    
    \vspace{0.3cm}
    
    \begin{columns}
        \begin{column}{0.5\textwidth}
            \begin{keybox}[Bagging vs Boosting]
                \textbf{Bagging}: Parallel, reduce variance
                
                \textbf{Boosting}: Sequential, reduce bias
            \end{keybox}
        \end{column}
        \begin{column}{0.5\textwidth}
            \begin{funnybox}
                Boosting: ``You got this wrong? Let me send my friend who's good at exactly that!''
            \end{funnybox}
        \end{column}
    \end{columns}
\end{frame}

% -----------------------------------------------------------------------------
% AdaBoost
% -----------------------------------------------------------------------------
\begin{frame}{AdaBoost: Adaptive Boosting}
    \begin{defbox}[AdaBoost (Freund \& Schapire, 1997)]
        \begin{enumerate}
            \item Initialize sample weights: $w_i = 1/N$
            \item For $t = 1, \ldots, T$:
            \begin{enumerate}
                \item Train weak learner on weighted data
                \item Compute weighted error: $\epsilon_t = \sum_{wrong} w_i$
                \item Compute model weight: $\alpha_t = \frac{1}{2} \ln\left(\frac{1-\epsilon_t}{\epsilon_t}\right)$
                \item Update sample weights: increase for wrong, decrease for correct
            \end{enumerate}
            \item Final: $H(x) = \text{sign}\left(\sum_t \alpha_t h_t(x)\right)$
        \end{enumerate}
    \end{defbox}
    
    \begin{keybox}
        Better weak learners get higher $\alpha$. Misclassified samples get higher weight.
    \end{keybox}
\end{frame}

% -----------------------------------------------------------------------------
% Gradient Boosting
% -----------------------------------------------------------------------------
\begin{frame}{Gradient Boosting: Boosting as Gradient Descent}
    \begin{defbox}[Gradient Boosting]
        Each new model fits the \textbf{residuals} (gradient of loss):
        
        \begin{enumerate}
            \item Initialize: $F_0(x) = \arg\min_c \sum_i L(y_i, c)$
            \item For $t = 1, \ldots, T$:
            \begin{enumerate}
                \item Compute residuals: $r_i = -\frac{\partial L(y_i, F_{t-1}(x_i))}{\partial F_{t-1}}$
                \item Fit tree $h_t$ to residuals
                \item Update: $F_t(x) = F_{t-1}(x) + \eta \cdot h_t(x)$
            \end{enumerate}
        \end{enumerate}
    \end{defbox}
    
    \vspace{0.2cm}
    
    \begin{columns}
        \begin{column}{0.5\textwidth}
            \textbf{For MSE loss:}
            
            Residual = $y_i - F_{t-1}(x_i)$
            
            (actual minus prediction)
        \end{column}
        \begin{column}{0.5\textwidth}
            \begin{funnybox}
                Each tree says: ``Here's how much you're off by. Let me fix that!''
            \end{funnybox}
        \end{column}
    \end{columns}
\end{frame}

% -----------------------------------------------------------------------------
% XGBoost
% -----------------------------------------------------------------------------
\begin{frame}{XGBoost: Gradient Boosting on Steroids}
    \begin{defbox}[XGBoost (Chen \& Guestrin, 2016)]
        Key improvements over vanilla gradient boosting:
        
        \begin{itemize}
            \item \textbf{Regularized objective}: $L = \sum_i l(y_i, \hat{y}_i) + \sum_k \Omega(f_k)$
            \item \textbf{Second-order approximation}: Uses Hessian
            \item \textbf{Efficient implementation}: Parallel, cache-aware
            \item \textbf{Handling missing values}: Built-in
            \item \textbf{Column subsampling}: Like random forest
        \end{itemize}
    \end{defbox}
    
    \vspace{0.2cm}
    
    \begin{successbox}
        XGBoost often wins Kaggle competitions!
        
        Also: \textbf{LightGBM} (Microsoft) and \textbf{CatBoost} (Yandex) are popular alternatives.
    \end{successbox}
\end{frame}

% -----------------------------------------------------------------------------
% Comparison
% -----------------------------------------------------------------------------
\begin{frame}{Ensemble Methods Comparison}
    \centering
    \begin{tabular}{lccc}
        & \textbf{Random Forest} & \textbf{AdaBoost} & \textbf{XGBoost} \\
        \hline
        Training & Parallel & Sequential & Sequential \\
        Reduces & Variance & Bias & Both \\
        Overfitting & Robust & Can overfit & Regularized \\
        Tuning & Easy & Medium & Complex \\
        Interpretable & Somewhat & Somewhat & Less \\
        Speed & Fast & Medium & Fast \\
    \end{tabular}
    
    \vspace{0.5cm}
    
    \begin{keybox}[Rules of Thumb]
        \begin{itemize}
            \item \textbf{Start with}: Random Forest (robust, few hyperparams)
            \item \textbf{For maximum performance}: XGBoost/LightGBM
            \item \textbf{For interpretability}: Single tree (with pruning)
        \end{itemize}
    \end{keybox}
\end{frame}

% -----------------------------------------------------------------------------
% Code
% -----------------------------------------------------------------------------
\begin{frame}{Ensemble Methods in Python}
    \begin{successbox}[Scikit-learn \& XGBoost]
        {\small\ttfamily
        from sklearn.ensemble import RandomForestClassifier\\
        from sklearn.ensemble import GradientBoostingClassifier\\
        import xgboost as xgb\\[0.3em]
        \# Random Forest\\
        rf = RandomForestClassifier(n\_estimators=100, max\_depth=10)\\
        rf.fit(X\_train, y\_train)\\[0.3em]
        \# Gradient Boosting\\
        gb = GradientBoostingClassifier(n\_estimators=100,\\
        \quad\quad\quad\quad\quad\quad\quad\quad\quad\quad learning\_rate=0.1)\\
        gb.fit(X\_train, y\_train)\\[0.3em]
        \# XGBoost\\
        xgb\_model = xgb.XGBClassifier(n\_estimators=100,\\
        \quad\quad\quad\quad\quad\quad\quad\quad\quad\quad learning\_rate=0.1, max\_depth=6)\\
        xgb\_model.fit(X\_train, y\_train)
        }
    \end{successbox}
\end{frame}

% -----------------------------------------------------------------------------
% Key Takeaways
% -----------------------------------------------------------------------------
\begin{frame}{Key Takeaways: Ensemble Methods}
    \begin{keybox}
        \begin{enumerate}
            \item \textbf{Ensemble} = combine multiple weak learners
            \item \textbf{Bagging}: Train in parallel on bootstrap samples
            \begin{itemize}
                \item Reduces variance
            \end{itemize}
            \item \textbf{Random Forest}: Bagging + random feature subsets
            \begin{itemize}
                \item Decorrelates trees
            \end{itemize}
            \item \textbf{Boosting}: Train sequentially, focus on errors
            \begin{itemize}
                \item Reduces bias
            \end{itemize}
            \item \textbf{AdaBoost}: Weight samples and models
            \item \textbf{Gradient Boosting}: Fit residuals (gradients)
            \item \textbf{XGBoost}: Regularized, efficient GB
        \end{enumerate}
    \end{keybox}
    
    \centering
    \textit{Next: Statistical Learning Theory — why does any of this work?}
\end{frame}


% Section 18: Statistical Learning Theory
% =============================================================================
% Section 18: Statistical Learning Theory
% PAC Learning, VC Dimension, Generalization
% =============================================================================

\section{Statistical Learning Theory}

% -----------------------------------------------------------------------------
% Opening
% -----------------------------------------------------------------------------
\begin{frame}{Statistical Learning Theory: Why ML Works}
    \begin{columns}[T]
        \begin{column}{0.5\textwidth}
            \begin{funnybox}
                \textit{``We've been fitting models and hoping they work. Let's understand WHY they work (or don't)!''}
            \end{funnybox}
            
            \vspace{0.3cm}
            
            \textbf{The Big Questions:}
            \begin{itemize}
                \item When can we learn from data?
                \item How much data do we need?
                \item Why does training error $\neq$ test error?
            \end{itemize}
        \end{column}
        \begin{column}{0.5\textwidth}
            \centering
            \begin{tikzpicture}[scale=0.7]
                \draw[->] (0,0) -- (4,0) node[right] {samples};
                \draw[->] (0,0) -- (0,3) node[above] {error};
                
                % Generalization gap
                \draw[thick,neonpink,domain=0.3:3.8,samples=50] 
                    plot (\x, {0.3 + 1.5/\x});
                \node[neonpink,font=\tiny] at (3.5,1.3) {test};
                
                \draw[thick,neongreen,domain=0.3:3.8,samples=50] 
                    plot (\x, {0.2 + 0.8/\x});
                \node[neongreen,font=\tiny] at (3.5,0.7) {train};
                
                % Gap
                \draw[<->,thick,neonyellow] (2,0.6) -- (2,1.05);
                \node[neonyellow,right,font=\tiny] at (2.1,0.8) {gap};
            \end{tikzpicture}
            
            \textit{Generalization gap decreases with more data}
        \end{column}
    \end{columns}
\end{frame}

% -----------------------------------------------------------------------------
% PAC Learning
% -----------------------------------------------------------------------------
\begin{frame}{PAC Learning: Probably Approximately Correct}
    \begin{defbox}[PAC Learning (Valiant, 1984)]
        A concept class $\mathcal{C}$ is \glow{PAC-learnable} if there exists an algorithm that:
        
        For any distribution $\mathcal{D}$, any $\epsilon > 0$, any $\delta > 0$:
        
        Given $m$ samples from $\mathcal{D}$, the algorithm outputs $h$ such that:
        \[
            P\left[\text{error}(h) \leq \epsilon\right] \geq 1 - \delta
        \]
        
        with $m = \text{poly}(1/\epsilon, 1/\delta, n, \text{size}(c))$
    \end{defbox}
    
    \vspace{0.2cm}
    
    \begin{columns}
        \begin{column}{0.5\textwidth}
            \begin{keybox}[Translation]
                \textbf{Probably}: With high probability ($1-\delta$)
                
                \textbf{Approximately}: Low error ($\leq \epsilon$)
                
                \textbf{Correct}: On new data!
            \end{keybox}
        \end{column}
        \begin{column}{0.5\textwidth}
            \begin{funnybox}
                ``Give me enough data, and I'll \textit{probably} give you something \textit{approximately} right!''
            \end{funnybox}
        \end{column}
    \end{columns}
\end{frame}

% -----------------------------------------------------------------------------
% Generalization Error
% -----------------------------------------------------------------------------
\begin{frame}{Generalization Error Decomposition}
    \begin{defbox}[Error Types]
        \begin{itemize}
            \item \textbf{Training error}: Error on training data
            \[
                \hat{R}(h) = \frac{1}{n}\sum_{i=1}^{n} \mathbf{1}[h(x_i) \neq y_i]
            \]
            
            \item \textbf{Generalization error}: Expected error on new data
            \[
                R(h) = \mathbb{E}_{(x,y) \sim \mathcal{D}}[\mathbf{1}[h(x) \neq y]]
            \]
        \end{itemize}
    \end{defbox}
    
    \vspace{0.3cm}
    
    \begin{keybox}[The Gap]
        \[
            \underbrace{R(h)}_{\text{test error}} = \underbrace{\hat{R}(h)}_{\text{train error}} + \underbrace{(R(h) - \hat{R}(h))}_{\text{generalization gap}}
        \]
    \end{keybox}
\end{frame}

% -----------------------------------------------------------------------------
% Finite Hypothesis Classes
% -----------------------------------------------------------------------------
\begin{frame}{Generalization Bound (Finite Hypothesis Class)}
    \begin{thmbox}[Finite Hypothesis Bound]
        For a finite hypothesis class $|\mathcal{H}|$, with probability $\geq 1-\delta$:
        \[
            R(h) \leq \hat{R}(h) + \sqrt{\frac{\ln|\mathcal{H}| + \ln(1/\delta)}{2n}}
        \]
    \end{thmbox}
    
    \vspace{0.2cm}
    
    \begin{columns}
        \begin{column}{0.5\textwidth}
            \textbf{Implications:}
            \begin{itemize}
                \item More hypotheses → larger gap
                \item More data $n$ → smaller gap
                \item Higher confidence $1/\delta$ → larger gap
            \end{itemize}
        \end{column}
        \begin{column}{0.5\textwidth}
            \begin{alertbox}[Problem]
                Neural networks have infinite hypothesis classes!
                
                This bound doesn't directly apply...
            \end{alertbox}
        \end{column}
    \end{columns}
\end{frame}

% -----------------------------------------------------------------------------
% VC Dimension
% -----------------------------------------------------------------------------
\begin{frame}{VC Dimension: Measuring Model Complexity}
    \begin{defbox}[Shattering]
        A hypothesis class $\mathcal{H}$ \glow{shatters} a set of points if it can realize ALL possible labelings of those points.
    \end{defbox}
    
    \begin{defbox}[VC Dimension]
        The \glow{VC dimension} of $\mathcal{H}$ is the maximum number of points that can be shattered by $\mathcal{H}$.
    \end{defbox}
    
    \centering
    \begin{tikzpicture}[scale=0.6]
        % 3 points that can be shattered by linear classifiers
        \node[font=\small,accentcyan] at (2,3) {Linear classifiers in 2D: VC = 3};
        
        \foreach \config/\xshift in {1/0, 2/4, 3/8} {
            \begin{scope}[xshift=\xshift cm]
                \fill[neongreen] (0.5,0.5) circle (4pt);
                \fill[neongreen] (1.5,0.5) circle (4pt);
                \fill[neonpink] (1,1.5) circle (4pt);
                \draw[thick,neonyellow] (0,1) -- (2,1);
            \end{scope}
        }
        
        \node[below,font=\tiny,fgwhite] at (5,-0.3) {Can separate any 3 points...};
    \end{tikzpicture}
    
    \begin{tikzpicture}[scale=0.6]
        % 4 points cannot be shattered (XOR)
        \fill[neongreen] (0.5,0.5) circle (4pt);
        \fill[neongreen] (1.5,1.5) circle (4pt);
        \fill[neonpink] (0.5,1.5) circle (4pt);
        \fill[neonpink] (1.5,0.5) circle (4pt);
        
        \node[right,font=\tiny,fgwhite] at (2,1) {Can't separate XOR with a line!};
    \end{tikzpicture}
\end{frame}

% -----------------------------------------------------------------------------
% VC Bound
% -----------------------------------------------------------------------------
\begin{frame}{VC Generalization Bound}
    \begin{thmbox}[VC Bound]
        For hypothesis class with VC dimension $d$, with probability $\geq 1-\delta$:
        \[
            R(h) \leq \hat{R}(h) + \sqrt{\frac{d(\ln(2n/d) + 1) + \ln(4/\delta)}{n}}
        \]
    \end{thmbox}
    
    \vspace{0.3cm}
    
    \begin{columns}
        \begin{column}{0.5\textwidth}
            \begin{keybox}[Implications]
                \begin{itemize}
                    \item Higher VC dim → worse generalization
                    \item Need $n \gg d$ for good bounds
                    \item Model complexity matters!
                \end{itemize}
            \end{keybox}
        \end{column}
        \begin{column}{0.5\textwidth}
            \begin{infobox}[VC Examples]
                \begin{itemize}
                    \item Linear in $\mathbb{R}^d$: VC = $d+1$
                    \item Intervals on $\mathbb{R}$: VC = 2
                    \item Axis-aligned rectangles: VC = 4
                    \item Neural nets: Depends on depth/width
                \end{itemize}
            \end{infobox}
        \end{column}
    \end{columns}
\end{frame}

% -----------------------------------------------------------------------------
% Bias-Variance
% -----------------------------------------------------------------------------
\begin{frame}{Bias-Variance Tradeoff (Formal)}
    \begin{defbox}[MSE Decomposition]
        For any estimator $\hat{f}$:
        \[
            \mathbb{E}[(y - \hat{f}(x))^2] = \underbrace{\text{Bias}[\hat{f}(x)]^2}_{\text{systematic error}} + \underbrace{\text{Var}[\hat{f}(x)]}_{\text{sensitivity to training}} + \underbrace{\sigma^2}_{\text{irreducible}}
        \]
    \end{defbox}
    
    \centering
    \begin{tikzpicture}[scale=0.5]
        \draw[->] (0,0) -- (5,0) node[right] {complexity};
        \draw[->] (0,0) -- (0,3.5) node[above] {error};
        
        % Bias
        \draw[thick,neongreen,domain=0.3:4.7,samples=50] 
            plot (\x, {2.5/\x});
        \node[neongreen,font=\small] at (4.5,1) {Bias$^2$};
        
        % Variance
        \draw[thick,neonpink,domain=0.3:4.7,samples=50] 
            plot (\x, {0.1*\x*\x});
        \node[neonpink,font=\small] at (4.5,2.5) {Variance};
        
        % Total
        \draw[thick,neonyellow,domain=0.3:4.7,samples=50] 
            plot (\x, {2.5/\x + 0.1*\x*\x});
        \node[neonyellow,font=\small] at (2.5,3.2) {Total};
        
        % Optimal
        \draw[dashed,accentcyan] (2,0) -- (2,3);
        \node[accentcyan,above,font=\tiny] at (2,3) {optimal};
    \end{tikzpicture}
\end{frame}

% -----------------------------------------------------------------------------
% No Free Lunch
% -----------------------------------------------------------------------------
\begin{frame}{No Free Lunch Theorem}
    \begin{thmbox}[No Free Lunch (Wolpert \& Macready)]
        Averaged over ALL possible problems, every learning algorithm performs equally!
        
        \[
            \sum_f P(d_m | f, A_1) = \sum_f P(d_m | f, A_2)
        \]
        
        for any algorithms $A_1, A_2$ and any performance measure.
    \end{thmbox}
    
    \vspace{0.3cm}
    
    \begin{columns}
        \begin{column}{0.5\textwidth}
            \begin{funnybox}
                ``No algorithm is universally better. Some are just better for YOUR problem!''
            \end{funnybox}
        \end{column}
        \begin{column}{0.5\textwidth}
            \begin{keybox}[Implications]
                \begin{itemize}
                    \item Domain knowledge matters
                    \item Try multiple algorithms
                    \item Inductive bias is crucial
                \end{itemize}
            \end{keybox}
        \end{column}
    \end{columns}
\end{frame}

% -----------------------------------------------------------------------------
% Sample Complexity
% -----------------------------------------------------------------------------
\begin{frame}{Sample Complexity: How Much Data?}
    \begin{defbox}[Sample Complexity]
        Minimum samples $m$ needed to achieve error $\leq \epsilon$ with probability $\geq 1-\delta$.
    \end{defbox}
    
    \vspace{0.2cm}
    
    \textbf{For PAC-learnable classes:}
    \[
        m \geq \frac{1}{\epsilon}\left(d \ln\frac{1}{\epsilon} + \ln\frac{1}{\delta}\right)
    \]
    
    where $d$ is VC dimension.
    
    \vspace{0.3cm}
    
    \begin{columns}
        \begin{column}{0.5\textwidth}
            \begin{infobox}[Rule of Thumb]
                For neural nets, people often use:
                \[
                    n \geq 10 \times \text{(\# parameters)}
                \]
                
                (Very rough heuristic!)
            \end{infobox}
        \end{column}
        \begin{column}{0.5\textwidth}
            \begin{alertbox}[Modern Paradox]
                Deep learning often works with:
                \[
                    n \ll \text{(\# parameters)}
                \]
                
                Theory is still catching up!
            \end{alertbox}
        \end{column}
    \end{columns}
\end{frame}

% -----------------------------------------------------------------------------
% Key Takeaways
% -----------------------------------------------------------------------------
\begin{frame}{Key Takeaways: Statistical Learning Theory}
    \begin{keybox}
        \begin{enumerate}
            \item \textbf{PAC learning}: Probably Approximately Correct framework
            \item \textbf{Generalization gap}: Test error - Train error
            \item \textbf{VC dimension}: Measure of model complexity
            \begin{itemize}
                \item Max points a model can shatter
            \end{itemize}
            \item \textbf{Generalization bounds}: Test error bounded by train error + complexity term
            \item \textbf{Bias-Variance tradeoff}: 
            \begin{itemize}
                \item Simple models: high bias, low variance
                \item Complex models: low bias, high variance
            \end{itemize}
            \item \textbf{No Free Lunch}: No universal best algorithm
            \item \textbf{Sample complexity}: More complex model → more data needed
        \end{enumerate}
    \end{keybox}
    
    \centering
    \textit{Next: KL Divergence and Information Theory!}
\end{frame}


% Section 19: KL Divergence & Information Theory
% =============================================================================
% Section 19: KL Divergence & Information Theory
% Entropy, cross-entropy, and KL divergence in ML
% =============================================================================

\section{KL Divergence \& Information Theory}

% -----------------------------------------------------------------------------
% Opening
% -----------------------------------------------------------------------------
\begin{frame}{Information Theory: The Math of Surprise}
    \begin{columns}[T]
        \begin{column}{0.5\textwidth}
            \begin{funnybox}
                \textit{``Information theory tells us: rare events are surprising, common events are boring. Just like plot twists!''}
            \end{funnybox}
            
            \vspace{0.3cm}
            
            \textbf{Why for ML?}
            \begin{itemize}
                \item Cross-entropy loss
                \item Measuring distribution similarity
                \item Understanding predictions
            \end{itemize}
        \end{column}
        \begin{column}{0.5\textwidth}
            \begin{defbox}[Information Content]
                The \glow{information} (surprise) of event with probability $p$:
                \[
                    I(x) = -\log_2(p(x)) \text{ bits}
                \]
                
                \vspace{0.2cm}
                
                \begin{itemize}
                    \item $p = 1$ (certain): 0 bits
                    \item $p = 0.5$: 1 bit
                    \item $p = 0.01$: 6.64 bits
                \end{itemize}
            \end{defbox}
        \end{column}
    \end{columns}
\end{frame}

% -----------------------------------------------------------------------------
% Entropy
% -----------------------------------------------------------------------------
\begin{frame}{Entropy: Average Surprise}
    \begin{defbox}[Shannon Entropy]
        The \glow{entropy} of a distribution $P$ is the expected information:
        \[
            H(P) = -\sum_{x} p(x) \log p(x) = \mathbb{E}_{x \sim P}[-\log p(x)]
        \]
    \end{defbox}
    
    \vspace{0.2cm}
    
    \begin{columns}
        \begin{column}{0.5\textwidth}
            \textbf{Binary entropy:}
            \centering
            \begin{tikzpicture}[scale=0.6]
                \draw[->] (0,0) -- (4,0) node[right] {$p$};
                \draw[->] (0,0) -- (0,2.5) node[above] {$H$};
                
                \draw[thick,neonpink,domain=0.01:0.99,samples=50] 
                    plot (\x*4, {-\x*ln(\x)/ln(2) - (1-\x)*ln(1-\x)/ln(2)});
                
                \fill[neongreen] (0,0) circle (2pt);
                \fill[neongreen] (4,0) circle (2pt);
                \fill[neonyellow] (2,2) circle (2pt);
                
                \node[below,font=\tiny] at (2,0) {0.5};
                \node[above,font=\tiny,neonyellow] at (2,2) {max};
            \end{tikzpicture}
        \end{column}
        \begin{column}{0.5\textwidth}
            \begin{keybox}[Properties]
                \begin{itemize}
                    \item $H \geq 0$ always
                    \item Max when uniform
                    \item Min (= 0) when deterministic
                    \item Units: bits (log$_2$) or nats (ln)
                \end{itemize}
            \end{keybox}
        \end{column}
    \end{columns}
\end{frame}

% -----------------------------------------------------------------------------
% Cross-Entropy
% -----------------------------------------------------------------------------
\begin{frame}{Cross-Entropy: Comparing Distributions}
    \begin{defbox}[Cross-Entropy]
        The \glow{cross-entropy} between true distribution $P$ and predicted distribution $Q$:
        \[
            H(P, Q) = -\sum_{x} p(x) \log q(x) = \mathbb{E}_{x \sim P}[-\log q(x)]
        \]
    \end{defbox}
    
    \vspace{0.3cm}
    
    \begin{columns}
        \begin{column}{0.5\textwidth}
            \begin{keybox}[Interpretation]
                Average bits needed to encode data from $P$ using code optimized for $Q$.
                
                \vspace{0.2cm}
                
                If $Q \neq P$: We're using a ``wrong'' code!
            \end{keybox}
        \end{column}
        \begin{column}{0.5\textwidth}
            \begin{infobox}[In ML]
                \begin{itemize}
                    \item $P$ = true labels (one-hot)
                    \item $Q$ = predicted probabilities
                    \item $H(P,Q)$ = cross-entropy loss!
                \end{itemize}
            \end{infobox}
        \end{column}
    \end{columns}
    
    \vspace{0.2cm}
    
    \[
        \mathcal{L}_{CE} = -\sum_{c} y_c \log(\hat{y}_c) = H(\mathbf{y}, \hat{\mathbf{y}})
    \]
\end{frame}

% -----------------------------------------------------------------------------
% KL Divergence Definition
% -----------------------------------------------------------------------------
\begin{frame}{KL Divergence: How Different Are Two Distributions?}
    \begin{defbox}[Kullback-Leibler Divergence]
        The \glow{KL divergence} from $Q$ to $P$:
        \[
            D_{KL}(P \| Q) = \sum_{x} p(x) \log \frac{p(x)}{q(x)} = \mathbb{E}_{x \sim P}\left[\log \frac{p(x)}{q(x)}\right]
        \]
    \end{defbox}
    
    \vspace{0.2cm}
    
    \begin{keybox}[Key Relationship]
        \[
            D_{KL}(P \| Q) = H(P, Q) - H(P)
        \]
        
        Cross-entropy = Entropy + KL divergence
        
        \vspace{0.2cm}
        
        When minimizing cross-entropy with fixed $P$, we're minimizing KL divergence!
    \end{keybox}
\end{frame}

% -----------------------------------------------------------------------------
% KL Properties
% -----------------------------------------------------------------------------
\begin{frame}{Properties of KL Divergence}
    \begin{columns}[T]
        \begin{column}{0.5\textwidth}
            \begin{thmbox}[Gibbs' Inequality]
                \[
                    D_{KL}(P \| Q) \geq 0
                \]
                
                with equality iff $P = Q$ (almost everywhere)
            \end{thmbox}
            
            \begin{alertbox}[Not Symmetric!]
                \[
                    D_{KL}(P \| Q) \neq D_{KL}(Q \| P)
                \]
                
                KL divergence is NOT a distance metric!
            \end{alertbox}
        \end{column}
        \begin{column}{0.5\textwidth}
            \centering
            \begin{tikzpicture}[scale=0.7]
                % P distribution
                \draw[thick,neongreen,domain=-2:2,samples=50] 
                    plot (\x+2.5, {1.5*exp(-\x*\x)});
                \node[neongreen,font=\small] at (2.5,2) {$P$};
                
                % Q distribution (shifted)
                \draw[thick,neonpink,domain=-2:2,samples=50] 
                    plot (\x+3.5, {1.2*exp(-0.8*\x*\x)});
                \node[neonpink,font=\small] at (3.5,1.8) {$Q$};
                
                \draw[->] (0,0) -- (6,0) node[right] {$x$};
                \draw[->] (0,0) -- (0,2) node[above] {};
                
                % KL arrow
                \draw[<->,thick,neonyellow] (2.5,0.3) -- (3.5,0.3);
                \node[neonyellow,below,font=\tiny] at (3,0.3) {$D_{KL}$};
            \end{tikzpicture}
            
            \begin{funnybox}
                $D_{KL}(P\|Q)$ asks: ``How surprised is $Q$ by data from $P$?''
            \end{funnybox}
        \end{column}
    \end{columns}
\end{frame}

% -----------------------------------------------------------------------------
% Forward vs Reverse KL
% -----------------------------------------------------------------------------
\begin{frame}{Forward vs Reverse KL}
    \begin{columns}[T]
        \begin{column}{0.5\textwidth}
            \begin{defbox}[Forward KL: $D_{KL}(P \| Q)$]
                Minimize: $\mathbb{E}_P[\log P - \log Q]$
                
                \begin{itemize}
                    \item $Q$ must cover all of $P$
                    \item \textbf{Mean-seeking}
                    \item Used in variational inference (ELBO)
                \end{itemize}
                
                \centering
                \begin{tikzpicture}[scale=0.5]
                    % Bimodal P
                    \draw[thick,neongreen,domain=-2:2,samples=50] 
                        plot (\x, {0.7*exp(-4*(\x+0.7)*(\x+0.7)) + 0.7*exp(-4*(\x-0.7)*(\x-0.7))});
                    % Q covers both
                    \draw[thick,neonpink,domain=-2:2,samples=50] 
                        plot (\x, {0.6*exp(-0.5*\x*\x)});
                    \draw[->] (-2,0) -- (2,0);
                \end{tikzpicture}
            \end{defbox}
        \end{column}
        \begin{column}{0.5\textwidth}
            \begin{defbox}[Reverse KL: $D_{KL}(Q \| P)$]
                Minimize: $\mathbb{E}_Q[\log Q - \log P]$
                
                \begin{itemize}
                    \item $Q$ can ignore parts of $P$
                    \item \textbf{Mode-seeking}
                    \item Used in policy gradient, GANs
                \end{itemize}
                
                \centering
                \begin{tikzpicture}[scale=0.5]
                    % Bimodal P
                    \draw[thick,neongreen,domain=-2:2,samples=50] 
                        plot (\x, {0.7*exp(-4*(\x+0.7)*(\x+0.7)) + 0.7*exp(-4*(\x-0.7)*(\x-0.7))});
                    % Q picks one mode
                    \draw[thick,neonpink,domain=-2:2,samples=50] 
                        plot (\x, {0.9*exp(-4*(\x-0.7)*(\x-0.7))});
                    \draw[->] (-2,0) -- (2,0);
                \end{tikzpicture}
            \end{defbox}
        \end{column}
    \end{columns}
\end{frame}

% -----------------------------------------------------------------------------
% KL in ML Applications
% -----------------------------------------------------------------------------
\begin{frame}{KL Divergence in Machine Learning}
    \begin{enumerate}
        \item \textbf{Cross-entropy loss}: Minimizes $D_{KL}(\text{true} \| \text{pred})$
        
        \item \textbf{Variational Autoencoders (VAE)}:
        \[
            \mathcal{L}_{VAE} = \text{reconstruction} + D_{KL}(q(z|x) \| p(z))
        \]
        
        \item \textbf{Knowledge Distillation}:
        \[
            \mathcal{L}_{KD} = D_{KL}(\text{teacher} \| \text{student})
        \]
        
        \item \textbf{Reinforcement Learning}: Policy optimization bounds
        
        \item \textbf{Bayesian inference}: Variational approximations
    \end{enumerate}
    
    \begin{keybox}
        KL divergence is everywhere in modern ML — it measures how ``different'' one distribution is from another!
    \end{keybox}
\end{frame}

% -----------------------------------------------------------------------------
% Mutual Information
% -----------------------------------------------------------------------------
\begin{frame}{Mutual Information: Shared Information}
    \begin{defbox}[Mutual Information]
        \[
            I(X; Y) = D_{KL}(P(X,Y) \| P(X)P(Y)) = H(X) - H(X|Y)
        \]
        
        How much knowing $Y$ tells us about $X$.
    \end{defbox}
    
    \vspace{0.2cm}
    
    \begin{columns}
        \begin{column}{0.5\textwidth}
            \centering
            \begin{tikzpicture}[scale=0.7]
                % Venn diagram
                \draw[thick,fill=neongreen,opacity=0.3] (-0.5,0) circle (1.2);
                \draw[thick,fill=neonpink,opacity=0.3] (0.5,0) circle (1.2);
                
                \node[neongreen] at (-1.2,0) {$H(X)$};
                \node[neonpink] at (1.2,0) {$H(Y)$};
                \node[neonyellow] at (0,0) {$I$};
            \end{tikzpicture}
        \end{column}
        \begin{column}{0.5\textwidth}
            \begin{keybox}[Properties]
                \begin{itemize}
                    \item $I(X;Y) \geq 0$
                    \item $I(X;Y) = I(Y;X)$ (symmetric!)
                    \item $I(X;Y) = 0$ iff independent
                    \item $I(X;X) = H(X)$
                \end{itemize}
            \end{keybox}
        \end{column}
    \end{columns}
    
    \vspace{0.2cm}
    
    \begin{infobox}
        Used in: feature selection, information bottleneck, representation learning
    \end{infobox}
\end{frame}

% -----------------------------------------------------------------------------
% Summary Table
% -----------------------------------------------------------------------------
\begin{frame}{Information Theory Summary}
    \centering
    \begin{tabular}{lll}
        \textbf{Quantity} & \textbf{Formula} & \textbf{Meaning} \\
        \hline
        Information & $-\log p(x)$ & Surprise of event \\[0.5em]
        Entropy $H(P)$ & $\mathbb{E}_P[-\log p]$ & Average surprise \\[0.5em]
        Cross-entropy & $\mathbb{E}_P[-\log q]$ & Bits using wrong code \\[0.5em]
        KL divergence & $\mathbb{E}_P[\log p/q]$ & Distribution difference \\[0.5em]
        Mutual info & $D_{KL}(P_{XY} \| P_X P_Y)$ & Shared information \\
    \end{tabular}
    
    \vspace{0.5cm}
    
    \begin{keybox}[Key Relationships]
        \begin{align*}
            H(P,Q) &= H(P) + D_{KL}(P \| Q) \\
            I(X;Y) &= H(X) + H(Y) - H(X,Y) \\
            D_{KL}(P \| Q) &\geq 0 \text{ with equality iff } P = Q
        \end{align*}
    \end{keybox}
\end{frame}

% -----------------------------------------------------------------------------
% Key Takeaways
% -----------------------------------------------------------------------------
\begin{frame}{Key Takeaways: Information Theory}
    \begin{keybox}
        \begin{enumerate}
            \item \textbf{Information}: $-\log p$ — rare events have more info
            \item \textbf{Entropy}: Average information = uncertainty measure
            \item \textbf{Cross-entropy}: Using ``wrong'' distribution for encoding
            \begin{itemize}
                \item This is our classification loss!
            \end{itemize}
            \item \textbf{KL divergence}: Measures distribution difference
            \begin{itemize}
                \item Not symmetric, not a true distance
                \item Forward vs reverse have different behaviors
            \end{itemize}
            \item \textbf{Mutual information}: Shared information between variables
            \item Minimizing cross-entropy = minimizing KL from true distribution
        \end{enumerate}
    \end{keybox}
    
    \centering
    \textit{Next: The classic MNIST dataset!}
\end{frame}


% =============================================================================
% PART V: PUTTING IT ALL TOGETHER
% =============================================================================

% Section 20: MNIST Dataset
% =============================================================================
% Section 20: MNIST Dataset
% The "Hello World" of Machine Learning
% =============================================================================

\section{MNIST Dataset}

% -----------------------------------------------------------------------------
% Opening
% -----------------------------------------------------------------------------
\begin{frame}{MNIST: The ``Hello World'' of ML}
    \begin{columns}[T]
        \begin{column}{0.5\textwidth}
            \begin{funnybox}
                \textit{``Every ML journey starts with MNIST. It's tradition. Like 'Hello World' for programmers, but with more pixels.''}
            \end{funnybox}
            
            \vspace{0.3cm}
            
            \textbf{What is MNIST?}
            \begin{itemize}
                \item Handwritten digit images
                \item Collected from Census Bureau
                \item Created by Yann LeCun, 1998
                \item The benchmark for decades
            \end{itemize}
        \end{column}
        \begin{column}{0.5\textwidth}
            \centering
            \begin{tikzpicture}[scale=0.6]
                % Grid of digits (simplified representation)
                \foreach \i in {0,...,2} {
                    \foreach \j in {0,...,2} {
                        \draw[thick,accentcyan,rounded corners] (\i*1.5,\j*1.5) rectangle (\i*1.5+1.2,\j*1.5+1.2);
                        \pgfmathtruncatemacro{\digit}{mod(\i+\j*3,10)}
                        \node[font=\Large,fgwhite] at (\i*1.5+0.6,\j*1.5+0.6) {\digit};
                    }
                }
                \node[below,fgwhite,font=\small] at (1.8,-0.3) {28×28 grayscale images};
            \end{tikzpicture}
        \end{column}
    \end{columns}
\end{frame}

% -----------------------------------------------------------------------------
% Dataset Details
% -----------------------------------------------------------------------------
\begin{frame}{MNIST Dataset Details}
    \begin{defbox}[MNIST Specifications]
        \begin{columns}
            \begin{column}{0.5\textwidth}
                \begin{itemize}
                    \item \textbf{Training set}: 60,000 images
                    \item \textbf{Test set}: 10,000 images
                    \item \textbf{Image size}: 28 × 28 pixels
                    \item \textbf{Channels}: 1 (grayscale)
                    \item \textbf{Pixel values}: 0-255 (usually normalized)
                \end{itemize}
            \end{column}
            \begin{column}{0.5\textwidth}
                \begin{itemize}
                    \item \textbf{Classes}: 10 (digits 0-9)
                    \item \textbf{Input dimension}: 784 (28×28)
                    \item \textbf{Output}: One-hot or class label
                    \item \textbf{Format}: IDX file format
                    \item \textbf{Size}: ~12 MB total
                \end{itemize}
            \end{column}
        \end{columns}
    \end{defbox}
    
    \vspace{0.3cm}
    
    \begin{keybox}
        \textbf{As a vector:} Each image $\rightarrow$ 784-dimensional vector
        
        \textbf{Input shape:} $(N, 1, 28, 28)$ or $(N, 784)$ depending on model
    \end{keybox}
\end{frame}

% -----------------------------------------------------------------------------
% Image as Vector
% -----------------------------------------------------------------------------
\begin{frame}{Image as Data: Flattening}
    \centering
    \begin{tikzpicture}[scale=0.8]
        % 2D image
        \node[font=\small,accentcyan] at (0,3) {28×28 Image};
        \draw[thick,accentcyan] (-1.5,-1.5) rectangle (1.5,1.5);
        
        % Grid
        \foreach \i in {-1.2,-0.6,0,0.6,1.2} {
            \draw[accentcyan,opacity=0.3] (\i,-1.5) -- (\i,1.5);
            \draw[accentcyan,opacity=0.3] (-1.5,\i) -- (1.5,\i);
        }
        
        % Some pixels highlighted
        \fill[neonpink,opacity=0.5] (-1.5,0.9) rectangle (-0.9,1.5);
        \fill[neonpink,opacity=0.3] (-0.9,0.9) rectangle (-0.3,1.5);
        
        % Arrow
        \draw[->,ultra thick,neonyellow] (2,0) -- (4,0) node[midway,above] {flatten};
        
        % Vector
        \node[font=\small,neongreen] at (7,3) {784-d Vector};
        \draw[thick,neongreen] (5,1.5) rectangle (5.5,-1.5);
        
        % Vector elements
        \foreach \y in {1.2,0.9,0.6,0.3,0,-0.3,-0.6} {
            \draw[neongreen,opacity=0.3] (5,\y) -- (5.5,\y);
        }
        \fill[neonpink,opacity=0.5] (5,1.2) rectangle (5.5,1.5);
        \fill[neonpink,opacity=0.3] (5,0.9) rectangle (5.5,1.2);
        
        \node[font=\tiny,fgwhite] at (5.25,1.35) {$x_1$};
        \node[font=\tiny,fgwhite] at (5.25,1.05) {$x_2$};
        \node[font=\tiny,fgwhite] at (5.25,0) {...};
        \node[font=\tiny,fgwhite] at (5.25,-1.2) {$x_{784}$};
    \end{tikzpicture}
    
    \vspace{0.3cm}
    
    \begin{infobox}
        Pixel $(i, j) \rightarrow$ vector index $28 \cdot i + j$
        
        Row-major ordering: read left-to-right, top-to-bottom
    \end{infobox}
\end{frame}

% -----------------------------------------------------------------------------
% Data Preprocessing
% -----------------------------------------------------------------------------
\begin{frame}{Preprocessing MNIST}
    \begin{keybox}[Standard Preprocessing]
        \begin{enumerate}
            \item \textbf{Normalization}: Scale to $[0, 1]$ or standardize
            \[
                x' = \frac{x}{255} \quad \text{or} \quad x' = \frac{x - \mu}{\sigma}
            \]
            
            \item \textbf{Reshaping}: 
            \begin{itemize}
                \item MLP: $(N, 784)$ — flat vector
                \item CNN: $(N, 1, 28, 28)$ — keep spatial structure
            \end{itemize}
            
            \item \textbf{Labels}: One-hot encode for cross-entropy
            \[
                y = 3 \rightarrow \mathbf{y} = (0,0,0,1,0,0,0,0,0,0)
            \]
        \end{enumerate}
    \end{keybox}
    
    \begin{funnybox}
        MNIST statistics: $\mu \approx 0.1307$, $\sigma \approx 0.3081$
        
        (These numbers are ML folk knowledge at this point!)
    \end{funnybox}
\end{frame}

% -----------------------------------------------------------------------------
% Historical Performance
% -----------------------------------------------------------------------------
\begin{frame}{MNIST Performance Through History}
    \centering
    \begin{tikzpicture}[scale=0.9]
        \draw[->] (0,0) -- (10,0) node[right] {year};
        \draw[->] (0,0) -- (0,4) node[above] {error \%};
        
        % Timeline markers
        \foreach \x/\year in {1/1998, 3/2003, 5/2010, 7/2015, 9/2020} {
            \draw (\x,0.1) -- (\x,-0.1) node[below,font=\tiny] {\year};
        }
        
        % Error rate line
        \draw[thick,neonpink] (1,3.5) -- (2,2.8) -- (3,2.0) -- (4,1.5) -- (5,1.0) -- (6,0.6) -- (7,0.4) -- (8,0.3) -- (9,0.2);
        
        % Key points
        \fill[neonyellow] (1,3.5) circle (3pt);
        \node[above,font=\tiny,neonyellow] at (1,3.5) {SVM};
        
        \fill[neonyellow] (3,2.0) circle (3pt);
        \node[above,font=\tiny,neonyellow] at (3,2.0) {LeNet};
        
        \fill[neonyellow] (6,0.6) circle (3pt);
        \node[above,font=\tiny,neonyellow] at (6,0.6) {CNN+aug};
        
        \fill[neongreen] (9,0.2) circle (3pt);
        \node[above,font=\tiny,neongreen] at (9,0.2) {0.17\%!};
        
        % Human performance
        \draw[dashed,accentcyan] (0,1.2) -- (10,1.2);
        \node[accentcyan,font=\tiny,right] at (10,1.2) {human};
    \end{tikzpicture}
    
    \vspace{0.2cm}
    
    \begin{infobox}
        \textbf{State-of-art}: 0.17\% error (99.83\% accuracy)
        
        \textbf{Human performance}: ~1-2\% error (some digits are genuinely ambiguous!)
    \end{infobox}
\end{frame}

% -----------------------------------------------------------------------------
% Why MNIST?
% -----------------------------------------------------------------------------
\begin{frame}{Why Start with MNIST?}
    \begin{columns}[T]
        \begin{column}{0.5\textwidth}
            \begin{successbox}[Advantages]
                \begin{itemize}
                    \item Small and fast to train
                    \item Easy to visualize
                    \item Well-studied baseline
                    \item Works on CPU
                    \item Great for learning
                    \item Available everywhere
                \end{itemize}
            \end{successbox}
        \end{column}
        \begin{column}{0.5\textwidth}
            \begin{alertbox}[Limitations]
                \begin{itemize}
                    \item ``Too easy'' for modern methods
                    \item Not representative of real problems
                    \item Grayscale, centered, clean
                    \item Limited variability
                    \item ``MNIST-easy'' is a term!
                \end{itemize}
            \end{alertbox}
        \end{column}
    \end{columns}
    
    \vspace{0.3cm}
    
    \begin{keybox}[Next Steps After MNIST]
        \begin{itemize}
            \item \textbf{Fashion-MNIST}: Clothes instead of digits (same format)
            \item \textbf{CIFAR-10/100}: 32×32 color images, 10/100 classes
            \item \textbf{ImageNet}: 1000 classes, real images
        \end{itemize}
    \end{keybox}
\end{frame}

% -----------------------------------------------------------------------------
% Loading in PyTorch
% -----------------------------------------------------------------------------
\begin{frame}{Loading MNIST in PyTorch}
    \begin{successbox}[PyTorch Implementation]
        {\small\ttfamily
        import torch\\
        from torchvision import datasets, transforms\\
        from torch.utils.data import DataLoader\\[0.3em]
        transform = transforms.Compose([\\
        \quad transforms.ToTensor(),\\
        \quad transforms.Normalize((0.1307,), (0.3081,))\\
        ])\\[0.3em]
        train\_data = datasets.MNIST(root='./data', train=True,\\
        \quad\quad\quad\quad\quad download=True, transform=transform)\\
        test\_data = datasets.MNIST(root='./data', train=False,\\
        \quad\quad\quad\quad\quad download=True, transform=transform)\\[0.3em]
        train\_loader = DataLoader(train\_data, batch\_size=64, shuffle=True)\\
        test\_loader = DataLoader(test\_data, batch\_size=64, shuffle=False)
        }
    \end{successbox}
\end{frame}

% -----------------------------------------------------------------------------
% Simple Model
% -----------------------------------------------------------------------------
\begin{frame}{A Simple MNIST Classifier}
    \begin{successbox}[MLP for MNIST]
        {\small\ttfamily
        import torch.nn as nn\\[0.3em]
        class MNISTClassifier(nn.Module):\\
        \quad def \_\_init\_\_(self):\\
        \quad\quad super().\_\_init\_\_()\\
        \quad\quad self.flatten = nn.Flatten()\\
        \quad\quad self.layers = nn.Sequential(\\
        \quad\quad\quad nn.Linear(784, 256), nn.ReLU(), nn.Dropout(0.2),\\
        \quad\quad\quad nn.Linear(256, 128), nn.ReLU(), nn.Dropout(0.2),\\
        \quad\quad\quad nn.Linear(128, 10)\\
        \quad\quad )\\[0.3em]
        \quad def forward(self, x):\\
        \quad\quad x = self.flatten(x)\\
        \quad\quad return self.layers(x)\\[0.3em]
        \# Typical accuracy: approx 98\% with this simple model!
        }
    \end{successbox}
\end{frame}

% -----------------------------------------------------------------------------
% Key Takeaways
% -----------------------------------------------------------------------------
\begin{frame}{Key Takeaways: MNIST}
    \begin{keybox}
        \begin{enumerate}
            \item \textbf{MNIST}: 70,000 handwritten digits (60k train, 10k test)
            \item \textbf{Format}: 28×28 grayscale → 784-dim vector
            \item \textbf{Preprocessing}:
            \begin{itemize}
                \item Normalize: divide by 255 or use $\mu=0.1307, \sigma=0.3081$
                \item One-hot encode labels for cross-entropy
            \end{itemize}
            \item \textbf{Baseline performance}:
            \begin{itemize}
                \item Simple MLP: ~98\%
                \item CNN: ~99\%
                \item State-of-art: 99.8\%+
            \end{itemize}
            \item \textbf{Perfect for learning} but not representative of real challenges
            \item \textbf{Graduate to}: Fashion-MNIST, CIFAR-10, ImageNet
        \end{enumerate}
    \end{keybox}
    
    \centering
    \textit{Next: Putting it all together — Training Pipeline!}
\end{frame}


% Section 21: Neural Network Training Pipeline
% =============================================================================
% Section 21: Neural Network Training Pipeline
% Putting it all together - the complete training loop
% =============================================================================

\section{Training Pipeline}

% -----------------------------------------------------------------------------
% Opening
% -----------------------------------------------------------------------------
\begin{frame}{The Complete Training Pipeline}
    \begin{funnybox}
        \textit{``We've learned all the ingredients. Now it's time to bake the neural network cake! CAKE''}
    \end{funnybox}
    
    \vspace{0.3cm}
    
    \begin{keybox}[The Training Recipe]
        \begin{enumerate}
            \item \textbf{Data}: Load, preprocess, batch
            \item \textbf{Model}: Define architecture
            \item \textbf{Loss}: Choose objective function
            \item \textbf{Optimizer}: Select update rule
            \item \textbf{Train}: Forward → Loss → Backward → Update
            \item \textbf{Evaluate}: Test set performance
            \item \textbf{Save}: Checkpoint best model
        \end{enumerate}
    \end{keybox}
\end{frame}

% -----------------------------------------------------------------------------
% Pipeline Overview
% -----------------------------------------------------------------------------
\begin{frame}{Training Pipeline Overview}
    \centering
    \begin{tikzpicture}[
        node distance=0.8cm,
        box/.style={rectangle,draw,thick,rounded corners,minimum width=2cm,minimum height=0.7cm,font=\small},
        arrow/.style={->,thick}
    ]
        % Nodes - using text=black for contrast on light backgrounds
        \node[box,fill=neongreen!20,draw=neongreen,text=black] (data) {Data};
        \node[box,fill=accentcyan!20,draw=accentcyan,text=black,right=of data] (model) {Model};
        \node[box,fill=neonpink!20,draw=neonpink,text=black,right=of model] (forward) {Forward};
        \node[box,fill=neonyellow!20,draw=neonyellow,text=black,right=of forward] (loss) {Loss};
        
        \node[box,fill=neonpink!20,draw=neonpink,text=black,below=of loss] (backward) {Backward};
        \node[box,fill=accentcyan!20,draw=accentcyan,text=black,below=of forward] (update) {Update};
        \node[box,fill=neongreen!20,draw=neongreen,text=black,below=of model] (eval) {Evaluate};
        \node[box,fill=neonyellow!20,draw=neonyellow,text=black,below=of data] (save) {Save};
        
        % Arrows
        \draw[arrow,neongreen] (data) -- (model);
        \draw[arrow,accentcyan] (model) -- (forward);
        \draw[arrow,neonpink] (forward) -- (loss);
        \draw[arrow,neonyellow] (loss) -- (backward);
        \draw[arrow,neonpink] (backward) -- (update);
        \draw[arrow,accentcyan] (update) -- (eval);
        \draw[arrow,neongreen] (eval) -- (save);
        
        % Loop back
        \draw[arrow,dashed,neonyellow] (update.west) -- ++(-0.5,0) |- (data.south);
        \node[font=\tiny,neonyellow] at (-1.5,-0.8) {next batch};
    \end{tikzpicture}
    
    \vspace{0.3cm}
    
    \begin{infobox}
        \textbf{Epoch}: One pass through entire training set
        
        \textbf{Iteration/Step}: One batch update
    \end{infobox}
\end{frame}

% -----------------------------------------------------------------------------
% The Training Loop
% -----------------------------------------------------------------------------
\begin{frame}{The Training Loop}
    \begin{successbox}[PyTorch Training Loop]
        {\scriptsize\ttfamily
        def train\_epoch(model, loader, criterion, optimizer, device):\\
        \quad model.train() \# Set to training mode\\
        \quad total\_loss = 0\\
        \quad correct = 0\\[0.2em]
        \quad for batch\_idx, (data, target) in enumerate(loader):\\
        \quad\quad data, target = data.to(device), target.to(device)\\[0.2em]
        \quad\quad optimizer.zero\_grad() \# Clear gradients\\
        \quad\quad output = model(data) \# Forward pass\\
        \quad\quad loss = criterion(output, target)\\
        \quad\quad loss.backward() \# Backward pass\\
        \quad\quad optimizer.step() \# Update weights\\[0.2em]
        \quad\quad total\_loss += loss.item()\\
        \quad\quad pred = output.argmax(dim=1)\\
        \quad\quad correct += (pred == target).sum().item()\\[0.2em]
        \quad return total\_loss / len(loader), correct / len(loader.dataset)
        }
    \end{successbox}
\end{frame}

% -----------------------------------------------------------------------------
% Training vs Evaluation Mode
% -----------------------------------------------------------------------------
\begin{frame}{Training vs Evaluation Mode}
    \begin{columns}[T]
        \begin{column}{0.5\textwidth}
            \begin{defbox}[\texttt{model.train()}]
                \begin{itemize}
                    \item Dropout \textbf{active}
                    \item BatchNorm uses \textbf{batch} statistics
                    \item Gradients computed
                \end{itemize}
                
                \vspace{0.2cm}
                
                Use during: \textbf{Training}
            \end{defbox}
        \end{column}
        \begin{column}{0.5\textwidth}
            \begin{defbox}[\texttt{model.eval()}]
                \begin{itemize}
                    \item Dropout \textbf{disabled}
                    \item BatchNorm uses \textbf{running} statistics
                    \item Can disable gradient computation
                \end{itemize}
                
                \vspace{0.2cm}
                
                Use during: \textbf{Validation/Test}
            \end{defbox}
        \end{column}
    \end{columns}
    
    \vspace{0.3cm}
    
    \begin{alertbox}[Common Mistake!]
        Forgetting to switch between train/eval mode is a very common bug!
        
        Always call \texttt{model.eval()} before testing and \texttt{model.train()} before training.
    \end{alertbox}
\end{frame}

% -----------------------------------------------------------------------------
% Evaluation Loop
% -----------------------------------------------------------------------------
\begin{frame}{The Evaluation Loop}
    \begin{successbox}[PyTorch Evaluation]
        {\scriptsize\ttfamily
        @torch.no\_grad() \# Disable gradient computation\\
        def evaluate(model, loader, criterion, device):\\
        \quad model.eval() \# Set to evaluation mode\\
        \quad total\_loss = 0\\
        \quad correct = 0\\[0.2em]
        \quad for data, target in loader:\\
        \quad\quad data, target = data.to(device), target.to(device)\\
        \quad\quad output = model(data)\\
        \quad\quad loss = criterion(output, target)\\
        \quad\quad total\_loss += loss.item()\\
        \quad\quad pred = output.argmax(dim=1)\\
        \quad\quad correct += (pred == target).sum().item()\\[0.2em]
        \quad accuracy = correct / len(loader.dataset)\\
        \quad avg\_loss = total\_loss / len(loader)\\
        \quad return avg\_loss, accuracy
        }
    \end{successbox}
\end{frame}

% -----------------------------------------------------------------------------
% Complete Training Script
% -----------------------------------------------------------------------------
\begin{frame}{Complete Training Script}
    \begin{successbox}[Full Training Pipeline]
        {\scriptsize\ttfamily
        device = torch.device('cuda' if torch.cuda.is\_available() else 'cpu')\\
        model = MNISTClassifier().to(device)\\
        criterion = nn.CrossEntropyLoss()\\
        optimizer = torch.optim.Adam(model.parameters(), lr=0.001)\\
        scheduler = torch.optim.lr\_scheduler.StepLR(optimizer, step\_size=5, gamma=0.5)\\[0.2em]
        best\_acc = 0\\
        for epoch in range(num\_epochs):\\
        \quad train\_loss, train\_acc = train\_epoch(model, train\_loader,\\
        \quad\quad\quad\quad\quad\quad\quad\quad\quad\quad criterion, optimizer, device)\\
        \quad val\_loss, val\_acc = evaluate(model, val\_loader, criterion, device)\\
        \quad scheduler.step()\\
        \quad print(f'Epoch \{epoch\}: Train Loss=\{train\_loss:.4f\}, Val Acc=\{val\_acc:.4f\}')\\[0.2em]
        \quad if val\_acc > best\_acc:\\
        \quad\quad best\_acc = val\_acc\\
        \quad\quad torch.save(model.state\_dict(), 'best\_model.pth')
        }
    \end{successbox}
\end{frame}

% -----------------------------------------------------------------------------
% Monitoring Training
% -----------------------------------------------------------------------------
\begin{frame}{Monitoring Training Progress}
    \begin{columns}[T]
        \begin{column}{0.5\textwidth}
            \centering
            \textbf{Loss Curves}
            
            \begin{tikzpicture}[scale=0.65]
                \draw[->] (0,0) -- (5,0) node[right] {epoch};
                \draw[->] (0,0) -- (0,3.5) node[above] {loss};
                
                % Training loss
                \draw[thick,neongreen,domain=0:4.5,samples=50] 
                    plot (\x, {2.5*exp(-0.5*\x) + 0.3});
                \node[neongreen,font=\tiny,right] at (4.5,0.5) {train};
                
                % Validation loss (with some overfitting)
                \draw[thick,neonpink,domain=0:4.5,samples=50] 
                    plot (\x, {2.7*exp(-0.4*\x) + 0.4 + 0.1*\x});
                \node[neonpink,font=\tiny,right] at (4.5,1.2) {val};
            \end{tikzpicture}
            
            \begin{keybox}[Watch For]
                \begin{itemize}
                    \item Val loss increasing → overfitting
                    \item Loss plateau → lower LR
                    \item Loss spikes → LR too high
                \end{itemize}
            \end{keybox}
        \end{column}
        \begin{column}{0.5\textwidth}
            \centering
            \textbf{Accuracy Curves}
            
            \begin{tikzpicture}[scale=0.65]
                \draw[->] (0,0) -- (5,0) node[right] {epoch};
                \draw[->] (0,0) -- (0,3.5) node[above] {acc};
                
                % Training accuracy
                \draw[thick,neongreen,domain=0:4.5,samples=50] 
                    plot (\x, {3*(1-exp(-0.8*\x))});
                \node[neongreen,font=\tiny,right] at (4.5,2.9) {train};
                
                % Validation accuracy
                \draw[thick,neonpink,domain=0:4.5,samples=50] 
                    plot (\x, {2.5*(1-exp(-0.7*\x))});
                \node[neonpink,font=\tiny,right] at (4.5,2.3) {val};
                
                % Gap indicating overfitting
                \draw[<->,dashed,neonyellow] (4,2.85) -- (4,2.2);
                \node[neonyellow,font=\tiny,right] at (4.1,2.5) {gap};
            \end{tikzpicture}
            
            \begin{infobox}[Gap = Overfitting]
                Large train-val gap means model memorizing, not learning!
            \end{infobox}
        \end{column}
    \end{columns}
\end{frame}

% -----------------------------------------------------------------------------
% Checkpointing
% -----------------------------------------------------------------------------
\begin{frame}{Saving and Loading Models}
    \begin{successbox}[Model Checkpointing]
        {\scriptsize\ttfamily
        \# Save model weights only\\
        torch.save(model.state\_dict(), 'model\_weights.pth')\\[0.2em]
        \# Load weights\\
        model = MNISTClassifier()\\
        model.load\_state\_dict(torch.load('model\_weights.pth'))\\[0.2em]
        \# Save complete checkpoint (for resuming training)\\
        checkpoint = \{\\
        \quad 'epoch': epoch,\\
        \quad 'model\_state\_dict': model.state\_dict(),\\
        \quad 'optimizer\_state\_dict': optimizer.state\_dict(),\\
        \quad 'loss': loss,\\
        \quad 'best\_acc': best\_acc\\
        \}\\
        torch.save(checkpoint, 'checkpoint.pth')\\[0.2em]
        \# Load and resume\\
        checkpoint = torch.load('checkpoint.pth')\\
        model.load\_state\_dict(checkpoint['model\_state\_dict'])\\
        optimizer.load\_state\_dict(checkpoint['optimizer\_state\_dict'])
        }
    \end{successbox}
\end{frame}

% -----------------------------------------------------------------------------
% Best Practices
% -----------------------------------------------------------------------------
\begin{frame}{Training Best Practices}
    \begin{keybox}[Tips for Successful Training]
        \begin{enumerate}
            \item \textbf{Start simple}: Small model, verify it can overfit one batch
            \item \textbf{Monitor everything}: Loss, accuracy, gradients, learning rate
            \item \textbf{Use validation set}: Never tune on test set!
            \item \textbf{Checkpoint regularly}: Save best model by validation metric
            \item \textbf{Reproducibility}: Set random seeds
            \item \textbf{Normalize data}: Huge impact on training stability
            \item \textbf{Learning rate}: Often the most important hyperparameter
        \end{enumerate}
    \end{keybox}
    
    \begin{funnybox}
        ``If in doubt, lower the learning rate.'' — Ancient ML Wisdom
    \end{funnybox}
\end{frame}

% -----------------------------------------------------------------------------
% Debugging Tips
% -----------------------------------------------------------------------------
\begin{frame}{Debugging Training Issues}
    \begin{columns}[T]
        \begin{column}{0.5\textwidth}
            \begin{alertbox}[Common Problems]
                \begin{itemize}
                    \item \textbf{Loss = NaN}
                    \begin{itemize}
                        \item LR too high
                        \item Divide by zero
                        \item Log of zero
                    \end{itemize}
                    \item \textbf{Loss doesn't decrease}
                    \begin{itemize}
                        \item LR too low
                        \item Bug in forward pass
                        \item Wrong loss function
                    \end{itemize}
                    \item \textbf{Overfitting}
                    \begin{itemize}
                        \item Add regularization
                        \item More data
                        \item Smaller model
                    \end{itemize}
                \end{itemize}
            \end{alertbox}
        \end{column}
        \begin{column}{0.5\textwidth}
            \begin{successbox}[Debugging Checklist]
                \begin{enumerate}
                    \item Can model overfit 1 batch?
                    \item Are labels correct?
                    \item Is data normalized?
                    \item Are gradients flowing?
                    \item Is loss function appropriate?
                    \item Is train/eval mode correct?
                    \item Are shapes matching?
                \end{enumerate}
            \end{successbox}
        \end{column}
    \end{columns}
\end{frame}

% -----------------------------------------------------------------------------
% Key Takeaways
% -----------------------------------------------------------------------------
\begin{frame}{Key Takeaways: Training Pipeline}
    \begin{keybox}
        \begin{enumerate}
            \item \textbf{Training loop}: Forward → Loss → Backward → Update
            \item \textbf{Modes}: Always toggle train/eval appropriately
            \item \textbf{Evaluation}: Use \texttt{@torch.no\_grad()} for efficiency
            \item \textbf{Monitoring}: Track train \& val loss/accuracy
            \item \textbf{Checkpointing}: Save best model, enable resuming
            \item \textbf{Best practices}:
            \begin{itemize}
                \item Start simple, verify overfitting
                \item Never tune on test set
                \item Learning rate is crucial
            \end{itemize}
        \end{enumerate}
    \end{keybox}
    
    \centering
    \vspace{0.3cm}
    {\Large \textbf{Congratulations! You've completed Day 0! PARTY}}
    
    \vspace{0.2cm}
    
    \textit{Next: Day 1 — Hands-on implementation and advanced topics!}
\end{frame}

% -----------------------------------------------------------------------------
% Workshop Summary
% -----------------------------------------------------------------------------
\begin{frame}{Day 0 Summary: The Journey So Far}
    \begin{columns}[T]
        \begin{column}{0.5\textwidth}
            \begin{keybox}[Mathematical Foundations]
                \begin{itemize}
                    \item Numbers \& counting
                    \item Real \& complex numbers
                    \item Functions \& parameters
                    \item Limits \& continuity
                    \item Differentiation
                    \item Integration
                    \item Graph theory
                \end{itemize}
            \end{keybox}
        \end{column}
        \begin{column}{0.5\textwidth}
            \begin{keybox}[Machine Learning]
                \begin{itemize}
                    \item Neural network basics
                    \item Forward propagation
                    \item Loss functions
                    \item Backpropagation
                    \item Optimization algorithms
                    \item Regularization
                    \item Decision trees \& boosting
                    \item Statistical learning
                    \item Information theory
                \end{itemize}
            \end{keybox}
        \end{column}
    \end{columns}
    
    \vspace{0.3cm}
    
    \begin{successbox}
        You now have the foundation to understand, implement, and debug neural networks from scratch!
    \end{successbox}
\end{frame}


% =============================================================================
% CLOSING
% =============================================================================

\begin{frame}[plain]
    \begin{tikzpicture}[remember picture,overlay]
        \shade[top color=bgdark,bottom color=bgblack]
            (current page.north west) rectangle (current page.south east);
    \end{tikzpicture}
    \centering
    \vfill
    \glowtitle[accentcyan]{Thank You!}
    \vspace{1cm}
    
    {\Large\color{fgwhite}Questions?}
    
    \vspace{1cm}
    
    \begin{tikzpicture}
        \shade[left color=bgblack,right color=bgblack,middle color=neonpink]
            (-4cm,0) rectangle (4cm,1pt);
    \end{tikzpicture}
    
    \vspace{0.5cm}
    
    {\color{softgray}\small Shuvam Banerji Seal | ML Workshop 2026}
    
    \vfill
\end{frame}

\end{document}
